\chapter{Proposta}\label{cap3}



Essa seção apresenta as etapas de desenvolvimento do sistema,  bem como o seu funcionamento geral, desde a preparação dos documentos até a entrega dos históricos de ocorrência ao usuário. Inicialmente será descrito a seleção e pré-processamento. Em seguida, será relatado como as técnicas de mineração de texto e resgate de informação são utilizadas nesse trabalho.


O objetivo do sistema é permitir ao usuário consultar uma coleção de documentos de reuniões a fim de obter todo o histórico de ocorrências de um determinado tema pesquisado, podendo identificar nos documentos onde o tema foi mencionado como informe ou onde houve uma decisão sobre o tema. Para isso, o sistema é divido em dois módulos principais: Módulo de preparação e manutenção e Módulo de consulta.




\section{Preparação dos documentos}

As atas são normalmente armazenadas em arquivos do tipo \textit{pdf}, \textit{doc}, \textit{docx} ou \textit{odt} que normalmente possuem formato binário.
O texto deve ser preparado para os métodos de MT e RI. Inicialmente, o texto puro é extraído e passa por processos de transformação conforme apresentados a seguir.



\begin{enumerate}

%  Cabeçalhos e rodapés
\item Remoção de cabeçalhos e rodapés: as atas contém trechos que podem ser considerados pouco informativos e descartados durante o pré-processamento, como cabeçalhos e rodapés que se misturam aos tópicos tratados na reunião, podendo ser  inseridos no meio de um tópico prejudicando tanto os algoritmos de MT e RI, quanto a leitura do texto pelo usuário.

%  Identificação de sentenças
\item Identificação de finais sentenças: devido ao estilo de pontuação desses documentos, como encerrar sentenças usando um \textit{";"} e inserção de linhas extras, foram usadas as regras especiais para identificação de finais de sentença. Cada final de sentença é identificado e marcado com uma \textit{string} especial, esse processo é melhor descrito na Subseção~\ref{subsec:indentificacaosentencas}.


% Os detalhes sobre essas regras estão disponíveis para consulta em \urlsoftwares.

%  Remoção de termos
\item Redução de termos: eliminou-se as \textit{stop words} por meio de uma lista de 438 palavras. Além disso, eliminou-se a acentuação, sinais de pontuação, numerais e todos os \textit{tokens} menores que três caracteres. 

%  Stemming
\item \textit{Stemming}: extraiu-se o radical de cada palavra. Para isso, as letras foram convertidas em caixa baixa e aplicou-se o algoritmo \textit{Orengo}\footnote{\urlorengo} para remoção de sufixos.

\end{enumerate}



Ao final ...


\subsection{Segmentação}


Como já mencionado, uma ata registra a sucessão de assuntos discutidos em uma reunião, porém apresenta-se com poucas quebras de parágrafo e sem marcações de estrutura, como capítulos, seções ou quaisquer indicações sobre o assunto do texto. Portanto, faz-se necessário descobrir quando há uma mudança de assunto no texto da ata. Para isso, considerou-se os algoritmos \textit{TextTiling} e \textit{C99)} os quais foram alvaliados no contexto das atas de reunião conforme mostrado a seguir.

% usando um conjunto de documentos e uma segmentação manual fornecida por participantes das reuniões. 




%  ==========   ==========   ==========   ==========   ==========   


\subsection{Avaliação dos Segmentadores}


%  Critérios de avaliação


Para fins de avaliação desse trabalho, um bom método de segmentação é aquele cujo resultado melhor se aproxima de uma segmentação manual, sem a obrigatoriedade de estar perfeitamente alinhado com tal. Ou seja, visto o contexto das atas de reunião, e a subjetividade da tarefa, não é necessário que os limites entre os segmentos (real e hipótese) sejam idênticos, mas que se assemelhem em localização e quantidade.

Para que se possa avaliar um segmentador automático de textos é preciso uma referência, isto é, um texto com os limites entre os segmento conhecidos. Essa referência, deve ser confiável, sendo uma segmentação legítima que é capaz de dividir o texto em porções relativamente independentes, ou seja, uma segmentação ideal.

Os algoritmos foram comparados com a segmentação fornecida pelos participantes das reuniões. Calculou-se as medidas mais aplicadas à segmentação textual, P$_k$ e \textit{WindowDiff}. Além dessas, computou-se também as medidas tradicionais acurácia, precisão, revocação e $F^1$ para comparação com outros trabalhos que as utilizam.


% --------------------------


De maneira geral, o algoritmo \textit{C99} apresentou melhores resultados em relação ao \textit{TextTiling}, sobre tudo quando aplicado o pré-processamento, contudo, testes estatísticos realizados indicaram que não houve diferença significativa entre os métodos. A etapa de pré-processamento proporciona melhora de desempenho quando aplicada, porém o seu maior benefício é a diminuição do custo computacional, uma vez que não prejudica a qualidade dos resultados.


As medidas de avaliação tradicionais, podem não ser confiáveis, por não considerarem a distância entre os limites, mas penalizam o algoritmo sempre que um limite que não coincide perfeitamente com a referência. Essas medidas podem ser mais adequadas quando necessita-se de segmentações com maior exatidão. 


As medidas \textit{WindowDiff} e P$_k$, consideram a quantidade e proximidade entre os limites, sendo mais tolerantes a pequenas imprecisões. Essa é uma característica desejável, visto que as segmentações de referência possuem diferenças consideráveis. 
%
\textit{WindowDiff} equilibra melhor os falsos positivos em relação a \textit{near misses}, ao passo que P$_k$ os penaliza com peso maior. Isso significa que segmentadores melhores avaliados em P$_k$ ajudam a selecionar as configurações que erram menos ao separar trechos de texto com o mesmo assunto, enquanto \textit{WindowDiff} é mais tolerante nesse aspecto.

Observa-se melhores resultados de \textit{WindowDiff} quando os algoritmos aproximam a quantidade de segmentos automáticos da quantidade de segmentos da referência. %em torno de 10. 
Por outro lado, observa-se que P$_k$ avalia melhor as configurações que retornam menos segmentos. A configuração do tamanho do passo (P) e da proporção de segmentos em relação ao número de candidatos (S), influenciam os algoritmos na quantidade de segmentos extraídos. Contudo, não é possível definir um valor adequado, uma vez que os segmentadores humanos frequentemente apontam valores diferentes.











