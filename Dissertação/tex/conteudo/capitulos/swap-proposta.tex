


% O \textit{TextTiling} obteve o melhor desempenho em revocação, utilizando-se janelas de tamanho igual a $20$ e passo igual a $3$, onde registrou-se uma média de $0,917$ para essa medida.
%
%
%Embora haja melhora de desempenho com os valores apresentados, os testes apontam que não há diferença significativa de performance entre as configurações. O pré-processamento é capaz de reduzir o tamanho do texto mantendo a qualidade final dos algoritmos, sobre tudo em textos longos e grandes coleções de documentos.
%
%
%%%%%%%%%%
% Definição do que é um bom algoritmo de segmentação
%%%%%%%%%%
%Para fins de avaliação desse trabalho, um bom método de segmentação é aquele cujo resultado melhor se aproxima de uma segmentação manual, sem a obrigatoriedade de estar perfeitamente alinhado com tal. 
%
%
%
%Dado o contexto das atas de reunião, e a subjetividade da tarefa de segmentação, não é necessário que os limites entre os segmentos (real e hipótese) sejam idênticos, mas que se assemelhem em localização e quantidade.
% 
%
%Nesse sentido, 


% Desempenho dos métodos 
% 
% C99 é um pouco melhor mas tem mais complexidade
% 



% Na maioria das medidas, o algoritmo \textit{C99} sobressaiu-se em relação ao \textit{TextTiling}, contudo, testes estatísticos realizados indicaram que não houve diferença significativa. 

%%%%%%%%%%%%%%%
% O Impacto do Preprocessamento 
%%%%%%%%%%%%%%%
%Da mesma forma, a etapa de preprocessamento proporciona melhora de performance quando aplicada, porém o seu maior benefício é a diminuição do custo computacional, uma vez que não prejudica a qualidade dos resultados.



%Em ambos os algoritmos os parâmetros que mais influenciam na quantidade de segmentos extraídos são para o \textit{TextTiling} o tamanho da janela e para o \textit{C99} a proporção de segmentos em relação a quantidade de candidatos. Observa-se que a quantidade de segmentos automáticos melhora os resultados principalmente e P$_k$ e \textit{WindowDiff}. 


%Melhores resultados foram percebidos quando os algoritmos aproximam a quantidade de segmentos automáticos da quantidade de segmentos da referência. Embora os parâmetros \textit{S} e \textit{J} influenciem na quantidade de segmentos extraídos, não se conhece a 

%melhores resultados de foram percebido com X e Y segmntos extraidos
%
%o que reforça o 
%
%Contudo, não possível conhecer esses valores e o parâmetros baseiam-se na quantidade de candidatos e 
%
%O parâmetro \textit{J} influência diretamente 
%
%
%
%A quantidade de segmentos extraídos
%
%A configuração da proporção de segmentos em relação ao número de candidatos (S) influencia diretamente n
%
%

%
%a quantidade de segmentos de referência é subjetiva


%e Precisão melhores resultados quando retorna-se menos segmentos o que pode ser explica 
%
%Contudo, não possível conhecer esses valores e o parâmetros baseiam-se na quantidade de candidatos e 

%Os resultados indicam que os algoritmos apresentam desempenho similar a outros trabalhos~\cite{arabi} 



% Critcar as medidas tradicionais

% Não houve diferença crítica
% Não são confiáveis
% Preferir o windiff por resolver alguns problenas do Pk. O Pk pode ser coniderádo se o falso-positivo for mais importante.
% O Pk penaliza mais quando um limite é identificado quando não existe.

% a metodologia + as ferra  mentas disponiblizadas podem ser úteis para outros trabalhos

% Poderia colocar mais uma tabela referente ao terceiro teste, mas acho desnecessário



% refletir melhor a performance 








% Na Tabela~\ref{tab:texttilingsempreprocess} são apresentadas, para cada medida, as configurações que otimizam o \textit{TextTiling} e a média computada considerando as bases, onde \textbf{J} é o tamanho da janela e \textbf{P} é o passo.






