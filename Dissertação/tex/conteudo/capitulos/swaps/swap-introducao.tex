
% Reuniões são tarefas presentes em atividades corporativas, ambientes de gestão e organizações de um modo geral, onde discute-se problemas, soluções, propostas, planos, questionamentos, alterações de projetos e frequentemente são tomadas decisões importantes. A comunicação entre os membros da reunião é feita de forma majoritariamente verbal. Para que seu conteúdo possa ser registrado e externalizado, adota-se a prática de registrar seu conteúdo em documentos, chamamos atas. 

% A navegação pelo documento pode ser aprimorada, em especial na utilização por usuários com deficiência visual, os quais utilizam  sintetizadores de texto como ferramenta de acessibilidade.




% -- Motivação
% -< Classificação
% -- Tópicos
% -- Gaps da área
% -- Objetivos


% -- Objetivos 
Assim, este trabalho tem como foco principal empregar técnicas de Recuperação de informação para encontrar histórico de assuntos mencionados em atas de reunião. Tal sistema tem três principais tarefas: 
1) Descobrir quando há uma mudança de assunto. 
2) Descobrir quais são esses assuntos. 
3) Fornecer um mecanismo de busca ao usuário. 



Uma ata frequentemente trada de vários assuntos

Uma vez que a ata trata de vários assuntos, 
há interesse em um sistema que aponte trechos de uma ata que tratam de um assunto específico. 

Assim, este trabalho tem como foco principal, a detecção de mudança de assuntos, que pode ser atendida pela segmentação automática de textos~\cite{Chen2017,Naili2016,Cardoso2017}, técnicas de mineração de texto a fim de extrair informações interessantes e técnicas de resgate de informação a fim de apresentar resultados mais relevantes. 



Quanto ao conteúdo, registram os principais assuntos tratados durante reuniões de uma organização como a sequência de acontecimentos sobre o que foi decidido e debatido durante as reuniões. Frequentemente são mantidas em arquivos digitais e disponibilizadas para consulta posterior.  Além disso, o estilo formal contribui para a escrita de textos com poucos detalhes, pois o ambiente dá preferência a textos sucintos.  Assim, as atas são utilizadas como fonte de consulta, referência e embasamento para o funcionamento da organização, bem como para outros documentos e decisões.



% 1) transfere certa complexidade da tarefa ao usuário.  % 2) buscas em grandes coleções de documentos tornam-se mais lentas.  % 3) não há suporte a padrões mais flexíveis como a proximidade entre as palavras ou palavras que estejam na mesma sentença.  % 4) o retorno ao usuário são os documento integrais, o que pode exigir uma segunda busca dentro de um documento para encontrar o trecho desejado.  % 5) não permite o ranqueamento dos resultados, que se apresentam sem indicações da relevância com a intenção do usuário.  % 6) não trata a string de busca % 7) sinonímia e polissemia 

não trata a string de busca --> falar da string de busca?

% -< ... que tambem é observando na produção de atas de reunião

%        ========== ==========   Descrição das Atas   ========== ==========
%        ========== ==========   Busca manual   ========== ==========



% de termos exatos que devem necessariamente estar contidos no trecho onde está o assunto.
% falar de Boolean Retrieval Models?



Usualmente, são feitas buscas manuais guiadas pela memória ou com uso de ferramentas computacionais baseadas em localização de palavras-chave. Normalmente esse tipo de busca exige a inserção de termos exatos e apresentam ao usuário um documento com as palavras-chave em destaque, mantendo o texto longo, o que dificulta por exemplo o ranqueamento por relevância. 


, e buscas pelo conteúdo dos textos são desaviadoras e exigem ferramentas especializadas.


Reuniões são tarefas presentes em atividades corporativas, ambientes de gestão e organizações de um modo geral, onde discute-se problemas, soluções, propostas, planos, questionamentos, alterações de projetos e decisões são tomadas. 
% controle,
% discussões,
% agendamento
% recomendações, sugestões



As atas, apresentam-se como uma sucessão de tópicos. Assim, o objetivo desse trabalho é identificar, automaticamente, onde há a mudança de um tópico para seus adjacentes.



%        ========== ==========   Motivação   ========== ==========
% conclusão da motivação?

As atas são documentos textuais que em geral descrevem dados não estruturados. Assim, um sistema que responde a consultas do usuário ao conteúdo das atas, retornando trechos de textos relevantes à sua intenção, é um desafio que envolve a compreensão de seu conteúdo~\cite{Bokaei2015}. 


%        ========== ==========   Busca manual   ========== ==========

A busca por informações específicas nesses documentos torna-se uma tarefa custosa se realizada manualmente, especialmente considerando o seu crescimento em uma instituição~\cite{Lee2011, Masakazu2013, Miriam2013}.





Não evita que o texto circundante seja exibido, caso pertença a outro assunto
Além disso, não evita que textos próximos sejam retornados.

A tarefa de segmentação textual consiste em encontrar pontos onde há mudança de tópicos no texto.  %Em outras palavras é identificar divisões entre unidade de informação sucessivas



 Os documentos apresentam um texto com poucas quebras de parágrafo e sem marcações de estrutura, como capítulos, seções ou quaisquer indicações sobre o assunto do texto. É comum a presença de cabeçalhos, rodapés e numeração de páginas e linhas o que pode prejudicar tanto similaridade entre sentenças como a apresentação dos segmentos ao usuário. %Esses elementos podem ser reduzidos ou eliminados como mostrado na Subseção~\ref{subsec:preprocessamento}, sobre preprocessamento.



O estilo de escrita favorece a escrita de textos sucintos com poucos detalhes, pois o ambiente dá preferência a textos curtos. Ao redigir o documento tem-se o cuidado de não repetir ideias e palavras. Tal característica enfraquece a coesão léxica e portanto o cálculo da similaridade é prejudicado. Por exemplo, duas sentenças diferem se uma contiver a palavra \textit{computadores} e na seguinte \textit{equipamentos}, mesmo que se refiram à mesma ideia. Além disso, o documento compartilha um certo vocabulário próprio do ambiente onde os assuntos são discutidos e com isso nota-se que os segmentos, embora tratem de assuntos diferentes, são semelhantes em vocabulário.


Para evitar um texto monótono à leitura, ao redigir o documento tem-se o cuidado de não repetir ideias e palavras em favor da elegância do documento. Tal característica enfraquece a coesão léxica e portanto o cálculo da similaridade é prejudicado. Por exemplo, duas sentenças diferem se uma contiver a palavra \textit{computadores} e na seguinte \textit{equipamentos}, mesmo que se refiram à mesma ideia. Além disso, o documento compartilha um certo vocabulário próprio do ambiente onde os assuntos são discutidos e com isso nota-se que os segmentos, embora tratem de assuntos diferentes, são semelhantes em vocabulário.


"Em outras palavras, uma ata é a narração de uma reunião na forma de documento textual."

A presença de ruídos como cabeçalhos, rodapés e numeração de páginas e linhas prejudicam tanto similaridade entre sentenças como a apresentação final ao usuário. Porém, esses ruídos podem ser reduzidos ou eliminados como mostrado na Subseção~\ref{subsec:preprocessamento}, sobre preprocessamento.


 % ------------------
  Cabeçalhos e rodapés
  % É comum a presença de cabeçalhos, rodapés e numeração de páginas e linhas o que pode prejudicar tanto similaridade entre sentenças como a apresentação dos segmentos ao usuário. %Esses elementos podem ser reduzidos ou eliminados como mostrado na Subseção~\ref{subsec:preprocessamento}, sobre preprocessamento.
 % ------------------


% O teor das atas é diversificado, contendo menções a assuntos como agendamentos, sugestões, reclamações, informes e, com maior atenção, as que foram decisões tomadas pelos participantes. 





Citação [7]

Pesquisar sobre o IBIS --> um método pra apoioar discussões em grupo.
IBIS exige um modelo retórico para garantir seus resultados.

 esse registro torna-se oficial, e serve de referência para outros membros. 

 com o passar do tempo, as decisões eventos perdem-se pela memória ou pelos membros que saem



Para essa classe de problema, frequentemente tem sido utilizadas técnicas de mineração de texto, por meio de diversas abordagens.  


