o módulo de preparação/manutenção recebe como entrada os documentos que compõe a base de consultas e produz uma estrutura de dados interna que será utilizada pelo módulo de consulta. Resumidamente, divide cada documento em segmentos de texto, que contêm um assunto predominante. Em seguida cada segmento será classificado de acordo o tipo da ocorrência de cada assunto, que pode ser, por exemplo, uma decisão ou um informe. Assim, a estrutura de dados interna registrará quais assuntos foram tratados na reunião, bem como o trecho do documento onde é discutido e seu tipo. Os tipos de ocorrência, que serão tratadas pelo sistema, serão ainda estabelecidos em conjunto com usuários especialistas, com base em uma amostra de documentos.

Uma vez que os segmentos estejam rotulados e representados computacionalmente, a tarefa é identificar qual assunto está presente em cada trecho e qual é o tipo de ocorrência. A fim de encontrar o principal assunto abordado em um trecho, pretende-se usar uma ferramenta de extração de tópicos para produzir uma base de trecho/tópicos, a qual será parte da estrutura de dados interna.

Essa estrutura irá descrever cada segmento de texto. Essa descrição irá conter, para cada segmento, dados do documento do qual foi extraído, bem como seus rótulos que indicarão o tipo da ocorrência e os tópicos que são tratados nesse fragmento de documento.

De maneira geral, esse módulo divide cada documento em segmentos de texto, que contêm um assunto predominante. 


Em seguida cada segmento será classificado de acordo o tipo da ocorrência de cada assunto, que pode ser, por exemplo, uma decisão ou um informe. Assim, a estrutura de dados interna registrará quais assuntos foram tratados na reunião, bem como o trecho do documento onde é discutido e seu tipo. Os tipos de ocorrência, que serão tratadas pelo sistema, serão ainda estabelecidos em conjunto com usuários especialistas, com base em uma amostra de documentos.



% O módulo de preparação e manutenção, tem como funções principais, 
% receber as atas, extrair e segmentar o texto, realizar o pré-processamento,  produzir uma representação matricial dos segmentos das atas originais, modelar um classificador/extrair os tópicos e entregar ao módulo de consulta a estrutura de dados interna. 


















Entre os principais trabalhos da literatura podemos citar o  \textit{TextTiling}~\cite{Hearst1994} e o \textit{C99}~\cite{Choi2000}.
%
%%%%%%%%%%%%%%%%%%%%%%%%%%%%%%%%%%%%%%%%%%%%%%%
%%%              TextTiling                 %%%
%%%%%%%%%%%%%%%%%%%%%%%%%%%%%%%%%%%%%%%%%%%%%%%
O \textit{TextTiling} é um algoritmo baseado em janelas deslizantes, em  que, para cada candidato a limite, analisa-se o texto circundante. Um limite ou quebra de segmento é identificado sempre que a similaridade cai abaixo de um limiar.
% \textit{threshold}. %TODO - explicar como se encontra os vales


O \textit{TextTiling} recebe uma lista de candidatos a limite, usualmente finais de parágrafo ou finais de sentenças. Para cada posição candidata são construídos 2 blocos, um contendo sentenças que a precedem e outro com as que a sucedem. O tamanho desses blocos é um parâmetro a ser fornecido ao algoritmo e determina o tamanho mínimo de um segmento.
%
Em seguida, os blocos de texto são representados por vetores que contém as frequências de suas palavras. Então, usa-se cosseno (Equação~\ref{equ:cosine}) para calcular a similaridade entre os blocos adjacentes a cada candidato e identifica-se uma transição entre tópicos pelos vales na curva de dissimilaridade.

%TODO como apresentado na Figura~\ref{fig:curvadedissimilaridade}.


O \textit{TextTiling} possui baixa complexidade computacional. Por outro lado, algoritmos mais complexos, como os baseados em matrizes de similaridade, apresentam acurácia relativamente superior como apresentado em~\cite{Choi2000, Kern2009, Misra2009}.

%%%%%%%%%%%%%%%%%%%%%%%%%%%%%%%%%%%%%%%%%%%%%%%
%%%                  C99                    %%%
%%%%%%%%%%%%%%%%%%%%%%%%%%%%%%%%%%%%%%%%%%%%%%%

O C99 é um algoritmo baseado em ranking.
% Choi \cite{Choi2000} apresenta um algoritmo baseado em ranking, o \textit{C99}. 
%
Embora muitos trabalhos utilizem matrizes de similaridades para pequenos segmentos, o cálculo de suas similaridades não é confiável, pois uma ocorrência adicional de uma palavra causa um impacto que pode alterar significativamente o cálculo da similaridade~\cite{Choi2000}.
%
Além disso, o estilo da escrita pode não ser constante em todo o texto. Por exemplo, textos iniciais dedicados a introdução costumam apresentar menor coesão do que trechos dedicados a um tópico específico. Portanto, comparar a similaridade entre trechos de diferentes regiões não é apropriado.
% Complexidade O(n²)
Devido a isso, as similaridades não podem ser comparadas em valores absolutos. Então, contorna-se esse problema fazendo uso de \textit{rankings} de similaridade para encontrar os segmentos de texto. 


Inicialmente é construída uma matriz que contém as similaridades de todas as unidades de texto. Em seguida, cada valor na matriz de similaridade é substituído por seu ranking local. Para cada elemento da matriz, seu \textit{ranking} é o número de elementos vizinhos com valor de similaridade menor que o seu.
% O que é a máscara
Então, cada elemento e comparado com seus vizinhos dentro de uma região denominada máscara.
%

Na Figura~\ref{fig:a} é destacado um quadro 3~x~3 de uma matriz em que cada elemento é a similaridade entre duas unidades de informação. 
%
Tomando como exemplo o elemento com valor $0,5$, a mesma posição na matriz de \textit{rankings} terá o valor $4$, pois esse é o número de vizinhos com valores inferiores a $0,5$ dentro do quadro analisado na matriz de similaridades. Da mesma forma, na Figura~\ref{fig:b} para o valor $0,2$ a matriz de \textit{rankings} conterá o valor $1$ na mesma posição.









% ----------------------------------------------------------------------------- 


% O \textit{TextTiling} obteve o melhor desempenho em revocação, utilizando-se janelas de tamanho igual a $20$ e passo igual a $3$, onde registrou-se uma média de $0,917$ para essa medida.
%
%
%Embora haja melhora de desempenho com os valores apresentados, os testes apontam que não há diferença significativa de performance entre as configurações. O pré-processamento é capaz de reduzir o tamanho do texto mantendo a qualidade final dos algoritmos, sobre tudo em textos longos e grandes coleções de documentos.
%
%
%%%%%%%%%%
% Definição do que é um bom algoritmo de segmentação
%%%%%%%%%%
%Para fins de avaliação desse trabalho, um bom método de segmentação é aquele cujo resultado melhor se aproxima de uma segmentação manual, sem a obrigatoriedade de estar perfeitamente alinhado com tal. 
%
%
%
%Dado o contexto das atas de reunião, e a subjetividade da tarefa de segmentação, não é necessário que os limites entre os segmentos (real e hipótese) sejam idênticos, mas que se assemelhem em localização e quantidade.
% 
%
%Nesse sentido, 


% Desempenho dos métodos 
% 
% C99 é um pouco melhor mas tem mais complexidade
% 



% Na maioria das medidas, o algoritmo \textit{C99} sobressaiu-se em relação ao \textit{TextTiling}, contudo, testes estatísticos realizados indicaram que não houve diferença significativa. 

%%%%%%%%%%%%%%%
% O Impacto do Preprocessamento 
%%%%%%%%%%%%%%%
%Da mesma forma, a etapa de preprocessamento proporciona melhora de performance quando aplicada, porém o seu maior benefício é a diminuição do custo computacional, uma vez que não prejudica a qualidade dos resultados.



%Em ambos os algoritmos os parâmetros que mais influenciam na quantidade de segmentos extraídos são para o \textit{TextTiling} o tamanho da janela e para o \textit{C99} a proporção de segmentos em relação a quantidade de candidatos. Observa-se que a quantidade de segmentos automáticos melhora os resultados principalmente e P$_k$ e \textit{WindowDiff}. 


%Melhores resultados foram percebidos quando os algoritmos aproximam a quantidade de segmentos automáticos da quantidade de segmentos da referência. Embora os parâmetros \textit{S} e \textit{J} influenciem na quantidade de segmentos extraídos, não se conhece a 

%melhores resultados de foram percebido com X e Y segmntos extraidos
%
%o que reforça o 
%
%Contudo, não possível conhecer esses valores e o parâmetros baseiam-se na quantidade de candidatos e 
%
%O parâmetro \textit{J} influência diretamente 
%
%
%
%A quantidade de segmentos extraídos
%
%A configuração da proporção de segmentos em relação ao número de candidatos (S) influencia diretamente n
%
%

%
%a quantidade de segmentos de referência é subjetiva


%e Precisão melhores resultados quando retorna-se menos segmentos o que pode ser explica 
%
%Contudo, não possível conhecer esses valores e o parâmetros baseiam-se na quantidade de candidatos e 

%Os resultados indicam que os algoritmos apresentam desempenho similar a outros trabalhos~\cite{arabi} 



% Critcar as medidas tradicionais

% Não houve diferença crítica
% Não são confiáveis
% Preferir o windiff por resolver alguns problenas do Pk. O Pk pode ser coniderádo se o falso-positivo for mais importante.
% O Pk penaliza mais quando um limite é identificado quando não existe.

% a metodologia + as ferra  mentas disponiblizadas podem ser úteis para outros trabalhos

% Poderia colocar mais uma tabela referente ao terceiro teste, mas acho desnecessário



% refletir melhor a performance 








% Na Tabela~\ref{tab:texttilingsempreprocess} são apresentadas, para cada medida, as configurações que otimizam o \textit{TextTiling} e a média computada considerando as bases, onde \textbf{J} é o tamanho da janela e \textbf{P} é o passo.






