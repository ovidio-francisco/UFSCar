\chapter{Introdução}\label{chap:introducao}

% -- Motivação
% -< Classificação
% -- Tópicos
% -- Gaps da área
% -- Objetivos

%%%%%%%%%%
% Descrição das Reuniões
%%%%%%%%%%
Reuniões são tarefas presentes em atividades corporativas, ambientes de gestão e organizações de um modo geral. Seu conteúdo é frequentemente registrado em texto na forma de atas para fins de documentação e posterior consulta. 
A organização e consulta manual desses arquivos torna-se uma tarefa custosa, especialmente considerando o seu crescimento em uma instituição~\cite{Lee2011, Masakazu2013,Miriam2013}. 


%%%%%%%%%%
% Atas são documentos não estruturados e
% diferencial do sistema
%%%%%%%%%%
As atas são documentos textuais que em geral descrevem dados não estruturados. Assim, um sistema que responde a consultas do usuário ao conteúdo das atas, retornando trechos de textos relevantes à sua intenção, é um desafio que envolve a compreensão de seu conteúdo~\cite{Bokaei2015}. 

%%%%%%%%%%%%%%%
% Busca manual por Assuntos
%%%%%%%%%%%%%%% 
%  
%  É custosa 
Devido a fatores como a não estruturação e volume dos textos, a localização de assunto em uma ata é uma tarefa custosa. 
%  
%  Memória e Buscas por KeyWords
Usualmente, o que se faz são buscas manuais guiadas pela memória ou com uso de ferramentas computacionais baseadas em localização de palavras-chave.
%  
%  === Problemas na busca por palavras-chave === 
%      O usuário deve inserir palavras acertadas
Normalmente esse tipo de busca exige a inserção de termos exatos e 
% de termos exatos que devem necessariamente estar contidos no trecho onde está o assunto.
%  
%  O texto continua longo, apenas é apresentado com as palavras destacadas
apresentam ao usuário um documento com as palavras buscadas em destaque, mantendo o texto longo, 
%  
%  Não é possível rankear os resultados
o que dificulta por exemplo o ranqueamento por relevância. 

% Não evita que o texto circundante seja exibido, caso pertença a outro assunto
% Além disso, não evita que textos próximos sejam retornados.



%%%%%%%%%%
% Segmentar e extrair tópicos
%%%%%%%%%%
Uma vez que a ata registra a sucessão de assuntos discutidos na reunião, há interesse em um sistema que aponte trechos de uma ata que tratam de um assunto específico. Tal sistema tem duas principais tarefas: 1) Descobrir quando há uma mudança de assunto. 2) Descobrir quais são esses assuntos. Este trabalho tem como foco principal, a detecção de mudança de assuntos, que pode ser atendida pela segmentação automática de textos~\cite{Chen2017,Naili2016,Cardoso2017}. E a extração de automática de tópicos. 





