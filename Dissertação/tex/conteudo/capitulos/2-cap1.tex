\chapter{Introdução}\label{chap:introducao}

% -- Motivação
% -< Classificação
% -- Tópicos
% -- Gaps da área
% -- Objetivos


%        ========== ==========   Reuniões   ========== ==========

Reuniões são tarefas presentes em atividades corporativas, ambientes de gestão e organizações de um modo geral. Seu conteúdo é frequentemente registrado em texto na forma de atas para fins de documentação e consulta posterior. 



%        ========== ==========   Descrição das Atas   ========== ==========

As atas de reunião possuem características particulares, frequentemente apresentam um texto com poucas quebras de parágrafo e sem marcações de estrutura, como capítulos, seções ou quaisquer indicações sobre o tema do texto. Além disso, o estilo formal contribui para a escrita de textos com poucos detalhes, pois o ambiente dá preferência a textos sucintos. 



Devido a fatores como a não estruturação e volume dos textos, a localização de um assunto em uma coleção de atas é uma tarefa custosa, especialmente considerando o seu crescimento em uma instituição.  Usualmente, são feitas buscas manuais guiadas pela memória ou com uso de ferramentas computacionais baseadas em localização de palavras-chave.  Normalmente esse tipo de busca exige a inserção de termos exatos e 
% falar de Boolean Retrieval Models?
% de termos exatos que devem necessariamente estar contidos no trecho onde está o assunto.
apresentam ao usuário um documento com as palavras buscadas em destaque, mantendo o texto longo, o que dificulta por exemplo o ranqueamento por relevância. 





%        ========== ==========|   Segmentação  |========== ==========


Uma vez que a ata registra a sucessão de assuntos discutidos na reunião, há interesse em um sistema que aponte trechos de uma ata que tratam de um assunto específico. Tal sistema tem duas principais tarefas: 1) Descobrir quando há uma mudança de assunto. 2) Descobrir quais são esses assuntos. Este trabalho tem como foco principal, a detecção de mudança de assuntos, que pode ser atendida pela segmentação automática de textos~\cite{Chen2017,Naili2016,Cardoso2017}. E a extração automática de tópicos. 

A tarefa de segmentação textual consiste em dividir um texto em partes que contenham um significado relativamente independente. Em outras palavras, é identificar as posições nas quais há uma mudança significativa de assuntos. 

A segmentação de textos é útil em aplicações que trabalham com textos sem indicações de quebras de assunto, ou seja, não apresentam seções ou capítulos, como transcrições automáticas de áudio, vídeos e grandes documentos que contêm vários assuntos como atas de reunião e notícias.


Pode ser usada para melhorar o acesso a informação solicitada por meio de uma consulta, onde é possível oferecer porções menores de texto mais relevantes ao invés de exibir um documento grande que pode conter informações menos pertinentes. 
%
%
A navegação pelo documento pode ser aprimorada, em especial na utilização por usuários com deficiência visual, os quais utilizam  sintetizadores de texto como ferramenta de acessibilidade~\cite{Choi2000}. 
%
Além disso, encontrar pontos onde o texto muda de assunto, pode ser útil como etapa de pré-processamento em aplicações voltadas ao entendimento do texto, principalmente em textos longos.




