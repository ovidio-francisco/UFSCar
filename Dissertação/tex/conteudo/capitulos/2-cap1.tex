\chapter{Introdução}\label{chap:introducao}



% -- Motivação

% -- Classificação

% -- Tópicos

% -- Gaps da área



% # Introdução
	
  % - Popularização dos computadores e crescimento de arquivos digitais.
  % - Custo para mantere documentos manualmente. Necessidade ferramentas para resgate de informações digitais.
  % - Importância de processos de extração de conhecimento em coleções textuais.
  % - Dificuldade em processar informações não estruturadas, volumosas e com texto irrelevante.
  % - Produção de textos é popular e presente em muitos ambientes.
    % - A tarefa de consulta manual é custosa e falha. 
  % - Descrição das atas.
    % - Conteúdo
    % - Formato textual usado para consultas posteriores.
    % - Usadas como registro das reuniões usadas como base para outras reuniões e decisões.
  % - Departamento de computação como exemplo. [ex: credenciamento de docentes no programa]
    % - O mesmo assunto pode ser mensionado várias vezes ao longo do tempo, podendo conter alterações. 
    % - É desejavel saber quais decisões foram tomadas a respeito e o histórico do que já foi discutido.
  % - Dificuldade em manter as atas e localizar assuntos discutidos divido a:
    % - Volume de texto.
    % - Falta de estruturação.
    % - As consultas são feitas normalemente com apoio de ferramentas de busca por palavra chave e/ou memória dos usuários.
  % - Técnicas de Mineração de Texto são empregradas a esse tipo de problema. Por exemplo:
    % - Exemplos onde é empregada.
    % - Técnicas que são utilizadas. 
  % - Justificativa do trabalho 


	  
% # Objetivos

  % - O objetivo é desenvolver um sistema de consultas a conteúdos de atas de reuniões.
  % - Justificativa e contribuição do trabalho.
      % - Testar se a aplicação de técnicas de MT nas atas ajuda a responder as consultas do usuário.
  % - Estudar e Empregar tecnicas de MT e IR em sub-documentos das atas de reunião.
      
