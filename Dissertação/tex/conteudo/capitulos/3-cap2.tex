\chapter{Conceituação Teórica}\label{cap2}


A popularidade dos computares permite a criação e compartilhamento de textos onde a quantidade de informação facilmente extrapola a capacidade de humana de leitura e análise de coleções de documentos, estejam eles disponíveis na Internet ou em computadores pessoais. A necessidade de simplificar e organizar grandes coleções de documentos criou uma demanda por modelos de aprendizado de máquina para extração de conhecimento em bases textuais. Para esse fim, foram desenvolvidas técnicas para descobrir, extrair e agrupar textos de grandes coleções, entre essas, a modelagem de tópicos~\cite{Hofmann1999,Deerwester1990,Lee1999,Blei2012}.  %--> não falar só de modelagem de tópicos. Falar de RI com referências



% -> Recuperação de Informação

\section{Modelos de Recuperação de Informação}

% Necessidade de encontrar informação

Devido à popularização dos computadores e a grande disponibilidade de documentos em formato digital, em especial na \textit{web}, a área da Recuperação de Informação (RI) tem recebido atenção de pesquisadores nas últimas décadas. Recuperação de Informação é a área da computação que envolve a aplicação de métodos computacionais no tratamento e busca de informação em bases de dados não estruturados, usualmente grandes coleções de documentos textuais armazenados em dispositivos eletrônicos. De fato, não há dados completamente não estruturados ao se considerar a estrutura linguística latente em documentos textuais. O termo ``não estruturado'' se refere a dados que não oferecem uma estrutura clara para sistemas computadorizados~\cite{Manning2008, Gutierrez2016}.
% , a exemplo de documentos textuais

% Tratamento == Classificação e Agrupamento?

A tarefa central da recuperação de informação é encontrar informações de interesse dos usuários e exibí-las. 
% Essa necessidade motiva o desenvolvimento de sistemas de recuperação de informação (SRI).
Nesses sistemas o usuário expressa sua necessidade por meio da formulação de uma consulta, usualmente composta por um conjunto de palavras-chave. Então, o sistema apresenta os resultados da busca, frequentemente documentos, em ordem de relevância com a consulta.  % TODO: Melhorar


% \subsection{Modelos de Recuperação de Informação}

Um modelo de recuperação de informação deve criar representações de documentos e consultas a fim de predizer a necessidade expressa nos termos da consulta. Com base na entrada do usuário esses modelos buscam por documentos similares aos termos da consulta. Segue abaixo a descrição dos três modelos clássicos para recuperação de informação.

\subsection{Modelo Booleano}

O modelo booleano ou modelo lógico foi um dos primeiros modelos aplicados à Recuperação Informação sendo utilizado a partir de 1960. Nesse modelo uma consulta é considerada uma sequencia de termos conectador por operadores lógicos como AND, OR e NOT. Como resultado, classifica cada documento como relevante ou não relevante à consulta, sem gradação de relevância. Esses operadores lógicos podem ser manipulados por usuários com algum conhecimento em álgebra booleana para aumentar a quantidade de resultados ou restringí-la.

Uma desvantagem desse modelo é que não é possível medir a relevância de um documento em relação a consultado do usuário, devido a essa limitação não há informação que permita a ordenação dos resultados, que é uma característica esperada para muitos sistemas de RI.
Já as vantagens desse modelo são a facilidade de implementação e a possibilidade de usuários experientes usarem os operadores lógicos como uma forma de controle sobre os resultados da busca. Por outro lado, para usuários inexperientes isso pode ser considerado uma desvantagem, uma vez que o uso de expressões lógicas não é intuitivo. Apesar dos problemas apresentados, visto sua simplicidade, esse modelo foi largamente utilizado em sistemas comerciais. 

% Falar do Booleano Estendido???


\subsection{Modelo Espaço Vetorial}
\label{subsec:modeloespacovetorial}

Uma das formas mais comuns para representação textual é conhecida como Modelo Espaço Vetorial (\textit{Vectorial Space Model} - VSM), onde os documentos e consultas são representados como vetores em um espaço Euclidiano $m$-dimensional em que cada termo extraído da coleção é representado por uma dimensão~\cite{Rezende2003}. 
% 
Considera-se que um documento pode ser representado pelo seu conjunto de termos, onde cada termo $t_i$ de um documento $d_j$ associa-se um peso $w_{ij}\geq0$ que indica a importância desse termo no documento. 
%
De forma similar, para uma consulta $q$, associa-se um peso $w_{i,q}$ a cada termo consulta. 
%
Assim o vetor associado ao documento $d_j$ é dado por $\vec{d}_{j} = (w_{j,1}, w_{j,2}, ..., w_{j,p})$ 
%
e o vetor associado a consulta $q$ é dado por $\vec{q} = (w_{q,1}, w_{q,2}, ..., w_{q,l})$.
%
% -- Movido para "Medidas de Proximidade " ↓↓↓↓
No modelo vetorial, a similidade entre um documento $d_j$ e uma consulta $q$ é calculada pela correlação entre os vetores $\vec{d_j}$ e $\vec{q}$, a qual pode ser medida pelo cosseno (Equação~\ref{equ:cosine}) do ângulo entre esses vetores, conforme mostrado adiante na Seção~\ref{subsec:MedidasProximidade}. 



% \begin{equation}
% sim(d_j, q) = \frac{ \vec{d_j} \bullet \vec{q} }
                   % { |\vec{d_j}| \times | \vec{q}|}
            % = \frac{ \sum_{i=1}^{t} w_{i,j} \cdot w_{i,q} }
                   % { \sqrt{\sum_{i=1}^{t} w_{i,j}^2} \times \sqrt{\sum_{i=1}^{t} w_{i,q}^2 } }                   \label{equ:cosseno-doc-consulta}		                   
% \end{equation} 


% \begin{equation}
	% sim(d_j, q) = cosseno(d_j, q) 
% 
% = \frac{ \vec{d_j} \bullet \vec{q} } { |\vec{d_j}| \times | \vec{q}|}
% 
% \frac{ \vec{d_j} \bullet \vec{q} }
				   % { |\vec{d_j}| \times | \vec{q}|}
% = \frac{ \sum_{i=1}^{t} w_{i,j} \cdot w_{i,q} }
	   % { \sqrt{\sum_{i=1}^{t} w_{i,j}^2} \times \sqrt{\sum_{i=1}^{t} w_{i,q}^2 } }
% 
% \label{equ:cosseno-doc-consulta}		                   
% \end{equation} 

% Valores de cosseno próximos a 0 indicam um ângulo próximo a 90º entre $\vec{d_j}$ e $\vec{q}$, ou seja, o documento $d_j$ compartilha poucos termos com a consulta $q$, enquanto valores próximos a 1 indicam um ângulo próximo a 0º, ou seja, $d_j$ e $q$ compartilham termos e são similares~\cite{Tan2005,Feldman2006}.



Avaliar a relevância de um documento sob uma consulta é fundamental para os modelos de RI. Para isso pode-se utilizar medidas estatísticas simples como a frequência do termo, conhecida como TF (do inglês \textit{Term Frequency}) e a frequência de documentos, conhecida como DF (do inglês \textit{Document Frequency}). A frequência do termo indica o número de vezes que um termo ocorre na coleção de documentos. A frequência de documentos, indica o número de documentos que contém ao menos uma ocorrência de um determinado termo. Considera-se que os termos que ocorrem frequentemente em muitos documentos, em geral, não trazem informações úteis para discriminar a relevância dos documentos, então, a fim de diminuir o peso de termos altamente frequentes, usa-se o fator IDF (\textit{Inverted Document Frequency}), que é o inverso da número de documentos que contem um termo. O IDF é a medida de informação que um termo fornece com base em quão raro ou comum esse termo é para a coleção. Seja $n$ o número de documentos de uma coleção e $N_i$ o número de documentos onde o termo $t_i$ ocorre, o cálculo de IDF é dado por: 

	\begin{equation}
		IDF(t_i) = log\frac{n}{N_i}~.
		\label{equ:IDF}
	\end{equation}

Entre as medidas mais populares para ranqueamento de buscas está a TF-IDF (\textit{Term Frequency-Inverted Document Frequency}) que pondera a frequência de um termo em um documento com sua frequência na coleção total de documentos. Assim, a relevância de um termo para um documento é dada por:

\begin{equation}
	w_{i,j} = freq_{i,j} \cdot IDF(t_i)~,
\end{equation}


\noindent
onde $freq_{i,j}$ é a frequência do termo $t_i$ no documento $d_j$. A medida TF-IDF atribui valores altos para termos que ocorrem frequentemente em um documentos, e valores menores para termos que ocorrem poucas vezes em um documento ou em muitos documentos da coleção. A ideia da medida TF-IDF e quantificar a importância de um termo em um documento com base em sua frequência no próprio documento e sua distribuição ao longo da coleção de documentos~\cite{Croft2009,Salton1988,Shamsinejadbabki2012,Salton:1994}.


Uma vez que o modelo, por meio da Equação~\ref{equ:cosine}, calcula a similaridade entre os documentos e a consulta do usuário, é possível ranquear os resultados por ordem de relevância. Além disso, sua relativa simplicidade e flexibilidade, favorecem a aplicação desse modelo em sistemas de Recuperação de Informação~\cite{Tan2005,Croft2009,Manning2008}.

% ->-----------------------------------------------------------------------


\subsection{Modelo Probabilístico}

 
O modelo probabilístico é baseado no princípio da ordenação probabilística (\textit{Probability Ranking Principle}) onde dada um consulta $q$ e um documento $d_j$ relevante a $q$, o modelo tenta estimar a probabilidade do usuário encontrar o documento $d_j$. O modelo assume que para uma consulta $q$ há um conjunto de documentos $R_q$ que contém exatamente os documentos relevantes e nenhum outro, sendo este um conjunto resposta ideal que maximiza a probabilidade do usuário encontrar um documento $d_j$ relevante a $q$. 

Seja $\overline{R_q}$ o complemento de $R$ de forma que $\overline{R_q}$ contém todos os documentos não relevantes à consulta $q$. Seja $P(R_q|d_j)$ a probabilidade do documento $d_j$ ser relevante à consulta $q$ e $P(\overline{R_q}|d_j)$ a probabilidade de $d_j$ não ser relevante à $q$. A similaridade entre um documento $d_j$ e uma consulta $q$ é definida por:





\begin{equation}
	sim(d_j, q) = \frac{P(R_q|dj)}{P(\overline{R_q}|dj)} 
	\label{equ:simprob}
\end{equation}


A fim de obter-se uma estimativa numérica das probabilidades, o modelo assume o documento como uma combinação de palavras e seus pesos aos quais atribui-se valores binários que indicam a presença ou ausência de um termo, isto é, $w_{ij} \in \{0,1\}$ e $w_{iq} \in \{0,1\}$. Seja $p_i = P(t_i|R_q)$ a probabilidade do termo $k_i$ ocorrer em um documento relevante à consulta $q$, e $s_i    = P(k_i|\overline{R_q})$ a probabilidade do termo $k_i$ estar presente em um documento não relevante. 
Seja ainda $\prod_{i:d_i=1}$ o produto dos termos com valor 1. 
Então, pode-se calcular:

\begin{equation}
	sim(d_j, q) = 	
	\prod_{i:d_i=1} \frac{p_i}{s_i} 
	\cdot
	\prod_{i:d_i=0} \frac{1 - p_i}{1 - s_i}~,
	\label{equ:simprob-numeric}
\end{equation}


\noindent
onde $\prod_{i:d_i=1}$ significa o produto dos termos com valor 1.\\


%
%
O modelo também supõe que os termos ocorrem independentemente no documento, ou seja, a ocorrência de um termo não influencia a ocorrência de outro. 
Partindo dessas suposições, a Equação~\ref{equ:simprob-numeric} passa por transformações que incluem aplicação da regra de Bayes e simplificações matemáticas, e chega-se a Equação~\ref{equ:robertson} conhecida como equação de Robertson-Spark Jones a qual é considerada a expressão clássica para ranqueamento no modelo probabilístico. Detalhes da dedução dessa equação podem ser encontrados em~\cite{Croft2009, Manning2008, Rijsbergen1979},



\begin{equation}
	sim(d_j,q) = \sum_{i=1}^{t} w_{i,j} \cdot w_{i,q}  \cdot \sigma_{i/R}~,
	\label{equ:robertson}
\end{equation}



\noindent
onde $t$ é o número total de termos da coleção e 



\begin{equation}
	\sigma_{i/R} = \log \frac{p_i}{1-p_i} + \log \frac{1-s_i}{s_i}~.
\end{equation} 


Esse modelo tem com principal desvantagem a necessidade de estimar a separação inicial entre $R_q$ e $\overline{R_q}$, pois não se conhece inicialmente o conjunto dos documentos relevantes a uma consulta, o qual deve ser aprimorado por meio de interações com o usuário. Além disso, o modelo não leva em consideração a frequência dos termos na indexação do documento. O modelo apresenta como vantagem a característica de atribuir probabilidades as similaridades entre documentos e consultas, o que permite ranquear dos resultados por ordem de relevância. 















% -> Segmentação 
\documentclass[12pt]{article}

\usepackage{sbc-template}

\usepackage{graphicx,url}

%\usepackage[brazil]{babel}   
\usepackage[latin1]{inputenc}  

%\usepackage{subfig}
\usepackage{subfigure}
     
\sloppy

\title{Segmenta��o Textual Autom�tica de Atas de Reuni�o}

%\author{Luciana P. Nedel\inst{1}, Rafael H. Bordini\inst{2}, Fl�vio Rech Wagner\inst{1}, Jomi F. H�bner\inst{3} }


\address{%Instituto de Inform�tica -- Universidade Federal do Rio Grande do Sul (UFRGS)\\  Caixa Postal 15.064 -- 91.501-970 -- Porto Alegre -- RS -- Brazil
%\nextinstitute
%  Department of Computer Science -- University of Durham\\
%  Durham, U.K.
%\nextinstitute
%  Departamento de Sistemas e Computa��o\\
%  Universidade Regional de Blumenal (FURB) -- Blumenau, SC -- Brazil
%  \email{\{nedel,flavio\}@inf.ufrgs.br, R.Bordini@durham.ac.uk,
%  jomi@inf.furb.br}
}




\newcommand{\urlatas}{http://www.ppgccs.net/?page\_id=1150}

\newcommand{\urlbrowcorpus}{http://clu.uni.no/icame/brown/bcm.html}
\newcommand{\urlreuterscorpus}{http://trec.nist.gov/data/reuters/reuters.html}
\newcommand{\urlicsi}{http://www1.icsi.berkeley.edu/Speech/mr}
% LDC -- Linguistic Data Consortium
\newcommand{\urltdt}{https://catalog.ldc.upenn.edu/LDC98T25}
\newcommand{\urlwsj}{https://catalog.ldc.upenn.edu/ldc2000t43}
\newcommand{\urlcaloproject}{http://www.ai.sri.com/project/CALO}
\newcommand{\urlami}{http://groups.inf.ed.ac.uk/ami/corpus}

\newcommand{\urlorengo}{http://www.inf.ufrgs.br/~viviane/rslp/}



\usepackage[linesnumbered,ruled,portuguese]{algorithm2e}

% Define o caminho das figuras
\graphicspath{{images/}}


\begin{document} 

\maketitle

\begin{abstract}


A tarefa de segmentação topical consiste em dividir um texto em porções com significado relativamente independente, de maneira que cada segmento contenha um assunto. 
%
%
A segmentação de atas de reuniões é útil na organização desses documentos e para facilitar o seu acesso.
%
%
Este artigo traz uma revisão bibliográfica dos métodos segmentação textual e se concentra nos algoritmos mais influentes adaptando-os ao contexto das atas de reunião em português, bem como a encontrar um modelo que ofereça segmentos coesos e contribua a sistemas de recuperação de informação para que respondam melhor à buscas do usuário.
%
%
Para a avaliação, um conjunto de atas foi manualmente segmentado por participantes das reuniões, então comparou-se as divisões manuais com as geradas automaticamente. Os testes registraram precisão de 0.7106, revocação de 0.8516, as medidas P$_k$ e \textit{WindowDiff} mostram respectivamente 0.1163 e 0.3800 de dissimilaridade.





% na adaptação dos algoritmos mais influentes ao contexto das atas de 

%a fim de melhor responder a buscas do usuário em sistemas de recuperação de informação.

%divida um documento em partes .



% os influentes tratam muito de discursos e textos longos
% as atas apresentam texto mais enxuto/compacto e com estilo próprio.
% as atas são uma paráfrase da reunião

\end{abstract}

	Frequentemente atas de reunião tem a característica de apresentar um texto com poucas quebras de parágrafo e sem marcações de estrutura, como capítulos, seções ou quaisquer indicações sobre o tema do texto. 


% Definição 

A tarefa de segmentação textual consiste dividir um texto em partes que contenham um significado relativamente independente. Em outras palavras, é identificar as posições onde há uma mudança significativa de tópicos.

% Usos
É útil em aplicações que trabalham com textos sem quebras de assunto, ou seja, não apresentam parágrafos, seções ou capítulos, como transcrições automáticas de áudio e grandes documentos que contêm assuntos não idênticos como atas de reunião e noticias.


% Interesses
O interesse por segmentação textual tem crescido em em aplicações voltadas a recuperação de informação %citar o [15] ...
e sumarização de textos. % ... e [2] do "Efficient Linear T S"
Essa técnica pode ser usada para aprimorar o acesso a informação quando essa é solicitada por um usuário por meio de uma consulta, onde é possível oferecer porções menores de texto mais relevante ao invés de exibir um documento maior que pode conter informações menos pertinente. A sumarização de texto também pode ser aprimorada ao processar segmentos separados por tópicos ao invés de documentos inteiros.




% As Atas
Assim, esse trabalho trata da adaptação e avaliação de algoritmos tradicionais ao contexto de documentos em português do Brasil, com ênfase especial nas atas de reuniões.


%As atas, como frequentemente são, apresentam-se como uma sucessão de tópicos. Assim, o objetivo desse trabalho é identificar, automaticamente, onde há a mudança de um tópico para seus adjacentes.


% Diversas aplicações fazem uso de segmentação textual, incluindo 

% Entre as principais mais frequentes de segmentação textual estão a tra



%É principalmente utilizada em aplicações que processam textos longos como transcrições de áudio e documentos longos, além de aprimorar técnicas de sumarização e information retrievel.



% Usos:
%	* quando não há identificações
%	* em transcrições de áudio
%	* em documentos longos
% 	* text summarization (ver a referencia [2] de Efficient Linear Text...)




%Isto é, dado um texto, identificar onde há mudança de tópicos.


% Interest in automatic text segmentation has blossomed over the last few years, with applications ranging from information retrieval to text summariza-tion to story segmentation of video feeds. [A Critique and Improvement of an Evaluation Metric for Text Segmentation]



%Em outras palavras é identificar divisões entre unidade de informação sucessivas

%A tarefa de segmentação textual consiste em encontrar pontos onde há mudança de tópicos no texto.



%[ The task of linear text segmentation is to split a large text document into shorter fragments, usually blocks of consecutive sentences. ]


% **Segmentação é identificar divisiões entre unidades de informação sucessivas (Beeferman, Berger, and Lafferty (1997))**

%   [Text segmentation is the task of determining the positions at which topics change in a stream of text]





	\section{Referencial Teórico}
	\label{sec:referencial}
	
%%%%%%%%%%
% Retomada das Definições
%%%%%%%%%%	

Um documento textual, sobre tudo quando longo, é frequentemente uma sucessão de tópicos. 
%%%%%%%%%%
% Segmentação
%%%%%%%%%%
A segmentação textual ou segmentação topical é a tarefa de dividir um texto mantendo em cada parte um tópico com seu significado completo.
	
%%%%%%%%%%
% Segmento
%%%%%%%%%%	
Um segmento pode ser visto como uma sucessão de unidades de informação que compartilham um tópico essas unidades podem ser, por exemplo, palavras, sentenças ou parágrafos. Sendo a menor parte de um segmento, portanto são consideradas candidatas a limite entre segmentos.

%%%%%%%%%%
% Coesão léxica como **presuposto básico**
%%%%%%%%%%
Trabalhos anteriores se apoiam na ideia de que a mudança de tópicos em um texto é acompanhada de uma proporcional mudança de vocabulário, essa ideia, chamada de coesão léxica, sugere que a distribuição das palavras é um forte indicador da estrutura do texto. A partir disso, vários algoritmos foram propostos baseados na ideia de que um segmento pode ser identificado e delimitado pela análise das palavras que o compõe~\cite{Galley2003}~\cite{Boguraev2000}.



%, bem como é necessário mensurar suas similaridades. { asunidades de medida}

%%%%%%%%%%
% Porque cosseno
%%%%%%%%%%
Uma vez que a coesão léxica é pressuposto básico da maioria dos algoritmos, o cálculo da similaridade entre unidades de informação (comumente sentenças) é fundamental. Uma medida de similidade frequentemente utilizada é o cosseno, a qual pode ser vista na Equação~\ref{equ:cosine}, sendo $f_{x,j}$ a frequência da palavra $j$ na sentença $x$ e $f_{y,j}$ sendo a frequência da palavra $j$ na sentença $y$.


\begin{equation}
Sim(x,y) = \frac
{\Sigma_j f_{x,j} \times f_{y,j}}
{\sqrt{\Sigma_j f^2_{x,j} \times \Sigma f^2_{y,j}}}
\label{equ:cosine}
\end{equation}


\subsection{Principais algoritmos}
	\label{subsec:principaisalgoritimos}

Entre os principais trabalhos da literatura podemos citar o  \textit{TextTiling}~\cite{Hearst1994} e o \textit{C99}~\cite{Choi2000}, os quais são mostrados a seguir.

%%%%%%%%%%%%%%%%%%%%%%%%%%%%%%%%%%%%%%%%%%%%%%%
%%%              TextTiling                 %%%
%%%%%%%%%%%%%%%%%%%%%%%%%%%%%%%%%%%%%%%%%%%%%%%
O \textit{TextTiling} é um algoritmo baseado em janelas deslizantes, onde para cada candidato a limite, analisa-se o texto circundante. Um limite ou quebra de segmento é identificado quando a similaridade entres os blocos apresenta uma queda considerável.

%Ela propõe um algoritmo baseado em janelas deslizante, para analisar blocos de texto adjacentes e identificar os limites com base nas similaridades dos blocos.

O \textit{TextTiling} recebe uma lista de candidatos a limite, usualmente finais de parágrafo ou finais de sentenças. Para cada posição candidata são construídos 2 blocos, um contendo sentenças que a precedem e outro com as que a sucedem. O tamanho desses blocos é um parâmetro a ser fornecido ao algoritmo e determina o tamanho mínimo de um segmento.
%
Em seguida, os blocos de texto são representados por vetores que contém as frequências de suas palavras. Então, usa-se cosseno (Equação~\ref{equ:cosine}) para calcular a similaridade entre os blocos adjacentes a cada candidato e identifica-se uma transição entre tópicos pelos picos na curva se dissimilaridade.

%TODO como apresentado na Figura~\ref{fig:curvadedissimilaridade}.

%sempre que a similaridade cai abaixo de um \textit{threshold}.

O \textit{TextTiling} apresenta baixa complexidade computacional, devido a simplicidade do algoritmo e baixa eficiência quando comparado a outros métodos mais sofisticados como apresentados em~\cite{Choi2000, Kern2009, Misra2009}.



%%%%%%%%%%%%%%%%%%%%%%%%%%%%%%%%%%%%%%%%%%%%%%%
%%%                  C99                    %%%
%%%%%%%%%%%%%%%%%%%%%%%%%%%%%%%%%%%%%%%%%%%%%%%

Choi \cite{Choi2000} apresenta um esquema de ranking em seu algoritmo, o \textit{C99}. 
%
Embora muitos trabalhos utilizem matrizes de similaridades, o autor traz obervações.
%
Ele aponta que para pequenos segmentos, o cálculo de suas similaridades não é confiável, pois uma ocorrência adicional de uma palavra causa um impacto desproporcional no cálculo.
%
Além disso, o estilo da escrita pode não ser constante em todo o texto. Choi sugere que, por exemplo, textos iniciais dedicados a introdução costumam apresentar menor coesão do que trechos dedicados a um tópico específico. 
%

Portanto, comparar a similaridade entre trechos de diferentes regiões, não é apropriado.
% Complexidade O(n²)
Devido a isso, as similaridades não podem ser comparadas em valores absolutos. Então, o autor apresenta um esquema de \textit{rankings} para contornar esse problema.


%

% 1 cria uma matrix de similaridades
% 2 cria a matrix de ranking
% 3 aplica divisive clustering

% mask = quadro
% pegar um exemplo --> mostrar os numeros dentro do quatro e pq o resultado foi aquele

% colocar os passos na imagem


Inicialmente é construída uma matriz que contém as similaridades de todas as unidades de texto. Em seguida, cada valor na matriz de similaridade é substituído por seu ranking local. Onde para cade elemento da matiz, seu \textit{ranking} é o número de elementos vizinhos com valor de similaridade menor ao seu.%, o qual é calculado com a Equação~\ref{equ:ranklocal}. 

Na Figura~\ref{fig:exemplomatrixrank} vemos um quadro de dimensões 3~x~3 destacado na matriz de similaridades, que contém os valores  $\{0,3; 0,4; 0,4; 0,6; 0,5; 0,2; 0,9; 0,5; 0,7\}$, tomando como exemplo o elemento com valor $0,5$, a mesma posição na matriz de \textit{ranks} terá o valor $4$, pois esse é número de vizinhos com valores inferiores a $0,5$ dentro do quadro analisado na matriz de similaridades. 


%nesse exemplo, o valor $0,5$ é substituído por $4$ na matriz de ranks pois há 4 vizinhos com valor inferior

%o valor $0,5$ é comparado com seus elementos vizinhos, o


%Um exemplo é mostrado na Figura \ref{fig:exemplomatrixrank} abaixo, onde utiliza-se uma máscara de largura igual a 3.



  \begin{figure}[!h]

	\centering
	\includegraphics[width=0.45\textwidth]{exemplo-matrix-rank-noborder.jpg}
	\caption{Exemplo de construção de uma matriz de rank.~\cite{Choi2000}}
	\label{fig:exemplomatrixrank}

  \end{figure}




%\begin{equation}
%r(x,y) = \frac
%{Numero\ de\ elementos\ com\ similaridade\ menor}
%{Numero\ de\ elementos\ examinados}
%\label{equ:ranklocal}
%\end{equation}


% Clustering Reynar maximization
	
Finalmente, utiliza um método de divisão por \textit{clustering} baseado no algoritmo de maximização de Reynar~\cite{Reynar1998} para identificar os limites entre os segmentos. %Essa abordagem apresenta uma redução taxa de erros de 22\% para 10\%. Por outro lado, exige que a quantidade de segmentos seja conhecida.
%Como melhoramento, os autores apresentam posteriormente uma versão do \textit{C99} que utiliza \textit{Latent Semantic Analisys} (LSA) para calcular as similaridades ao invês de cosseno~\cite{Choi2001-LSA}.




\subsection{Medidas de Avaliação}

%%%%%%%%%%
%	Avaliações baseadas em hits 
%%%%%%%%%%

As medidas de avaliação tradicionais, baseiam se na contagem de acertos. No contexto da segmentação de textos um acerto é quando um o limite hipotético coincide com um limite de referência.

Essas medidas de avaliação tentam computar os erros do algoritmo, isto é falsos positivos e falsos negativos, a fim de calcular sua eficiência. 
%
% falso positivo
Um falso positivo é um limite identificado pelo algoritmo que não corresponde a nenhum limite na segmentação de referência. 
%
% falso negativo
Um falso negativo é quando o algoritmo não identifica um limite existente na segmentação de referência.


Nesse sentido, 
%
a precisão, que é a proporção de limites corretamente identificados pelo algoritmo, e 
%
a revocação, que é a proporção de limites verdadeiros que foram identificados pelo algoritmo,
%
trazem alguns problemas na avaliação de segmentadores automáticos.
 	
	
Conforme o algoritmo aponta mais segmentos no texto, este tende a melhorar a revocação e ao mesmo tempo, reduzir a precisão. Esse problema de avaliação pode ser contornado utilizado a medida F-1 que é uma média harmônica entre precisão e revocação onde ambas tem a o mesmo peso. Por por outro lado, tem a desvantagem de ser mais difícil de interpretar. 

As medidas apresentadas acima falham ao não serem sensíveis a \textit{near misses}, ou seja, quando um limite não coincide exatamente com o esperado, mas está próximo a ele~\cite{Kern2009}.

Na Figura~\ref{fig:exemplosegmentacaozoom} é apresentado um exemplo com duas segmentações hipotéticas e uma referência. Na Figura~\ref{fig:exemplosegmentacao}, em ambos os casos não há nenhum verdadeiro positivo, o que implica em zero para os valores de precisão, acurácia, e revocação, embora a segunda hipótese possa ser considerada superior à primeira se levado em conta a proximidade dos limites.



  \begin{figure}[!h]

	\centering
	\includegraphics[width=0.47\textwidth]{windiffzoom.jpg}
	\caption{Exemplos de \textit{near missing} e falso positivo puro. Os blocos indicam uma unidade de informação e as linha verticais representam os limites entre segmentos de texto representando um tópico do texto. }
	\label{fig:exemplosegmentacaozoom}

  \end{figure}
  
  \begin{figure}[!h]

	\centering
	\includegraphics[width=0.47\textwidth]{windiff.jpg}
	\caption{
	Exemplo de duas segmentações hipotéticas em comparação a uma ideal. 
	}
	\label{fig:exemplosegmentacao}

  \end{figure}
  
  
Entre as medidas mais utilizadas para avaliar segmentadores estão:

\begin{enumerate}

	\item P$_k$. A fim de resolver o problema de \textit{near misses}, Beeferman \textit{et. al.}~\cite{Beeferman1999} apresentam uma medida chamada P$_k$ que atribui 
%
valores parciais a \textit{near misses}, 
%
ou seja, limites sempre receberão um peso proporcional à sua proximidade, desde que dentro de um janela de tamanho~$k$.
%
Para isso, esse método move uma janela de tamanho $k$ e a cada posição e verifica se o início e o final da janela estão ou não dentro do mesmo segmento e penaliza o algoritmo em caso de discrepância. Ou seja, dado duas palavras de distancia $k$, uma discrepância é computada quando o algoritmo e a referência não concordam se as palavras estão ou não no mesmo segmento.

O valor de $k$ é calculado como a metade da média dos comprimentos dos segmentos reais. Como resultado, é retornado a contagem de discrepâncias divido pelo quantidade de segmentações analisadas. Esse valor serve como medida de dissimilaridade entre as segmentações e pode ser interpretada como a probabilidade de duas sentenças extraídas aleatoriamente pertencerem ao mesmo segmento.

\item \textit{WindowDiff}. Pevzner~\cite{Pevzner2002} aponta problemas na avaliação mais tradicional, P$_k$~\cite{Beeferman1999}. Eles apontam que esse método penaliza demasiadamente os falsos negativos em relação aos falsos positivos e a \textit{near misses}, além disso, desconsidera o tamanho dos segmentos. Como solução, propõem um método, que traz duas diferenças principais: a dobra da penalidade para os falsos positivos a fim de diminuir o problema da subestimação dessa medida e, diferente de P$_k$, ao mover a janela pelo texto, penaliza-se o algoritmo sempre que o número de limites proposto pelo algoritmo não coincidir com o número de limites esperados para aquela janela de texto. 

Com isso, demonstram em seu trabalho que, em relação a P$_k$, consegue resolver seus principais problemas e mantém sua proposta inicial de sensibilidade a \textit{near misses}, penalizando-os menos que os falsos positivos puros.


\end{enumerate}














%apontam que a coesão léxica é um forte indicador da estrutura do texto, isto é, a mudança de tópicos é acompanhada de uma proporcional mudança de vocabulário. A partir disso, vários algoritmos foram propostos baseados na ideia de que um segmento pode ser identificado e delimitado pela análise das palavras que o compõe~\cite{Galley2003}~\cite{Boguraev2000}.


%Finalmente, os limites são identificados sempre que a similaridade entre blocos adjacentes entre cada candidato ultrapassa um determinado \textit{threshold}.



% Mensionar que existem duas abordagens principais - Baseada em coesão léxiam e em discursos [ver a pg 2 do Text Segmentation With Topic Moeling and Entity Coherence]




%Há ainda outros critérios para segmentação como a segmentação temática 

%outros tipos de abordagem
%	Segmentação funcional
%	Segementação temática
	
	% Ideia básica dos algorítmos (Coesão léxica ) como **presuposto básico**

Os principais algoritmos de segmentação textual baseiam-se na ideia de coesão léxica entre assuntos. Isto é, a mudança de tópicos é acompanhada de uma proporcional mudança de vocabulário. A partir disso, vários algoritmos foram propostos. Dessa forma, assumem o pressuposto que um segmento pode ser identificado e delimitado pela análise das palavras que o compõe.







% A coesão léxica é um termômetro para as mudanças de tópicos, e portanto, um indicador para quebras de segmento.

 
 
% Nesse artigo, os principais serão analisados na perspectiva de atas de reunião.


%Os principais algoritmos de segmentação textual assumem o pressuposto que um segmento pode ser identificado e delimitado pela análise de seu vocabulário





%Os entre os principais trabalhos relacionados a segmentação textual estão o \textit{TextTiling} e o \textit{C99}



%\subsubsection{TextTiling}
%	O algoritmo TextTiling, proposto por 
	
%
%\subsubsection{C99}



Entre os mais influentes podemos citar o \textit{TextTiling}~\cite{Hearst1994} 




Semelhante a esse trabalho, outras abordagens foram propostas como ...

\cite{Banerjee200657} faz uma adaptação do \textit{TextTiling} ao contexto das conversas em reuniões com múltiplos participantes.  



%%%%%%%
% C99 %
%%%%%%%

Choi \cite{Choi2000} apresenta um trabalho que usa \textit{cosine} como medida similaridade e apresenta um esquema de ranking em seu algoritmo, o C99.
%
Embora muitos dos melhores trabalho utilizarem matrizes de similaridades, o autor traz obervações.
%
Ele aponta que para pequenos segmentos, o cálculo de suas similaridades não é confiável. Pois uma ocorrência adicional de uma palavra causa um impacto desproporcional no cálculo.
%
Além disso, o estilo da escrita pode não ser constante em todo o texto. Choi sugere que, por exemplo, textos iniciais dedicados a introdução costumam apresentar menor coesão do que trechos dedicados a um tópico específico. Portanto comparar a similaridade entre trechos de diferentes regiões, não é apropriado.
% Complexidade O(n²)
Devido a isso, as similaridades não podem ser comparadas em valores absolutos. O autor apresenta um esquema de ranking para contornar esse problema.

Cada valor na matriz similaridade é substituída por seu ranking local. O ranking é o número de elementos vizinhos com similaridade menor, conforme a imagem abaixo.


\begin{equation}
Sim(x,y) = \frac
{\Sigma_j f_{x,j} \times f_{y,j}}
{\sqrt{\Sigma_j f^2_{x,j} \times \Sigma f^2_{x,j}}}
\end{equation}



\begin{equation}
r(x,y) = \frac
{Numero\ de\ elementos\ com\ similaridade\ menor}
{Numero\ de\ elementos\ examinados}
\end{equation}













	\section{Proposta: Segmenta��o Linear Autom�tica de Atas de Reuni�o}
	\label{sec:proposta}






%%%%%%%%%%
% TextTiling e C99 criados para ingl�s e independente de dom�nio
%%%%%%%%%%
Os algoritmos \textit{TextTiling} e \textit{C99} foram propostos para o ingl�s, independentemente de dom�nio, ou seja, a proposta inicial dos autores � trabalhar em qualquer texto nessa l�ngua.
%%%%%%%%%%
% Adapatar para Atas em portugu�s
%%%%%%%%%%
Neste trabalho, prop�e-se aplic�-los ao contexto das atas de reuni�o em portugu�s do Brasil, ou seja, em uma l�ngua diferente e dentro de um contexto espec�fico. As implementa��es utilizadas, bem como ferramentas utilizadas e os resultados completos est�o dispon�veis para utiliza��o e consulta em \url{link}. As subse��es seguintes tratam da aplica��o desses algoritmos para esse nicho mais espec�fico. 


\subsection{Cole��o de documentos}


\section{Conjunto de documentos}
	\label{sec:conjutodedocumentos} 
 
%%%%%%%%%%
% A concatenação de textos favorece a segmentação
%%%%%%%%%%
A maioria dos trabalhos da literatura enquadram-se em duas categorias no que se refere ao \textit{corpus} estudado. A primeira utiliza a concatenação de textos de diferentes fontes e assuntos. 
Ocorre que essas características favorecem o processo de segmentação uma vez que os documentos produzidos pela concatenação contém tópicos obtidos de domínios distintos que compartilham pouco vocabulário. Por outro lado, essas características não estão presentes no contexto das atas, onde um documento é redigido por um mesmo autor, e todos tópicos foram produzidos no mesmo domínio, e dessa forma, têm estilo de escrita e vocabulário similares.


%%%%%%%%%%
% Diferenças entre atas formais e discursos falados
%%%%%%%%%%
A segunda estuda a segmentação automática de textos obtidos da transcrição de áudios como de vídeos e de gravações de reuniões com grupos de pessoas. 
A transcrição da fala de pessoas, difere desse trabalho, pois este possui caracteristicas típicas de um estilo formal, por exemplo, o texto sucinto, onde o autor se limita a relatar apenas o essencial de uma discussão e omite detalhes da discussão, poupando o leitor de diálogos repetitivos ou falas durante a transição de tópicos, o que resulta em segmentos mais curtos os quais desfavorecem algoritmos baseados em coesão léxica~\cite{Choi2000}. 
Mais detalhes referentes às diferenças entre as atas e os textos analisados na maioria dos trabalhos são mostrados na Subseção~\ref{subsec:conjunto-de-documentos}.

% Em outras palavras, uma ata é a narração de uma reunição que poupa



	A fim de obter um conjunto de documentos segmentados que possam servir como referência na avaliação, seis atas de reunião foram coletadas junto ao departamento de pós-graduação da UFSCar-Sorocaba\footnote{\urlatas}. Os documentos foram oferecidos à profissionais que participam de reuniões desse departamento e por meio de um \textit{software} segmentaram o texto das atas conforme o julgamento de cada um. Os segmentos gerados manualmente foram comparados à segmentação automática conforme os critérios descritos a seguir.
	
	As atas de reunião diferem dos textos comumente estudados em outros trabalhos em alguns pontos. O estilo de escrita favorece textos sucintos com poucos detalhes de maneira que o ambiente dá preferência a textos curtos. Segundo Choi~\cite{Choi2001-LSA}, o segmentador tem a acurácia reduzida em segmentos curtos (em torno de 3 a 5 sentenças).
	
	Para evitar um texto monótono à leitura, a redação do documento tem o cuidado de não repetir ideias e palavras em favor da elegância do texto. Tal característica enfraquece a coesão léxica e portanto o cálculo da similaridade é prejudicado. Por exemplo, duas sentenças diferem se uma contiver a palavra \textit{computadores} e na seguinte \textit{equipamentos}, mesmo que se refiram à mesma ideia. Além disso, o documento compartilha um certo vocabulário próprio do ambiente onde os assuntos são discutidos e com isso nota-se que os segmentos, embora tratem de assuntos diferentes, são semelhantes em vocabulário.
	
A presença de ruídos como cabeçalhos, rodapés e numeração de páginas e linhas prejudicam tanto similaridade entre sentenças como a apresentação final ao usuário. Porém, esses ruídos podem ser reduzidos ou eliminados como mostrado na Subseção~\ref{subsec:preprocessamento}, sobre preprocessamento.





\subsection{Pr�-processamento}
	\label{subsec:preprocessamento}

As atas a serem segmentadas s�o extra�das de documentos do tipo \textit{pdf}, \textit{doc}, \textit{docx} ou \textit{odt} que normalmente possuem formato bin�rio. A fim de extrair o texto desses documentos, aplicou-se um processo que transforma esses formatos em aquivos de texto plano. 
%Ap�s isso, obteve-se uma m�dia de 906 \textit{tokens} por ata. 
A fim de preparar o texto e selecionar as palavras mais significativas, as atas passaram por processos de transforma��o os quais ser�o apresentados a seguir.
	

\begin{enumerate}


% Cabe�alhos e rodap�s
\item Remo��o de cabe�alhos e rodap�s: as atas cont�m trechos que podem ser considerados pouco informativos e descartados durante o pr�-processamento, como cabe�alhos e rodap�s que se misturam aos t�picos tratados na reuni�o, podendo ser  inseridos no meio de um t�pico prejudicando tanto o algoritmo de segmenta��o, quanto a leitura do texto pelo usu�rio.

% Identifica��o de senten�as
\item Identifica��o de finais senten�as: Devido ao estilo de pontua��o desses documentos, como encerrar senten�as usando um \textit{";"} e inser��o de linhas extras, foram usadas as regras especiais para identifica��o de finais de senten�a. Os detalhes sobre essas regras est�o dispon�veis para consulta em \url{link}.


%cada final de senten�a � identificado e marcado com uma \textit{string} especial, esse processo � melhor descrito na Subse��o~\ref{subsec:indentificacaosentencas}.

% Remo��o de termos
\item Redu��o de termos: Eliminou-se a acentua��o, sinais de pontua��o, numerais e todos os \textit{tokens} menores que tr�s caracteres. 
Palavras de uso muito frequente como artigos, preposi��es e pronomes, chamadas de \textit{stop words}, foram removidas utilizando-se uma lista de 438 palavras.

% Stemming
\item \textit{Stemming}:
Extraiu-se o radical de cada palavra. Para isso, as letras foram convertidas em caixa baixa e aplicou-se o algoritmo \textit{Orengo}\footnote{\urlorengo} para remo��o de sufixos.

\end{enumerate}
	


%Ap�s os passo, reduziu-se a quantidade de tokens conforme mostrado na Tabela~\ref{tab:preprocessamento}.

Na Tabela~\ref{tab:preprocessamento} � mostrado, para cada ata, a quantidade de \textit{tokens} ap�s a extra��o dos documentos e durante o pr�-processamento.

%\begin{table}[!h]
%	\centering
%
%	\begin{tabular}{|l|c|}
%	
%		\hline
%		\textbf{Processo}      &  \textbf{M�dia de \textit{tokens}}\\		
%
%		\hline
%
%		Extra��o do texto                    & 906 \\ \hline
%		Remo��o de Cabe�alhos Rodap�s        & 813 \\ \hline
%		Identifica��o de finais de senten�a  & 813 \\ \hline
%%		Limpeza                              & 535 \\ \hline
%%		Remo��o de Numerais                  & 526 \\ \hline
%%		Remo��o de \textit{Stop Words}       & 441 \\ \hline
%		Redu��o de termos     				 & 441 \\ \hline
%		\textit{Stemming}                    & 441 \\ \hline
%		
%		
%	\end{tabular}
%	
%	\caption{Quantidade m�dia de \textit{tokens} extra�dos ap�s cada passo do pr�-processamento.}
%	\label{tab:preprocessamento}
%\end{table}

%Ata 29 - 25a Reuni�o Odin�ria PPGCCS.pdf.txt    Ata 1 - 18 senten�as
%	 On Load = 			809
%	 Header Removal = 	665
%	 Term Reduction = 	461
%Ata 30 - 26a Reuni�o Odin�ria PPGCCS.pdf.txt    Ata 2 - 26 senten�as
%	 On Load = 			851
%	 Header Removal = 	704
%	 Term Reduction = 	489
%Ata 32 - 28a Reuni�o Odin�ria PPGCCS.pdf.txt    Ata 3 - 24 senten�as 
%	 On Load = 			1043
%	 Header Removal = 	840
%	 Term Reduction = 	566
%Ata 33 - 29a Reuni�o Odin�ria PPGCCS.pdf.txt    Ata 4 - 32 senten�as
%	 On Load = 			1407
%	 Header Removal = 	1247
%	 Term Reduction = 	872
%Ata 36 - 31a Reuni�o Odin�ria PPGCCS.pdf.txt Ata 5 - 25 senten�as 
%	 On Load = 			834
%	 Header Removal = 	708
%	 Term Reduction = 	485
%Ata 35 - 5a Reuni�o Extraodin�ria PPGCCS.pdf.txt  Ata 6 - 10 senten�as
%	 On Load =	 		496
%	 Header Removal = 	392
%	 Term Reduction = 	286

\begin{table}[!h]
	\centering
	\begin{tabular}{|l|c|c|c|c|c|c|}
		\hline
		\textbf{Processo}   &  
		\textbf{Ata 1}		&  
		\textbf{Ata 2}		&  
		\textbf{Ata 3}		&  
		\textbf{Ata 4}		&	  
		\textbf{Ata 5}		&  
		\textbf{Ata 6}		\\	\hline

		Extra��o do texto                    & 809 & 851 & 1043 & 1407 & 834 & 496 \\ \hline
		Remo��o de Cabe�alhos e Rodap�s      & 665 & 704 & 840  & 1247 & 708 & 392 \\ \hline
%		Identifica��o de finais de senten�a  & 665 & 704 & 840  & 1247 & 708 & 392 \\ \hline
		Redu��o de termos				     & 461 & 489 & 566  &  872 & 485 & 286 \\ \hline
%		\textit{Stemming}                    & 461 & 489 & 566  &  872 & 485 & 286 \\ \hline
		
		
	\end{tabular}
	
	\caption{Quantidade de \textit{tokens} por ata.}
	\label{tab:preprocessamento}
\end{table}






%A Figura~\ref{fig:exemplopreprocessamento} mostra a etapa de pr�-processamento em uma senten�a em portugu�s.
%	
%
%
%  \begin{figure*}
%	\centering
%	\includegraphics[width=1\textwidth]{pre-processamento.jpg}
%	\caption{Exemplo de pr�-processamento.}
%	\label{fig:exemplopreprocessamento}
%  \end{figure*}
%

%\subsection{Identifica��o de candidatos}
%	\label{subsec:indentificacaosentencas}
%	
%%%%%%%%%%%	
%% Indicar unidade m�nima de Segmento
%%%%%%%%%%%
%	
%	
%%	Como entrada para os 
%%	Os algoritmos de segmenta��o devem ser
%	� preciso fornecer aos algoritmos os candidatos iniciais a limites de segmento. Para isso, � necess�rio escolher qual ser� a unidade de informa��o m�nima que constitui um segmento. %Baseando-se no estilo de escrita e considerando as pontua��es de um texto, � poss�vel, em alguns casos, indicar quebras de par�grafo, finais de senten�as ou mesmo palavras como elementos que encerram um segmento. 
%
%	Ocorre que em atas de reuni�o � uma pr�tica comum redig�-las de forma que o conte�do discutido fica em par�grafo �nico. Al�m disso, as quebras de par�grafo s�o usadas para formata��o de outros elementos como espa�o para assinaturas. 
%%
%% Tamb�m n�o � conveniente indicar todo ponto entre \textit{token} como candidato pois obrigaria a ajustar posteriormente os segmentos de maneira a n�o quebrar uma ideia ou frase. 
%%	
%	Assim, neste trabalho, os finais de senten�a s�o considerados unidades de informa��o e portanto, pass�veis a limite entre segmentos. 
%	
%	Devido ao estilo de pontua��o desses documentos, como encerrar senten�as usando um \textit{";"} e inser��o de linhas extras, foram usadas as regras apresentadas no Algoritmo 1 para identificar os finais de senten�as.  
%
%
%\begin{algorithm}
%	\SetKwInOut{Input}{Entrada}
%	\SetKwInOut{Output}{Sa�da}
%	\SetKwBlock{Inicio}{in�cio}{fim}
%	\SetKwFor{ParaTodo}{para todo}{}{fim para todo}
%	\SetKwIF{Se}{SenaoSe}{Senao}{}{}{senao se}{senao}{fim se}
%	\SetKwFor{Para}{}{}{}
%%	\SetKwAlgorithm{Algorithm}{Algoritmo}{}
%
%	
%	\Input{Texto}
%	\Output{Texto com identifica��es de finais de senten�a}
%	
%	\ParaTodo {token, marc�-lo como final de senten�a se:} {	
%
%	Terminar com um \texttt{!}\\
%	Terminar com um \texttt{.} e n�o for uma abrevia��o\\
%	Terminar em \texttt{.?;} e:
%		\Para{}{
%			For seguido de uma quebra de par�grafo ou tabula��o\\
%			O pr�ximo \textit{token} iniciar com  \texttt{(\{["'}\\
%			O pr�ximo \textit{token} iniciar com letra mai�scula\\
%			O pen�ltimo caracter  for \texttt{)\}]"'}\\
%		}
%	}
%	
%	\caption{Identifica��o de finais de senten�a}
%	\label{alg:identificacaofinaisdesent}
%\end{algorithm}
%
 


  



\subsection{Segmenta��o de Refer�ncia}
	\label{subsec:segmetacaoreferencia}


% \subsection{Segmenta��o de Refer�ncia}
	% \label{subsec:segmetacaoreferencia}


%%%%%%%%%% % Como foram obtidas (software e especialistas) %%%%%%%%%%




A fim de obter um conjunto de documentos segmentados que possam servir como refer�ncia na avalia��o, os documentos coletados foram segmentados manualmente por profissionais que participam de reuni�es. Para isso, utilizou-se um \textit{software}, desenvolvido com esse objetivo especifico, que permitiu aos volunt�rios visualizar um documento, e indicar livremente as divis�es entre segmentos. Com o uso desse \textit{software} foram coletados os dados de seis atas segmentadas por dois participantes das reuni�es, os quais serviram como refer�ncia para a avalia��o dos algoritmos.
O \textit{software} desenvolvido para segmenta��o manual est� dispon�vel para utiliza��o e consulta em~\urlsoftwares

%Os arquivos gerados foram tratados para que os segmentos sempre terminem em uma senten�a reconhecida pelo algoritmo, uma vez que as senten�as s�o a unidade m�nima de informa��o nesse trabalho.

A Tabela~\ref{tab:segmentacaoreferencia} cont�m, para cada ata, a quantidade de senten�as e a quantidade de segmentos identificadas pelos participantes.



% Refer�ncia ������������������������������������������������������������
% Documento	|	Segmentos Esp 1 |	Segmentos Esp 2
% Doc1		|	7 				|	15
% Doc2		|	9 				|	20
% Doc3		|	7 				|	15
% Doc4		|	9 				|	17
% Doc5		|	4 				|	9 
% Doc6		|	11				|	17
\begin{table}[!h]
	\centering
	\begin{tabular}{|l|c|c|c|} \hline
		\textbf{Ata} & \textbf{Senten�as}  & 
		\textbf{Participante 1}  & 
		\textbf{Participante 2} \\	\hline

		Ata 1 & 18 & 7  & 15 \\ \hline 
		Ata 2 & 26 & 9  & 20 \\ \hline 
		Ata 3 & 24 & 7  & 15 \\ \hline 
		Ata 4 & 32 & 9  & 17 \\ \hline 
		Ata 5 & 25 & 11 & 17 \\ \hline 
		Ata 6 & 10 & 4  & 9  \\ \hline 

	\end{tabular}
	\caption{Quantidade de senten�as e segmentos de refer�ncia por ata.}
	\label{tab:segmentacaoreferencia}
\end{table}




















% A fim de obter um conjunto de documentos segmentados que possam servir como refer�ncia na avalia��o, foram coletadas as atas de reuni�es do Conselhos de P�s Gradua��o, Conselho de Cursos e Conselho de Departamento de da UFSCar-Sorocaba. Os documentos foram oferecidos � profissionais que participam de reuni�es desse departamento para segment�-las. Para isso, utilizou-se um \textit{software} que permitiu aos volunt�rios visualizar um documento, e indicar livremente as divis�es entre segmentos com um assunto relativamente independente. Ao final, o software coletou os dados da segmenta��o de doze das os quais serviram como refer�ncia para a avalia��o dos algoritmos.

% Os arquivos gerados foram tratados para que os segmentos sempre terminem em uma senten�a reconhecida pelo algoritmo, uma vez que as senten�as s�o a unidade m�nima de informa��o nesse trabalho.
  
	
	
%%%%%%%%%% % Como foram obtidas (software e especialistas) %%%%%%%%%%
%A fim de obter um conjunto de documentos segmentados que possam servir como refer�ncia na avalia��o, os documentos foram oferecidos � profissionais que participam de reuni�es para segment�-las. Para isso, utilizou-se um \textit{software} que permitiu aos volunt�rios visualizar um documento, e indicar livremente as divis�es entre segmentos com um assunto relativamente independente. Ao final, o software coletou os dados de seis atas segmentatas por dois participantes das reuni�es, os quais serviram como refer�ncia para a avalia��o dos algoritmos.

%Os arquivos gerados foram tratados para que os segmentos sempre terminem em uma senten�a reconhecida pelo algoritmo, uma vez que as senten�as s�o a unidade m�nima de informa��o nesse trabalho.
  




\subsection{Configura��o experimental}
	\label{subsec:configuracaoexperimental}

%%%%%%%%%%
% Par�metros
%%%%%%%%%%
As implementa��es dos algoritmos permitem ao usu�rio a configura��o de seus par�metros. 
%
O \textit{TextTiling} permite ajustarmos dois par�metros, sendo o tamanho da janela e o passo. Por meio de testes emp�ricos escolheu-se os valores os valores 20, 40 e 60 para o tamanho da janela e 3, 6, 9 e 12 para o passo. Gerando ao final 20 configura��es.
%

O \textit{C99} permite o ajuste de tr�s par�metros, sendo, o primeiro a quantidade segmentos desejados, uma vez que, n�o se conhece o n�mero ideal de segmentos e os documentos n�o apresentam muitos candidatos, calculou-se uma propor��o dos candidatos a limite. Para isso atribuiu-se os valores {0,2; 0,4; 0,6; 0,8}. O segundo par�metro, o tamanho do quadro utilizado para gerar a matriz de ranking, atribuiu-se os valores 9 e 11, sendo 11 o valor padr�o da apresentado pelo autor. O algoritmo permite ainda indicar se as senten�as ser�o representados por vetores contendo a frequ�ncia ou o peso de cada termo. Ambas as representa��es foram utilizadas. Considerando todos os par�metros, foram geradas 16 configura��es para o algoritmo \textit{C99}.



\subsection{Crit�rios de avalia��o}

% \subsection{Crit�rios de avalia��o}

%%%%%%%%%%
% Defini��o do que � um bom algoritmo de segmenta��o
%%%%%%%%%%
Para fins de avalia��o desse trabalho, um bom m�todo de segmenta��o � aquele cujo resultado melhor se aproxima de uma segmenta��o manual, sem a obrigatoriedade de estar perfeitamente alinhado com tal. Ou seja, visto o contexto das atas de reuni�o, e a subjetividade da tarefa, n�o � necess�rio que os limites entre os segmentos (real e hip�tese) sejam id�nticos, mas que se assemelhem em localiza��o e quantidade.


Os algoritmos foram comparados com a segmenta��o fornecida pelos participantes das reuni�es. Calculou-se as medidas mais aplicadas � segmenta��o textual, P$_k$ e \textit{WindowDiff}. Al�m dessas, computou-se tamb�m as medidas tradicionais acur�cia, precis�o, revoca��o e $F^1$ para compara��o com outros trabalhos que as utilizam.

Inicialmente, calculou-se as medidas configurando cada algoritmo conforme mostrado na Subse��o~\ref{subsec:configuracaoexperimental}, sem aplicar o pr�-processamento. O teste de Friedman com p�s-teste de Nemenyi foi utilizado para gerar um ranking das melhores configura��es para cada medida calculada. Com isso, foi poss�vel descobrir quais valores otimizam um algoritmo para uma medida, 	desconsiderando o pr�-processamento. 

A fim de conhecer o impacto do pr�-processamento, repetiu-se os testes com o texto pr�-processado. Com isso, descobriu-se quais valores otimizam os algoritmos para cada medida, considerando essa etapa.

Com os testes anteriores obteve-se, para cada medida, 4 configura��es, levando em conta ambos os algoritmos e a presen�a ou aus�ncia do pr�-processamento. Novamente utilizou-se o teste de Friedman e Nemenyi e descobriu-se, para cada medida, qual configura��o a otimiza. Os resultados completos est�o dispon�veis para consulta em~\urlsoftwares.




% Para mim o friedman funcionou como uma ferramenta para ranquear as configura��es

% Como tenho muitas configura��es, dividi em v�rios testes...

% Testei os algoritmos separados

% repeti considerando o preprocess

% isso me deu 4 rankings para cada medida...

% repeti o teste denovo com as 4 melhores para cada medida para encontrar o 'melhor dos melhres'

% Fiz assim por 2 motivos: 1- Comparar os algs e a influ�ncia do preprocess
% e 2� porque o Keel n�o deixa eu testar com mais de 20 configs









% Deseja-se avaliar os algoritmos TextTiling e C99 no contexto das atas, bem como encontrar os valores que otimizam seus par�metros e o impacto do pr�-processamento. 



% A fim de avaliar os algoritmos \textit{TextTiling} e \textit{C99} no contexto das atas, bem como encontrar os valores que otimizam seus par�metros e o impacto do pr�-processamento, esses

% Para isso, utilizou-se o teste de Friedman com p�s teste de Nemenyi para ranquear o desempenho de um conjunto de configura��es.

% Em um primeiro momento calculou-se as medidas para os algoritmos desconsiderando o pr�-processamento, a fim de saber 

% Inicialmente calculou-se as medidas para os algoritmos configurando seus par�metros conforme mostrado na Subse��o~\ref{subsec:configuracaoexperimental} sem aplicar o preprocessamento, afim de conhecer quais par�metros melhor configuram os algoritmos considerando o texto integral e por meio do teste de Friedman gerou-se um ranking das melhores configura��es para cada medida e para cada algoritmo.




\subsection{Resultados}
	\label{subsec:resultados}
\section{Análise dos Resultados}
	\label{sec:resultados}









	
%	\input{project/proposta}	
%	
\section{Avaliação}
	\label{sec:avaliacao}



%%%%%%%%%% 
% Necessidade de uma referência
%%%%%%%%%%
Para que se possa avaliar um segmentador automático de textos, é preciso uma referência, isto é, um texto com os limites entre os segmento conhecidos. Essa referência, deve ser confiável, sendo uma segmentação legítima que é capaz de dividir o texto em porções relativamente independentes, mantendo um conteúdo legível, ou seja, uma segmentação ideal.
%

Entre as abordagens mais comuns para se conseguir essas referências, encontramos: A concatenação aleatória de documentos distintos, onde o ponto entre o final de um texto e o inicio do seguinte é um limite entre eles. A segmentação manual dos documentos, nesse caso, pessoas capacitadas, também chamadas de juízes, ou mesmo o autor do texto, são consultadas e indicam manualmente onde há uma quebra de segmento. Em transcrição de conversas faladas em reuniões com múltiplos participantes, um mediador é responsável por encerrar um assunto e iniciar um novo, nesse caso o mediador anota manualmente o tempo onde há uma transição de tópico. Em aplicações onde a segmentação é tarefa secundária, é possível, ao invés de avaliar o segmentador, analisar seu impacto na aplicação final.


%%%%%%%%%%
% As 2 principais dificuldades na avaliação
%%%%%%%%%%
De acordo com \cite{Pevzner2002} há duas principais dificuldades na avaliação de segmentadores automáticos. A primeira é conseguir um referência, já que juízes humanos costumam não concordar entre si, sobre onde os limites estão e outras abordagens podem não se aplicar ao contexto. A segunda é que tipos diferentes de erros devem ter pesos diferentes de acordo com a aplicação. Há casos onde certa imprecisão é tolerável e outras, como a segmentação de notícias, onde a precisão é mais importante.


%%%%%%%%%%
% Definição do que é um bom algoritmo de segmentação
%%%%%%%%%%
Para fins de avaliação desse trabalho, um bom método de segmentação é aquele cujo resultado melhor se aproxima do ideal, sem a obrigatoriedade de estar perfeitamente alinhado com tal. Ou seja, visto o contexto das atas de reunião, e a subjetividade da tarefa, não é necessário que os limites entre os segmentos (real e hipótese) sejam idênticos, mas que se assemelhem em localização e quantidade.


%Para quantificar a eficiência dos algoritmos, segue uma revisão das principais métricas aplicáveis.

As próximas subseções mostram o conjunto de atas e a segmentação usada como referência, uma revisão das principais métricas aplicáveis à segmentação e os testes realizados para avaliar os métodos.

\subsection{Conjunto de documentos}
	A fim de obter um conjunto de documentos segmentados que possam servir como referência na avaliação, seis atas de reunião foram coletadas junto ao departamento de computação da UFSCar-Sorocaba. Os documentos foram oferecidos à profissionais que participam de reuniões desse departamento e por meio de um \textit{software} segmentaram o texto das atas conforme o julgamento de cada um. Os segmentos gerados manualmente foram comparados à segmentação automática conforme os critérios descritos a seguir.
	
	As atas de reunião diferem dos textos comumente estudados em outros trabalhos em alguns pontos. O estilo de escrita favorece textos sucintos com poucos detalhes de maneira que o ambiente dá preferência a textos curtos. Segundo Choi~\cite{Choi2001-LSA}, o segmentador tem a acurácia reduzida em segmentos curtos (em torno de 3 a 5 sentenças).
	
	Para evitar um texto monótono à leitura, a redação do documento tem o cuidado de não repetir ideias e palavras em favor da elegância do texto. Tal característica enfraquece a coesão léxica e portanto o cálculo da similaridade é prejudicado. Por exemplo, duas sentenças diferem se uma contiver a palavra \textit{computadores} e na seguinte \textit{equipamentos}, mesmo que se refiram à mesma ideia.
	
	Além disso, o documento compartilha um certo vocabulário próprio do ambiente onde os assuntos são discutidos e com isso nota-se que os segmentos, embora tratem de assuntos diferentes, são semelhantes em vocabulário.
	
A presença de ruídos como cabeçalhos, rodapés e numeração de páginas e linhas prejudicam tanto similaridade entre sentenças como a apresentação final ao usuário. Porém, esses ruídos podem ser reduzidos ou eliminados como mostrado na Subseção~\ref{subsec:preprocessamento}, sobre preprocessamento.


\subsection{Medidas de Avaliação}


	As medidas de avaliação tradicionalmente utilizadas em \textit{information retrieval} como precisão e revocação trazem alguns problemas na avalização de segmentadores automáticos.  
Conforme o algoritmo aponta mais segmentos no texto, tende a melhorar a revocação e ao mesmo tempo, reduzir a precisão, um problema que pode ser contornado usando \textit{F-measure} que faz uma combinação da duas levando em conta seus pesos, o que por outro lado é mais difícil de interpretar. 
Essas medidas falham ao não serem sensíveis a \textit{near misses}, ou seja, quando um limite não coincide exatamente com o esperado, mas fica próximo a ele~\cite{Kern2009}.

A Figura~\ref{fig:exemplosegmentacaozoom} mostra um exemplo com duas segmentações hipotéticas e uma referência. Na Figura~\ref{fig:exemplosegmentacao}, em ambos os casos não há nenhum verdadeiro positivo, o que implica em zero para os valores de precisão, acurácia, e revocação, embora a segunda hipótese possa ser considerada superior à primeira se levado em conta a proximidade dos limites.



  \begin{figure}[!h]

	\centering
	\includegraphics[width=0.47\textwidth]{windiffzoom.jpg}
	\caption{Exemplos de \textit{near missing} e falso positivo puro. Os blocos indicam uma unidade de informação e as linha verticais representam os limites entre segmentos. }
	\label{fig:exemplosegmentacaozoom}

  \end{figure}
  
  \begin{figure}[!h]

	\centering
	\includegraphics[width=0.47\textwidth]{windiff.jpg}
	\caption{
	Exemplo de duas segmentações hipotéticas em comparação a uma ideal. 
	}
	\label{fig:exemplosegmentacao}

  \end{figure}
  
  
Entre as medidas mais utilizadas para avaliar segmentadores estão:

\subsubsection{P$_k$}
A fim de resolver o problema de \textit{near misses}, Beeferman \textit{et. al.}~\cite{Beeferman1999} apresentam uma nova medida chamada P$_k$ que atribui valores parciais a \textit{near misses}. Esse método move uma janela de tamanho $k$ e a cada posição e verifica se o início e o final da janela estão ou não dentro do mesmo segmento e penaliza o algoritmo em caso de discrepância. 

Ou seja, dado duas palavras de distancia $k$, uma discrepância é computada quando o algoritmo e a referência não concordam se as palavras estão ou não no mesmo segmento.

O valor de $k$ é calculado como a metade da média dos comprimentos dos segmentos reais. Como resultado, é retornado a contagem de discrepâncias divido pelo quantidade de segmentações analisadas. Esse valor serve como medida de dissimilaridade entre as segmentações e pode ser interpretada como a probabilidade de duas sentenças extraídas aleatoriamente pertencerem ao mesmo segmento.



\subsubsection{WindowDiff}

Pevzner~\cite{Pevzner2002} aponta problemas na avaliação mais tradicional P$_k$~\cite{Beeferman1999}. Eles apontam que esse método penaliza demasiadamente os falsos negativos em relação aos falsos positivos e a \textit{near misses}, além disso, desconsidera o tamanho e a quantidade de segmentos, entre outros problemas.

Como solução, propõem um novo método, o qual chamam de \textit{WindowDiff} que traz duas diferenças principais: a dobra a penalidade para os falsos positivos a fim de diminuir o problema da subestimação dessa medida e, diferente de P$_k$, ao mover a janela pelo texto, penaliza o algoritmo sempre que o número de limites proposto pelo algoritmo não coincidir com o número de limites esperados para aquela janela de texto. 

Com isso, demonstram em seu trabalho que, em relação a P$_k$, consegue resolver seus principais problemas e mantém sua proposta inicial de sensibilidade a \textit{near misses}, penalizando-os menos que os falsos positivos puros.


  

%Falar do software para segmentação manual????


\subsection{Avaliação dos segmentadores}


%%%%%%%%%%
% Parâmetros
%%%%%%%%%%
As implementações dos algoritmos permitem ao usuário a configuração de seus parâmetros. 
%
O \textit{TextTiling} permite ajustarmos dois parâmetros, sendo, o tamanho da janela (distância entre a primeira e a última sentença) para o qual atribuiu-se os valores 20, 40 e 60. Para o segundo parâmetro, o passo (distância que a janela desliza), atribuiu-se os valores 3, 6, 9 e 12. Gerando ao final 20 modelos.
%

O \textit{C99} permite ajustarmos três parâmetros, sendo, a quantidade segmentos desejados, o qual é calculado como uma proporção dos candidatos a limite. Para isso atribuiu-se as proporções de 0,2 a 1,0 em intervalos de 0,2 O segundo parâmetro, o tamanho da máscara utilizada para gerar a matriz de ranking, atribuiu-se os valores 9 e 11. Permite ainda, definirmos se as sentenças serão representados por vetores contendo a frequência ou o peso de cada termo, onde ambas as representações foram utilizadas. Gerando ao final 20 modelos.



%%%%%%%%%%
% Cálculo das medidas para cada modelo
%%%%%%%%%%
Pela comparação dos resultados com a segmentação fornecida pelos especialistas, calculou-se para cada modelo as medidas tradicionais acurácia, precisão, revocação, F-medida. Além dessas, computou-se também as métricas mais aplicadas à segmentação textual P$_k$ e \textit{WindowDiff}.



%%%%%%%%%%
% Teste de Fiedman e CD
% 1ª Etapa
%%%%%%%%%%
Em seguida aplicou-se o teste de Friedman a fim de saber se há diferenças significativas entre a eficácia dos modelos. O pós-teste de Nemenyi foi aplicado para descobrir quais diferenças são significativas. 
%
Exite diferença quando seus \textit{rankings} médios diferirem em um valor mínimo, chamado de diferença critica (CD). 
%

%%%%%%%%%%
% Dados Obtidos
%%%%%%%%%%
Com isso foi possível, pela análise do diagrama de diferença crítica, verificar qual é o melhor modelo para cada medida
% e quão significativamente 
em relação aos demais. 


A tabela~\ref{tab:mediasC99} mostra os dados obtidos com o \textit{C99}, onde \texttt{S} é a proporção de segmentos em relação a quantidade de candidatos, \texttt{M} é o tamanho da máscara utilizada para criar a matriz de \textit{ranking} e \texttt{W} indica se os segmentos são representados por vetores contendo a frequência ou um peso das palavras. 



\begin{table}[!h]
	\centering

	\begin{tabular}{|c|c|c|c|c|}
	
		\hline
		Medida & \texttt{S} & \texttt{M} & \texttt{W} & \textbf{Média}\\		
		\hline

		Acuracy		& 40	& 11 & Sim & 0.6199	\\ \hline	
		F1			& 60	& 9	 & Sim & 0.6167	\\ \hline	
		Precision	& 40	& 11 & Sim & 0.7106	\\ \hline			
		Recall		& 100	& 9	 & Não & 0.8516	\\ \hline		
		Pk			& 40	& 11 & Sim & 0.1163	\\ \hline	
		Windiff		& 40	& 11 & Sim & 0.3800	\\ \hline		

		
	\end{tabular}
	
	\caption{Médias das medidas obtidas com \textit{C99}}
	\label{tab:mediasC99}
\end{table}


A tabelas~\ref{tab:mediasTextTiling} mostra os dados obtidos com o \textit{TextTiling}, onde \texttt{J} é o tamanho da janela e \texttt{P} é o passo.

\begin{table}[!h]
	\centering

	\begin{tabular}{|c|c|c|c|}
	
		\hline
		Medida & \texttt{J} & \texttt{P} & \textbf{Média}\\		
		\hline

		Acuracy		& 50 & 9 	& 0.5510 \\ \hline	
		F1			& 50 & 3 	& 0.5898 \\ \hline	
		Precision	& 60 & 12 	& 0.5746 \\ \hline			
		Recall		& 50 & 3 	& 0.7717 \\ \hline		
		Pk			& 30 & 9 	& 0.1572 \\ \hline	
		Windiff		& 50 & 9 	& 0.4489 \\ \hline		

		
	\end{tabular}
	
	\caption{Médias das medidas obtidas com o \textit{TextTiling}}
	\label{tab:mediasTextTiling}
\end{table}


Uma vez sabendo quais valores de parâmetros melhor configuram um algoritmo para uma medida, resta então saber qual dos dois algoritmos é mais eficiente segundo essa medida. Para isso aplicou-se novamente o teste de Friedman com pós-teste de Nemenyi, dessa vez, com os melhores modelos dos dois algoritmos para cada medida. O resultado segue na Tabela~\ref{tab:melhoresmodelos}

\begin{table}[!h]
	\centering
	
	\begin{tabular}{|c|c|c|c|c|}

		\hline
		Medida & Algoritmo & \texttt{S} & \texttt{M} & \texttt{W}\\		
		\hline
		
	
		Acuracy		& C99 & 40 	& 11	& Sim \\ \hline
		Precision	& C99 & 40 	& 11	& Sim \\ \hline
		Pk			& C99 & 40 	& 11	& Sim \\ \hline
		Windiff		& C99 & 40 	& 11	& Sim \\ \hline
		F1			& C99 & 60 	& 9		& Sim \\ \hline
		Recall		& C99 & 100 & 9		& Não \\ \hline
 	
	
	\end{tabular}

	\caption{Melhores modelos para cada medida segundo diagramas de diferença crítica}
	\label{tab:melhoresmodelos}	
	
\end{table}


Na análise do diagrama de diferença crítica verificou-se que o algoritmo \textit{C99} apresenta melhor eficiência em todas as medidas e os valores das quatro primeiras os valores de \texttt{S}, \texttt{M} e \texttt{W} se repetiram, sugerindo uma configuração otimizada para o problema da segmentação de atas de reunião.







	\section{Conclusão}
	\label{sec:conclusao}
	
	As atas de reunião, objeto de estudo desse artigo, apresentam características peculiares em relação à discursos em reuniões e textos em geral. Características como segmentos curtos e coesão mais fraca devida ao estilo que evita repetição de palavras e ideias em benefício da leitura por humanos, dificulta o processamento por computadores.

	Os algoritmos \textit{TextTiling} e \textit{C99} foram testados em um conjunto de atas reais coletadas do departamento de computação da UFSCar-Sorocaba. Por meio da análise dos dados chegou-se a um modelo cujos segmentos melhor se aproximaram as amostras de participantes das reuniões. Obteve-se resultados comparáveis aos vistos em discursos longos, o que pode ser justificado pelo estilo peculiar de escrita.	
	
	Em trabalhos futuros, serão investigadas técnicas para descrever os segmentos e com isso aprimorar o acesso ao conteúdo das atas de reunião.



\bibliographystyle{sbc}
\bibliography{bibs}
\end{document}


% -> Representação

\section{Representação de Textos} \label{subsection:RepTextos}

As etapas anteriores produzem fragmentos de documentos onde o texto esta em um estágio de processamento inicial, com menos atributos que as versões originais, onde cada fragmento está associado a um tema, porém, ainda desestruturado. Ocorre que as técnicas de mineração de texto exigem uma representação estruturada dos textos conforme será visto na Seção~\ref{subsection:RepTextos}.

Uma das formas mais comuns para que a grande maioria dos algoritmos de aprendizado de máquina possa extrair padrões das coleções de textos é a representação no formato matricial conhecido como Modelo Espaço Vetorial (\textit{Vectorial Space Model} - VSM)~\cite{Rezende2003}, onde os documentos são representados como vetores em um espaço Euclidiano $t$-dimensional em que cada termo extraído da coleção é representado por um dimensão. Assim, cada componente de um vetor expressa a relação entre os documentos e as palavras. Essa estrutura é conhecida como \textit{document-term matrix} ou matriz documento-termo. Uma das formas mais populares para representação de textos é conhecida como \textit{Bag Of Words} a qual é detalhada a seguir.
	
%\subsubsection{Espaço Vetorial} \cite{Rezende2003}

%\subsubsection{Extração de Termos}

\subsection{\textit{Bag Of Words}} \label{subsubsec:BOW}
Após a etapa de pré-processamento, onde as \textit{stop words} foram removidas e as palavras reduzidas ao seus radicais (\textit{stemming}), tem-se uma versão reduzida, com menos atributos, dos dados originais. Essa versão pode ser facilmente convertida em uma tabela ou matriz documento-termo. Essa representação, conhecida como \textit{bag-of-words}, onde cada palavra (termo) não eliminada é transformado em um atributo (\textit{feature})~\cite{Rezende2003}.	Essa representação é mostrada pela Tabela \ref{table:bagofwords}.
		

\begin{table}[!h]
	\centering

	\begin{tabular}{c|c|c|c|c|c|c|c}


	& $Term_1$ & \dots & \dots & $Term_k$ & \dots & \dots & $Term_n$ \\ \hline \hline
	$d_1$ & $a_{11}$ & \dots & \dots & $a_{1k}$ & \dots & \dots & $a_{1n}$   \\ \hline 
	\dots & \dots    & \dots & \dots & \dots    & \dots & \dots & \dots      \\ \hline 
	$d_j$ & $a_{j1}$ & \dots & \dots & $a_{jk}$ & \dots & \dots & $a_{jn}$   \\ \hline 
	\dots & \dots    & \dots & \dots & \dots    & \dots & \dots & \dots      \\ \hline 
	$d_m$ & $a_{m1}$ & \dots & \dots & $a_{mk}$ & \dots & \dots & $a_{mn}$   \\ 

	\end{tabular}
	\caption{Coleção de documentos na representação \textit{bag-of-words}}
	\label{table:bagofwords}\\ 
\end{table}



Essa forma de representação sintetiza a base de documentos em um contêiner de palavras, ignorando a ordem em que ocorrem, bem como pontuações e outros detalhes, preservando apenas o peso de determinada palavra nos documentos.	É uma simplificação de toda diversidade de informações contidas na base de documentos sem o propósito de ser uma representação fiel do documento, mas oferecer a relação entre as palavras e os documentos a qual é suficiente para a maioria dos métodos de aprendizado de máquina~\cite{Rezende2003}. 
%https://en.wikipedia.org/wiki/Document-term_matrix

%=====================
%Sumarizaç\~ao????
%Como Calcular a Avaliaç~ao? Precis~ao? Recall????
%=======================

%\subsubsection{Métricas de relevância dos termos} 

Após a extração dos termos, esses devem receber pesos de acordo com sua relevância dentro da base de documentos. As medidas mais tradicionais são a binária, onde o termo recebe o valor 1 se ocorre em determinado documento ou 0 caso contrário; \textit{document frequency}, é o número de documentos no qual um termo ocorre; \textit{term frequency - tf}, atribui-se ao peso a frequência do termo dentro de um determinado documento; \textit{term frequency-inverse document frequency, tf-idf}, pondera a frequência do termo pelo inverso do número de documentos da coleção em que o termo ocorre.

% (Manning et al., 2008; Feldman e Sanger, 2006) Tese Rafael		



% -> Extração de Tópicos
\section{Extração de Tópicos}


Os modelos de extração de tópicos foram propostos para simplificar e organizar grandes coleções de documentos. Nesse contexto, um tópico é uma estrutura com valor semântico que formam grupos de palavras que frequentemente ocorrem próximas. Essas palavra ajudam a descrever um documento ou sub-conjunto de documentos e ajudam a entender o tema ou assunto do texto onde essa estrutura está presente.

As técnicas de extração de tópicos são abordagens não-supervisionadas que visam encontrar a estrutura semântica de uma coleção de documentos a qual é latente, isto é, desconhecida. Os modelos de extração de tópicos baseiam-se na ideia de que um documento é produzido a partir de tópicos previamente definidos, os quais se deseja aborda no texto e determinam os termos a sem utilizados em um documento.

O processo de elaboração do documento a partir desses tópicos é chamado de processo generativo ou modelo generativo, o qual é desconhecido porém, pode ser estimado  com base nos termos presentes no documento, aqui chamados de variáveis observáveis. Assim, o processo de extração de tópicos consiste em estimar o modelo generativo que de origem ao documento.

Obtém-se ao final do processo de extração de tópicos uma representação documento-tópico que atribui um peso para cada tópico em cada documento da coleção e um representação termo-tópico que representa a probabilidade de ocorrência de um termo em um documento dado que o tópico está presente no documento.
% Convencionalmente uma matrix documento-termo, extremamente esparsa, é decomposta em outras duas matrizes, sendo documento-tópico e termo-tópico~\cite{Cheng2013}.

% Entre as principais abordagens da literatura está o \textit{Latent Dirichlet Allocation} (LDA) sendo referenciado em diversos trabalhos e considerado um dos primeiros modelos dessa área e base para outros trabalhos. 


% Essas são técnicas não-supervisionadas, ou seja, os resultados são extraídos a partir da análise dos documentos sem a necessidade de informações extras como 


% -- A maioria dos trabalho enquadram-se em duas duas principais categorias, os modelos não-probabilísticos e os modelos probabilísticos.

\subsection{Modelos Não Probabilísticos}

Nos modelos não-probabilísticos um tópico é um conjunto de termos e seus pesos que indicam o quão significante esses termos são para um assunto.

A maioria das técnicas não-probabilísticas baseiam-se em técnicas de fatoração de matrizes, onde a matrix documento-termo é projetada em um espaço com menor dimensionalidade chamado \textit{Latent Semantic Space}. Por exemplo, o \textit{Latente Semantic Indexing} (LSI) usa a técnica chamada \textit{Singular Value Decomposition} (SVD) para encontrar padrões no relacionamento entre conceitos e termos em uma coleção de texto não estruturada. Entretanto, esse método não fornece uma interpretação para elementos com valores negativos~\cite{Cheng2013}. % Trocar essa referência do Cheng2013 pela que ele usa na seção 2 do trabalho dele.
% Convencionalmente uma matrix documento-termo, extremamente esparsa, é decomposta em outras duas matrizes, sendo documento-tópico e termo-tópico~\cite{Cheng2013}.

% -- NMF
% Outra modelo popular é o \textit{Non-Negative Matrix Factorization} (NMF) o qual aborda a extração de tópicos como a fatoração de uma matriz documento-termo W, extremamente esparsa de dimensões m x n em uma matriz Z de dimensões n x k e um matriz A de dimensões k x m, onde n é o número de termos, m é o número de documentos da coleção e k é a quantidade de tópicos a serem extraídos. Diferente do LSI, o processo de fatoração garante que não as matrizes resultantes não possuem elementos negativos, permitindo uma interpretação mais intuitiva de seus valores. Além disso, o processo de fatoração proporciona a propriedade de \textit{clustering}, ou seja, automaticamente agrupar as colunas da matriz W, e dessa forma, oferece a característica interessante de agrupar os documentos da coleção. 
Outra modelo popular é o \textit{Non-Negative Matrix Factorization} (NMF) o qual aborda a extração de tópicos como a fatoração de uma matriz documento-termo W, extremamente esparsa em duas matrizes não negativas $Z$ e $A$, tal que a resultante de $ZA$ é uma aproximação da matriz $W$ original.  A corresponde a matriz documento-tópico e possui dimensão $k \times m$. Z corresponde a matriz termo-tópico e possui dimensão $n \times k$ onde $n$ é o número de termos, $m$ é o número de documentos da coleção e $k$ é a quantidade de tópicos a serem extraídos.  Diferente do LSI, no processo de fatoração apenas operações aditivas são permitidas, o que garante que as matrizes resultantes não possuem elementos negativos, permitindo uma interpretação mais intuitiva de seus valores. Além disso, o processo de fatoração proporciona a propriedade de \textit{clustering}, ou seja, automaticamente agrupar as colunas da matriz $W$, e dessa forma, oferece a característica interessante de agrupar os documentos da coleção. Uma vez que $k$ é menor que $n$ e $m$, então $A$ e $Z$ são menores que a matriz de entrada, o que resulta em uma versão comprimida da matriz original.

% -- Onde é utilizado





\subsection{Modelos Probabilísticos}












\section{Trabalhos Relacionados}
% -< O trabalho que mais se aproxima a este, e o ...

% Nessa seção são discutidos os principais trabalhos relacionados esta proposta. Na subseção ... são apresentados alguns 


