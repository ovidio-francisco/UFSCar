
\section{Representação de Textos} \label{subsection:RepTextos}

As etapas anteriores produzem fragmentos de documentos onde o texto esta em um estágio de processamento inicial, com menos atributos que as versões originais, onde cada fragmento está associado a um tema, porém, ainda desestruturado. Ocorre que as técnicas de mineração de texto exigem uma representação estruturada dos textos conforme será visto na Seção~\ref{subsection:RepTextos}.

Uma das formas mais comuns para que a grande maioria dos algoritmos de aprendizado de máquina possa extrair padrões das coleções de textos é a representação no formato matricial conhecido como Modelo Espaço Vetorial (\textit{Vectorial Space Model} - VSM)~\cite{Rezende2003}, onde os documentos são representados como vetores em um espaço Euclidiano $t$-dimensional em que cada termo extraído da coleção é representado por um dimensão. Assim, cada componente de um vetor expressa a relação entre os documentos e as palavras. Essa estrutura é conhecida como \textit{document-term matrix} ou matriz documento-termo. Uma das formas mais populares para representação de textos é conhecida como \textit{Bag Of Words} a qual é detalhada a seguir.
	
%\subsubsection{Espaço Vetorial} \cite{Rezende2003}

%\subsubsection{Extração de Termos}

\subsection{\textit{Bag Of Words}} \label{subsubsec:BOW}
Após a etapa de pré-processamento, onde as \textit{stop words} foram removidas e as palavras reduzidas ao seus radicais (\textit{stemming}), tem-se uma versão reduzida, com menos atributos, dos dados originais. Essa versão pode ser facilmente convertida em uma tabela ou matriz documento-termo. Essa representação, conhecida como \textit{bag-of-words}, onde cada palavra (termo) não eliminada é transformado em um atributo (\textit{feature})~\cite{Rezende2003}.	Essa representação é mostrada pela Tabela \ref{table:bagofwords}.
		

\begin{table}[!h]
	\centering

	\begin{tabular}{c|c|c|c|c|c|c|c}


	& $Term_1$ & \dots & \dots & $Term_k$ & \dots & \dots & $Term_n$ \\ \hline \hline
	$d_1$ & $a_{11}$ & \dots & \dots & $a_{1k}$ & \dots & \dots & $a_{1n}$   \\ \hline 
	\dots & \dots    & \dots & \dots & \dots    & \dots & \dots & \dots      \\ \hline 
	$d_j$ & $a_{j1}$ & \dots & \dots & $a_{jk}$ & \dots & \dots & $a_{jn}$   \\ \hline 
	\dots & \dots    & \dots & \dots & \dots    & \dots & \dots & \dots      \\ \hline 
	$d_m$ & $a_{m1}$ & \dots & \dots & $a_{mk}$ & \dots & \dots & $a_{mn}$   \\ 

	\end{tabular}
	\caption{Coleção de documentos na representação \textit{bag-of-words}}
	\label{table:bagofwords}\\ 
\end{table}



Essa forma de representação sintetiza a base de documentos em um contêiner de palavras, ignorando a ordem em que ocorrem, bem como pontuações e outros detalhes, preservando apenas o peso de determinada palavra nos documentos.	É uma simplificação de toda diversidade de informações contidas na base de documentos sem o propósito de ser uma representação fiel do documento, mas oferecer a relação entre as palavras e os documentos a qual é suficiente para a maioria dos métodos de aprendizado de máquina~\cite{Rezende2003}. 
%https://en.wikipedia.org/wiki/Document-term_matrix

%=====================
%Sumarizaç\~ao????
%Como Calcular a Avaliaç~ao? Precis~ao? Recall????
%=======================

%\subsubsection{Métricas de relevância dos termos} 

Após a extração dos termos, esses devem receber pesos de acordo com sua relevância dentro da base de documentos. As medidas mais tradicionais são a binária, onde o termo recebe o valor 1 se ocorre em determinado documento ou 0 caso contrário; \textit{document frequency}, é o número de documentos no qual um termo ocorre; \textit{term frequency - tf}, atribui-se ao peso a frequência do termo dentro de um determinado documento; \textit{term frequency-inverse document frequency, tf-idf}, pondera a frequência do termo pelo inverso do número de documentos da coleção em que o termo ocorre.

% (Manning et al., 2008; Feldman e Sanger, 2006) Tese Rafael		

