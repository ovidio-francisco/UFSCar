
% --> encerramento da seção de segmentação.
% Essa etapa produz fragmentos de documentos, aqui chamados de subdocumentos onde o texto, assim como no documento original, está em um estágio de processamento inicial, com menos atributos que as versões originais, onde cada fragmento está associado a um tema, porém, ainda desestruturado. Ocorre que as técnicas de aprendizado de máquina exigem uma representação estruturada dos textos conforme será visto na Seção~\ref{section:RepTextos}.

esses subdocumentos devem ser representados computacionalmente de forma mais estruturada


As etapas anteriores produzem fragmentos de documentos onde o texto esta em um estágio de processamento inicial, com menos atributos que as versões originais, onde cada fragmento está associado a um tema, porém, ainda desestruturado. Ocorre que as técnicas de mineração de texto exigem uma representação estruturada dos textos conforme será visto na Seção~\ref{subsection:RepTextos}.



%\subsubsection{Espaço Vetorial} \cite{Rezende2003}

%\subsubsection{Extração de Termos}



%https://en.wikipedia.org/wiki/Document-term_matrix

%=====================
%Sumarizaç\~ao????
%Como Calcular a Avaliaç~ao? Precis~ao? Recall????
%=======================

%\subsubsection{Métricas de relevância dos termos} 



\begin{table}[!h]
	\centering

	\begin{tabular}{|c|c|c|c|c|c|c|c|}

	\hline
	    & $t_1$      & $t_2$     & $t_3$    & \dots& $t_j$    & \dots & $t_n$      \\ \hline
	$d_1$ & $a_{11}$ & $a_{12}$  & $a_{13}$ & \dots& $a_{1j}$ & \dots & $a_{1n}$   \\ \hline 
	$d_2$ & $a_{21}$ & $a_{22}$  & $a_{23}$ & \dots& $a_{2j}$ & \dots & $a_{2n}$   \\ \hline 
	$d_3$ & $a_{31}$ & $a_{32}$  & $a_{33}$ & \dots& $a_{3j}$ & \dots & $a_{3n}$   \\ \hline 
	\dots & \dots    & \dots     & \dots    & \dots& \dots    & \dots & \dots      \\ \hline 
	$d_i$ & $a_{i1}$ & $a_{i2}$  & $a_{i3}$ & \dots& $a_{ij}$ & \dots & $a_{in}$   \\ \hline 
	\dots & \dots    & \dots     & \dots    & \dots& \dots    & \dots & \dots      \\ \hline 
	$d_m$ & $a_{m1}$ & $a_{m2}$  & $a_{m3}$ & \dots& $a_{mj}$ & \dots & $a_{mn}$   \\ \hline 

	\end{tabular}

	\caption{Coleção de documentos na representação \textit{bag-of-words}}
	\label{table:bagofwords}\\ 
\end{table}


% http://auburnbigdata.blogspot.com.br/2013/03/vector-creating-from-document-set.html


Após a extração dos termos, esses devem receber pesos de acordo com sua relevância dentro da base de documentos.


























