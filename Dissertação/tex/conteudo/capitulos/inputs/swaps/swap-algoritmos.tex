
***************************
# Bayesian Unsupervised Topic Segmentation
***************************

- Descreve uma abordagem Bayesiana à segmentação textual;
- 
- Impossibilidade de combinar cue-frases com as métricas tradicionais (cosine)
- {enquadra a coesão léxica em um contexto Bayesiano}
- usa programação dinâmica

"The Bayesian framework provides a principled way to incorporate additional features such as cue phrases, a powerful indicator of discourse structure"

%  ----------------------------------------

"We formalize lexical cohesion in a generative model in which the text for each segment is produced by a distinct lexical distribution." 

%  ----------------------------------------

"More formally, we treat the words in each sentence as draws from a language model associated with the topic segment. This is related to topic-modeling methods such as latent Dirichlet allocation (LDA; Blei et al. 2003), but here the induced topics are tied to a linear discourse structure. This property enables a dynamic programming solution to find the exact maximum-likelihood segmentation." 

%  ----------------------------------------


"Lexical cohesion can be placed in a probabilistic context by modeling the words in each topic segment as draws from a multinomial language model associated with the segment."

"Formally, if sentence t is in segment j, then the bag of words xt is drawn from the multinomial language model θ j."

%  ---------------------------------------- 

This is similar in spirit to hidden topic models such as latent Dirichlet allocation (Blei et al., 2003), but rather than assigning a hidden topic to each word, we constrain the topics to yield a linear segmentation of the document.

%  ---------------------------------------- 

"We model cue phrases as generated from a separate multinomial that is shared across all topics and documents in the dataset"
As frases-pista vem de um único modelo generativo.

%  ---------------------------------------- 

"This is achieved by biasing the selection of samples towards boundaries with known cue phrases"
{Isto é conseguido através da influência da seleção de amostras em direção a fronteiras com frases conhecidas}

%  ---------------------------------------- 

{As palavras que formam a sentença t em um segmento j provem de um modelo de linguagem multinomial θ j}

%  ---------------------------------------- 


"They then build a linear segmentation by adding a switching variable to indicate whether the topic distribution for each sentence is identical to that of its predecessor. Unlike Purver et al., we do not assume a dataset in which topics are shared across multiple documents; indeed, our model can be applied to single documents individually."

{Detecta-se um limite entre sentenças, quando a distribuição de tópicos entre elas for diferente}

%  ---------------------------------------- 


para as frases pistas, o modelo é adaptado para refletir essa ideia. Muda-se a probabilidade de ser um limite quando for uma pista.

fulano mostra que a coesão léxica pode ser enquadrada em um contexto bayesiano modelando as palavras em cada tópico como 







% Galley et al. (2003) encontra as pistas minerando segmentos rotulados em busca de palavras que ocorrem próximas aos limites; Elsner and Charniak (2008) construíram uma lista de frases-pista manualmente;

Fulano e fulana apresentam um método para aproveitar essas pistas

% -< Alguns termos são usados para expressar uma estrutura do texto
% -< Alguns termos são usados para expressar uma estrutura do texto

% -< Alguns termos são usados em tópicos específicos, mas outros são neutros em relação aos tópicos sendo usados para expressar uma estrutura do discurso.

% Beltrano mostra que essas palavras não pertencem a um tópico específico, mas são neutras em relação a algum assunto sendo usadas para expressar a estrutura do discurso.


Fulano e fulana mostram em seu trabalho que a coesão léxica pode ser enquadrada em um contexto bayesiano onde as palavras de um segmento surgem de um modelo de linguagem multinomial o qual é associado a um tópico.
% -< e com isso aproveitam as pistas.


onde as palavras de um segmento surgem de um modelo de linguagem multinomial o qual é associado a um tópico.

onde as palavras de um segmento surgem de modelo generativo em que o texto de cada segmento é produzido por uma distribuição léxica distinta. 











% O método BayesSeg~\cite{} aborda a coesão léxica em um contexto bayesiano onde as palavras de um segmento surgem de um modelo generativo em que o texto de cada segmento é produzido por uma distribuição léxica distinta. Assim, considera-se que as palavras de cada sentença surgem de um modelo de linguagem multinomial o qual está associado ao assunto.		








fulano e fulana apresentam uma abordagem baseada em um modelo bayesiano que permite aproveitar esses indicadores para encontrar segmentos melhores.







1 - TextSeg
2 - MinCut
3 - BayesSeg


Lexical cohesion e.g., 
	- Utiyama and Isahara,     2001; {TextSeg}
	- Galley et al.            2003; {LCSeg}
	- Malioutov and Barzilay,  2006  {}




% Utiyama ==>
% https://github.com/jacobeisenstein/bayes-seg/tree/master/baselines/textseg-1.211

