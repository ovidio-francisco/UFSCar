

Uma delas baseia-se na ideia de um processo generativo ou modelo generativo, 

As técnicas de fatoração de matrizes, analisam estatisticamente 


% Entre as principais está a fatoração de matrizes a qual analisa estatisticamente os termos presentes no documento para descobrir os tópicos relacionados. 


% --> nos probabilísticos também uai
--> Um tópico por sua vez, é uma estrutura com valor semântico que representa um conjunto de termos significantes para o entendimento do tema ao qual o tópico trata.
 % + 
--> Nos modelos não-probabilísticos um tópico é um conjunto de termos e seus pesos que indicam o quão significante esses termos são para um assunto.
% =
% --> Um tópico por sua vez, é uma estrutura com valor semântico que representada por um conjunto de termos e seus pesos que indicam o quão significante esses termos são para um assunto e pode ser útil para o entendimento do tema ao qual o tópico trata.

significantes para o entendimento do tema ao qual o tópico trata.






Nos modelos não-probabilísticos um tópico é definido como uma distribuição sobre um vocabulário fixo. Os termos que o compõe recebem pesos ou probabilidades que indicam o quão significantes esses termos são para um assunto.






% Esses métodos analisam estatisticamente os termos presentes nos documento para descobrir os tópicos relacionados. Seja 


% --> Modelo generativo
% Isso é só uma categoria de agoritmos de extração de tópicos. Ainda tem os baseados em fatorização de matrizes e os baseados em agrupamentos. --> Rafael
Os modelos probabilísticos consideram os documentos como uma mistura de tópicos e um tópico como uma distribuição probabilística sobre os termos. O processo de elaboração do documento a partir desses tópicos é chamado de processo generativo ou modelo generativo, o qual é desconhecido porém pode ser estimado com base nos termos presentes no documento, também chamados de variáveis observáveis. Assim, o processo de extração de tópicos consiste em estimar o modelo generativo que deu origem ao documento.


















% -? As definições de Tópicos e documentos estão corretas? Estão no lugar certo?


é um conjunto de termos e seus pesos 
Um tópico pode ser definido como uma distribuição sobre um vocabulário fixo, tendo palavras com maiores e menores probabilidades.

Os modelos de extração de tópicos são usados para simplificar e organizar grandes coleções de documentos. Nesse contexto, um tópico é uma estrutura com valor semântico que formam grupos de palavras que frequentemente ocorrem próximas. Essas palavra ajudam a descrever um documento ou sub-conjunto de documentos e ajudam a entender o tema ou assunto do texto onde essa estrutura está presente.


% usam analisam quantitativamente os termos 
% encontrar palavras que frequentemente ocorrem próximas
% aprendem decompondo uma matriz documento-termo em matrizes menores usualmente 
% Convencionalmente uma matrix documento-termo, extremamente esparsa, é decomposta em outras duas matrizes, sendo documento-tópico e termo-tópico~\cite{Cheng2013}.
Mais formalmente, os dados podem ser expressos como:
% -> W ~ Ŵ = Z·A
% -> Z·A = Ŵ ~ W

Dado um \textit{corpus} com $N$ 
Os documentos podem ser expressos como vetores: $ d \in D = \{d_1,\dots,d_N\}$ onde $N$ é o numero documentos na coleção. 



[Minicurso4] 
"Na modelagem, cada documento é representado como uma combinação de tópicos e cada tópico é representado por um conjunto de termos, ambos com probabilidades associadas. Assim, cada tópico extraído da coleção possui termos mais relevantes. Analogamente, cada documento possui tópicos mais relevantes de acordo com as respectivas probabilidades."
"Para essa tarefa foram desenvolvidas técnicas probabilísticas chamadas de modelagem de tópicos, que são utilizadas para descobrir, extrair e agrupar documentos de grandes coleções em estruturas temáticas." [Blei 2012]
"Após a aplicação da modelagem dos tópicos, normalmente o resultado é um conjunto de termos que nos indicam ou induzem ao(s) tema(s) ou assunto(s) de uma coleção."
"A modelagem de tópicos, apesar de ser um método estatístico para descobrir temas na estrutura de um corpus, também é vista como uma “clusterização” 1 fuzzy ou soft."[De Oliveira et al. 2007]

% --> a extração de topicos resulta em um conjunto de termos que indicam aos temas ou assuntos de uma coleção de documentos



[Wikipedia - LDA]
"For example, if observations are words collected into documents, it posits that each document is a mixture of a small number of topics and that each word's creation is attributable to one of the document's topics."

"A ideia básica dos modelos de tópicos é descobrir, nas relações entre documentos e termos, padrões latentes que sejam significativos para o entendimento dessas relações. Por exemplo, tais modelos podem ranquear um conjunto de termos como importantes para um ou mais temas. Bem como ranquear documentos como tendo relevância para um ou mais temas."


% --> Definição de Tópico
[Modelos probabilísticos de tópicos: desvendando o Latent Dirichlet Allocation]
"Os tópicos são estruturas com valor semântico e que, no contexto de mineração de texto, formam grupos de palavras que frequentemente ocorrem juntas. Esses grupos de palavras quando analisados, dão indícios a um tema ou assunto que ocorre em um subconjunto de documentos. A expressão tópico é usada levando-se em conta que o assunto tratado em uma coleção de documentos é extraído automaticamente, ou seja, tópico é definido como um conjunto de palavras que frequentemente ocorrem em documentos semanticamente relacionados."





% Na modelagem, cada documento é representado como uma combinação de tópicos e cada tópico é representado por um conjunto de termos, ambos com probabilidades associadas.  [Minicurso4] 
% a maioria dos modelos convencionais como NMF e NMF
% Convencionalmente uma matrix documento-termo, extremamente esparsa, é decomposta em outras duas matrizes, sendo documento-tópico e termo-tópico~\cite{Cheng2013}.

% Entre as principais abordagens da literatura está o \textit{Latent Dirichlet Allocation} (LDA) sendo referenciado em diversos trabalhos e considerado um dos primeiros modelos dessa área e base para outros trabalhos. 

% Essas são técnicas não-supervisionadas, ou seja, os resultados são extraídos a partir da análise dos documentos sem a necessidade de informações extras como 



Links para "Learning Topics in Short Texts by Non-negative Matrix Factorization on Term Correlation Matrix" --> [http://locus.siam.org/doi/abs/10.1137/1.9781611972832.83]
Página do autor "Xiaohui Yan" --> [http://xiaohuiyan.github.io/]



% -- pLSA
% que liga os tópicos aos documentos e as palavras a um tópico atribuindo as respectivas probabilidades a essas ligações.

% Os modelos de extração de tópicos podem se utilizados para mensurar a relevância de termo ou conjunto de termos para determinado assunto ou documento.

% ============= LSI =============


o LSI usa a técnica chamada \textit{Singular Value Decomposition} (SVD) para encontrar padrões no relacionamento entre conceitos e termos em uma coleção de texto não estruturada.

baseia-se na premissa que palavras que ocorrem no mesmo contexto tendem a ter significado similar.


% Os modelos probabilísticos analisam a frequência dos termos a fim de descobrir os assuntos que melhor representam os documentos.

% modelos probabilísticos convencionais são o PLSA e o LDA 

"'Assim, a modelagem probabilística de tópicos é uma abordagem para atacar o problema do agrupamento e organização de dados, principalmente de conteúdo textual e cujo objetivo principal é a descoberta de tópicos e a anotação de grandes coleções de documentos por classificação temática. Tais métodos analisam quantitativamente as palavras dos textos originais para descobrir os temas presentes nos mesmos. Os algoritmos de modelagem de tópicos não requerem nenhum conhecimento prévio dos elementos e os tópicos emergem da análise dos textos originais [Blei 2012].'"













NMF
% Outra modelo popular é o \textit{Non-Negative Matrix Factorization} (NMF) o qual aborda a extração de tópicos como a fatoração de uma matriz documento-termo W, extremamente esparsa de dimensões m x n em uma matriz Z de dimensões n x k e um matriz A de dimensões k x m, onde n é o número de termos, m é o número de documentos da coleção e k é a quantidade de tópicos a serem extraídos. Diferente do LSI, o processo de fatoração garante que não as matrizes resultantes não possuem elementos negativos, permitindo uma interpretação mais intuitiva de seus valores. Além disso, o processo de fatoração proporciona a propriedade de \textit{clustering}, ou seja, automaticamente agrupar as colunas da matriz W, e dessa forma, oferece a característica interessante de agrupar os documentos da coleção. 


% Sobre o PLSA, Removido por Rafael. -->
% Além disso, não permite calcular as probabilidades de um novo documento acrescentado após a criação da modelo.


Implementação do LDA
[https://github.com/datquocnguyen/jLDADMM]
