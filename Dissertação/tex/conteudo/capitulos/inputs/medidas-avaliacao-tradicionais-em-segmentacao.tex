


As medidas de avaliação tradicionais como precisão e revocação são permitem medir o desempenho de modelos de Recuperação de Informação e Aprendizado de Máquina por meio da comparação dos valores produzidos pelo modelo com os valores observados em uma referência. 
%	Avaliações baseadas em hits 
Usa-se uma tabela, chamada matriz de confusão, para visualizar o desempenho de um algoritmo. Na Tabela~\ref{tab:matrizconfusao} é apresentada uma matriz de confusão para duas classes (Positivo e Negativo). 

\begin{table}[!h]
\centering

\begin{tabular}{|c|c|c|}
  \hline
				& Predição Positiva        & Predição Negativa        \\ \hline
  Positivo real & VP (Verdadeiro Positivo) & FN (Falso Negativo)      \\ \hline
  Negativo real & FP (Falso Positivo)      & VN (Verdadeiro Negativo) \\ \hline

\end{tabular}

\caption{Matriz de confusão.}
\label{tab:matrizconfusao}

\end{table}


%
% Falso Positivo 
% Falso Negativo 
% Verdadeiro Positivo 
% O Verdadeiro Negativo, que não existe 
%%% 
No contexto de segmentação textual, um falso positivo é um limite identificado pelo algoritmo que não corresponde a nenhum limite na segmentação de referência, ou seja, o algoritmo indicou que em determinado ponto há uma quebra de segmento, mas na segmentação de referência, no mesmo ponto, não há. De maneira semelhante, um falso negativo é quando o algoritmo não identifica um limite existente na segmentação de referência, ou seja, em determinado ponto há, na segmentação de referência, um limite entre segmentos, contudo, o algoritmo não o identificou.  Um verdadeiro positivo é um ponto no texto indicado pelo algoritmo e pela segmentação de referência como uma quebra de segmentos, ou seja, o algoritmo e a referência concordam que em determinado ponto há uma transição de assunto.  Na avaliação de segmentadores, não há o conceito de verdadeiro negativo. Este seria um ponto no texto indicado pelo algoritmo e pela segmentação de referência onde não há uma quebra de segmentos. Uma vez que os algoritmos apenas indicam onde há um limite, essa medida não é necessária. % Não há ou não e necessário?


% 
% Precisão 
%%%
Nesse sentido, a precisão indica a proporção de limites corretamente identificados pelo algoritmo, ou seja, correspondem a um limite real na segmentação de referência. 
Porém, não diz nada sobre quantos limites reais existem. 
É calculada dividindo-se o número de limites identificados automaticamente pelo número de candidatos a limite (Equação~\ref{equ:precisao}).
 
 \begin{equation}
	 Precis\tilde{a}o = \frac{VP}{VP+FP}
	 \label{equ:precisao}
 \end{equation}

%
% Revocação 
%%%
 A revocação, é a proporção de limites verdadeiros que foram identificados pelo algoritmo. Porém não diz nada sobre quantos limites foram identificados incorretamente. É calculada dividindo-se o número de limites identificados automaticamente pelo número limites verdadeiros (Equação~\ref{equ:revocacao}).
 
 \begin{equation}
	 Revoca\c{c}\tilde{a}o = \frac{VP}{VP+FN}
	 \label{equ:revocacao}
 \end{equation}

 Existe uma relação inversa entre precisão e revocação. Conforme o algoritmo aponta mais segmentos no texto, este tende a melhorar a revocação e ao mesmo tempo, reduzir a precisão. Esse problema de avaliação pode ser contornado utilizado a medida $F^1$ que é a média harmônica entre precisão e revocação onde ambas tem o mesmo peso (Equação~\ref{equ:f1}). 

 \begin{equation}
	 F^1 = \frac{2 \times Precis\tilde{a}o \times Revoca\c{c}\tilde{a}o}
		        {Precis\tilde{a}o + Revoca\c{c}\tilde{a}o}
	 \label{equ:f1}
 \end{equation}



%-> Precisão é a fração de instâncias recuperadas que são relevantes, 
%-> Revocação é a fração de instâncias relevantes que são recuperadas.
%-> https://pt.wikipedia.org/wiki/Precis%C3%A3o_e_revoca%C3%A7%C3%A3o


























