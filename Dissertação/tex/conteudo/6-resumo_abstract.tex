%================================= Resumo e Abstract ========================================
\setlength{\absparsep}{18pt} % ajusta o espaçamento dos paragrafos do resumo 
                     
% --- 
% resumo em português
% --- 
\begin{resumo}
Dissertação (do latim, \emph{disertatio}), é uma modalidade de redação ou tipo de composição, escrita em prosa sobre um tema que se devem apresentar e discutir argumentos, provas, exemplos etc. Nos meios universitários equivale a tese diferenciando-se pelo volume de material, a dissertação seria o material que envolvesse poucas páginas até o limite de cem, enquanto a tese rotularia os textos que ultrapassassem esse número; e pelo aspecto qualitativo, a dissertação pressupõe a capacidade de aplicação de um método de análise e interpretação, enquanto a tese implica a originalidade do tema ou da abordagem à luz do qual é exposta e discutida (fonte: wikipedia).

\textbf{Palavras-chaves}: Dissertação. Texto. Mestrado.

\end{resumo}
% --- 


% --- 
% resumo em ingl�s
% --- 
\begin{resumo}[Abstract]
 \begin{otherlanguage*}{english}

A thesis or dissertation is a document submitted in support of candidature for an academic degree or professional qualification presenting the author's research and findings. In some contexts, the word ``thesis'' or a cognate is used for part of a bachelor's or master's course, while ``dissertation'' is normally applied to a doctorate, while in other contexts, the reverse is true. Dissertations and theses may be considered as grey literature. The required complexity and/or quality of research of a thesis or dissertation can vary by country, university and/or program, therefore, the required minimum study period may vary highly significant in duration. The word dissertation can at times be used to describe a treatise without relation to obtaining an academic degree. The term thesis is also used to refer to the general claim of an essay or similar work (Source: wikipedia).

\textbf{Key-words}: Dissertation. Text. Master.

 \end{otherlanguage*}
\end{resumo}
% --- 

%===========================================================================================
