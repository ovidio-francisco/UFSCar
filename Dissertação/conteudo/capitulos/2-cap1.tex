\chapter{Introdução}\label{cap1}

\let\cleardoublepage\clearpage


A popularização dos computadores possibilitou o armazenamento cada vez maior de conteúdos digitais, sendo bastante comum, o formato textual como livros, documentos, e-mails, redes sociais e páginas web. A produção de textos gera informações em volumes crescentes que superaram a capacidade humana de análise manual.  % Referências ...
%
Além disso, dados textuais quase sempre apresentam-se em formato não estruturado, caracterizado pela ausência de uma organização pré-definida que facilite a busca em sistemas computacionais.
% para que possam ser recuperados.  % Referências ...
%
Contudo, dados textuais possuem uma organização linguística intrínseca o que possibilita a pesquisa com o uso de ferramentas automáticas para manipulação e consulta a bases dados textuais não estruturadas~\cite{Cao:2017, Manning2008}. 
% Essa dificuldade incentiva a pesquisa de ferramentas automáticas para manipulação e consulta a bases dados textuais não estruturadas~\cite{Cao:2017, Manning2008}. 



Assim, os processos de extração automática de conhecimento em coleções textuais são essenciais, e ao mesmo tempo, constituem um desafio, devido às suas características como o formato não estruturado e trechos com diferentes níveis de importância, desde informações essenciais até textos pouco informativos ou mesmo irrelevantes~
\cite{Aggarwal2012, Jeong:2010, Tagarelli2013}. 
 % -< falar disso mais adiante -->

Além dos tipos de informações mais comuns que são armazenados no formato textual, como e-mails, relatórios, artigos e postagens em redes sociais, têm-se também o armazenamento das atas de reuniões, as quais permitem às organizações a documentação oficial de reuniões em arquivos digitais, facilitando a sua confecção e compartilhamento, bem como consulta às decisões tomadas.
% 
% 
%        ========== ==========   Reuniões   ========== ==========
% 
%Seu conteúdo é frequentemente registrado em texto na forma de atas para fins de documentação e consulta posterior. 
Reuniões são tarefas presentes em ambientes de gestão e organizações de um modo geral, onde discute-se problemas, soluções, propostas, alterações de projetos e frequentemente são tomadas decisões importantes onde a comunicação entre os membros da reunião é feita de forma majoritariamente verbal. 

Para que seu conteúdo possa ser registrado e externalizado, adota-se a prática de escrever seu conteúdo em atas~\cite{Miriam2013, Lee2011}. Por exemplo, nas reuniões do conselho de um programa de pós-graduação de uma universidade, são decididos, quais são os critérios para credenciamento e permanência de docentes no programa. Ao longo do tempo, esse tema pode ser discutido e mencionado diversas vezes, podendo os critérios inclusive passar por significativas alterações, devido a diversos fatores. O coordenador do programa pode desejar recuperar qual foi a decisão mais recente, para poder aplicar os critérios a um potencial novo membro do programa, ou os membros do conselho podem desejar rever o histórico de tudo o que já foi discutido/decidido sobre o tema, para poder propor alterações nas regras, de forma mais adequada.

As atas de reunião possuem características particulares. Frequentemente apresentam um texto com poucas quebras de parágrafo e sem marcações de estrutura, como capítulos, seções ou quaisquer indicações sobre o tema do texto. Devido a fatores como a não estruturação e volume dos textos, a localização de um assunto em uma coleção de atas é uma tarefa custosa, especialmente considerando o seu crescimento de seu número em uma instituição. 
%%  -->


As organizações costumam manter seus documentos eletrônicos organizados em pastas e nomeá-los com informações básicas sobre a reunião a que se refere como a data e alguma referência cronológica, por exemplo \textit{``37ª Reunião Ordinária do Conselho ...''}. Essa organização facilita a localização dos arquivos com ferramentas que fazem buscas pelo nome dos arquivos e pastas, sem levar em conta o teor dos documentos. 
%
Também é comum o uso de ferramentas que fazem buscas nos conteúdos dos documentos, buscando por ocorrências de palavras-chave nos textos. Essas ferramentas permitem buscas combinadas com operadores lógicos como \textit{and}, \textit{or} e \textit{not} ou ainda suporte a expressões regulares. Esse recurso, conhecido como \textit{grepping}\footnote{O nome \textit{grepping} é uma referência ao comando \texttt{grep} do Unix}, produz resultados satisfatórios em muitos casos. Por outro lado, traz algumas desvantagens como: 
1) transfere certa complexidade da tarefa ao usuário 
2) não há suporte a padrões mais flexíveis como a proximidade entre as palavras ou palavras que estejam na mesma sentença 
3) informa apenas se um documento casa ou não com a consulta do usuário com base na presença ou ausência dos termos da consulta~\cite{Aggarwal2012,Manning2008}. 
 

Ainda nesse contexto, usa-se outras técnicas de Recuperação de Informação como o Modelo de Espaço Vetorial para ranquear documentos atribuindo pontuações para a similaridade de cada par documento/consulta. Com isso, é possível apresentar os documentos ordenados conforme a sua relevância com a consulta~\cite{Gutierrez2016, Croft2009, Manning2008}. 
% Utiliza-se também dicionários de sinônimos (\textit{thesaurus}) para expandir a consulta do usuário por meio de conhecimento externo adicionado ao sistema~\cite{X,Y,Z}. % e com isso ...
% TODO: Referências
%
Contudo, essas técnicas baseiam-se na frequência de palavras, em que os documentos e consultas são vistos como conjuntos de termos sem levar em conta relações entre termos que compartilham um mesmo tópico dentro do domínio~\cite{WEIXING}. Por exemplo, as consultas ``\textit{alunos bolsa CAPES}'' e ``\textit{suporte financeiro a pesquisa}'' podem estar relacionados a um assunto em comum,
% tratam de um mesmo assunto, 
nesse caso, a transferência de valores monetários como apoio a carreira acadêmica.
% -> ou ainda "assistência téncina em computadores" e "manutenção de PCs" que tratam de assuntos muito próximos, "conservação de equipamentos de informática".
% -> ou ainda "Workshop desenvolvimento de sites" e "oficina webdesign" que tratam de assuntos muito próximos, "".
Utilizando-se das técnicas até agora mencionadas, obteria-se resultados distintos para cada caso, uma vez que as consultas não compartilham termos e não há relação direta ente eles. Como efeito, os resultados de cada consulta limitariam-se a documentos que compartilham termos com a consulta.
% com termos em comum a consulta.
Essas técnicas produzem resultados melhores a medida que o usuário fornece termos mais acertados na consulta, o que por vezes é dependente de certo conhecimento e familiaridade com o domínio no qual a coleção de documentos está inserida. 
% 
Além disso, o retorno ao usuário é uma lista documentos integrais, o que pode exigir uma segunda busca dentro de um documento para encontrar o trecho desejado.





% Essas técnicas permitem melhorar a busca por informações em atas de reunião~. 
%  Encontra variaveis latentes e agrega informações.
% 
%        ==========   Necessidade de consultas  ==========
% 
Uma vez que a ata registra a sucessão de assuntos discutidos na reunião, um sistema de recuperação de informação idealmente deve retornar ao usuário apenas os trechos que tratem do assunto pesquisado ao invés de documentos inteiros. Assim, cada trecho com um assunto predominate pode ser considerado um subdocumento. Portanto, em primeiro lugar, há a necessidade de descobrir onde há mudanças de assunto no texto. 
% Falar do 'segundo lugar' que é indentificar os tópicos !


% Esse trabalho tem 3 tarefas principais: 
% 1. Segmentar as atas em trechos que tratem de um único assunto.
% 2. Identificar o assunto de cada trecho.
% 3. Ranquear os trechos conforme a relevância com a consulta do usuário.


%        ========== ==========|   Segmentação  |========== ==========

Técnicas de segmentação automática de textos (segmentação textual) podem ser aplicadas com esse propósito. Elas podem dividir um documento em segmentos que contenham um assunto relativamente independente e gerar um conjunto de subdocumentos derivado da coleção de atas original~\cite{Aggarwal2018, bokaei2015a, sakahara2014a, misra2009a, Eis2008}.
%--> onde é usada e como pode ser usada aqui



%        ========== ==========|   Extração de Tópicos  |========== ==========

Contudo, a segmentação textual apenas indica as transições de assuntos ao longo do texto,  sem indicações sobre o teor dos segmentos. O assunto de cada trecho pode ser estimado por meio de modelos de extração de tópicos. Essa técnica possibilita a formação de grupos de segmentos que compartilham o mesmo assunto bem como indicar palavras que melhor descrevem o grupo~\cite{Wei2007}. Com isso, obtém-se uma organização da coleção de documentos que favorece técnicas para navegação e consulta à coleção de documentos~\cite{Maracini2010}. Tais modelos podem eleger um conjunto de termos importantes para um ou mais assuntos, bem como ranquear documentos por sua relevância para determinado tema~\cite{Faleiros2016,Xing2009}.




%        ========== ==========|   Recuperação de Informação  |========== ==========
% ->> RI
% --> tópico é o assunto extraído automaticamente 
% --> segmento é o trecho extraído automaticamente



%        ========== ==========|   Unindo as coisas  |========== ==========

% --> Falando do problema ??
Devido às características das atas, como a multiplicidade de assuntos, e ausência de meta-informação, as técnicas de segmentação podem ser empregadas em conjunto com modelos de extração de tópicos para criar uma estrutura de dados derivada da coleção de documentos original. 
Essa abordagem visa, em primeiro lugar, identificar os assuntos tratados em cada ata e gerar uma coleção de subdocumentos derivados da coleção de atas originais e, a partir disso, utilizar modelos de extração de tópicos para encontrar relações latentes entre os subdocumentos e termos da coleção.
Como resultado, obtém-se uma organização da coleção em que os segmentos são agrupados por assuntos e acrescidos de um conjunto de termos que representam os principais tópicos ou assuntos identificados na coleção de documentos, dessa forma, incorporando conhecimento aos dados originais. Esse novos atributos e a organização das atas por temas podem ser usados para 
% os descrevem, 
% podendo assim 
expandir o espaço de busca a fim de aprimorar técnicas de recuperação de informação em um sistema para extração de conhecimento em coleções de atas de reunião.
% como texto seccionado e \textit{hyperlinks} 



%        ======== ========   Essa abordagem traz como vantagens   ======== ========

Essa abordagem traz vantagens para a recuperação de informação em coleções de documentos com múltiplos tópicos como atas de reunião.
% -> 1 Segmentação
A segmentação das atas permite ao sistema criar uma base de documentos mais simples, em que os assuntos estão isolados em segmentos, formando assim, um \textit{corpus} mais adequado aos métodos de aprendizado de máquina e recuperação de informação, empregados nesse trabalho. Além disso, permite ao sistema final exibir apenas os trechos onde o assunto pesquisado está presente ao invés de entregar documentos integrais~\cite{Tagarelli2013, Jeong:2010, Prince2007, Huang2003}. 

% -> 2 Extração de tópicos -- descritores
Modelos de extração de tópicos podem se integrar ao processo de recuperação de informação usando os agrupamentos e seus descritores como uma forma de descrição da coleção~\cite{Zhai2017, Xing2009}, bem como relacionar termos distintos com um determinado tópico em comum~\cite{WEIXING}. A busca por palavras-chave em descritores transfere esforço computacional da varredura dos documentos para a etapa de extração de tópicos. Essa estratégia evita processamento e lentidão no momento da pesquisa de maneira semelhante à criação de índices para aumentar a eficiência em sistemas de recuperação de informação. Além disso, encontra relações entre termos e documentos sem necessidade de conhecimento externo sobre o domínio. Assim, nesse trabalho os modelos de extração de tópicos são utilizados para incorporar informação aos segmentos com a finalidade de aprimorar o ranqueamento dos resultados~\cite{Maracini2010, WEIXING}. 
% 
% -> 3 Extração de tópicos -- agrupamento
Além disso, ao agrupar os segmentos por tópicos, tem-se uma organização dos documentos que permite a visualização de segmentos semelhantes permitindo a navegação e exploração aos grupos além das consultas por palavras-chave.
%


%        ========== ==========   Objetivos   ========== ==========

Diante desse cenário, o objetivo desse trabalho de mestrado é propor o desenvolvimento uma ferramenta para identificar, organizar e apresentar assuntos registrados em atas de reunião utilizando a estrutura latente de documentos segmentados em conjunto com técnicas de recuperação de informação. 
%
Como objetivos específicos, esse trabalho visa  
% 1)
dar início a investigação de métodos de segmentação textual, extração de tópicos e recuperação de informação no contexto de atas de reunião. Para conhecer a eficiência das técnicas de segmentação textual, seus resultados foram analisados tendo como referência anotações coletadas de profissionais que desempenham atividades ligadas a atas e reuniões.
% 2)
Avaliar junto ao usuário a qualidade dos subdocumentos apresentados quanto ao agrupamento e relevância das informações contidas. Para isso, foi feito uma coleta de dados sobre a percepção de usuários sobre a qualidade dos resultados apresentados em uma consulta a coleção de atas por meio do sistema.
% 
Dessa forma, busca-se ajudar a suprir a necessidade de ferramentas para para esse cenário e contribuindo com uma metodologia de extração de informação em documentos com múltiplos assuntos. Além disso, disponibilizar o sistema implementado bem como os dados coletados durante a análise das técnicas de forma a contribuir com novos trabalhos relacionados a esse contexto.

% --> Disponibilisação do sistema final.



% \section{Organização da Dissertação}

No Capítulo~\ref{cap2} são apesentados os principais conceitos da Recuperação de Informação, de Mineração de Textos, Segmentação Textual e Extração de Tópicos.
%
No Capítulo~\ref{cap3} é proposto um sistema para extração de conhecimento em atas de reunião e inicia-se o seu desenvolvimento com base nas técnicas mencionadas. 
Ainda nesse capítulo, é mostrado um estudo de caso da aplicação das técnicas de Segmentação Textual e Extração de Tópicos em uma coleção de atas. 
%
No Capítulo~\ref{cap-segmentadores} é apresentada uma avaliação objetiva em algoritmos de Segmentação Textual a fim de analisar seus resultados e escolher um segmentador a ser utilizado no sistema proposto.
anotações
%
No Capítulo~\ref{cap-extratores} é apresentada uma avaliação subjetiva dos resultados dos modelos de Extração de Tópicos sob um \textit{corpus} composto por segmentos de atas. Adicionalmente, avaliou-se o segmentador utilizado, uma vez que este está ligado aos resultados avaliados. 
%
Por fim as conclusões do trabalho, as limitações dessa pesquisa e os trabalhos futuros são apresentados no último capítulo dessa dissertação.





% "Diante do cenário descrito na seção anterior, a hipótese levantada neste trabalho é que a qualidade da recomendação é melhorada quando são utilizadas hierarquias de tópicos como informações contextuais."











% // https://verbosus.com/bibtex-style-examples.html
