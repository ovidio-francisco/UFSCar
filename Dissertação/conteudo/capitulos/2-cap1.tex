\chapter{Introdução}\label{cap1}

\let\cleardoublepage\clearpage


A popularização dos computadores possibilitou o armazenamento cada vez maior de conteúdos digitais, sendo bastante comum, o formato textual como livros, documentos, e-mails, redes sociais e páginas web. A produção de textos gera informações em volumes crescentes que superaram a capacidade humana de análise manual.  % Referências ...
%
Além disso, dados textuais quase sempre apresentam-se em formato não estruturado, caracterizado pela ausência de uma organização pré-definida que facilite a busca em sistemas computacionais.
% para que possam ser recuperados.  % Referências ...
%
Contudo, dados textuais possuem uma organização linguística intrínseca o que possibilita a pesquisa de ferramentas automáticas para manipulação e consulta a bases dados textuais não estruturadas~\cite{Cao:2017, Manning2008}. 
% Essa dificuldade incentiva a pesquisa de ferramentas automáticas para manipulação e consulta a bases dados textuais não estruturadas~\cite{Cao:2017, Manning2008}. 



Assim, os processos de extração automática de conhecimento em coleções textuais são essenciais, e ao mesmo tempo, constituem um desafio, devido às suas características como o formato não estruturado e trechos com diferentes níveis de importância, desde informações essenciais até textos pouco informativos ou mesmo irrelevantes. 
 % -< falar disso mais adiante -->

Além dos tipos de informações mais comuns que são armazenados no formato textual, como e-mails, relatórios, artigos e postagens em redes sociais, têm-se também o armazenamento das atas de reuniões, as quais permitem às organizações a documentação oficial de reuniões em arquivos digitais, facilitando a confecção, compartilhamento e consulta às decisões tomadas.
% 
% 
%        ========== ==========   Reuniões   ========== ==========
% 
%Seu conteúdo é frequentemente registrado em texto na forma de atas para fins de documentação e consulta posterior. 
Reuniões são tarefas presentes em ambientes de gestão e organizações de um modo geral, onde discute-se problemas, soluções, propostas, alterações de projetos e frequentemente são tomadas decisões importantes onde a comunicação entre os membros da reunião é feita de forma majoritariamente verbal. Para que seu conteúdo possa ser registrado e externalizado, adota-se a prática de escrever seu conteúdo em atas~\cite{Miriam2013, Lee2011}.

Por exemplo, nas reuniões do conselho de um programa de pós-graduação de uma universidade, são decididos, quais são os critérios para credenciamento e permanência de docentes no programa. Ao longo do tempo, esse tema pode ser discutido e mencionado diversas vezes, podendo os critérios inclusive passar por significativas alterações, devido a diversos fatores. O coordenador do programa pode desejar recuperar qual foi a decisão mais recente, para poder aplicar os critérios a um potencial novo membro do programa, ou os membros do conselho podem desejar rever o histórico de tudo o que já foi discutido/decidido sobre o tema, para poder propor alterações nas regras, de forma mais adequada.

As atas de reunião possuem características particulares. Frequentemente apresentam um texto com poucas quebras de parágrafo e sem marcações de estrutura, como capítulos, seções ou quaisquer indicações sobre o tema do texto. Devido a fatores como a não estruturação e volume dos textos, a localização de um assunto em uma coleção de atas é uma tarefa custosa, especialmente considerando o seu crescimento em uma instituição. 
%%  -->


As organizações costumam manter seus documentos eletrônicos organizados em pastas e nomeá-los com informações básicas sobre a reunião a que se refere como a data e alguma referência cronológica, por exemplo \textit{``37ª Reunião Ordinária do Conselho ...''}. Essa organização facilita a localização dos arquivos com ferramentas que fazem buscas pelo nome dos arquivos e pastas, sem levar em conta o teor dos documentos. 
%
Também é comum o uso de ferramentas que fazem buscas nos conteúdos dos documentos, buscando por ocorrências de palavras-chave nos textos. Essas ferramentas permitem buscas combinadas com operadores lógicos como \textit{and}, \textit{or} e \textit{not} ou ainda suporte a expressões regulares. Esse recurso, conhecido como \textit{grepping}\footnote{O nome \textit{grepping} é uma referência ao comando \texttt{grep} do Unix}, produz resultados satisfatórios em muitos casos. Por outro lado, traz algumas desvantagens como: 
1) transfere certa complexidade da tarefa ao usuário 
2) não há suporte a padrões mais flexíveis como a proximidade entre as palavras ou palavras que estejam na mesma sentença 
3) informa apenas se um documento casa ou não com a consulta do usuário com base na presença ou ausência dos termos da consulta~\cite{Aggarwal2012,Manning2008}. 
 

Ainda nesse contexto, usa-se outras técnicas de Recuperação de Informação como o Modelo de Espaço Vetorial para ranquear documentos atribuindo pontuações para a similaridade de cada par documento/consulta. Com isso, é possível apresentar os documentos ordenados conforme a sua relevância com a consulta. Utiliza-se também dicionários de sinônimos (\textit{thesaurus}) para expandir a consulta do usuário por meio de conhecimento externo adicionado ao sistema~\cite{X,Y,Z}. % e com isso ...
% TODO: Referências
%
Contudo, essas técnicas baseiam-se na frequência de palavras, em que os documentos e consultas são vistos como conjuntos de termos sem levar em conta relações entre termos que compartilham um mesmo tópico dentro do domínio~\cite{WEIXING}. Por exemplo, as consultas ``\textit{alunos bolsa CAPES}'' e ``\textit{suporte financeiro a pesquisa}'' 
podem estar relacionados a um assunto em comum,
% tratam de um mesmo assunto, 
nesse caso, a transferência de valores monetários como apoio a carreira acadêmica.
% -> ou ainda "assistência téncina em computadores" e "manutenção de PCs" que tratam de assuntos muito próximos, "conservação de equipamentos de informática".
% -> ou ainda "Workshop desenvolvimento de sites" e "oficina webdesign" que tratam de assuntos muito próximos, "".
Utilizando-se das técnicas até agora mencionadas, obteria-se resultados distintos para cada caso, uma vez que as consultas não compartilham termos e não há relação direta ente eles. Como efeito, os resultados de cada consulta limitariam-se a documentos que compartilham termos com a consulta.
% com termos em comum a consulta.
Essas técnicas produzem resultados melhores a medida que o usuário fornece termos mais acertados na consulta, o que por vezes é dependente de certo conhecimento e familiaridade com o domínio no qual a coleção de documentos está inserida. 

Além disso, o retorno ao usuário é uma lista documentos integrais, o que pode exigir uma segunda busca dentro de um documento para encontrar o trecho desejado.




% TODO: Referências de trabalhos que têm feito isso   ↓↓↓↓↓↓↓
% Para superar essas limitações têm sido utilizadas técnicas de aprendizado de máquina por meio de diversas abordagens. Por exemplo, elas vêm sendo empregadas na organização, gerenciamento, recuperação de informação e extração de conhecimento, como a extração de tópicos e a categorização de automática de documentos~\cite{Purver2006}.  % Estava sobrando :: Removido em 13-05-2018

% Essas técnicas permitem melhorar a busca por informações em atas de reunião~. 
%  Encontra variaveis latentes e agrega informações.
% 
%        ==========   Necessidade de consultas  ==========
% 
Uma vez que a ata registra a sucessão de assuntos discutidos na reunião, um sistema de recuperação de informação idealmente deve retornar ao usuário apenas os trechos que tratem do assunto pesquisado ao invés de documentos inteiros. Assim, cada trecho com um assunto predominate pode ser considerado um subdocumento. Portanto, em primeiro lugar, há a necessidade de descobrir onde há mudanças de assunto no texto. 
% Falar do 'segundo lugar' que é indentificar os tópicos !


%--> explicar cada um dos 3, e no final colocar esse texto com os objetivos.

%        ========== ==========|   Segmentação  |========== ==========

% --> 1) Descobrir quando há uma mudança de assunto. 
Técnicas de segmentação automática de textos (segmentação textual) podem ser aplicadas com esse propósito. A tarefa de segmentação automática de textos, ou segmentação textual consiste em dividir um texto em partes que contenham um significado relativamente independente. Em outras palavras, é identificar as posições nas quais há uma mudança significativa de assunto. É útil em aplicações que trabalham com textos sem indicações de quebras de assunto, ou seja, não apresentam seções ou capítulos, como transcrições automáticas de áudio, vídeos e grandes documentos que contêm vários assuntos como atas de reunião e notícias~\cite{ Bokaei2015, Sakahara2014, Misra2009, Eisenstein2008, Choi2000}.
% <-- Referências
Pode ser usada para melhorar o acesso a informação solicitada por meio de uma consulta, onde é possível oferecer porções menores de texto mais relevantes ao invés de exibir um documento grande que pode conter informações menos pertinentes.  Além disso, encontrar pontos onde o texto muda de assunto, pode ser útil como etapa de pré-processamento em aplicações voltadas ao entendimento do texto, principalmente em documentos longos~\cite{Choi2000}.

%--> onde é usada e como pode ser usada aqui




%        ========== ==========|   Extração de Tópicos  |========== ==========

Contudo, a segmentação textual apenas indica as transições de assuntos ao longo do texto,  
% sem indicações do que cada segmento trata.
sem indicações sobre o teor dos segmentos.
% pela qual obtém-se trechos com um assunto principal e relativamente independente do texto integral. 
O assunto de cada trecho pode ser estimado por meio de técnicas de extração de tópicos.  
Os modelos de extração de tópicos fornecem uma estratégia que visa encontrar nas relações entre documentos, padrões latentes que sejam significativos para o entendimento dessas relações~\cite{Wei2007}. Tais modelos podem ranquear um conjunto de termos importantes para um ou mais assuntos, bem como ranquear documentos por sua relevância para determinado tema~\cite{Faleiros2016,Xing2009}.
Atualmente, destacam-se os modelos probabilísticos de extração de tópicos como LDA~\cite{Blei2003} e PLSA~\cite{Hofmann1999}. São abordagens amplamente utilizadas ~\cite{DZhu20122} e frequentemente referenciadas em trabalhos que buscam extrair conhecimento e organizar bases textuais ~\cite{Aggarwal2018, OCallaghan2015, Steyvers2007}.  
%
%
Nesse trabalho, a expressão \textit{tópico} é usada para designar um assunto considerando que o mesmo foi extraído por meio de técnicas automáticas, ficando a expressão \textit{assunto} utilizada como seu teor popular. 

O processo de extração de tópicos atribui um peso a cada documento-tópico e uma relação termo-tópico que pode representar a probabilidade de ocorrência de um termo em um documento dado que o tópico está presente. A partir dessas representações, é possível agrupar documentos que compartilham o mesmo tópico bem como os termos que melhor descrevem o tópico~\cite{Aggarwal2018}. Com isso, obtém-se uma organização da coleção de documentos que favorece técnicas para navegação e consulta à coleção de documentos~\cite{Maracini2010}. 
% 
Além disso, essas abordagens de extração de tópicos fornecem a construção de novos atributos que representam os principais tópicos ou assuntos identificados na coleção de documentos, sendo uma oportunidade de incorporar conhecimento de domínio aos dados~\cite{Guyon2003}. 
% com o mesmo teor 



%        ========== ==========|   Recuperação de Informação  |========== ==========
% ->> RI
% --> tópico é o assunto extraído automaticamente 
% --> segmento é o trecho extraído automaticamente



%        ========== ==========|   Unindo as coisas  |========== ==========


Devido as características das atas como ausência de meta-informação e a multiplicidade de assuntos, as técnicas de segmentação podem ser empregadas em conjunto com a extração de tópicos para aprimorar a recuperação de informação em coleções de atas de reunião. 
% como texto seccionado e \textit{hyperlinks} 

Nesse trabalho, a segmentação textual é empregada na identificação dos assuntos tratados em cada ata. Os segmentos de atas gerados, aqui considerados como sub-documentos, são entregues a um extrator de tópicos. A tarefa do extrator é agrupar os segmentos com um mesmo tópico e eleger palavras da coleção que melhor descrevem cada grupo. Os descritores gerados pelo extrator de tópicos bem com os grupos são aproveitados nas técnicas de recuperação de informação a fim de ranquear os resultados e auxiliar o usuário a encontrar segmentos semelhantes por meio da navegação pelos grupos. Essa abordagem visa, em primeiro lugar, identificar os assuntos tratados em cada ata e gerar uma coleção de subdocumentos derivados da coleção de atas originais e, a partir disso, encontrar relações latentes entre os subdocumentos e termos da coleção. Com isso, tem-se uma estrutura que permite às técnicas de recuperação de informação fazer buscas à coleção de atas valendo-se da estrutura criada a partir das técnicas de segmentação textual e extração de tópicos. 




%        ======== ========   Essa abordagem traz como vantagens   ======== ========

% Essa abordagem traz vantagens em relação às técnicas de recuperação de informação 
%
%
Essa abordagem traz vantagens para a recuperação de informação em coleções de documentos com múltiplos tópicos como atas de reunião.
% -> 1 Segmentação
A segmentação das atas permite ao sistema criar uma base de dados interna mais simples, em que os assuntos estão isolados em segmentos, formando assim, uma base de dados mais adequada aos métodos de aprendizado de máquina e recuperação de informação, empregados nesse trabalho. Além disso, permite ao sistema final exibir apenas os trechos onde o assunto pesquisado está presente ao invés de entregar documentos integrais~\cite{Tagarelli2013, Jeong:2010, Prince2007, Huang2003}. 
%

% -> 2 Extração de tópicos -- descritores

Modelos de extração de tópicos podem se integrar ao processo de recuperação de informação usando os agrupamentos e seus descritores como uma forma de descrição da coleção~\cite{Zhai2017, Xing2009}, bem como relacionar termos distintos com um determinado tópico em comum~\cite{WEIXING}.  
A busca por palavras-chave em descritores transfere processamento da varredura dos documentos para o processo de extração de tópicos. Esse processo evita processamento e lentidão no momento da pesquisa. Essa estratégia é semelhante a criação de índices para aumentar a eficiência em sistemas de recuperação de informação. Além disso, encontra relações entre termos e documentos sem necessidade de conhecimento externo sobre o domínio.
%
Assim, nesse trabalho os modelos de extração de tópicos são utilizados para incorporar informação aos segmentos com a finalidade de aprimorar o ranqueamento dos resultados~\cite{Maracini2010, WEIXING}. 
% 
% -> 3 Extração de tópicos -- agrupamento
Além disso, ao agrupar os segmentos por tópicos, tem-se uma organização dos documentos que permite a visualização de segmentos semelhantes permitindo a navegação e exploração aos grupos além das consultas por palavras-chave.
%



% // https://verbosus.com/bibtex-style-examples.html




%        ========== ==========   Aplicando na UFSCar   ========== ==========
% --> Seleção das atas
% --> 1º Teste Segmentação, rotulação, apontamento de descritores
% --> 2º Teste Percepção do usuário.
% --> Disponibilisação do sistema final.




\section{Objetivos}
%        ========== ==========   Final   ========== ==========

A Busca por informação em dados textuais implica basicamente em consultar uma base de dados não estruturada. O que se objetiva nesse trabalho de mestrado é utilizar a estrutura latente de documentos segmentados em conjunto com técnicas tradicionais de recuperação de informação, tendo como 
% O objetivo geral desse trabalho de mestrado é desenvolver uma 
% Esse trabalho tem como 
proposta o desenvolvimento de uma ferramenta para identificar, organizar e apresentar assuntos registrados em atas de reunião. O projeto se divide em dois objetivos específicos: 1) Avaliar os métodos de segmentação topical, extração de tópicos e recuperação de informação no contexto de atas de reunião. 2) Analisar a qualidade dos subdocumentos apresentados quanto ao agrupamento, e relevância das informações contidas.
Dessa forma, busca-se ajudar a suprir a necessidade de ferramentas para para esse cenário e contribuir com uma metodologia de extração de informação em documentos com assuntos que pode ser útil em trabalhos relacionados a outros domínios como documentos jurídicos, notícias e transcrições de conversas.


% Assim, nesse contexto, este trabalho propõe a investigação do uso de mineração de texto e as técnicas que constituem o estado da arte na área para o desenvolvimento de uma ferramenta para extração automática de históricos de decisão em atas de reuniões.  % -- <-- 'POBRE'



% "Diante do cenário descrito na seção anterior, a hipótese levantada neste trabalho é que a qualidade da recomendação é melhorada quando são utilizadas hierarquias de tópicos como informações contextuais."






\section{Organização da Dissertação}





