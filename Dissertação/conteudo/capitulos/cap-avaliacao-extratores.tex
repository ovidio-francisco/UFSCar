\chapter{Avaliação dos Extratores de Tópicos}\label{cap-extratores}

% Dar uma geral nesse primeiro parágrafo
	% Escolha dos Algorítmos
	% Escolha do método de avaliação (subjetivo com questionario)

% -- 1 - Motivação para o experimento.
Nesse capítulo, as técnicas de extração de tópicos são analisadas. O objetivo é comparar os algoritmos de extração de tópicos na tarefa de extração de padrões no contexto das atas de reunião no que tange a qualidade dos agrupamentos e seus descritores bem como sua capacidade de representar os segmentos. Escolheu-se os modelos LDA, PLSA e K-Means para essa análise devido a popularidade desses métodos os quais são amplamente utilizados~\cite{DZhu20122} e frequentemente referenciados em trabalhos voltados a organização de bases textuais~\cite{Aggarwal2018, OCallaghan2015, Steyvers2007}.
Os algoritmos foram inicialmente configurados com base em avaliações internas~\cite{Hassani2017} e observações empíricas nas quais escolheu-se os melhores valores para seus parâmetros. Os resultados desses modelos foram submetidos a uma avaliação subjetiva a fim de analisá-los junto a usuários com afinidade com atas de reuniões. 

Nessa avaliação, a técnica de segmentação textual também foi avaliada, uma vez que é a etapa anterior a extração de tópicos está diretamente ligada a os resultados apresentados ao avaliador bem como pode interferir no funcionamento dos modelos de extração de tópicos. Assim, a técnica de segmentação textual foi avaliada subjetivamente em complemento a análise estatística apresentada no Capítulo~\ref{cap-segmentadores}.

A avaliação se deu por meio de questionários onde profissionais com afinidade com atas de reunião forneceram suas percepções em relação aos resultados dos modelos de extração de tópicos. Por fim, os dados obtidos dos experimentos serviram de base para as análises dos algoritmos e de sua aplicação no contexto das atas de reuniões.
% os dados serviram como base para análise dos algoritmos 

\section{Configuração experimental}

% A qtd de tópicos é um parametro importante.  % Todos com 70 tópicos e 5 extratores;
Durante os primeiros testes empíricos a qualidade dos resultados mostrou-se sensível à quantidade de tópicos extraídos.
Inicialmente, realizou-se um teste prévio utilizando uma versão não-paramétrica dos algorítimos a fim de automaticamente obter valores ótimos para esse parâmetro por meio da análise das medidas Silhueta e Coesão. Essa configuração automática resulta valores em torno de 20 tópicos. Contudo, apresenta grupos com muitos segmentos (em torno de 100) o que os tornam pouco coesos além de certa dificuldade em julgar capacidade dos descritores em representar bem o tópico.  
Em observações diretas, valores próximos a entre 60 e 80 mostraram melhores resultados. Nesse trabalho, optou-se por configurar os algoritmos para extrair 70 tópicos da coleção de segmentos por apresentar melhores resultados aparentes na visão de usuário.

Outro fator importante é a quantidade de descritores selecionados para cada tópico. Com base no experimento de anotações em segmentação, descrito no Capítulo~\ref{cap2}, os anotadores selecionaram em média 5 palavras para descrever os segmentos, sendo esse valor adotado para essa avaliação.




\section{Critérios de avaliação}


% - Explicar que foram 2 consultas (quais) e mostrar os extratores.
% - 4 - Procedimento (2 consultas, técnicas em ordens diferentes).
Após a configuração, cada um dos modelos de extração de tópicos foi submetido a duas consultas: ``\textit{compra de equipamentos}'' e ``\textit{defesa de dissertação}'' gerando 6 cenários distintos a serem analisados. 
Para cada cenário, o sistema seleciona o tópico com maior relevância com a consulta e em seguida exibe 5 segmentos desse tópico escolhidos aleatoriamente. 
Vale dizer que nessa avaliação as técnicas de ranqueamento dos resultados não são aplicadas para que estas não interfiram na avaliação dos extratores, contudo, o sistema final poderá ranquear também os segmentos com maior relevância de um ou mais tópicos por meio de técnicas de recuperação de informação. 
Os resultados desses cenários foram apresentados a um grupo de avaliadores que individualmente avaliaram a qualidade das técnicas de extração de tópicos. 
%

% -- 3 - Escolha dos participantes.
O perfil dos avaliadores é de profissionais da area acadêmica/escolar devido à sua afinidade com o ambiente de gestão e conhecimentos de assuntos relacionados ao \textit{corpus} estudado nesse trabalho. O grupo convidado a participar do experimento é formado por 24 profissionais da UFSCar campus Sorocaba, 13 profissionais de escolas técnicas e 3 profissionais de escolas do Ensino Fundamental, sendo 11 ocupantes de cargos de gestão como coordenadores de curso, diretores, 17 membros de conselhos, 5 profissionais administrativos e 3 professores, totalizando 40 avaliadores em que a maioria afirma ter afinidade com atas e reuniões e 3 declararam nenhuma afinidade com esses documentos. Os avaliadores foram divididos em dois grupos onde cada grupo avaliou as técnicas de extração tópicos a partir de uma consulta (palavras-chave), ou seja, cada indivíduo avaliou 3 cenários distintos. A avaliação consistiu de um documento impresso contendo uma breve apresentação do trabalho, seguido de uma cópia dos resultados das técnicas de extração de tópicos e questões avaliativas sobre cada técnica.

% X profissionais de gestão de instituições 

% -- 2 - Escolha das questões.
O questionário foi formado por questões envolvendo aspectos os extratores de tópicos e questões referentes à técnica de segmentação textual empregada
As respostas seguiram a escala \textit{Likert}~\cite{Norman2010} com 5 alternativas. 
% , conforme mostrado:

\begin{enumerate}
	\item Todos os trechos apresentados compartilham um mesmo assunto.
	\item As palavas \textit{<descritores>} resumem bem o assunto tratado nos trechos.
	\item Existem trechos que não tratam de um único assunto?
	\item Existem trechos incompletos e insuficientes para compreensão do assunto do trecho?
\end{enumerate}


As questões 1 e 2 estão relacionadas ao extrator de tópicos. A primeira refere-se ao agrupamento dos segmentos pela qual foi avaliada a semelhança dos trechos em termos de assunto. A segunda questão diz respeito aos descritores selecionados, ao respondê-la o avaliador indicou o quão bem esses termos representam aquele grupo.
As questões 3 e 4 estão ligadas à técnica de segmentação utilizada, o BayesSeg conforme já mencionado no Capítulo~\ref{cap-segmentadores}. A questão 3 está ligada à coesão de cada segmento, levando em conta a homogeneidade do texto em relação a um assunto. A questão 4 refere-se a completude dos segmentos, ou seja, o quão bem os segmentos podem ser bem compreendidos independentemente da leitura do documento integral.
Para afastar a hipótese de que os resultados das técnicas fossem influenciados pela ordem apresentada, essas foram apresentadas aos avaliadores em ordem aleatórias.
% -- 5 - Escolha dos temas para consulta: 




% A primeira refere-se ao agrupamento dos segmentos pela qual foi avaliada a semelhança dos trechos em termos de assunto.

\section{Resultados}

Nessa seção, os dados coletados das avaliações são apresentados e analisados. Os modelos de extração de tópicos discutidos nesse trabalhos são comparados de acordo com os critérios mencionados anteriormente: 
(1) comparar algoritmos de extração de tópicos na tarefa de extração de padrões no contexto das atas de reunião, 
(2) analisar a qualidade dos agrupamentos no que tange a navegação por grupos com mesmo tópico, 
(3) analisar a qualidade dos descritores extraídos para recuperar os documentos dos grupos,
(4) validar a performance do segmentador empregado.


Na Figura~\ref{fig:Q1} é apresentado as frequência das respostas coletadas sobre a primeira questão a qual refere-se a qualidade do agrupamento levando em conta a semelhança dos segmentos em termos de assunto. Verifica-se que o K-Means tem resultados similares ao LDA enquanto o PLSA se mostrou menos eficiente nesse critério uma vez que mais avaliadores rejeitaram a afirmação de que todos os segmentos tratam de um único assunto em comum. 

\begin{figure}[!h] \centering     %%% not \center

	\subfigure{ \label{fig:kmeans}
		\includegraphics[width=.31\textwidth]{conteudo/capitulos/figs/figuras-experimento/Q1-KMeans.png}
	}	
	\subfigure{ \label{fig:lda}
		\includegraphics[width=.31\textwidth]{conteudo/capitulos/figs/figuras-experimento/Q1-LDA.png}
	}
	\subfigure{ \label{fig:plsa}
		\includegraphics[width=.31\textwidth]{conteudo/capitulos/figs/figuras-experimento/Q1-PLSA.png}
	} 
	\caption{Contagem de respostas referente a primeira questão cujo enunciado foi:\textit{``Todos os trechos apresentados compartilham um mesmo assunto.''}. O eixo vertical indica a frequência das alternativas representadas no eixo horizontal por:
		a=\textit{``Discordo Totalmente''}
		b=\textit{``Discordo Parcialmente''}
		c=\textit{``Não Concordo, nem Discordo''}
		d=\textit{``Concordo Parcialmente''}
		e=\textit{``Concordo Totalmente''}.
	}
	\label{fig:Q1}
\end{figure}



% A segunda questão diz respeito aos descritores selecionados, ao respondê-la o avaliador indicou o quão bem esses termos representam aquele grupo.
Outro ponto importante a ser analisado é a capacidade representativa dos descritores, ou seja, o quão bem os descritores podem representar o tópico ao qual os segmentos foram atribuídos. Observa-se na Figura~\ref{fig:Q2} no K-Means a maior diferença entre as opiniões positivas e negativas, sendo a maioria concordante com a afirmação de que os descritores extraídos são bons atributos para descrever o teor dos segmentos extraídos.

\begin{figure}[!h] \centering     %%% not \center

	\subfigure{ \label{fig:kmeans}
		\includegraphics[width=.31\textwidth]{conteudo/capitulos/figs/figuras-experimento/Q2-KMeans.png}
	}	
	\subfigure{ \label{fig:lda}
		\includegraphics[width=.31\textwidth]{conteudo/capitulos/figs/figuras-experimento/Q2-LDA.png}
	}
	\subfigure{ \label{fig:plsa}
		\includegraphics[width=.31\textwidth]{conteudo/capitulos/figs/figuras-experimento/Q2-PLSA.png}
	}
	\caption{Contagem de respostas referente a segunda questão cujo enunciado foi:\textit{``As palavas <descritores> resumem bem o assunto tratado nos trechos.''}. O eixo vertical indica a frequência das alternativas representadas no eixo horizontal por:
		a=\textit{``Discordo Totalmente''}
		b=\textit{``Discordo Parcialmente''}
		c=\textit{``Não Concordo, nem Discordo''}
		d=\textit{``Concordo Parcialmente''}
		e=\textit{``Concordo Totalmente''}.
	}
	\label{fig:Q2}
\end{figure}


% -- Conclusão : extração de padrões	
Ao analisar os resultados, verifica-se que de maneira geral os modelos K-Means e LDA podem ser considerados satisfatórios na tarefa de agrupar e representar os segmentos das atas.
Verifica-se também que para o modelo K-Means os avaliadores identificaram que, na maioria dos casos, houve resultados satisfatórios, principalmente quanto a representatividade dos descritores. Embora a avaliação aponte imperfeições, esse modelo apresenta maior uniformidade em relação aos demais.


Esse experimento foi aproveitado para validar o segmentador empregado nesse trabalho. Uma vez que esta uma etapa anterior à extração de tópicos a princípio, a modelo que selecionou os segmentos não interfere em sua avaliação. Assim, a Figura~\ref{fig:Q3e4} mostra as respostas do avaliadores considerando todos os cenários. As respostas referentes a terceira questão, na qual se averígua a homogeneidade de cada segmento quanto ao seu assunto central, apontam que poucos segmentos contém mais de um assunto.
Ainda sobre a qualidade da segmentação, a quarta questão investiga a integridade de cada segmento, isto é, sua capacidade de informar o usuário sobre o assunto que trata sem necessidade de se recorrer a leitura do documento integral. Nesse critério, a maioria das avaliações indicam que nenhum ou poucos segmentos apresentam texto insuficiente para leitura.  Uma análise mais detalhada das questões relacionadas a segmentação das atas foi discutida no Capítulo~\ref{cap-segmentadores}, ficando aqui análises de pontos onde a segmentação influencia os extratores e os resultados finais apresentados ao usuário.


\begin{figure}[!h] \centering     %%% not \center

	\subfigure{ \label{fig:seg1}
		\includegraphics[width=.48\textwidth]{conteudo/capitulos/figs/figuras-experimento/Q3-Seg.png}
	}	
	\subfigure{ \label{fig:seg2}
		\includegraphics[width=.48\textwidth]{conteudo/capitulos/figs/figuras-experimento/Q4-Seg.png}
	}
	\caption{Contagem de respostas referente a terceira e quarta questão. Os eixos verticais indicam as frequências das alternativas representadas no eixos horizontais por:
		a=\textit{``Nenhum''}
		b=\textit{``Poucos''}
		c=\textit{``Nem Muitos, nem Poucos''}
		d=\textit{``Muitos''}
		e=\textit{``Todos''}.
	}
	\label{fig:Q3e4}
\end{figure}


Outra questão analisada foi o comportamento dos modelos nas diferentes consultas. Ao se isolar as respostas das questões referentes a uma consulta específica, nota-se certa alteração nas respostas dos modelos. 
Os gráficos apresentados na Figura~\ref{fig:c12-q1}, mostram na primeira superior as respostas para cada modelo considerando-se os segmentos extraídos na primeira consulta e na linha inferior aqueles referentes a segunda consulta. O K-Means apresenta uma diminuição considerável na segunda consulta em relação a primeira no que se refere a respostas afirmando que os todos os segmentos compartilham um único assunto, e um aumento de respostam indicando discordância total com a afirmação da questão.
De forma semelhante, o PLSA apresenta diminuição de respostas positivas para essa afirmação e na proporção que há um aumento de respostas negativas. 
Por outro lado o LDA mantém resultados semelhantes em ambas as consultas, nas quais apresenta resultados equilibrados entre respostas positivas e negativas. Em outras palavras, os modelos K-Means e PLSA sofreram perda de desempenho enquanto o LDA manteve-se estável em ambas consultas.
Ao analisar separadamente a segunda questão, referente a representatividade dos descritores observa-se na Figura~\ref{fig:c12-q2} que todos os modelos apresentam perca de performance nesse critério, contudo, de forma acentuada no PLSA. 



% indicam que o Algoritmo 





\begin{figure}[!h] \centering     %%% not \center

	\subfigure{ \label{fig:c1q1kmeans}
		\includegraphics[width=.31\textwidth]{conteudo/capitulos/figs/figuras-experimento/C1-Q1-KMeans.png}
	}	
	\subfigure{ \label{fig:c1q1lda}
		\includegraphics[width=.31\textwidth]{conteudo/capitulos/figs/figuras-experimento/C1-Q1-LDA.png}
	}
	\subfigure{ \label{fig:c1q1plsa}
		\includegraphics[width=.31\textwidth]{conteudo/capitulos/figs/figuras-experimento/C1-Q1-PLSA.png}
	}
	\subfigure{ \label{fig:c2q1kmeans}
		\includegraphics[width=.31\textwidth]{conteudo/capitulos/figs/figuras-experimento/C2-Q1-KMeans.png}
	}	
	\subfigure{ \label{fig:c2q1lda}
		\includegraphics[width=.31\textwidth]{conteudo/capitulos/figs/figuras-experimento/C2-Q1-LDA.png}
	}
	\subfigure{ \label{fig:c2q1plsa}
		\includegraphics[width=.31\textwidth]{conteudo/capitulos/figs/figuras-experimento/C2-Q1-PLSA.png}
	}

	\caption{Contagem das respostas referentes a Primeira questão. A primeira consulta, ``\textit{compra de equipamentos}'', é mostrada na linha superior e a segunda consulta, ``\textit{defesa de dissertação}'', na linha inferior.}
	\label{fig:c12-q1}
\end{figure}






\begin{figure}[!h] \centering     %%% not \center

	\subfigure{ \label{fig:c1q2kmeans}
		\includegraphics[width=.31\textwidth]{conteudo/capitulos/figs/figuras-experimento/C1-Q2-KMeans.png}
	}	
	\subfigure{ \label{fig:c1q2lda}
		\includegraphics[width=.31\textwidth]{conteudo/capitulos/figs/figuras-experimento/C1-Q2-LDA.png}
	}
	\subfigure{ \label{fig:c1q2plsa}
		\includegraphics[width=.31\textwidth]{conteudo/capitulos/figs/figuras-experimento/C1-Q2-PLSA.png}
	}
	\subfigure{ \label{fig:c2q2kmeans}
		\includegraphics[width=.31\textwidth]{conteudo/capitulos/figs/figuras-experimento/C2-Q2-KMeans.png}
	}	
	\subfigure{ \label{fig:c2q2lda}
		\includegraphics[width=.31\textwidth]{conteudo/capitulos/figs/figuras-experimento/C2-Q2-LDA.png}
	}
	\subfigure{ \label{fig:c2q2plsa}
		\includegraphics[width=.31\textwidth]{conteudo/capitulos/figs/figuras-experimento/C2-Q2-PLSA.png}
	}

	\caption{Contagem das respostas referentes a Segunda questão. A primeira consulta, ``\textit{compra de equipamentos}'', é mostrada na linha superior e a segunda consulta, ``\textit{defesa de dissertação}'', na linha inferior.}
	\label{fig:c12-q2}
\end{figure}



{
\small
TODO:
--> O segmentador, sendo uma etapa anterior a extração de tópicos, influencia na qualidade do grupos??

--> O K-means traz melhores segmentos (Sem misturar assuntos) em relação aos outros métodos.  'Aparentemente o K-means seleciona os trechos mais coesos, mas precisa de mais experimentos, mais amostras ...'


--> Com os avaliadores das ETECs T1 > T2 > T3
--> Incluindo os da UFSCar   T1 > T2 = T3;
}




% -- Conclusão : do conjunto das técnicas de extração e segmentação

Nessa Seção, analisou-se as técnicas de segmentação textual e extração de tópicos na criação de uma estrutura derivada do \textit{corpus} original como uma representação estruturada da coleção de atas a qual foi organizada e acrescida de atributos para sua descrição. As análises sugerem que tais técnicas podem oferecer a sistemas de recuperação uma representação estruturada que preserva o conteúdo dos documentos ao mesmo tempo que cria atributos adicionais que incorporam informação à base de dados e podem ser inseridas no espaço de busca.  

% Na maioria dos casos, os dados coletados apontam 





