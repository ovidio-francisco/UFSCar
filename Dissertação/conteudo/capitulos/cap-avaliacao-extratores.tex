
\chapter{Avaliação dos Extratores de Tópicos}\label{cap-extratores}



% Dar uma geral nesse primeiro parágrafo
% Escolha dos Algorítmos
% Escolha do método de avaliação (subjetivo com questionario)
% 

Nesse capítulo, as técnicas de extração de tópicos são analisadas. O objetivo é comparar os algoritmos de extração de tópicos na tarefa de extração de padrões no contexto das atas de reunião no que tange a qualidade dos agrupamentos e seus descritores bem como sua capacidade de representar os segmentos. Escolheu-se os modelos LDA, PLSA e K-Means para essa análise devido a popularidade desses métodos sendo amplamente utilizados~\cite{DZhu20122} e frequentemente referenciados em trabalhos voltados a organização de bases textuais~\cite{Aggarwal2018, OCallaghan2015, Steyvers2007}.


Os métodos foram inicialmente configurados com base em avaliações internas~\cite{Hassani2017} e observações empíricas nas quais escolheu-se os melhores valores para seus parâmetros. Os resultados desses modelos foram submetidos a uma avaliação subjetiva a fim de analisa-los junto a usuários com afinidade com atas de reuniões.


% e comparações
% com melhores resultados.  
% Optou-se por uma avaliação subjetiva 


\section{Configuração experimental}

% Parâmetros 
	% Silueta e Coesão para o K-Means
	% Alpha e Beta do LDA
	% PLSA ???



\section{Critérios de avaliação}

% Nesse trabalho, os algoritmos de extração de tópicos foram avaliados subjetivamente. 

\section{Resultados}


