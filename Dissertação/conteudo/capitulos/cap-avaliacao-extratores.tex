\chapter{Avaliação dos Extratores de Tópicos}\label{cap-extratores}

% 4 - Procedimento (2 consultas, técnicas em ordens diferentes).
% 5 - Medidas para avaliação.
% 6 - Conclusão e discussão.


% Dar uma geral nesse primeiro parágrafo
% Escolha dos Algorítmos
% Escolha do método de avaliação (subjetivo com questionario)

% -- 1 - Motivação para o experimento.
Nesse capítulo, as técnicas de extração de tópicos são analisadas. O objetivo é comparar os algoritmos de extração de tópicos na tarefa de extração de padrões no contexto das atas de reunião no que tange a qualidade dos agrupamentos e seus descritores bem como sua capacidade de representar os segmentos. Escolheu-se os modelos LDA, PLSA e K-Means para essa análise devido a popularidade desses métodos sendo amplamente utilizados~\cite{DZhu20122} e frequentemente referenciados em trabalhos voltados a organização de bases textuais~\cite{Aggarwal2018, OCallaghan2015, Steyvers2007}.
Os algoritmos foram inicialmente configurados com base em avaliações internas~\cite{Hassani2017} e observações empíricas nas quais escolheu-se os melhores valores para seus parâmetros. Os resultados desses modelos foram submetidos a uma avaliação subjetiva a fim de analisá-los junto a usuários com afinidade com atas de reuniões. 
% A etapa de segmentação textual é avaliada também é avaliada por conveniencia e proximidade dos assuntos.
A avaliação se deu por meio de questionários onde profissionais com afinidade com atas de reunião forneceram suas percepções em relação aos resultados dos modelos de extração de tópicos. 


% -- Explicar que foram 2 consultas (quais) e mostrar os extratores.


% -- 2 - Escolha das questões.
O questionário foi formado por questões envolvendo aspectos os extratores de tópicos e questões referentes à técnica de segmentação textual empregada, conforme mostrado:

\begin{enumerate}
	\item Todos os trechos apresentados compartilham um mesmo assunto.
	\item As palavas compra, material, verba, permanente e valor resumem bem o assunto tratado nos trechos.
	\item Existem trechos que não tratam de um único assunto?
	\item Existem trechos incompletos e insuficientes para compreensão do assunto do trecho?
\end{enumerate}


A questão 1 está relacionada ao agrupamento dos segmentos. Pretende-se avaliar a Semelhança dos trechos em termos de assunto. 

% -- 3 - Escolha dos participantes.

% -- 4 - Procedimento (2 consultas, técnicas em ordens diferentes).
% -- 5 - Escolha dos temas para consulta: 


Os dados coletados serviram como base para análise dos algoritmos avaliados 


\section{Configuração experimental}

% A qtd de tópicos é um parametro importante.
% Todos com 70 tópicos e 5 extratores;
Durante os primeiros testes empíricos a qualidade dos resultados mostrou-se sensível à quantidade de tópicos extraídos.
Inicialmente, realizou-se um teste prévio utilizando uma versão não-paramétrica dos algorítimos a fim de automaticamente obter valores ótimos para esse parâmetro por meio da análise das medidas Silhueta e Coesão. Essa configuração automática resulta valores em torno de 20 tópicos. Contudo, apresenta grupos com muitos segmentos (em torno de 100) o que os tornam pouco coesos além de certa dificuldade em julgar capacidade dos descritores em representar bem o tópico.  
Em observações diretas, valores próximos a entre 60 e 80 mostraram melhores resultados. Nesse trabalho, optou-se por configurar os extratores para extrair 70 tópicos da coleção de segmentos por apresentar melhor resultados aparente na visão de usuário.

Outro fator importante é a quantidade de descritores selecionados para cada tópico. Com base no experimento de anotações em segmentação, descrito no Capítulo~\ref{cap-segmentadores}, os anotadores selecionaram em média 5 palavras para descrever os segmentos, sendo esse valor adotado para essa avaliação.



% LDA - Gibbs Sampling
% Parâmetros 
	% Silueta e Coesão para o K-Means
	% Alpha e Beta do LDA
	% PLSA ???



\section{Critérios de avaliação}

% Nesse trabalho, os algoritmos de extração de tópicos foram avaliados subjetivamente. 

\section{Resultados}


