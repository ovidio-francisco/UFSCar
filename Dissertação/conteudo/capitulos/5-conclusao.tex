\chapter*[Conclusão]{Conclusão}
\label{cap:conclusao}
\addcontentsline{toc}{chapter}{Conclusão}



% -------------------- Introdução -------------------- 

Em um contexto em que a grande parte das informações armazenadas pelas organizações está em formato textual, o desenvolvimento de ferramentas computacionais para extração e organização automática dessas informações é uma tarefa que retem atenção e relevância.
Dentre os textos usados para registros, identificou-se documentos multi-temáticos, os quais abordam assuntos diversos no mesmo texto, como exemplo, as atas de reuniões apresentam com característica a ausência de meta informações ou mesmo quebras de parágrafos que ajudariam a separar e identificar seus conteúdos. 
Entre as abordagens aplicáveis a este problema estão a segmentação textual e os modelos de extração de tópicos as quais em conjunto são capazes de separar, agrupar e idenficar os assuntos contidos nessa categoria de documentos.



% -------------------- Retomada do que foi feito -------------------- 




