
\subsection{Medidas de Proximidade}
\label{subsec:MedidasProximidade}
No modelo espaço vetorial, a similidade entre um documentos $x$ e $y$ pode ser calculada utilizando-se a medida Cosseno. Essa medida é definida pela correlação entre os vetores $\vec{x}$ e $\vec{y}$, a qual pode ser calculada pelo cosseno do  ângulo entre esses vetores. Dados dois documentos $x = (x_1, x_1, \dots, x_t)$ e $y = (y_1, y_1, \dots, y_t)$, calcula-se: 


% conforme mostrado na Equação~\ref{equ:cosseno-doc-consulta}.


\begin{equation}
cosseno(x, y) = \frac{ \vec{x} \bullet \vec{y} }
                   { |\vec{x}| \times | \vec{y}|}
            = \frac{ \sum_{i=1}^{t} x_i \cdot y_i}
                   { \sqrt{\sum_{i=1}^{t} x_i^2} \times \sqrt{\sum_{i=1}^{t} y_i^2 } }                      \label{equ:cosine}
\end{equation} 


Valores de Cosseno próximos a 0 indicam um ângulo próximo a 90º entre $\vec{x}$ e $\vec{y}$, ou seja, o documento $x$ compartilha poucos termos com a consulta $y$, enquanto valores próximos a 1 indicam um ângulo próximo a 0º, ou seja, $x$ e $y$ compartilham termos e são similares.
%
Outra medida utilizada para medir a similaridade entre documentos é conhecida como Jaccard, a qual é definida por:

\begin{equation}
jaccard(x_i, x_j) = \frac{f_{11}}{f_{01} + f_{10} + f_{11}}~,
\end{equation}

\noindent
onde 
$f_{11}$ é o número de termos presentes em ambos os documentos,
$f_{01}$ é o número de termos ausentes em $x_i$ e presentes em $x_j$ e
$f_{10}$ é o numero de termos ausentes em $x_j$ e presentes em $x_i$.
%
Semelhante a Cosseno, Jaccard é uma medida de similaridade que retorna um valor no intervalo [0,1], sendo que valores próximo a 1 indicam similaridade máxima~\cite{Maracini2010,Tan2005,Feldman2006}.
