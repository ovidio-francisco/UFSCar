
\section{Segmentação Textual} \label{sec:segmentacao}


% -- Definição da Tarefa e de Segmentação
% A tarefa de segmentação textual consiste em dividir um texto em partes ou segmentos que contenham um significado relativamente independente. Em outras palavras, é identificar as posições nas quais há uma mudança significativa de assuntos. 


% -- #################### SAI ####################

%         ==========|   Segmentação  |========== 

A tarefa de segmentação automática de textos, ou segmentação textual consiste em dividir um texto em partes que contenham um significado relativamente independente. Em outras palavras, é identificar as posições nas quais há uma mudança significativa de assunto. É útil em aplicações que trabalham com textos sem indicações de quebras de assunto, ou seja, não apresentam seções ou capítulos, por exemplo, transcrições automáticas de áudio, vídeos e grandes documentos que contêm vários assuntos como atas de reunião e notícias~\cite{Aggarwal2018, bokaei2015a, sakahara2014a, misra2009a, Eis2008}.
% <-- Referências

Pode ser usada para melhorar o acesso a informação solicitada por meio de uma consulta, onde é possível oferecer porções menores de texto mais relevantes ao invés de exibir um documento grande que pode conter informações menos pertinentes.  Além disso, encontrar pontos onde o texto muda de assunto, pode ser útil como etapa de pré-processamento em aplicações voltadas ao entendimento do texto, principalmente em documentos longos~\cite{Choi2001-LSA}.

As técnicas de segmentação textual consideram um texto como uma sequência linear de unidades de informação que podem ser, por exemplo, cada termo presente no texto, os parágrafos ou as sentenças. Cada unidade de informação é um elemento do texto que não será dividido no processo de segmentação e cada ponto entre duas unidades é considerado um candidato a limite entre segmentos. Nesse sentido, um segmento pode ser visto como uma sucessão de unidades de informação que compartilham o mesmo assunto.


Nessa seção serão apresentados os algoritmos frequentemente referenciados na literatura  com diferentes características. A seguir são detalhados os algoritmos baseados em coesão léxica, \textit{TextTiling} e \textit{C99}, o \textit{BayesSeg} e \textit{TextSeg} que trazem abordagens probabilísticas e o \textit{MinCutSeg} baseado em particionamento de grafos.


% -- Algoritmos

\subsection{Algoritmos}
	\label{subsec:principaisalgoritimos}
% Lembrando que tem muuuuitos outros. Principalmente os baseados em tópicos. -- Rafael % -? antes falar das técnicas (janelas, agrupamento, tópicos, etc..) ?
% falar um pouco sobre TT e C99, sobre como foram pioneiros, influenciáram muitos outros e são empregrados até hoje;
% Nem mencionou janelas deslizantes anteriormente. --> Rafael
% -? Criar um texto sobre janelas deslizantes?
% \textit{threshold}. %TODO - explicar como se encontra os vales


% ==========  TextTiling  ==========

Entre os trabalhos tradicionais da literatura podemos citar o  \textit{TextTiling}~\cite{Hearst1994} e o \textit{C99}~\cite{Choi2000}.
% TT pioneiro e simples, mas pouco acurado
% C99 utiliz




O \textit{TextTiling} é um algoritmo baseado em janelas deslizantes, em  que, para cada candidato a limite, analisa-se o texto circundante. Um limite ou quebra entre segmentos é identificado sempre que a similaridade entre as unidades que antecedem e precedem o ponto candidato cai abaixo de um limiar. O \textit{TextTiling} recebe uma lista de candidatos a limite, usualmente finais de parágrafo ou finais de sentenças. Para cada posição candidata são construídos 2 blocos, um contendo sentenças que a precedem e outro com as que a sucedem. O tamanho desses blocos é um parâmetro a ser fornecido ao algoritmo e determina o tamanho mínimo de um segmento. Em seguida, os blocos de texto são representados por vetores que contém as frequências de suas palavras. Então, usa-se cosseno (Equação~\ref{equ:cosine}) para calcular a similaridade entre os blocos adjacentes a cada candidato e identifica-se uma transição entre tópicos pelos vales na curva de dissimilaridade.
% Rafael --> Podem ter vários tipos de cursa. Ser mais objetivo. Quando a similaridade fica abaixo de um limiar fornecido pelo usuário.? Na verdade, dá pra fazer uma mescla. 
%TODO como apresentado na Figura~\ref{fig:curvadedissimilaridade}.



O TextTiling apresenta como vantagens a facilidade de implementação e baixa complexidade computacional, favorecendo a implementação de trabalhos similares ~\cite{Naili2016,Bokaei2015,CHAIBI2014,Kern2009,Galley2003}, e usado com \texttit{base line} em outros trabalhos~\cite{Cardoso2017,Dias2007}. Por outro lado, algoritmos mais complexos, como os baseados em matrizes de similaridade, apresentam acurácia relativamente superior como apresentado posteriormente  em~\cite{Choi2000, Kern2009, Misra2009}.


% ==========  C99  ==========

Outro algoritmo frequentemente referenciado na literatura é o C99 o qual é baseado em \textit{ranking}. Embora muitos trabalhos utilizem matrizes de similaridades para pequenos segmentos, o cálculo de suas similaridades não é confiável, pois uma ocorrência adicional de uma palavra pode causar certo impacto e alterar o cálculo da similaridade~\cite{Choi2000}. Além disso, o estilo da escrita normalmente não ser constante em todo o texto. Por exemplo, textos iniciais dedicados a introdução costumam apresentar menor coesão do que trechos dedicados a um tópico específico. Portanto, comparar a similaridade entre trechos de diferentes regiões não é apropriado. Devido a isso, as similaridades não podem ser comparadas em valores absolutos. Então, contorna-se esse problema fazendo uso de \textit{rankings} de similaridade para encontrar os segmentos de texto. Para isso, o C99 constrói uma matriz que contém as similaridades de todas as unidades de informação (normalmente sentenças ou parágrafos). Em seguida, cada valor na matriz de similaridade é substituído por seu \textit{ranking local}. Para cada elemento da matriz, seu \textit{ranking} é o número de elementos vizinhos com valor de similaridade menor que o seu. Então, cada elemento e comparado com seus vizinhos dentro de uma região denominada máscara.

Na Figura~\ref{fig:a} é destacado um quadro 3~x~3 de uma matriz em que cada elemento é a similaridade entre duas unidades de informação. Tomando como exemplo o elemento com valor $0,5$, a mesma posição na matriz de \textit{rankings} terá o valor $4$, pois esse é o número de vizinhos com valores inferiores a $0,5$ dentro do quadro analisado na matriz de similaridades. Da mesma forma, na Figura~\ref{fig:b} para o valor $0,2$ a matriz de \textit{rankings} conterá o valor $1$ na mesma posição.

\begin{figure}[!h]
	\centering     %%% not \center

	\subfigure[Passo 1]{\label{fig:a}\includegraphics[width=60mm]{conteudo/capitulos/figs/exemplo-matrix-rank-A.png}}
	\subfigure[Passo 2]{\label{fig:b}\includegraphics[width=60mm]{conteudo/capitulos/figs/exemplo-matrix-rank-B.png}}
	
	\caption{Exemplo de construção de uma matriz de rankings.%~\cite{Choi2000}.
	}
	\label{fig:exemplomatrixrank}
\end{figure}
% -< Colocar uma explicação mais detalhada para esses passos



Finalmente, com base na matriz de \textit{ranking}, o C99 utiliza um método de \textit{clustering} baseado no algoritmo de maximização de Reynar para identificar os limites entre os segmentos. 
% ~\cite{Reynar1998} 
% Rafael: Esse passo aqui tá meio obscuro ainda. Não dá pra fazer uma figura ilustrativa?






% ==========  Utiyama  ==========





% ==========  BayesSeg  ==========



Os métodos baseados em coesão léxica que utilizam métricas como cosseno quantificam a similaridade entre sentenças baseando-se apenas na frequência das palavras, Essa abordagem, ignora certas características do texto que podem dar pistas sobre a estrutura do texto. Por exemplo, frases como "Prosseguindo", "Dando continuidade", "Ao final da reunião" podem dar "pistas" de inicio ou final de segmento. A fim de aproveitar esses indicadores, usa-se um framework bayesiano que permite incorporar fontes externas ao modelo. O método BayesSeg~\cite{Eisenstein2008} aborda a coesão léxica em um contexto bayesiano onde as palavas de um segmento surgem de um modelo de linguagem multinomial o qual é associado a um assunto. 

Essa abordagem é similar à métodos probabilísticos de extração de tópicos como o Latent Dirichlet Allocation (LDA)~\cite{Blei2003}, com a diferença que ao invés de atribuir tópicos ocultos a cada palavra, esses são usados para segmentar o documento. Nesse sentido, detecta-se um limite entre sentenças quando a distribuição de tópicos entre elas for diferente.

Baseia-se na ideia que alguns termos são usados em tópicos específicos enquanto outros são neutros em relação aos tópicos do documento e são usados para expressar uma estrutura do documento, ou seja, as "frases-pista" vem de um único modelo generativo. A fim de refletir essa ideia, o modelo é adaptado para influenciar a probabilidade da sentença de ser uma final ou início de segmento conforme a presença de "frases pista".













% -- Medidas
\subsection{Medidas de Avaliação}





%
% Utilizadas em RI e Classificação 
%%% 
As medidas de avaliação tradicionais como precisão e revocação são usadas em recuperação de informação e classificação automática para medir o desempenho de modelos de classificação e predição. São baseadas na comparação dos valores produzidos por uma hipótese com os valores reais. 

%	Avaliações baseadas em hits 
Esses valores são apresentados em uma tabela que permite a visualização do desempenho de um algoritmo, a qual é chamada de matriz de confusão. Na Tabela~\ref{tab:matrizconfusao} é apresentada a matriz de confusão para duas classes (Positivo e Negativo). Uma matriz de confusão é uma tabela que permite a visualização do desempenho de um algoritmo. 

\begin{table}[!h]
	\centering
	
	\begin{tabular}{|c|c|c|}
		\hline
		                & Predição Positiva         & Predição Negativa        \\ \hline
		Positivo real   & VP (Verdadeiro Positivo)  & FN (Falso Negativo)      \\ \hline
		Negativo real   & FP (Falso Positivo)       & VN (Verdadeiro Negativo) \\ \hline
	
	\end{tabular}
	
	\caption{Matriz de confusão.}
	\label{tab:matrizconfusao}

\end{table}





% Falso Positivo 
No contexto de segmentação textual, um falso positivo é um limite identificado pelo algoritmo que não corresponde a nenhum limite na segmentação de referência, ou seja, o algoritmo indicou que em determinado ponto há uma quebra de segmento, mas na segmentação de referência, no mesmo ponto, não há. 
%
% Falso Negativo 
%%% 
 De maneira semelhante, um falso negativo é quando o algoritmo não identifica um limite existente na segmentação de referência, ou seja, em determinado ponto há, na segmentação de referência, um limite entre segmentos, contudo, o algoritmo não o identificou.
%
%
% Verdadeiro Positivo 
%%% 
 Um verdadeiro positivo é um ponto no texto indicado pelo algoritmo e pela segmentação de referência como uma quebra de segmentos, ou seja, o algoritmo e a referência concordam que em determinado ponto há uma transição de assunto.
%
%
% O Verdadeiro Negativo, que não existe 
%%%
 Na avaliação de segmentadores, não há o conceito de verdadeiro negativo. Este seria um ponto no texto indicado pelo algoritmo e pela segmentação de referência onde não há uma quebra de segmentos. Uma vez que os algoritmos apenas indicam onde há um limite, essa medida não é necessária. % Não há ou não e necessário?
%
%
%
 Nesse sentido, a precisão, é a proporção de limites corretamente identificados pelo algoritmo. É calculada dividindo-se o número de limites identificados automaticamente pelo número de candidatos a limite (Equação~\ref{equ:precisao}).
 
 \begin{equation}
	 Presis\tilde{a}o = \frac{VP}{VP+FP}
	 \label{equ:precisao}
 \end{equation}
%
% Ideia 
%%% 
 Essa medida varia entre $0,0$ e $1,0$, que indica a proporção de limites identificados pelo algoritmo que são corretos, ou seja, correspondem a um limite real na segmentação de referência. Porém não diz nada sobre quantos limites reais existem. 
%
%
%
%%%%%%%%%%%%%%%
% Revocação 
%%%%%%%%%%%%%%% 
%
% Definição 
%%% 
 A revocação, é a proporção de limites verdadeiros que foram identificados pelo algoritmo.
%
% Cálculo 
%%%
 É calculada dividindo-se o número de limites identificados automaticamente pelo número limites verdadeiros.
%
%
% 
 \begin{equation}
	 Revoca\c{c}\tilde{a}o = \frac{VP}{VP+FN}
	 \label{equ:revocacao}
 \end{equation}
%
% Ideia 
%%% 
 Pode variar entre $0,0$ e $1,0$, onde indica que a proporção de limites corretos que foram identificados. Porém não diz nada sobre quantos limites foram identificados incorretamente. 
% Relação inversa entre precisão e revocação 
 Existe uma relação inversa entre precisão e revocação. Conforme o algoritmo aponta mais segmentos no texto, este tende a melhorar a revocação e ao mesmo tempo, reduzir a precisão. 
%
% Pode ser contornado com F1 
%%% 
 Esse problema de avaliação pode ser contornado utilizado a medida $F^1$ que é a média harmônica entre precisão e revocação onde ambas tem o mesmo peso. 
%

































%  Medidas de avaliação tradicionais 
As medidas de avaliação tradicionais, precisão e revocação, podem não ser confiáveis, por não considerarem a distância entre os limites, mas penalizam o algoritmo sempre que um limite que não coincide perfeitamente com a referência. Essas medidas podem ser mais adequadas quando necessita-se de segmentações com maior exatidão. Em outras palavras, computam apenas os erros do algoritmo quando se detecta falsos positivos ou falsos negativos, o que nesse contexto de segmentação textual pode não ser suficiente, dado a subjetividade da tarefa. Além dessas medidas, que consideram apenas se um segmento foi perfeitamente definido conforme uma referência, pode-se também considerar a distância entre o segmento extraído automaticamente e o segmento de referência~\cite{Kern2009}. Chama-se \textit{near misses} o caso em que um limite identificado automaticamente não coincide exatamente com a referência, mas é necessário considerar a proximidade entre eles.

%  Medidas que consideram a distancia entre os segmentos 
Na Figura~\ref{fig:exemplosegmentacaozoom} é apresentado um exemplo com duas segmentações extraídas automaticamente e uma referência. Em ambos os casos não há nenhum verdadeiro positivo, o que implica em zero para os valores de precisão, acurácia, e revocação, embora o resultado do algoritmo A possa ser considerado superior ao primeiro se levado em conta a proximidade dos limites.
% Para não confundir hipótese com algoritmo, escrever: "A hipótese A, produzida pelo algoritmo A e a hipótese B, produzida pelo algoritmo B".
% Ou trocar hipótese por resultado do algoritmo A, B

  \begin{figure}[!h]

	\centering
	\includegraphics[width=0.7\textwidth]{conteudo/capitulos/figs/windiffzoom.jpg}
	\caption{Exemplos de \textit{near missing} e falso positivo puro. Os blocos indicam uma unidade de informação e as linha verticais representam uma transição de assunto. }
	\label{fig:exemplosegmentacaozoom}

  \end{figure}
  

 
% === PK ===

Considerando o conceito de \textit{near misses}, algumas soluções foram propostas sendo as mais utilizadas a P$_k$ e \textit{WindowDiff}. Proposta por~\cite{Beeferman1999}, P$_k$ atribui valores parciais a \textit{near misses}, ou seja, limites sempre receberão um peso proporcional à sua proximidade, desde que dentro de um janela de tamanho~$k$.  Para isso, esse método move uma janela de tamanho $k$ ao longo do texto.  A cada passo verifica, na referência e na hipótese, se o início e o final da janela estão ou não dentro do mesmo segmento, então, penaliza o algoritmo caso não concorde com a referência. Ou seja, dado duas palavras de distância $k$, o algoritmo é penalizado quando não concordar com a segmentação de referência se as palavras estão ou não no mesmo segmento.  O valor de $k$ é calculado como a metade da média dos comprimentos dos segmentos reais. Como resultado, é retornado a contagem de discrepâncias divida pelo quantidade de segmentações analisadas.  P$_k$ é uma medida de dissimilaridade entre as segmentações e pode ser interpretada como a probabilidade de duas sentenças extraídas aleatoriamente pertencerem ao mesmo segmento.  % TODO: Rafael: Não dá pra fazer uma fórmula matemática pra deixar o funcionamento dessa medida mais claro?



% === WindowDiff ===

\textit{WindowDiff} é uma medida alternativa à P$_k$. De maneira semelhante, move uma janela pelo texto e penaliza o algoritmo sempre que o número de limites proposto pelo algoritmo não coincidir com o número de limites esperados para aquela janela. Ou seja, o algoritmo é penalizado quando não concordar com a segmentação de referência quanto ao número de segmentos na janela.  Assim, consegue manter a sensibilidade a \textit{near misses} e além disso, considerar o tamanho das janelas.  A fim de melhor equilibrar o peso dos falsos positivos em relação a \textit{near misses}, dobra-se a penalidade para falsos positivos, evitando-se a supervalorização dessa medida.  % OBS: Os problemas de Pk ficaram subentendidos aqui :/ 

As medidas \textit{WindowDiff} e P$_k$, consideram a quantidade e proximidade entre os limites, sendo mais tolerantes a pequenas imprecisões. Essa é uma característica desejável, visto que as segmentações de referência possuem diferenças consideráveis. \textit{WindowDiff} equilibra melhor os falsos positivos em relação a \textit{near misses}, ao passo que P$_k$ os penaliza com peso maior. Isso significa que segmentadores melhores avaliados em P$_k$ ajudam a selecionar as configurações que erram menos ao separar trechos de texto com o mesmo assunto, enquanto \textit{WindowDiff} é mais tolerante nesse aspecto.  De maneira geral, observa-se  melhores resultados de \textit{WindowDiff} quando os algoritmos aproximam a quantidade de segmentos automáticos da quantidade de segmentos da referência. Por outro lado, P$_K$ avalia melhor as configurações que retornam menos segmentos. Contudo, não é possível definir um valor adequado, uma vez que os segmentadores humanos frequentemente apontam segmentações diferentes.













