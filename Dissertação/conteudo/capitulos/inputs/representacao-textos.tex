\subsection*{Representação de Textos} 
\label{section:RepTextos}

Os dados textuais se diferenciam de outros formatos estruturados como bancos de dados relacionais em que um dado é facilmente encontrado. Uma das etapas mais importantes para as tarefas de Mineração Textos é a criação de uma representação adequada dos dados. Essa representação deve prover uma maneira estruturada para que os textos possam ser processados e utilizados por algoritmos de Aprendizado de Máquina. 




Uma das formas mais comuns para que a grande maioria dos algoritmos possa extrair padrões das coleções de textos é a representação no formato matricial conhecido como Modelo Espaço Vetorial (\textit{Vectorial Space Model} - VSM)~\cite{Rezende2003}, onde os documentos são representados como vetores em um espaço Euclidiano $m$-dimensional em que cada termo extraído da coleção é representado por uma dimensão. Assim, cada componente de um vetor expressa a relação entre os documentos e as palavras. Essa estrutura é conhecida como matriz documento-termo (\textit{document-term matrix}). Uma das formas mais populares dessa matriz é conhecida como \textit{Bag Of Words} a qual é detalhada a seguir.
	

\subsection*{\textit{Bag Of Words}}
% \subsection{\textit{Bag Of Words}} \label{subsubsec:BOW}
		
Nessa representação, cada termo é transformado em um atributo  (\textit{feature}), em que $a_{ij}$ é o peso do termo $t_j$ no documento $d_i$ e indica a sua relevância dentro da base de documentos. As medidas mais tradicionais para o cálculo desses pesos são a binária, onde o termo recebe o valor 1 se ocorre em determinado documento ou 0 caso contrário; \textit{document frequency}, que é o número de documentos no qual um termo ocorre; \textit{term frequency - tf}, atribui-se ao peso a frequência do termo dentro de um determinado documento; \textit{term frequency-inverse document frequency, tf-idf}, pondera a frequência do termo pelo inverso do número de documentos da coleção em que o termo ocorre.
Essa representação é mostrada pela Tabela \ref{table:bagofwords}.

\begin{table}[!h]
	\centering

	\begin{tabular}{|c|ccccc|}

	\hline
	    & $t_1$      & $t_2$     & $t_j$    & \dots & $t_m$      \\ \hline \hline
	$d_1$ & $a_{11}$ & $a_{12}$  & $a_{1j}$ & \dots & $a_{1m}$   \\  %\hline 
	$d_2$ & $a_{21}$ & $a_{22}$  & $a_{2j}$ & \dots & $a_{2m}$   \\  %\hline 
	$d_i$ & $a_{i1}$ & $a_{i2}$  & $a_{ij}$ & \dots & $a_{im}$   \\  %\hline 
	\vdots & \vdots    & \vdots     & \vdots    & \ddots & \vdots      \\  %\hline 
	$d_n$ & $a_{n1}$ & $a_{n2}$  & $a_{nj}$ & \dots & $a_{nm}$   \\  \hline 

	\end{tabular}

	\caption{Coleção de documentos na representação \textit{bag-of-words}}
	\label{table:bagofwords} 
\end{table}


Essa forma de representação sintetiza a base de documentos em um contêiner de palavras, ignorando a ordem em que ocorrem, bem como pontuações e outros detalhes, preservando apenas o peso de determinada palavra nos documentos. É uma simplificação de toda diversidade de informações contidas na base de documentos sem o propósito de ser uma representação fiel do documento, mas oferecer a relação entre as palavras e os documentos a qual é suficiente para a maioria dos métodos de Aprendizado de Máquina~\cite{Aggarwal2018, Feldman2006, Rezende2003}. 






% (Manning et al., 2008; Feldman e Sanger, 2006) Tese Rafael		

