Diogo --

1 Introdução
17
1.1 Motivação . . . . . . . . . . . . . . . . . . . . . . . . . . . . . . . . . . . . 17
1.2 Objetivos, Métodos e Resultados Esperados . . . . . . . . . . . . . . . . . 19
1.3 Organização da Dissertação . . . . . . . . . . . . . . . . . . . . . . . . . . 20

B APÊNDICE 2 - Email enviado convidando usuários a testarem o sistema 91



Camila
1 Introdução
1
1.1 Hipótese e Objetivo . . . . . . . . . . . . . . . . . . . . . . . . . . . . . . . 3
1.2 Proposta e Contribuições . . . . . . . . . . . . . . . . . . . . . . . . . . . . 4
1.3 Organização do Texto 5


simbata
1
Introdução p. 1
1.1 Relações semânticas . .. . . . . . . . . . . . . . . . . . . . . . . . . . . . p. 3
1.2 Motivação . . . . . . .  . . . . . . . . . . . . . . . . . . . . . . . . . . . p. 3
1.3 Objetivos . . . . . . .  . . . . . . . . . . . . . . . . . . . . . . . . . . . p. 5
1.4 Organização do texto . . . . . . . . . . . . . . . . . . . . . . . . . . . . . p. 5





LCHOliveira

Introdução 1
1.1 Recuperação de Informação . . . . . . . . . . . .. . . . . . . . . . . . 1
1.2 Meta-Modelos Formais em Recuperação de Informação. . . . . . . . . . . . 3
1.3 Objetivos e Contribuições  . . . . . . . . . . . . . . . . . . . . . . . 4
1.4 Organização da Dissertação . . . . . . . . . . . . . . . . . . . . . . . 6

Proposta de uma Ferramenta para Avaliação de Desempenho de SRI 69
6.1 Introdução . . . . . . . . . . . . . . . . . . . . . .. . . . . . . . . 69
6.2 Especificação dos Modelos Funcionais . . . . . . . . .. . . . . . . . . 71
6.3 Especificação da Coleção de Referência . . . . . . . .. . . . . . . . . 72
6.4 Especificação da Medida de Avaliação . . . . . . . . .. . . . . . . . . 73
6.4.1 Precisão . . . . . . . . . . . . . . . . . . . . . .. . . . . . . . . 74
6.4.2 Revocação . . . . . . . . . . . . . . . . . . . . . . . . . . . . . . 75
6.4.3 Precisão nos X primeiros . . . . . . . . . . . . . .. . . . . . . . . 75
6.4.4 Precisão-R . . . . . . . . . . . . . . . . . . . . .. . . . . . . . . 75
6.4.5 Medida-E . . . . . . . . . . . . . . . . . . . . . .. . . . . . . . . 75
6.5 Processo de Recuperação . . . . . . . . . . . . . . . . . . . . . . . . 76
6.6 Processo de Comparação Relativa ou Avaliação de Resultados  . . . . . . 77
Conclusões e Trabalhos Futuros 79
7.1 Conclusões . . . . . . . . . . . . . . . . . . . . . . . .  . . . . . . 79
7.2 Trabalhos Futuros . . . . . . . . . . . . . . . . . . . . . . . . . . . 80




Marcelo Bonfin

INTRODUÇÃO ............................................................. 1
CONSIDERAÇÕES INICIAIS ................................................. 1
MOTIVAÇÃO .............................................................. 1
OBJETIVO DA PESQUISA ................................................... 2
ESTRUTURA DA DISSERTAÇÃO ............................................... 2





Takao

1 INTRODUÇÃO....................................................................1
1.1 Apresentação ..................................................................1
1.2 O Problema da Pesquisa .................................................1
1.3 Justificativa e Relevância ...............................................2
1.4 Métodos de Desenvolvimento da Pesquisa ........................................3
1.5 Limitações do Trabalho ..................................................4
1.6 Estrutura do Trabalho .....................................................4

