
Os modelos de extração de tópicos fornecem uma estratégia que visa encontrar nas relações entre documentos, padrões latentes que sejam significativos para o entendimento dessas relações. Tais modelos podem ranquear um conjunto de termos importantes para um ou mais assuntos, bem como ranquear documentos por sua relevância para determinado tema~\cite{Faleiros}.


estruturas latentes em coleções de textos. Atualmente, destacam-se os modelos de extração de tópicos como LDA~\cite{Blei2003} e PLSA~\cite{Hofmann1999} são abordagens amplamente utilizadas ~\cite{DZhu20122} e frequentemente referenciadas em trabalhos que buscam extrair conhecimento e organizar bases textuais ~\cite{Steyvers2007, OCallaghan2015}. 

"A ideia básica dos modelos de tópicos é descobrir, nas re- lações entre documentos e termos, padrões latentes que sejam significativos para o entendimento dessas relações. Por exemplo, tais modelos podem ranquear um conjunto de termos como impor- tantes para um ou mais temas. Bem como ranquear documentos como tendo relevância para um ou mais temas. Se for associado um vetor A j a um documento d j e um vetor B i para um termo w i , sendo A j e B i K-dimensionais, pode-se considerar que cada uma dessas dimensões caracterizam um fator latente relacionado com um documento A j e um termo B i . Assim o produto interno A j · B i pode modelar a importância desses fatores na relação documento d j e termo w i ."




"Nesta tese, a expressão tópico é usada levando-se em conta que o assunto tratado em uma coleção de documentos é extraído automaticamente, ou seja, tópico é definido como um conjunto de palavras que frequentemente ocorrem em documentos semanticamente relacionados. Esses conjuntos de palavras (que definem os tópicos) são obtidos por um processo de pós-processamento realizado a partir das dimensões latentes descobertas pela aplicação dos métodos de modelo de tópicos."  



e com isso obter uma organização da coleção de documentos.



% os principais tópicos ou assuntos na coleção de documentos 


% A extração de tópicos é uma técnica que visa encontrar . 
% Um extrator de tópicos é 

% Conhecendo 

% descoberta pelos modelos de extração de tópicos em documentos segmentados em conjunto com técnicas tradicionais de recuperação de informação. (aproveitando-se da estrutura dos tópicos e delimitação dos segmentos)




"O resultado do processo de extração de tópico é uma representação documento-tópico que determinam um peso de cada tópico para cada documento e uma representação termo-tópico. Esta última está relacionada com o modelo generativo que foi escolhido, e pode representar uma probabilidade da ocorrência do termo quando um tópico ocorre em um documento, a frequência esperada desse termo, ou mesmo um peso estimado matematicamente que não pode ser “traduzido” com algum significado para o contexto linguı́stico."



"Atualmente, os modelos de extração de tópicos probabilı́sticos como o Latent Dirichlet Allocation (LDA) são abor- dagens amplamente aplicadas (Zhu et al., 2012), sendo referenciadas na grande maioria dos trabalhos da literatura como sinônimos de modelos de extração de tópicos (Stey- vers and Griffiths, 2007; O’Callaghan et al., 2015)."  

(Zhu et al., 2012)
Intuitive topic discovery by incorporating word-pair’s connection into lda

( Steyvers and Griffiths, 2007; )
Probabilistic Topic Models

( O’Callaghan et al., 2015 )
An analysis of the coherence of descriptors in topic modeling.







Essas abordagens de extração de tópicos também fornecem uma estratégia
de redução da dimensionalidade visando a construção de novas dimensões que representam
os principais tópicos ou assuntos identificados na coleção de documentos. Ainda, uma or-
ganização baseada em tópicos agrupa termos com mesmo significado em um mesmo tópico
(sinonı́mia) e permite que um mesmo termo ocorra em mais de um tópico caso ele possa
ter significado diferente em diferentes contextos (polissemia). Uma representação cons-
truı́da com os novos atributos extraı́dos é uma oportunidade de incorporar conhecimento
de domı́nio aos dados (Guyon and Elisseeff, 2003).


para construção de uma representação estruturada da coleção de documentos.





























Diogo --

1 Introdução
17
1.1 Motivação . . . . . . . . . . . . . . . . . . . . . . . . . . . . . . . . . . . . 17
1.2 Objetivos, Métodos e Resultados Esperados . . . . . . . . . . . . . . . . . 19
1.3 Organização da Dissertação . . . . . . . . . . . . . . . . . . . . . . . . . . 20

B APÊNDICE 2 - Email enviado convidando usuários a testarem o sistema 91



Camila
1 Introdução
1
1.1 Hipótese e Objetivo . . . . . . . . . . . . . . . . . . . . . . . . . . . . . . . 3
1.2 Proposta e Contribuições . . . . . . . . . . . . . . . . . . . . . . . . . . . . 4
1.3 Organização do Texto 5


simbata
1
Introdução p. 1
1.1 Relações semânticas . .. . . . . . . . . . . . . . . . . . . . . . . . . . . . p. 3
1.2 Motivação . . . . . . .  . . . . . . . . . . . . . . . . . . . . . . . . . . . p. 3
1.3 Objetivos . . . . . . .  . . . . . . . . . . . . . . . . . . . . . . . . . . . p. 5
1.4 Organização do texto . . . . . . . . . . . . . . . . . . . . . . . . . . . . . p. 5





LCHOliveira

Introdução 1
1.1 Recuperação de Informação . . . . . . . . . . . .. . . . . . . . . . . . 1
1.2 Meta-Modelos Formais em Recuperação de Informação. . . . . . . . . . . . 3
1.3 Objetivos e Contribuições  . . . . . . . . . . . . . . . . . . . . . . . 4
1.4 Organização da Dissertação . . . . . . . . . . . . . . . . . . . . . . . 6

Proposta de uma Ferramenta para Avaliação de Desempenho de SRI 69
6.1 Introdução . . . . . . . . . . . . . . . . . . . . . .. . . . . . . . . 69
6.2 Especificação dos Modelos Funcionais . . . . . . . . .. . . . . . . . . 71
6.3 Especificação da Coleção de Referência . . . . . . . .. . . . . . . . . 72
6.4 Especificação da Medida de Avaliação . . . . . . . . .. . . . . . . . . 73
6.4.1 Precisão . . . . . . . . . . . . . . . . . . . . . .. . . . . . . . . 74
6.4.2 Revocação . . . . . . . . . . . . . . . . . . . . . . . . . . . . . . 75
6.4.3 Precisão nos X primeiros . . . . . . . . . . . . . .. . . . . . . . . 75
6.4.4 Precisão-R . . . . . . . . . . . . . . . . . . . . .. . . . . . . . . 75
6.4.5 Medida-E . . . . . . . . . . . . . . . . . . . . . .. . . . . . . . . 75
6.5 Processo de Recuperação . . . . . . . . . . . . . . . . . . . . . . . . 76
6.6 Processo de Comparação Relativa ou Avaliação de Resultados  . . . . . . 77
Conclusões e Trabalhos Futuros 79
7.1 Conclusões . . . . . . . . . . . . . . . . . . . . . . . .  . . . . . . 79
7.2 Trabalhos Futuros . . . . . . . . . . . . . . . . . . . . . . . . . . . 80




Marcelo Bonfin

INTRODUÇÃO ............................................................. 1
CONSIDERAÇÕES INICIAIS ................................................. 1
MOTIVAÇÃO .............................................................. 1
OBJETIVO DA PESQUISA ................................................... 2
ESTRUTURA DA DISSERTAÇÃO ............................................... 2





Takao

1 INTRODUÇÃO....................................................................1
1.1 Apresentação ..................................................................1
1.2 O Problema da Pesquisa .................................................1
1.3 Justificativa e Relevância ...............................................2
1.4 Métodos de Desenvolvimento da Pesquisa ........................................3
1.5 Limitações do Trabalho ..................................................4
1.6 Estrutura do Trabalho .....................................................4

