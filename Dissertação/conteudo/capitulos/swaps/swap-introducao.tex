




o agrupamento dos segmentos onde cada grupo contém 









Detalhes sobre as técnicas de segmentação textual e extração de tópicos são apresentadas no Capítulo~\ref{cap2}.

Extração de tópicos evita redundâncias. \\ Exibir o tópico que só contém cabeçalhos??


% -- #################### SAI ####################

%         ==========|   Segmentação  |========== 


A tarefa de segmentação automática de textos, ou segmentação textual consiste em dividir um texto em partes que contenham um significado relativamente independente. Em outras palavras, é identificar as posições nas quais há uma mudança significativa de assunto. É útil em aplicações que trabalham com textos sem indicações de quebras de assunto, ou seja, não apresentam seções ou capítulos, como transcrições automáticas de áudio, vídeos e grandes documentos que contêm vários assuntos como atas de reunião e notícias~\cite{Aggarwal2018, bokaei2015, sakahara2014, misra2009, eisenstein2008, choi2000}.
% <-- Referências
Pode ser usada para melhorar o acesso a informação solicitada por meio de uma consulta, onde é possível oferecer porções menores de texto mais relevantes ao invés de exibir um documento grande que pode conter informações menos pertinentes.  Além disso, encontrar pontos onde o texto muda de assunto, pode ser útil como etapa de pré-processamento em aplicações voltadas ao entendimento do texto, principalmente em documentos longos~\cite{Choi2000}.


% pela qual obtém-se trechos com um assunto principal e relativamente independente do texto integral. 

%        ==========|   Extração de Tópicos  |==========

Os modelos de extração de tópicos fornecem uma estratégia que visa encontrar nas relações entre documentos, padrões latentes que sejam significativos para o entendimento dessas relações~\cite{Wei2007}. Tais modelos podem ranquear um conjunto de termos importantes para um ou mais assuntos, bem como ranquear documentos por sua relevância para determinado tema~\cite{Faleiros2016,Xing2009}.
Atualmente, destacam-se os modelos probabilísticos de extração de tópicos como LDA~\cite{Blei2003} e PLSA~\cite{Hofmann1999}. São abordagens amplamente utilizadas ~\cite{DZhu20122} e frequentemente referenciadas em trabalhos que buscam extrair conhecimento e organizar bases textuais ~\cite{Aggarwal2018, OCallaghan2015, Steyvers2007}.  
%
%
Nesse trabalho, a expressão \textit{tópico} é usada para designar um assunto considerando que o mesmo foi extraído por meio de técnicas automáticas, ficando a expressão \textit{assunto} utilizada como seu teor popular. 

O processo de extração de tópicos atribui um peso a cada documento-tópico e uma relação termo-tópico que pode representar a probabilidade de ocorrência de um termo em um documento dado que o tópico está presente. A partir dessas representações, é possível agrupar documentos que compartilham o mesmo tópico bem como os termos que melhor descrevem o tópico~\cite{Aggarwal2018}. Com isso, obtém-se uma organização da coleção de documentos que favorece técnicas para navegação e consulta à coleção de documentos~\cite{Maracini2010}. 
% 
Além disso, essas abordagens de extração de tópicos fornecem a construção de novos atributos que representam os principais tópicos ou assuntos identificados na coleção de documentos, sendo uma oportunidade de incorporar conhecimento de domínio aos dados~\cite{Guyon2003}. 
% com o mesmo teor 




% -- #################### SAI ####################

















% -- fiz ---↓↓↓↓ ?


Essas abordagens de extração de tópicos fornecem a construção de novas atributos que representam os principais tópicos ou assuntos identificados na coleção de documentos, sendo uma oportunidade de incorporar conhecimento de domínio aos dados. 

Ainda, uma organização baseada em tópicos agrupa termos com mesmo significado em um mesmo tópico (sinonímia) e permite que um mesmo termo ocorra em mais de um tópico caso ele possa ter significado diferente em diferentes contextos (polissemia). 


Uma representação construída com os novos atributos extraídos é uma oportunidade de incorporar conhecimento de domínio aos dados (Guyon and Elisseeff, 2003).




--> { Um sistema capaz de construir e utilizar tópicos para aprimorar a recuperação de informação em coleções de atas. }

--> { dado uma grande coleção de documentos, selecionar os trechos mais relevantes a um determinado tópico/consulta~\cite{Zhai2017}. }

--> Além disso, é possível atribuir um tópico à consulta e utilizá-lo para expandir a consulta~\cite{Xing2009}. 

--> Os modelos de extração de tópicos são utilizados para expandir os documentos e consultas~\cite{WEI2007}.

# --> Uma representação baseada em tópicos pode relacionar termos distintos com forte conexão com um determinado tópico~\cite{Wei2006}.

--> A utilização de modelos de extração tópicos na recuperação de informação parte da ideia que documentos que compartilham tópicos com a consulta tem maior relevância~\cite{Xing2009}. 

# --> Assim, os modelos de extração de tópicos são utilizados para incorporar informação aos documentos e consultas com a finalidade de aprimorar o ranqueamento dos resultados~\cite{WEI2007}.


" With the topic models derived from previous methods, texts are reformulated (i.e. usually expanded) to improve the retrieval effectiveness. "  
--> Os modelos de extração de tópicos são utilizados para expandir os documentos e consultas~\cite{WEI2007}.
.



" Using the LDA-based representation, this document is closely related to two topics that have strong connections with the term “leverage” "
Uma representação baseada no LDA pode relacionar dois tópicos com forte conexão com um determinado termo~\cite{Wei2006}.




Liu and Croft (2004) showed that document clustering can improve retrieval effectiveness in the language modeling framework


" intuitively relevant documents have topic distributions that are likely to have generated the set of words associated with the query."[1, 2]
--> { parte da ideia que documentos que compartilham tópicos com  a consulta tem maior probabilidade de compartilhar termos semelhantes~\cite{Xing2009}. } 

"n fact, early research on topic models suggested that they might be used for information retrieval (IR)"[2]



% -> 2 Extração de tópicos -- descritores

% é possível expander a consulta atribuindo a ela um tópico e utilizá-lo pa
% os descritores podem ser usados para expandir a consulta. 

e o teor dos segmentos
--> { Diferente de trabalhos anteriores, usa-se os descritores para expandir o texto dos segmentos ao invés da consulta. }

" In the first, a document is represented by itself and the topics to which it belongs "
os grupos extraídos e seus descritores podem ser vistos como uma forma de descrição da coleção.
os tópicos ajudam a representar o documento

" A second approach is to calculate a query related topic by using topic models and use it for query expansion "
A segunda forma é atribuir um tópico a consulta e usá-lo para expandir a consulta.
~\cite{Xing2009}

 " The second is the query expansion approach, where we topics similarly add words to the query and run the revised query "



"text retrieval serves for the purpose of converting the very large raw text collection into a much smaller more relevant set of documents that would be actually needed for a particular application"
~\cite{Zhai2017}
--> { dado uma grande coleção de documentos, selecionar os trechos mais relevantes a um determinado tópico/consulta }













[ https://www.google.com.br/search?q=topic+extraction+for+information+retrieval&source=lnt&tbs=qdr:y&sa=X&ved=0ahUKEwiCncC215DbAhWHI5AKHcrKD_8QpwUIIQ&biw=1366&bih=669 ]



Manning et al.: Chap. 1-5
• Croft et al.: Chap. 4-5


--> Xing2009 :
Modelos de extração de tópicos podem se integrar ao processo de recuperação de informação

podendo ser usados para descrever o conteúdo da coleção
" These topics can be used to describe the contents of a collection: the high probability topics and words within the topics can be viewed as a loose description of the collection, with better topic models providing better descriptions "






% localização rápida de objetos. 
% , evitando porções de texto irrelevantes 
% , que frequentemente tratam de assuntos diversos.

% Navegação e exploração
% Ao agrupar os segmentos por tópicos, tem-se uma organização dos documentos que permite ao usuário visualizar segmentos semelhantes ao explorar os grupos após uma consulta por palavras-chave ou.
% Ao apresentar os segmentos agrupados por tópicos 
% após o retorno de uma busca inicial por palavras chave


com subdocumentos 

Ao utilizar a estrutura gerada pelas técnicas descritas, 
ao mesmo tempo que vale-se das vantagens encontradas nas relações



% --> Gera uma estrutura aos texto e à coleção de textos
% --> encontra relação entre termos sem necessidade de conhecimento externo sobre o domínio.
% Evita-se o uso de indices, ao invês disso, a busca se dá pelos descritores dos tópicos, o permite também uma exploração aos grupos em busca de documentos relacionados que não foram retornados na busca 
% o agrupamento de documentos permite a visualização de documentos semelhantes indepentemente de consultas.  %%%  --- Trazendo uma experiência mais próxima de uma exploração que de consultas.

% e com isso obter uma organização da coleção de documentos.

% descoberta pelos modelos de extração de tópicos em documentos segmentados em conjunto com técnicas tradicionais de recuperação de informação. (aproveitando-se da estrutura dos tópicos e delimitação dos segmentos)




% Consultar uma base de dados textual implica em atender à necessidade do usuário no que tange o acesso as informações mais relevantes.

% No caso de documentos como atas de reunião, a busca analisa apenas o conteúdo dos documentos 

% As atas não apresentam meta-informação significantes como texto seccionado, \textit{hyperlinks} e nome de autores.   


% Índice invertido é uma técnica popular

% Ao buscar por \textit{``bolsa de estudo''}, o usuário pode também estar interessado em \textit{``processo seletivo''}, \textit{``CNPQ''} ou \textit{``FAPESP''}






% -->
Essas abordagens de extração de tópicos também fornecem uma estratégia de redução da dimensionalidade visando a construção de novas dimensões que representam os principais tópicos ou assuntos identificados na coleção de documentos. Ainda, uma organização baseada em tópicos agrupa termos com mesmo significado em um mesmo tópico (sinonímia) e permite que um mesmo termo ocorra em mais de um tópico caso ele possa ter significado diferente em diferentes contextos (polissemia). Uma representação construída com os novos atributos extraídos é uma oportunidade de incorporar conhecimento de domínio aos dados (Guyon and Elisseeff, 2003).


para construção de uma representação estruturada da coleção de documentos.





"The art of machine learning starts with the design of appropriate data representations.  Better performance is often achieved using features derived from the original input. Building a feature representation is an opportunity to incorporate domain knowledge into the data and can be very application specific. Nonetheless, there are a number of generic feature construction methods, including: clustering; basic linear transforms of the input variables (PCA/SVD, LDA); more sophisticated linear transforms like spectral transforms (Fourier, Hadamard), wavelet transforms or convolutions of kernels; and applying simple functions to subsets of variables, like products to create monomials"





% --> Já falei o que é uma consulta??
% --> Falei de Boolean Retrieval
% --> Passar para Information retrieval com indexação, ranqueamento e TF-ITF
% --> É possível atribuir similaridades usando variáveis latentes?
% --> Gaps estão suficientemente evidendes?


--> Gera uma estrutura aos texto e à coleção de textos



expansão de consulta para reformular a consulta

%--> colocar isso no final. Isso se refere a IR de modo geral, não apenas a Boolean Retrieval 
2) as buscas em grandes coleções de documentos podem ser mais lentas 
4) o retorno ao usuário são os documentos integrais, o que pode exigir uma segunda consulta dentro de um documento para encontrar o trecho desejado

% <-- --- --- --- --- --- --- --- --- --- ---

6) Não reconhece variáveis latentes...
7) O usuário pode não estar familiarizado com o vocabulário do domínio, % mas pode conhecer o tópico ao qual está interessado.

Thesaurus depende de conhecimento externo sobre o domínio.




% Nesse trabalho, a segmentação textual é empregada na identificação dos assuntos tratados em cada ata. Os segmentos de atas gerados, aqui considerados como sub-documentos, são entregues a um extrator de tópicos. A tarefa do extrator é agrupar os segmentos com um mesmo tópico e eleger palavras da coleção que melhor descrevem cada grupo. Os descritores gerados pelo extrator de tópicos bem com os grupos são aproveitados nas técnicas de recuperação de informação a fim de ranquear os resultados e auxiliar o usuário a encontrar segmentos semelhantes por meio da navegação pelos grupos.

% O uso de indices e ... traz soluções que podem resolver problemas como lentidão, ranquear e ... mas ....
% Essa técnica, chamada de \textit{Boolean Retrieval} informa para cada documento se este casa ou não com a consulta do usuário. 
% Essa técnica, chamada de \textit{Boolean Retrieval} enxerga os documentos como conjuntos de palavras e guiam-se 
% pela presença ou ausência dos termos da pesquisa nos documentos. 
% verificando se os termos da pesquisa estão presentes nos documentos.
% analisa nos documentos a presença ou ausência dos termos da pesquisa e informa

% com significado próximo como ``computador'' e ``equipamento''.


% Outra forma comum é o uso de ferramentas que fazem buscas nos conteúdos dos documentos, buscando por ocorrências de palavras-chave nos textos. Essas ferramentas permitem buscas combinadas com operadores lógicos como \textit{and}, \textit{or} e \textit{not} ou ainda suporte a expressões regulares. Esse recurso, conhecido como \textit{grepping}\footnote{O nome \textit{grepping} é uma referência ao comando \texttt{grep} do Unix}, traz resultados satisfatórios em muitos casos. Por outro lado, traz algumas desvantagens como: 1) transfere certa complexidade da tarefa ao usuário 2) as buscas em grandes coleções de documentos podem ser mais lentas 3) não há suporte a padrões mais flexíveis como a proximidade entre as palavras ou palavras que estejam na mesma sentença 4) o retorno ao usuário são os documentos integrais, o que pode exigir uma segunda busca dentro de um documento para encontrar o trecho desejado~\cite{Aggarwal2012,Manning2008}. 


% :: Segmentos --> trechos com o assunto desejado e nada mais
% Essa abordagem visa desenvolver um sistema de busca que entrega trechos de diferentes documentos relacionados à consultado do usuário que proporciona

% um retorno mais direcionado



----- ----- ----- ----- ----- ----- ----- ----- ----- -----


Buscando por ocorrências <--> analisando a frequência das palavras
usuário usam ferramentas básicas fornecidas junto com OS ?


1) Transfere certa complexidade da tarefa ao usuário.

2) As buscas em grandes coleções de documentos podem ser mais lentas. 
% Complexidade computacional 
% Recursos limitados

3) Não há suporte a padrões mais flexíveis como a proximidade entre as palavras ou palavras que estejam na mesma sentença.

4) O retorno ao usuário são os documentos integrais, o que pode exigir uma segunda busca dentro de um documento para encontrar o trecho desejado~\cite{Aggarwal2012,Manning2008}. 

5) Dificuldade em ranquear ?? ( TF-IDF )


% contudo essa prática costuma ser insuficiente, pois em uma busca pelo conteúdo dos textos usa-se ferramentas computacionais baseadas em localização de palavras-chave que 

% além de encontrar ocorrências das palavras podem oferecer 



----- ----- ----- ----- ----- ----- ----- ----- ----- ----- 


Whereas traditional information retrieval only uses the content of documents to
retrieve results of queries, the Web requires stronger mechanisms for quality control because
of its open nature.



Consequentemente, o alvo da busca é refinado à medida que o usuário ganha mais informações sobre o domı́nio. Outra caracterı́stica das buscas é que o usuário potencialmente se interessa por resultados similares. Por exemplo, ao buscar por Cleópatra, o usuário também pode estar interessado em informações sobre Marco Antônio ou o Antigo Egito.



"A ideia básica dos modelos de tópicos é descobrir, nas re- lações entre documentos e termos, padrões latentes que sejam significativos para o entendimento dessas relações. Por exemplo, tais modelos podem ranquear um conjunto de termos como impor- tantes para um ou mais temas. Bem como ranquear documentos como tendo relevância para um ou mais temas. Se for associado um vetor A j a um documento d j e um vetor B i para um termo w i , sendo A j e B i K-dimensionais, pode-se considerar que cada uma dessas dimensões caracterizam um fator latente relacionado com um documento A j e um termo B i . Assim o produto interno A j · B i pode modelar a importância desses fatores na relação documento d j e termo w i ."




"Nesta tese, a expressão tópico é usada levando-se em conta que o assunto tratado em uma coleção de documentos é extraído automaticamente, ou seja, tópico é definido como um conjunto de palavras que frequentemente ocorrem em documentos semanticamente relacionados. Esses conjuntos de palavras (que definem os tópicos) são obtidos por um processo de pós-processamento realizado a partir das dimensões latentes descobertas pela aplicação dos métodos de modelo de tópicos."  



e com isso obter uma organização da coleção de documentos.



% os principais tópicos ou assuntos na coleção de documentos 


% A extração de tópicos é uma técnica que visa encontrar . 





"O resultado do processo de extração de tópico é uma representação documento-tópico que determinam um peso de cada tópico para cada documento e uma representação termo-tópico. Esta última está relacionada com o modelo generativo que foi escolhido, e pode representar uma probabilidade da ocorrência do termo quando um tópico ocorre em um documento, a frequência esperada desse termo, ou mesmo um peso estimado matematicamente que não pode ser “traduzido” com algum significado para o contexto linguı́stico."



"Atualmente, os modelos de extração de tópicos probabilı́sticos como o Latent Dirichlet Allocation (LDA) são abor- dagens amplamente aplicadas (Zhu et al., 2012), sendo referenciadas na grande maioria dos trabalhos da literatura como sinônimos de modelos de extração de tópicos (Stey- vers and Griffiths, 2007; O’Callaghan et al., 2015)."  


Essas abordagens de extração de tópicos também fornecem uma estratégia
de redução da dimensionalidade visando a construção de novas dimensões que representam
os principais tópicos ou assuntos identificados na coleção de documentos. Ainda, uma or-
ganização baseada em tópicos agrupa termos com mesmo significado em um mesmo tópico
(sinonı́mia) e permite que um mesmo termo ocorra em mais de um tópico caso ele possa
ter significado diferente em diferentes contextos (polissemia). Uma representação cons-
truı́da com os novos atributos extraı́dos é uma oportunidade de incorporar conhecimento
de domı́nio aos dados (Guyon and Elisseeff, 2003).


para construção de uma representação estruturada da coleção de documentos.





























Diogo --

1 Introdução
17
1.1 Motivação . . . . . . . . . . . . . . . . . . . . . . . . . . . . . . . . . . . . 17
1.2 Objetivos, Métodos e Resultados Esperados . . . . . . . . . . . . . . . . . 19
1.3 Organização da Dissertação . . . . . . . . . . . . . . . . . . . . . . . . . . 20

B APÊNDICE 2 - Email enviado convidando usuários a testarem o sistema 91



Camila
1 Introdução
1
1.1 Hipótese e Objetivo . . . . . . . . . . . . . . . . . . . . . . . . . . . . . . . 3
1.2 Proposta e Contribuições . . . . . . . . . . . . . . . . . . . . . . . . . . . . 4
1.3 Organização do Texto 5


simbata
1
Introdução p. 1
1.1 Relações semânticas . .. . . . . . . . . . . . . . . . . . . . . . . . . . . . p. 3
1.2 Motivação . . . . . . .  . . . . . . . . . . . . . . . . . . . . . . . . . . . p. 3
1.3 Objetivos . . . . . . .  . . . . . . . . . . . . . . . . . . . . . . . . . . . p. 5
1.4 Organização do texto . . . . . . . . . . . . . . . . . . . . . . . . . . . . . p. 5





LCHOliveira

Introdução 1
1.1 Recuperação de Informação . . . . . . . . . . . .. . . . . . . . . . . . 1
1.2 Meta-Modelos Formais em Recuperação de Informação. . . . . . . . . . . . 3
1.3 Objetivos e Contribuições  . . . . . . . . . . . . . . . . . . . . . . . 4
1.4 Organização da Dissertação . . . . . . . . . . . . . . . . . . . . . . . 6

Proposta de uma Ferramenta para Avaliação de Desempenho de SRI 69
6.1 Introdução . . . . . . . . . . . . . . . . . . . . . .. . . . . . . . . 69
6.2 Especificação dos Modelos Funcionais . . . . . . . . .. . . . . . . . . 71
6.3 Especificação da Coleção de Referência . . . . . . . .. . . . . . . . . 72
6.4 Especificação da Medida de Avaliação . . . . . . . . .. . . . . . . . . 73
6.4.1 Precisão . . . . . . . . . . . . . . . . . . . . . .. . . . . . . . . 74
6.4.2 Revocação . . . . . . . . . . . . . . . . . . . . . . . . . . . . . . 75
6.4.3 Precisão nos X primeiros . . . . . . . . . . . . . .. . . . . . . . . 75
6.4.4 Precisão-R . . . . . . . . . . . . . . . . . . . . .. . . . . . . . . 75
6.4.5 Medida-E . . . . . . . . . . . . . . . . . . . . . .. . . . . . . . . 75
6.5 Processo de Recuperação . . . . . . . . . . . . . . . . . . . . . . . . 76
6.6 Processo de Comparação Relativa ou Avaliação de Resultados  . . . . . . 77
Conclusões e Trabalhos Futuros 79
7.1 Conclusões . . . . . . . . . . . . . . . . . . . . . . . .  . . . . . . 79
7.2 Trabalhos Futuros . . . . . . . . . . . . . . . . . . . . . . . . . . . 80




Marcelo Bonfin

INTRODUÇÃO ............................................................. 1
CONSIDERAÇÕES INICIAIS ................................................. 1
MOTIVAÇÃO .............................................................. 1
OBJETIVO DA PESQUISA ................................................... 2
ESTRUTURA DA DISSERTAÇÃO ............................................... 2





Takao

1 INTRODUÇÃO....................................................................1
1.1 Apresentação ..................................................................1
1.2 O Problema da Pesquisa .................................................1
1.3 Justificativa e Relevância ...............................................2
1.4 Métodos de Desenvolvimento da Pesquisa ........................................3
1.5 Limitações do Trabalho ..................................................4
1.6 Estrutura do Trabalho .....................................................4





% TODO: Referências de trabalhos que têm feito isso   ↓↓↓↓↓↓↓
% Para superar essas limitações têm sido utilizadas técnicas de aprendizado de máquina por meio de diversas abordagens. Por exemplo, elas vêm sendo empregadas na organização, gerenciamento, recuperação de informação e extração de conhecimento, como a extração de tópicos e a categorização de automática de documentos~\cite{Purver2006}.  % Estava sobrando :: Removido em 13-05-2018


