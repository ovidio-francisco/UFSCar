


















As medidas tradicionais baseiam-se na matriz de confusão a qual considera um verdadeiro positivo somente quando duas segmentações colocam o final de um segmento no mesmo ponto, sem considerar pequenas diferenças. Além disso, sempre penalizam o algoritmo quando um segmento não não coincide perfeitamente com a referência. Assim essas medidas podem ser mais adequadas quando necessita-se medir a eficiência do algoritmo com maior exatidão. 
Por outro lado, essas medidas computam apenas os erros do algoritmo quando se detecta falsos positivos ou falsos negativos, o que nesse contexto de segmentação textual pode não ser suficiente, dado a subjetividade da tarefa. 
%
Além dessas medidas, que consideram apenas se um segmento foi perfeitamente definido conforme uma referência, pode-se também considerar a distância entre o segmento extraído automaticamente e o segmento de referência~\cite{Kern2009}. Chama-se \textit{near misses} o caso em que um limite identificado automaticamente não coincide exatamente com a referência, mas é necessário considerar a proximidade entre eles.





















podem não ser confiáveis por não considerarem as distâncias lineares dos limites entre segmentos.
Em outras palavras, 



Por consequência, 






As medidas de avaliação tradicionais, podem não ser confiáveis, por não considerarem a distância entre os limites, mas penalizam o algoritmo sempre que um limite que não coincide perfeitamente com a referência. Essas medidas podem ser mais adequadas quando necessita-se de segmentações com maior exatidão. 






, F$^1$, Acurácia
- Precisão...
- F1...

















Com base nos valores da matriz de contingência, pode-se calcular:
- Precisão...
- F1...










% Ou seja, dado dois termos de distância $k$, o algoritmo é penalizado caso não concorde com a segmentação de referência se as palavras estão ou não no mesmo segmento. Dadas uma segmentação de referência $ref$ e uma segmentação automática $hyp$, ambas com $N$ sentenças, P$_k$ é computada como: 

o algoritmo é penalizado caso não concorde com a segmentação de referência se as palavras estão ou não no mesmo segmento. 



ou seja o algoritmos verifica se a hipotese coloca as palavras no mesmo segmento ou e segmentos distintos. O algoritmo será penlizado caso a referência não concorde com a hipotese.




% As medidas de avaliação tradicionais, precisão e revocação, podem não ser confiáveis, por não considerarem a distância entre os limites, mas penalizam o algoritmo sempre que um limite que não coincide perfeitamente com a referência. Essas medidas podem ser mais adequadas quando necessita-se de segmentações com maior exatidão. Em outras palavras, computam apenas os erros do algoritmo quando se detecta falsos positivos ou falsos negativos, o que nesse contexto de segmentação textual pode não ser suficiente, dado a subjetividade da tarefa. Além dessas medidas, que consideram apenas se um segmento foi perfeitamente definido conforme uma referência, pode-se também considerar a distância entre o segmento extraído automaticamente e o segmento de referência~\cite{Kern2009}. Chama-se \textit{near misses} o caso em que um limite identificado automaticamente não coincide exatamente com a referência, mas é necessário considerar a proximidade entre eles.
