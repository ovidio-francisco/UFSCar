

Como deve ser uma avaliação de extratores de tópicos? --> 


% --> o processo de recuperação de informação filtrou um conjunto de segmentos. Dentre esses segmentos, 5 foram escolhidos aleatoriamente.




% - - - - - - - - - - - - - - - - - - - - - - - - - - - - - - 

Avaliação Subjetiva do extrator de tópicos:

1 - Comparar algoritmos de extração de tópicos na tarefa de extração de padrões no contexto
das atas de reunião. 
2 - Analisar a qualidade dos agrupamentos no que tange a navegação por grupos com mesmo tópico, 
3 - Analisar a qualidade dos descritores extraídos para recuperar os documentos dos grupos.


Tarefas:

Fazer uma busca no sistema pelos termos "concessão de bolsa".

Localizar segmentos que abordem esse assunto na lista ranqueada.

Localizar segmentos que abordem esse assunto utilizado o agrupamento.

Encontrar o documento original de um segmento que aborde esse assunto.

Responder:


* É possível identificar claramente os tópicos das coleções com bases nas respectivas categorias?


* A navegação pelos tópicos ajuda a conduzir o usuário para o documento desejado?


* Os descritores dos tópicos são bons representantes do assunto dos documentos?


* Os segmentos apresentados são suficientes para entender o tópico?


% - - - - - - - - - - - - - - - - - - - - - - - - - - - - - - 




 % ------------------------------


 - O K-Means se sobre-sai enquanto PLSA e LDA são equivalentes.


--> O segmentador, sendo uma etapa anterior a extração de tópicos, influencia na qualidade do grupos??

--> O K-means traz melhores segmentos (Sem misturar assuntos) em relação aos outros métodos.


'Aparentemente o K-means seleciona os trechos mais coesos, mas precisa de mais experimentos, mais amostras ...'




 1 - Motivação para o experimento.
 2 - Escolha das questões.
 3 - Escolha dos participantes.
 4 - Procedimento (2 consultas, técnicas em ordens diferentes).
 5 - Medidas para avaliação.
 6 - Conclusão e discussão.


% ------------------------------
% ------------------------------

--> Seria uma tecnica mais estável que  outra? 


--> Com os meus avaliadores T1 > T2 > T3
--> Incluindo os da Katti   T1 > T2 = T3;
19:10 --> "Informação importante"
quantos da Consulta 1 e da 2
as localiazações
Afinidade -- quantificar
Procedimento --> Ordem alternada das Técnicas


% ------------------------------




'Usando o K-Means as pessoas identificaram que na maioria dos casos foi .. não é perfeito mais ... tem uma uniformidade maior em termos dos assuntos '









% Neste capítulo, o processo de extração de tópicos em segmentos de atas de reunião é avaliado. A avaliação experimental neste trabalho é focada em três pontos principais: (1) Comparar algoritmos de extração de tópicos na tarefa de extração de padrões no contexto das atas de reunião.  (2) Analisar a qualidade dos agrupamentos no que tange a navegação por grupos com mesmo tópico, (3) Analisar a qualidade dos descritores extraídos para recuperar os documentos dos grupos.  %(padrões = descritores e agrupamento)  




Nas próximas seções, são apresentadas as coleções textuais a serem utilizadas na avaliação experimental. Em seguida, é descrita a configuração dos experimentos, com o ajuste de parâmetros dos algoritmos e os critérios de avaliação. Por fim, são apresentados os experimentos realizados e uma análise dos resultados.





Neste capítulo, o processo de extração de tópicos em sub-documentos de atas de reunião é avaliado. A avaliação experimental neste trabalho é focada em três pontos principais: 
(1) Analisar a qualidade dos agrupamentos no que tange a navegação por grupos com mesmo tópico, 
(2) Analisar a qualidade dos descritores extraídos para recuperar os documentos dos grupos.
% na representação dos sub-documentos.











0 - Nada;
1 - Pouco;
2 - Razoável;
3 - Bom;
4 - Excelente;



Avaliação da UX do sistema ...























1) Is it possible to identify clearly the topics of the collections
based on the respective categories?
2) Does the browsing through the topic hierarchy conduced
the user to a desired document set?
For both evaluations the used grades were:
0 - Nothing;
1 - Little;
2 - Fair;
3 - Good;
4 - Excellent


\item Os descritores dos tópicos são bons representantes do assunto dos documentos?
\item É possível indentificar claramente os 
\item A navegação pelos topicos ajuda a encontrar o assunto desejado?







% Nesse capítulo, as técnicas de extração de tópicos são analisadas em no que tange a qualidade dos agrupamentos e na capacidade dos descritores em representar os segmentos. 
%
% O objetivo é comparar algoritmos de extração de tópicos na tarefa de extração de padrões no contexto

% na capacidade dos descritores em representar os segmentos. 
%


