

Avaliação Subjetiva do extrator de tópicos:

1 - Comparar algoritmos de extração de tópicos na tarefa de extração de padrões no contexto
das atas de reunião. 
2 - Analisar a qualidade dos agrupamentos no que tange a navegação por grupos com mesmo tópico, 
3 - Analisar a qualidade dos descritores extraídos para recuperar os documentos dos grupos.


Tarefas:

Fazer uma busca no sistema pelos termos "concessão de bolsa".

Localizar segmentos que abordem esse assunto na lista ranqueada.

Localizar segmentos que abordem esse assunto utilizado o agrupamento.

Encontrar o documento original de um segmento que aborde esse assunto.

Responder:


* É possível identificar claramente os tópicos das coleções com bases nas respectivas categorias?


* A navegação pelos tópicos ajuda a conduzir o usuário para o documento desejado?


* Os descritores dos tópicos são bons representantes do assunto dos documentos?


* Os segmentos apresentados são suficientes para entender o tópico?





0 - Nada;
1 - Pouco;
2 - Razoável;
3 - Bom;
4 - Excelente;



Avaliação da UX do sistema ...























1) Is it possible to identify clearly the topics of the collections
based on the respective categories?
2) Does the browsing through the topic hierarchy conduced
the user to a desired document set?
For both evaluations the used grades were:
0 - Nothing;
1 - Little;
2 - Fair;
3 - Good;
4 - Excellent


\item Os descritores dos tópicos são bons representantes do assunto dos documentos?
\item É possível indentificar claramente os 
\item A navegação pelos topicos ajuda a encontrar o assunto desejado?






Nas próximas seções, são apresentadas as coleções textuais a serem utilizadas na avaliação experimental. Em seguida, é descrita a configuração dos experimentos, com o ajuste de parâmetros dos algoritmos e os critérios de avaliação. Por fim, são apresentados os experimentos realizados e uma análise dos resultados.





Neste capítulo, o processo de extração de tópicos em sub-documentos de atas de reunião é avaliado. A avaliação experimental neste trabalho é focada em três pontos principais: 
(1) Analisar a qualidade dos agrupamentos no que tange a navegação por grupos com mesmo tópico, 
(2) Analisar a qualidade dos descritores extraídos para recuperar os documentos dos grupos.
% na representação dos sub-documentos.






