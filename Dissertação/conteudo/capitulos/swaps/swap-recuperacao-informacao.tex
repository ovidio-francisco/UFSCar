













































% Uma das formas mais comuns para representação textual é conhecida como Modelo Espaço Vetorial (\textit{Vectorial Space Model} - VSM)~\cite{Rezende2003}, onde os documentos e consultas são representados como vetores em um espaço Euclidiano $t$-dimensional em que cada termo extraído da coleção é representado por uma dimensão. 
% % 
% Considera-se que um documento pode ser representado pelo seu conjunto de termos, onde cada termo $k_i$ de um documento $d_j$ associa-se um peso $w_{ij}\geq0$ que indica a importância desse termo no documento. 
% %
% De forma similar, para uma consulta $q$, associa-se um peso $w_{i,q}$ ao par termo consulta que representa a similaridade entre a necessidade do usuário e o termo $k_i$. 
% %
% Assim o vetor associado ao documento $d_j$ é dado por $\vec{d}_{j} = (w_{1,j}, w_{2,j}, ..., w_{t,j})$. 
% %
% De forma similar, o vetor associado a consulta $q$ é dado por $\vec{q} = (w_{1,q}, w_{2,q}, ..., w_{t,q})$.







a similaridade entre a necessidade do usuário e o termo $k_i$. 
% citar (BAEZA e RIBERO, 1999)


% Esse modelo apresenta como principal desvantagem a impossibilidade de ordenação dos resultados por relevância, uma vez que para muitos sistemas de RI o \textit{ranking} dos resultados é uma característica essencial, principalmente em grades bases de dados. 




% No modelo vetorial, a similidade entre um documento $d_j$ e uma consulta $q$ é calculada pela correlação entre os vetores $\vec{d_j}$ e $\vec{q}$, a qual pode ser medida pelo cosseno do  ângulo entre esses vetores, conforme mostrado na Equação~\ref{equ:cosseno-doc-consulta}.










