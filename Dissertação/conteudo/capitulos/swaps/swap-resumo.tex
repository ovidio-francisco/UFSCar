

















































experimentos com 

a fim de entender a composição dos documentos e distribui dos tópicos ao longo da coleção 

permite a exploração de padrões temáticos em \textit{corpora} extensos a partir de representações 



oferecendo.
 Contudo mais dados 

visualização da ...
criação de ... 
utilização de ...
dados sobre a composição temática de documentos obtido por meio da modelagem de tópicos

O sistema desenvolvido oferece a visualização sobre a composição temática da coleção de documentos obtida por meio da modelagem de tópicos.



y e z são as principais contribuições desse trabalho no intuito de dar aportes ao desenvolvimento de novos trabalhos.  

Avaliação de técnicas de Recuperação de Informação para Organização e Extração de Conhecimento de Documentos de Reuniões 


independente de inserção de conhecimentos externos 


relações entre documentos, padrões latentes que sejam significativos para o entendimento
dessas relações

--> que sejam significativos para o entendimento de coleções de documentos textuais, e incorporar conhecimento de domínio aos dados.




a fim de extrair conhecimento em coleções textuais.  
nos quais o conhecimento embutido nos texto é 




problema --> multi-temática
extrator de tópicos --> resolve parte do problema
segmentador --> resolve outra parte do problema
estrutura interna --> resultado
exploração e entendimento o corpus --> benefício.


Os resultados obtidos sugerem que 














% Com a grande disponibilidade de dados em formato textual, há um constante interesse no desenvolvimento de ferramentas computacionais para recuperação automática de informações úteis a partir desses dados. 

% Os modelos de extração de tópicos são recorrentemente referenciados/remetidos como capazes de estabelecer relações e encontrar padrões latentes que sejam significativos para o entendimento de coleções de documentos textuais, e incorporar conhecimento de domínio aos dados.


% tem se destacado nessa tarefa, contudo há um desafio adicional em documento que contém múltiplos assuntos. 


% falar um pouco também sobre a incapacidade humana em analisar grandes volumes de texto. 
% --> entre as soluções destacan-se a extração de tópicos ...
% problema --> multi-temática.
% ...

% Tratar grandes quantidades de dados é uma exigência dos modernos algoritmos de mineração de texto




Em um contexto em que a grande parte das informações armazenadas pelas organizações está em formato textual, o desenvolvimento de ferramentas computacionais para extração e organização automática dessas informações é uma tarefa que retem atenção e relevância.
%
Os modelos de extração de tópicos são recorrentemente empregados nessa tarefa. Esses modelos são capazes de estabelecer relações e encontrar padrões latentes em coleções de documentos textuais.
%
No entanto, há um desafio adicional em documentos constituídos por múltiplos assuntos. Para textos onde há transição de assuntos faz-se necessário, em primeiro lugar, encontrar porções de texto que tratam de um único assunto. Para isso, as técnicas de Segmentação Textual são utilizada para dividir um texto em segmentos com um assunto relativamente independente.
% Criação de estrutura. 
O uso combinado de algoritmos de segmentação textual e extração de tópicos, pode então ser aplicado para criar uma estrutura que ajuda a entender dados textuais, os quais são inerentemente não estruturados.
%
A criação de uma estrutura organizada em tópicos, que concentra informações latentes sobre o \textit{corpus} pode favorecer técnicas de Recuperação de Informação. Os dados incorporados à coleção original fornecem base para localização e \textit{ranking} dos resultados mais relevantes à uma consulta. 
% 
A pesquisa desse trabalho de mestrado investigou as técnicas de segmentação textual e extração de tópicos para o desenvolvimento de um sistema de recuperação de informação em uma coleção de atas de reunião. 
% 
% A ferramenta foi desenvolvida para 
Desenvolveu-se uma sistema para identificar, organizar e apresentar assuntos registrados em atas de reunião utilizando a estrutura latente de documentos segmentados em conjunto com técnicas de Recuperação de Informação.
% 
A metodologia apresenta nessa dissertação, descreve o estudo realizado sobre o corpus, bem como avaliações das técnicas utilizadas bem implementação das ferramentas implementadas. 
% 
Foram avaliados cinco técnicas de segmentação textual e três modelos de extração de tópicos da literatura, e proposto uma metodologia para extração de conhecimento em atas de reunião.












% A nova estrutura derivada da coleção original fornece base para localização 
% e \textit{ranking} dos resultados mais relevantes à uma consulta. 
% expansão




% Essa organização permite que técnicas de recuperação de informação expandam o espaço de busca além do conjunto de termos original de cada segmento, sendo assim, favorecidas quanto a identificação segmentos relacionados a consulta do usuário




% Desenvolveu-se uma sistema para identificar, organizar e apresentar assuntos registrados em atas de reunião utilizando a estrutura latente de documentos segmentados em conjunto com técnicas de Recuperação de Informação.










