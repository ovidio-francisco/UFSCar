Não entendi como o aprendizado de máquina adiciona conhecimento ao trecho. O AM vai adicionar conhecimento à recuperação de informação, não?









% Nos trabalhos consultados como referência a esse problema estão técnicas para dividir os documentos utilizando algoritmos de segmentação textual baseados em diversas abordagens e técnicas que simplesmente usam quebras de parágrafos ou mesmo sentenças para isolar os múltiplos assuntos em trechos. Então, esses trabalhos agrupam os trechos por sua similaridade a fim de encontrar relações entre os trechos ou os próprios documentos originais. Entre os trabalhos encontrados na literatura, poucos utilizam modelos de aprendizado de máquina para adicionar conhecimento aos segmentos sendo a maior parte baseada em representações tradicionais como \textit{BOW}. 

% Observou-se então na recuperação de informação em documentos multi-temáticos, uma área pouco explorada, sobre tudo para o idioma português. 
% Entre as abordagens aplicáveis a este problema estão a segmentação textual e os modelos de extração de tópicos as quais, em conjunto, são capazes de separar, identificar e agrupar os assuntos contidos nessa categoria de documentos além de proporcionar novos atributos aos dados originais os quais expandem o espaço de busca para técnicas de recuperação de informação.








sendo a maior parte baseada em representações tradicionais como \textit{BOW}.







Em geral, nesses trabalhos utiliza-se algoritmos de segmentação 

Entre as técnicas utilizadas 


% Entre as abordagens aplicáveis a este problema estão a segmentação textual e os modelos de extração de tópicos as quais em conjunto são capazes de separar, identificar e agrupar os assuntos contidos nessa categoria de documentos.

e agrupar seus fragmentos com base em similaridades utilizando representações tradicionais como BOW para recuperar os assuntos . 





Entre as técnicas utilizadas nos trabalhos consultados como referência a esse problema estão técnicas para dividir os documentos e agrupar seus fragmentos com base em similaridades utilizando representações tradicionais como BOW para recuperar os assuntos . 

A literatura consultada apresenta trabalhos que 

de agrupamento baseadas na similaridade entre f


Investigou-se também outras abordagem possíveis como 
A abordagem apresentada neste trabalho utiliza 





A literatura consultada contem/apresentaa




















































Eu incluiria também a busca tradicional, na qual compara-se os termos dos segmentos e não dos tópicos. Inclusive fazer uma avaliação comparando as duas abordgaens.





A metodologia apresentada neste projeto 












% De modo geral, os resultados obtidos com a utilização das informações extraídas pela metodologia apresentada nesse trabalho recebe uma base de dados constituída por documentos multi-temáticos não estruturada. Produz uma estrutura de dados interna mais organizada e acrescida de informações latentes extraídas do próprio \textit{corpus} original de forma não supervisionada. 
















% Considera-se como principais contribuições deste trabalho o método apresentado para extração de conhecimento em documentos multi-temáticos e os resultados produzidos durante a execução da proposta desse sistema. O método proposto permite a extração automática de informações em atas de reunião podendo ser utilizado nesse contexto e aproveitado para trazer avanços em aplicações como extração e recuperação de informação em outros domínios com as mesmas características. O \textit{corpus} anotado produzido está disponível para servir como base de dados para outros trabalhos. A ferramenta desenvolvida para esse propósito pode ser utilizada em qualquer corpus a ser anotado com propósito de pesquisa, bem como modificada livremente. 



sob a ótica de profissionais ligados ao contexto das atas de reuniões 
colocou os resultados das combi






Concluindo, este trabalho tem como principal contribuição para a área de sistemas de extração de conhecimento em bases textuais o método proposto para organizar e adicionar informação a coleção de documentos. Tal método se mostrou eficiente e o os dados obtidos possibilitaram bons resultados para exploração do \textit{corpus} composto por atas de reunião.





















% -------------------- Limitações ? -------------------- 





































A proposta inicial desse trabalho contempla a classificação dos segmentos em relação ao tipo de menção ao assunto, como decisão, orientação e irrelevante. Os dados colhidos no experimento mencionado na Seção~\ref{subsec:anotacoes} referentes ao tipo de menção e contexto do assunto serão utilizados para o gerar um classificador para categorizar automaticamente um segmento nas categorias. Para isso, as anotações coletadas podem ser utilizadas na etapa de treinamento dos classificadores.


em categorias como decisão, orientação e irrelevante, utilizado as anotações na etapa de treinamento de classificadores.





utilizando os 













% --> Mas há ganho em relação à metodos tradicionais?





% -------------------- Fechando o Assunto -------------------- 

contexto extraído possibilitou bons resultados para as recomendações geradas. "



extração de informações contextual. 


% utilização de fontes externas para melhorar os métodos de segmentação textual. Recursos como \textit{thesaurus} (dicionários de sinônimos) e \textit{clue words} (palavras-pista) podem adicionar conhecimento ao sistema . 
















...
...
O método proposto permite a extração automática de informações de contexto atas



	"O método, as features e os resultados apresentados podem ser utilizados e trazer avanços em aplicações como extração e recuperação de informação,"
	"A implementação da ferramenta proposta nesta dissertação."







	Considera-se como principais contribuições deste trabalho o método e os resultados apresentados para extração de conhecimento em documentos multi-temáticos os quais podem ser utilizados e trazer avanços em aplicações como extração e recuperação de informação em outros domínios com as mesmas características.
	% e os resultados produzidos. Os quais podem ser utilizados para trazer avanços em aplicações como extração e recuperação de informação em outros domínios com as mesmas características.
	% A metodologia desenvolvida pode ser utilizada para trazer 













% ==================================================



% ==================================================


Neste trabalho foi apresentado o processo proposto para extração de informações em documentos multi-temáticos.


Essas técnicas foram executadas em uma base de dados constituída por atas de reunião



De modo geral, os resultados obtidos com a utilização das informações extraídas pelo método
proposto neste trabalho, foram 



Considera-se contribuições deste trabalho o método proposto para extração de contexto e
...
...
O método proposto permite a extração automática de informações de contexto aptas


 Replicação dos experimentos deste trabalho em novas bases de dados;
 Uso de informações externas ao sistema como clue words etc




" Concluindo, este trabalho tem como principal contribuição para a área de sistemas de recomendações sensı́veis ao contexto o método proposto para extração de informa- ções contextual. Tal método se mostrou eficiente e o contexto extraı́do possibilitou bons resultados para as recomendações geradas. "
Camila


------------------------------

-- gap
-- solução proposta
	" Sendo assim, este trabalho de mestrado visou o estudo e a comparação das duas principais abordagens para extração automática de relações semânticas, ..."
	"O método, as features e os resultados apresentados podem ser utilizados e trazer avanços em aplicações como extração e recuperação de informação,"
trabalhos futuros:
	" Entre as continuações e futuras melhorias possı́veis para este trabalho, pode-se citar a definição de novas features que auxiliem os métodos de AM na classificação de relações. "


------------------------------



" A estrutura funcional permite

"A implementação da ferramenta proposta nesta dissertação."



------------------------------


% -------------------- Retomada do que foi feito -------------------- 

Objetivos propostos e como foram atingidos





% -------------------- Contribuições -------------------- 
Contribuições	
	- é a proposta do sistema 
	- a avaliação dos segmentos etc ... 
	- o corpus... 
	- a ferramenta para anotação ... 
	- tudo que for de contribuição vc ressalta lá 




% -------------------- Limitações ? -------------------- 
	A proposta inicial desse trabalho contempla a classificação dos segmentos em relação ao tipo de menção ao assunto, como decisão, orientação e irrelevante, utilizado 

	os dados colhidos no experimento mencionado na Seção referentes ao tipo de menção e contexto do assunto serão podem ser utilizados para o gerar um classificador para categorizar automaticamente um segmento em categorias como decisão, orientação e irrelevante, utilizado as anotações na etapa de treinamento de classificadores.



% -------------------- Trabalhos Futuros -------------------- 

	- Replicação dos experimentos deste trabalho em novas bases de dados;
	- Uso de informações externas ao sistema como clue words etc
	- Classificação
	- Exploração dos grupos
	- Hierarquia de tópicos


	- Utilização de dados externos
	- Novas bases outros domínios



Definir uma interface gráfica para possibilitar ao usuário
uma melhor análise dos resultados. {UX} testes com usuários em consultas livres, a fim de medir a satisfação dos resultados e usabilidade do sistema.






% -------------------- Fechando o Assunto -------------------- 





















% Coloque o texto aqui.
% \section*{Publicações} 
% \section*{Submissões}


















A avaliação dos segmentadores considerou os algoritmos e seus principais parâmetros para encontrar o modelo que melhor otimiza a tarefa de segmentação do corpus investigado. Comparou-se os resultados obtidos com a segmentação de referência e verificou-se que o algoritmo BayesSeg apresenta resultados melhores em relação às demais técnicas analisadas. Contudo, não há diferença significativa entre os métodos, que,  de maneira geral apresentam resultados satisfatórios.  
Os resultados mostram certa dificuldade devido as características das atas, como estilo de escrita e segmentos relativamente curtos, além da subjetividade intrínseca da tarefa.
Embora os métodos de segmentação textual tenham se mostrado suficientes, melhores resultados podem ser alcançados com o acréscimo das técnicas mencionadas e a construção de uma segmentação de referência mais robusta em termos concordância entre os anotadores que ajudaram a criá-la.


% Os algoritmos de segmentação textual foram avaliados comparando-se seus resultados a segmentação de referência considerando a . Após análise dos resultados, chegou-se a um modelo que 
% Os algoritmos de segmentação textual foram avaliados comparando-se seus resultados a segmentação de referência. Após análise dos resultados, chegou-se a um modelo que 

discretamente 
modestamente
levemente
% Os algoritmos de segmentação textual foram avaliados comparando-se seus resultados com uma segmentação de referência criada a partir das segmentações manuais coletadas dos anotadores. Após análise das avaliações, chegou-se a um modelo que 

A metodologia 
para segmentação das atas e agrupamento dos trechos por assunto.  
dos diversos assuntos contidos em uma coleção de atas de reunião.




% confirmou o BayesSeg
% K-Means é melhor
% Os modelos de extração de tópicos foram avaliados subjetivamente por meio de questionários em que 41 avaliadores responderam à questões referentes à qualidade dos segmentos apresentados como resultado à consultas à base de dados criada com essa metodologia. Os questionários foram formulados para obter a percepção dos avaliadores quanto a similaridade dos segmentos em relação ao mesmo assunto e quanto a representatividade dos descritores extraídos. 



% a estrutura ajuda a compreender entender analisar e visualizar a base de dados









