






Neste trabalho foi apresentado o processo proposto para extração de informações em documentos multi-temáticos.


Essas técnicas foram executadas em uma base de dados constituída por atas de reunião



De modo geral, os resultados obtidos com a utilização das informações extraídas pelo método
proposto neste trabalho, foram 



Considera-se contribuições deste trabalho o método proposto para extração de contexto e
...
...
O método proposto permite a extração automática de informações de contexto aptas


 Replicação dos experimentos deste trabalho em novas bases de dados;
 Uso de informações externas ao sistema como clue words etc




" Concluindo, este trabalho tem como principal contribuição para a área de sistemas de recomendações sensı́veis ao contexto o método proposto para extração de informa- ções contextual. Tal método se mostrou eficiente e o contexto extraı́do possibilitou bons resultados para as recomendações geradas. "
Camila


------------------------------

-- gap
-- solução proposta
	" Sendo assim, este trabalho de mestrado visou o estudo e a comparação das duas principais abordagens para extração automática de relações semânticas, ..."
	"O método, as features e os resultados apresentados podem ser utilizados e trazer avanços em aplicações como extração e recuperação de informação,"
trabalhos futuros:
	" Entre as continuações e futuras melhorias possı́veis para este trabalho, pode-se citar a definição de novas features que auxiliem os métodos de AM na classificação de relações. "




------------------------------



" A estrutura funcional permite

"A implementação da ferramenta proposta nesta dissertação."



------------------------------
Definir uma interface gráfica para possibilitar ao usuário
uma melhor análise dos resultados. {UX} testes com usuários em consultas livres, a fim de medir a satisfação dos resultados e usabilidade do sistema.


% -------------------- Retomada do que foi feito -------------------- 

Objetivos propostos e como foram atingidos





% -------------------- Contribuições -------------------- 
Contribuições	
	- é a proposta do sistema 
	- a avaliação dos segmentos etc ... 
	- o corpus... 
	- a ferramenta para anotação ... 
	- tudo que for de contribuição vc ressalta lá 




% -------------------- Limitações ? -------------------- 




% -------------------- Trabalhos Futuros -------------------- 




% -------------------- Fechando o Assunto -------------------- 


























% Coloque o texto aqui.
% \section*{Publicações} 
% \section*{Submissões}


