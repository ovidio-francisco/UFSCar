

fulano não considera partes do documento. 
não usa segmentação, mas só sentenças.


 % --- Multi-Topic Multi-Document Summarization ---

The method proposed in this paper is
based on spreading activation over documents
syntactically and semantically annotated with
GI)A (Global l)ocument Annotation) tags.
 
The method extracts important documents aald 
important parts therein, and creates a network
consisting of important entities and relations
among them. It also identifies cross-document
co-references to replace expressions with more
concrete ones.



Summarization of such a large event, or multiple 
documents about multiple topics, is the
concern of this paper.


Selection of multiple important topics (not
keywords) tbr nmltiple-topic summarization has
not; yet been really addressed in the previ-
ous literatm:e.

 
Barzilay et al., 1999; Mani and Bloedorn, 1999) deal with multiple docmnents about a single topic, but not about multiple topics 1.  




Um dos primeiros trabalhos da literatura a abordar a multiplicidade de tópicos em um documento foi proposto em~\cite{}. Esse trabalho foca na sumarização de múltiplos documentos sobre múltiplos tópicos. Os autores propuseram um método baseado em \textit{spreading activation} em uma base de documentos anotados semanticamente. O método extrai partes dos documentos consideradas importantes para criar uma rede que os relaciona. Essa abordagem foi capaz de identificar sentenças relacionadas bem como os documentos. Contudo essa abordagem não utilizada métodos de segmentação textual, considerando cada sentença como nós da rede. 

% visa extrair partes dos documentos consideradas importantes para criar uma rede que os relaciona. 


% ================================================================================  







% --- Multi-Topic Multi-Document Summarizer --- 

The present study introduces a new concept of centroid approach and reports new techniques for extracting summary sentences for multi-document.


Second, the system was applied to summarize multi-topic documents.







% ================================================================================  



 % --- Multi-document Topic Segmentation ---

We argue that revealing hidden
relations by jointly segmenting the documents, or, equivalently, 
predicting links between topically related segments in
different documents would help to visualize documents of interest 
and construct friendlier user interfaces. In this paper,
we refer to this problem as multi-document topic segmentation.




Discovery of these inherent relations between content in such
groups of documents could offer a great convenience to users:


In this paper, we argue that this hidden relation 
can be revealed by jointly segmenting documents, or,
equivalently, predicting links between topically related segments 
in different documents. We refer to such a problem
as multi-document topic segmentation.



Topic segmentation of the multiple related documents is
a novel and challenging problem,


We present a non-parametric Bayesian model for unsupervised 
joint segmentation and alignment of multiple documents


We studied the problem of multi-document topic segmentation, 
where the goal is to jointly segment multiple documents 
detecting both aligned and non-aligned segments.




% --> 
O algoritmo MultiSeg, proposto em~\cite{} visa descobrir descobrir ligações entre segmentos semanticamente relacionados. Os autores apresentam um modelo Bayesiano não paramétrico para inferir relação e agrupar segmentos de documentos. Essa abordagem se propõe a ajudar usuário a encontrar segmentos relacionados e detectar informações complementares à pesquisa inicial. Segundo os autores, essas relações ainda podem revelar tendências em fontes de dados.



% relações latentes entre documentos por meio do agrupamento de documentos segmentados. 


























% -> Fazer uma figura |Segmentação Textual| --> |Extração de Tópicos| --> |RI|

% -> falar sobre abordagens sobre os problemas de documentos com múltiplos temas, segmentação, extração de tópicos e recuperação de informação. Em seguida, falar sobre trabalhos que fazem a interseção dessas técnicas.



O modelo LDA foi usado por \cite{XINGWEI} como modelo para descrever documentos a fim de aprimorar método de recuperação de informação. Baseia-se na probabilidade do modelo gerar a cnoslta

















>> Text segmentation is a task of splitting a document into topically coherent segments.  All documents in the set are put into a text segmentation system to get a collection of segments.
>> Segment combination is a task of merging and combining all the segments to form a hierarchical structure of segments (a tree of segments) which reflects the hierarchical structure of information.
>> Title generation is a task of generating a title for each node in the tree of segments. A title is a phrase which reflects the content of segments belonging to the node.


file:///ext4Data/UFSCar/papers/Segmentação/recentes/A Study on Statistical Generation of a Hierarchical Structure of Topic-information for Multi-documents.pdf
~\cite{NGUYEN Viet Cuong}

=== === === === === === === === === === === === === === === === === === === === 



Inicialmente os documentos são modelados como um conjunto de segmentos de acordo com seus tópicos. Em seguida os segmentos são agrupados por meio do algoritmo \textit{Spherical k-Means}\cite{I. S. Dhillon and D. S. Modha,   Y. Zhao and G. Karypis}. Por fim, os conjuntos de segmentos são mapeados para conjuntos de documentos. Cada segmento é substituído por seu documento original, gerando assim, conjuntos sobrepostos de documentos. Os autores ainda propuseram variantes do SKM para 


\textit{Bisecting Spherical k-Means}\cite{Y. Zhao and G. Karypis}.
\textit{Bisecting Spherical k-Means}\cite{I. S. Dhillon and D. S. Modha,   Y. Zhao and G. Karypis}.

--> os autores utilizam segmentos de um determinado documentos para facilitar a atribuição deste a mais de um grupo (onde cada grupo é contém segmentos relevantes a um tópico).
--> 

% --> Na intro
--> A proposta de XX é, a partir de conjuntos de segmentos, derivar conjuntos sobrepostos de documentos, os quais podem pertencer a mais de um grupo.


% --> os documentos originais são decompostos em um conjunto de segmentos.
% --> os conjuntos de documentos são derivados dos conjuntos de segmentos.


e os documentos originais são classificados. Por fim, um classificador foi induzido a partir dos grupos de segmentos.



Nesse contexto, Tagarelli e Karypis, (2013) propuseram um \textit{framework} para documentos com múltiplos tópicos que para induzir um classificador a partir da identificação grupos de segmentos dos documentos originais.


que levam em conta 




tratar de documentos com múltiplos tópicos e um problema 

documentos com múltiplos tópicos que para induzir um classificador a partir da identificação grupos de segmentos dos documentos originais.



\cite{Tagarelli2013} 
" We propose a segment-based document clustering framework, which is designed to induce a classification of documents starting from the identification of cohesive groups of segment-based portions of the original documents. "


" -->"
The final stage in our framework is to map the segment-set clustering solution to a document clustering, in order to provide the user with a likely more useful organization of the input texts. In our study we obtain this document clustering by simply replacing the segment-sets of each cluster with their corresponding original document. Formally, given the set C S = {C 1 , . . . , C h } of clusters over S, the goal is to (d) (d) provide a set C D = {C 1 , . . . , C h } of clusters over (d) D such that C i = {d j ∈ D | S ∈ S j ∈ S and S ∈ C i ∈ C S }, for each i ∈ [1..h].  Although more refined mapping schemes could be devised (e.g. mapping segment-sets with documents on a “majority vote” basis), in this work we chose to pursue the above idea for the sake of its simplicity.



" we developed a novel clustering framework for multi-topic documents that works "

" Generally speaking, multi-topic documents have a multi-faceted communicative intention, thus reflecting different users’ informative needs "


"In other terms, partitioning the set S of segment-sets allows a possibly overlapping clustering of the original documents to be induced."



% -> Segmento é uma porção de texto indivisível.





% No trabalho de \cite{Tagarelli2013} foi proposto um \textit{framework} para induz um classificador de documentos que usa a identificação de grupos de 















