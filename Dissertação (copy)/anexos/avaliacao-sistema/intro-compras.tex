
Essa avaliação se refere a um sistema cujo objetivo é ajudar o usuário a fazer buscas por um assunto específico em uma coleção de atas de reunião. O sistema recebe uma consulta do usuário sobre um assunto e apresenta os trechos onde esse assunto é mencionado. Inicialmente, o sistema analisa todas atas e divide o texto de cada ata em trechos que contêm um assunto principal e relativamente independente. Ou seja, os diversos assuntos tratados em uma ata são separados em trechos com um único assunto. Em seguida, utiliza-se técnicas de inteligência artificial para identificar trechos com assuntos relacionados e agrupá-los. Cada grupo contém trechos de atas diferentes mas com assuntos relacionados. Além disso, o sistema seleciona um conjunto de palavras que indicam o assunto do grupo. Assim, espera-se que o agrupamento de trechos com assuntos similares extraídos de diferentes documentos facilite a navegação e busca por assuntos na coleção de atas.

Essa avaliação se refere a componentes internos do sistema e não em como seria o seu uso. Para isso, foi feita uma consulta ao sistema com os termos \textit{``compra de equipamentos''} e foram extraídos 
%
%resultados parciais 
%
alguns elementos 
para a avaliação dos componentes de interesse.

A seguir, você encontrará três grupos de cinco trechos de tamanhos variados. Para cada um dos grupos, há um conjunto de 4 questões simples, que você deve responder de forma a refletir a sua percepção quanto à qualidade do grupo e respectivos trechos. A avaliação deve ser feita na ordem em que os grupos e as questões foram apresentados, não importando a numeração de consulta e técnica fornecida (são apenas para controle interno).

\vspace{2em}
Antes de iniciar a avaliação, identifique aqui a sua afinidade com atas de reuniões (por exemplo, presidente de conselho, membro de conselho, secretário, coordenador de curso):




% --------------------


% --------------------

% Trata-se de um sistema para consultas em atas de reunião. O objetivo é ajudar o usuário a fazer buscas por um assunto específico em uma coleção de atas de reunião. O Sistema deve receber uma consulta do usuário sobre um assunto e apresentar os trechos onde esse assunto é mencionado. 

% Escolheu-se as atas por tratar-se de documentos que não possuem um assunto principal, mas contém diversos assuntos registrados em um mesmo documento. Essa multiplicidade de assuntos das atas constitui um desafio para sistemas de consulta e ao mesmo tempo motivou esse trabalho de mestrado.

% Para isso, visto a multiplicidade de assuntos uma mesma ata, o sistema inicialmente  analisa a coleção de atas e divide o texto de cada ata em trechos que contêm um assunto principal e relativamente independente. 
% Ou seja, os diversos assuntos tratados em uma ata são separados em trechos com um único assunto.
% Em seguida, utiliza-se técnicas de inteligência artificial para identificar trechos com assuntos relacionados e agrupá-los. Cada grupo contém trechos de atas diferentes mas com assuntos relacionados. Além disso, o sistema seleciona um conjunto de palavras descritoras que indicam o tópico do grupo. Nesse sistema, um grupo é formado por um conjunto de trechos e por um conjunto de palavras que o descreve.

% Assim, espera-se que o agrupamento de trechos com assuntos similares extraídos de diferentes documentos facilite a navegação e busca por assuntos na coleção de atas.

% Ao iniciar, o sistema apresenta os grupos anteriormente identificados os quais são representados por suas palavras descritoras. Ao selecionar um grupo, seus trechos são exibidos ao usuário para que possa verificar o que foi registrado sobre o assunto de cada grupo. 
% %
% O sistema também permite que o usuário faça consultas à coleção de atas por meio de um campo de pesquisa por palavras-chave. Nesse caso, o sistema  analisa as similaridades entre a consulta do usuário e os trechos extraídos das atas, bem como os grupos aos quais pertence. Então, apresenta os trechos ordenados pela relevância com a consulta do usuário. Para cada trecho é apresentado além do texto, um link para o arquivo original do qual foi extraído. Ao clicar sobre o texto de um trecho, seu grupo é destacado para que o usuário possa explorar outros trechos similares.


% Abaixo é mostrado a tela principal do sistema. A esquerda são apresentados os grupos com seus descritores e a direita os trechos atribuídos ao grupo selecionado. Acima está o campo para pesquisa.












% --> por que trechos? 

%Um trecho é representado por  texto  

%Com isso, espera-se que o agrupamento trechos de diferentes documentos, mas com assuntos similares, facilite a navegação e busca por assuntos na coleção de atas.


%Após realizar a busca, o sistema apresenta os trechos agrupados por tópicos, em que documentos com assuntos similares pertencem ao mesmo grupo. Ao selecionar um grupo, os trechos de atas são exibidos ao usuário.

%O sistema atribui um conjunto de palavras para cada grupo as quais servem para descrever o tópico do grupo.

%Cada grupo possui um conjunto de palavras 

%O sistema deve apresentar ao usuário os trechos de atas que melhor satisfazem a busca. 
 


%e documentos com assuntos diferentes ficam em gru
