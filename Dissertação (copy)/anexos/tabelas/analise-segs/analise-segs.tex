\documentclass{article} 
\usepackage{geometry} %[landscape]
\usepackage{longtable} 
\usepackage[table]{xcolor}\geometry{a4paper, left=10mm, right=10mm, top=10mm} 
\begin{document} 

Tabelas para Análise de de Parâmetros para os algoritmos de Segmentação Textual


Nesse anexo podem ser observadas tabelas com os valores de \textit{Window Diff} $P_k$, Acurácia, e F$^1$ com as variações dos principais parâmetros dos segmentadores \textit{TextTiling}, \textit{C99}, \textit{MinCutSeg}, \textit{BayesSeg}, \textit{TextSeg} e \textit{PseudoSeg}.

Todos os gráficos apresentados foram analisados para escolha e configuração do algoritmo de Segmentação Textual utilizado na avaliação experimental apresentada no Capítulo~\ref{cap:avaliacao-extratores}. 
Nas tabelas, cada linha apresenta a variação dos parâmetros e a média dos valores obtidos por meio da segmentação de referência apresentada na Seçao~\ref{sec:segreferencia}. Vale lembrar que todos os valores de \textit{WindowDiff} e $P_k$, representam a dissimilaridade entre entre uma segmentação automática e uma referência. 


\tiny
\center TextTiling
\begin{longtable}[c]{|c|c|c|c|c|c|c|c|c|c|c|c|} 
\hline 
 Step & Win Size & $WinDiff$ & $\sigma$$WinDiff$ & $P_k$ & $\sigma$$P_k$ & Acurácia & $\sigma$Acurácia & $F^1$ & $\sigma$$F^1$ & \#Segs & $\sigma$\#Segs\\ \hline 
 20 & 30 & 0.461 & 0.145 & 0.444 & 0.153 & 0.581 & 0.141 & \cellcolor{gray!20} \textbf{0.411} & \cellcolor{gray!20} \textbf{0.161} & 8.833 & 3.387  \\ \hline 
  20 & 35 & 0.462 & 0.111 & 0.443 & 0.119 & 0.582 & 0.116 & 0.401 & 0.168 & 8.750 & 3.767  \\ \hline 
  20 & 40 & 0.485 & 0.117 & 0.466 & 0.126 & 0.562 & 0.124 & 0.378 & 0.113 & 8.250 & 2.947  \\ \hline 
  20 & 45 & 0.480 & 0.101 & 0.458 & 0.089 & 0.572 & 0.081 & 0.369 & 0.149 & 8.250 & 3.031  \\ \hline 
  20 & 50 & 0.523 & 0.115 & 0.503 & 0.120 & 0.528 & 0.118 & 0.327 & 0.147 & 8.417 & 2.842  \\ \hline 
  20 & 55 & 0.491 & 0.144 & 0.474 & 0.149 & 0.549 & 0.139 & 0.331 & 0.195 & 8.250 & 3.515  \\ \hline 
  30 & 30 & 0.509 & 0.103 & 0.488 & 0.113 & 0.536 & 0.106 & 0.286 & 0.122 & 6.917 & 2.532  \\ \hline 
  30 & 35 & 0.500 & 0.094 & 0.479 & 0.101 & 0.551 & 0.098 & 0.318 & 0.102 & 7.167 & 2.764  \\ \hline 
  30 & 40 & 0.468 & 0.106 & 0.451 & 0.112 & 0.576 & 0.104 & 0.348 & 0.085 & 6.750 & 2.241  \\ \hline 
  30 & 45 & \cellcolor{gray!20} \textbf{0.450} & \cellcolor{gray!20} \textbf{0.103} & \cellcolor{gray!20} \textbf{0.435} & \cellcolor{gray!20} \textbf{0.109} & \cellcolor{gray!20} \textbf{0.596} & \cellcolor{gray!20} \textbf{0.110} & 0.373 & 0.087 & 6.417 & 2.465  \\ \hline 
  30 & 50 & 0.493 & 0.152 & 0.478 & 0.171 & 0.543 & 0.162 & 0.307 & 0.131 & 6.417 & 2.326  \\ \hline 
  30 & 55 & 0.481 & 0.135 & 0.463 & 0.154 & 0.558 & 0.137 & 0.346 & 0.086 & 7.083 & 2.361  \\ \hline 
  40 & 30 & 0.475 & 0.125 & 0.460 & 0.137 & 0.566 & 0.126 & 0.306 & 0.104 & 5.833 & 2.034  \\ \hline 
  40 & 35 & 0.501 & 0.125 & 0.482 & 0.138 & 0.542 & 0.127 & 0.268 & 0.104 & 6.083 & 2.629  \\ \hline 
  40 & 40 & 0.499 & 0.151 & 0.478 & 0.163 & 0.548 & 0.149 & 0.293 & 0.102 & 6.083 & 2.465  \\ \hline 
  40 & 45 & 0.488 & 0.134 & 0.471 & 0.150 & 0.551 & 0.137 & 0.275 & 0.098 & 5.500 & 1.936  \\ \hline 
  40 & 50 & 0.495 & 0.104 & 0.474 & 0.113 & 0.552 & 0.110 & 0.280 & 0.125 & 5.833 & 2.154  \\ \hline 
  40 & 55 & 0.476 & 0.084 & 0.453 & 0.103 & 0.567 & 0.093 & 0.310 & 0.072 & 6.083 & 2.100  \\ \hline 
  50 & 30 & 0.492 & 0.138 & 0.473 & 0.150 & 0.557 & 0.149 & 0.274 & 0.120 & 5.167 & 2.075  \\ \hline 
  50 & 35 & 0.504 & 0.138 & 0.484 & 0.147 & 0.549 & 0.143 & 0.268 & 0.097 & 5.583 & 2.985  \\ \hline 
  50 & 40 & 0.501 & 0.102 & 0.481 & 0.115 & 0.556 & 0.122 & 0.278 & 0.070 & 5.417 & 2.139  \\ \hline 
  50 & 45 & 0.508 & 0.092 & 0.484 & 0.107 & 0.549 & 0.111 & 0.264 & 0.089 & 5.500 & 1.803  \\ \hline 
  50 & 50 & 0.513 & 0.162 & 0.491 & 0.175 & 0.536 & 0.162 & 0.253 & 0.149 & 5.417 & 2.253  \\ \hline 
  50 & 55 & 0.509 & 0.143 & 0.487 & 0.156 & 0.543 & 0.150 & 0.276 & 0.130 & 5.833 & 2.511  \\ \hline 
  60 & 30 & 0.481 & 0.105 & 0.462 & 0.124 & 0.564 & 0.121 & 0.267 & 0.082 & 4.917 & 2.019  \\ \hline 
  60 & 35 & 0.503 & 0.120 & 0.483 & 0.136 & 0.549 & 0.139 & 0.250 & 0.118 & 5.083 & 1.935  \\ \hline 
  60 & 40 & 0.497 & 0.104 & 0.481 & 0.119 & 0.554 & 0.127 & 0.242 & 0.124 & 4.750 & 1.738  \\ \hline 
  60 & 45 & 0.465 & 0.108 & 0.448 & 0.127 & 0.577 & 0.121 & 0.271 & 0.134 & 4.500 & 1.658  \\ \hline 
  60 & 50 & 0.478 & 0.116 & 0.459 & 0.129 & 0.569 & 0.128 & 0.250 & 0.129 & 4.333 & 1.434  \\ \hline 
  60 & 55 & 0.474 & 0.101 & 0.457 & 0.116 & 0.568 & 0.111 & 0.269 & 0.121 & 5.000 & 1.871  \\ \hline 
 \caption{Valores das medidas de desempenho para análise do algoritmo \textit{TextTiling}, utilizando o texto pré-processado.}
 \end{longtable} 


 \newpage
\center C99
\begin{longtable}[c]{|c|c|c|c|c|c|c|c|c|c|c|c|c|} 
\hline 
 Seg Rate & Raking Size & Weitght & $WinDiff$ & $\sigma$$WinDiff$ & $P_k$ & $\sigma$$P_k$ & Acurácia & $\sigma$Acurácia & $F^1$ & $\sigma$$F^1$ & \#Segs & $\sigma$\#Segs\\ \hline 
 0.200 & 3 & true & 0.463 & 0.130 & 0.445 & 0.140 & 0.581 & 0.131 & 0.339 & 0.091 & 6.083 & 2.660  \\ \hline 
  0.300 & 3 & true & \cellcolor{gray!20} \textbf{0.434} & \cellcolor{gray!20} \textbf{0.089} & \cellcolor{gray!20} \textbf{0.407} & \cellcolor{gray!20} \textbf{0.101} & 0.607 & 0.084 & 0.457 & 0.070 & 9.250 & 3.961  \\ \hline 
  0.400 & 3 & true & 0.452 & 0.114 & 0.422 & 0.092 & 0.604 & 0.087 & 0.515 & 0.091 & 12.083 & 5.123  \\ \hline 
  0.500 & 3 & true & 0.499 & 0.162 & 0.458 & 0.098 & 0.577 & 0.085 & 0.539 & 0.112 & 15.500 & 6.397  \\ \hline 
  0.600 & 3 & true & 0.487 & 0.194 & 0.440 & 0.105 & 0.592 & 0.084 & 0.591 & 0.120 & 18.417 & 7.794  \\ \hline 
  0.700 & 3 & true & 0.485 & 0.225 & 0.431 & 0.130 & 0.602 & 0.111 & \cellcolor{gray!20} \textbf{0.633} & \cellcolor{gray!20} \textbf{0.134} & 21.417 & 8.949  \\ \hline 
  0.200 & 5 & true & 0.454 & 0.130 & 0.437 & 0.143 & 0.583 & 0.125 & 0.338 & 0.092 & 6.083 & 2.660  \\ \hline 
  0.300 & 5 & true & 0.454 & 0.121 & 0.434 & 0.116 & 0.595 & 0.111 & 0.446 & 0.093 & 9.250 & 3.961  \\ \hline 
  0.400 & 5 & true & 0.475 & 0.119 & 0.443 & 0.087 & 0.590 & 0.080 & 0.497 & 0.082 & 12.083 & 5.123  \\ \hline 
  0.500 & 5 & true & 0.460 & 0.147 & 0.421 & 0.091 & \cellcolor{gray!20} \textbf{0.609} & \cellcolor{gray!20} \textbf{0.079} & 0.571 & 0.107 & 15.500 & 6.397  \\ \hline 
  0.600 & 5 & true & 0.491 & 0.186 & 0.442 & 0.098 & 0.591 & 0.081 & 0.588 & 0.121 & 18.417 & 7.794  \\ \hline 
  0.700 & 5 & true & 0.525 & 0.251 & 0.449 & 0.106 & 0.576 & 0.094 & 0.609 & 0.132 & 21.417 & 8.949  \\ \hline 
  0.200 & 7 & true & 0.491 & 0.121 & 0.474 & 0.133 & 0.555 & 0.129 & 0.293 & 0.099 & 6.083 & 2.660  \\ \hline 
  0.300 & 7 & true & 0.486 & 0.097 & 0.469 & 0.097 & 0.565 & 0.098 & 0.395 & 0.117 & 9.250 & 3.961  \\ \hline 
  0.400 & 7 & true & 0.502 & 0.119 & 0.472 & 0.086 & 0.561 & 0.082 & 0.453 & 0.133 & 12.083 & 5.123  \\ \hline 
  0.500 & 7 & true & 0.460 & 0.143 & 0.421 & 0.085 & 0.604 & 0.078 & 0.561 & 0.125 & 15.500 & 6.397  \\ \hline 
  0.600 & 7 & true & 0.486 & 0.198 & 0.433 & 0.113 & 0.591 & 0.104 & 0.585 & 0.143 & 18.417 & 7.794  \\ \hline 
  0.700 & 7 & true & 0.547 & 0.248 & 0.470 & 0.113 & 0.551 & 0.108 & 0.586 & 0.141 & 21.417 & 8.949  \\ \hline 
  0.200 & 3 & false & 0.448 & 0.128 & 0.427 & 0.145 & 0.596 & 0.129 & 0.362 & 0.093 & 6.083 & 2.660  \\ \hline 
  0.300 & 3 & false & 0.454 & 0.125 & 0.426 & 0.127 & 0.594 & 0.111 & 0.445 & 0.098 & 9.250 & 3.961  \\ \hline 
  0.400 & 3 & false & 0.490 & 0.116 & 0.455 & 0.098 & 0.568 & 0.089 & 0.469 & 0.100 & 12.083 & 5.123  \\ \hline 
  0.500 & 3 & false & 0.529 & 0.145 & 0.481 & 0.091 & 0.543 & 0.083 & 0.503 & 0.104 & 15.500 & 6.397  \\ \hline 
  0.600 & 3 & false & 0.554 & 0.167 & 0.499 & 0.095 & 0.528 & 0.084 & 0.535 & 0.094 & 18.417 & 7.794  \\ \hline 
  0.700 & 3 & false & 0.565 & 0.204 & 0.496 & 0.075 & 0.526 & 0.070 & 0.570 & 0.103 & 21.417 & 8.949  \\ \hline 
  0.200 & 5 & false & 0.498 & 0.159 & 0.479 & 0.170 & 0.545 & 0.151 & 0.277 & 0.128 & 6.083 & 2.660  \\ \hline 
  0.300 & 5 & false & 0.505 & 0.136 & 0.482 & 0.139 & 0.540 & 0.123 & 0.369 & 0.110 & 9.250 & 3.961  \\ \hline 
  0.400 & 5 & false & 0.536 & 0.130 & 0.504 & 0.106 & 0.520 & 0.096 & 0.407 & 0.118 & 12.083 & 5.123  \\ \hline 
  0.500 & 5 & false & 0.540 & 0.161 & 0.490 & 0.091 & 0.529 & 0.082 & 0.485 & 0.121 & 15.500 & 6.397  \\ \hline 
  0.600 & 5 & false & 0.529 & 0.187 & 0.469 & 0.086 & 0.545 & 0.087 & 0.543 & 0.135 & 18.417 & 7.794  \\ \hline 
  0.700 & 5 & false & 0.542 & 0.245 & 0.464 & 0.101 & 0.549 & 0.108 & 0.584 & 0.147 & 21.417 & 8.949  \\ \hline 
  0.200 & 7 & false & 0.512 & 0.099 & 0.495 & 0.107 & 0.534 & 0.104 & 0.250 & 0.074 & 6.083 & 2.660  \\ \hline 
  0.300 & 7 & false & 0.527 & 0.093 & 0.506 & 0.095 & 0.522 & 0.090 & 0.336 & 0.090 & 9.250 & 3.961  \\ \hline 
  0.400 & 7 & false & 0.530 & 0.099 & 0.494 & 0.043 & 0.535 & 0.038 & 0.420 & 0.095 & 12.083 & 5.123  \\ \hline 
  0.500 & 7 & false & 0.503 & 0.164 & 0.454 & 0.076 & 0.571 & 0.068 & 0.523 & 0.132 & 15.500 & 6.397  \\ \hline 
  0.600 & 7 & false & 0.511 & 0.178 & 0.453 & 0.070 & 0.565 & 0.070 & 0.562 & 0.124 & 18.417 & 7.794  \\ \hline 
  0.700 & 7 & false & 0.559 & 0.239 & 0.476 & 0.087 & 0.535 & 0.096 & 0.572 & 0.138 & 21.417 & 8.949  \\ \hline 
 \caption{Valores das medidas de desempenho para análise do algoritmo \textit{C99}, utilizando o texto pré-processado.}
 \end{longtable} 



 \newpage
 \center MinCutSeg
\begin{longtable}[c]{|c|c|c|c|c|c|c|c|c|c|c|c|} 
\hline 
 Seg Rate & LenCutoff & $WinDiff$ & $\sigma$$WinDiff$ & $P_k$ & $\sigma$$P_k$ & Acurácia & $\sigma$Acurácia & $F^1$ & $\sigma$$F^1$ & \#Segs & $\sigma$\#Segs\\ \hline 
 0.200 & 5 & 0.523 & 0.127 & 0.499 & 0.136 & 0.530 & 0.130 & 0.241 & 0.087 & 5.833 & 2.609  \\ \hline 
  0.200 & 7 & 0.516 & 0.121 & 0.490 & 0.132 & 0.544 & 0.131 & 0.263 & 0.094 & 5.833 & 2.609  \\ \hline 
  0.200 & 9 & 0.516 & 0.107 & 0.490 & 0.121 & 0.545 & 0.127 & 0.268 & 0.091 & 5.833 & 2.609  \\ \hline 
  0.200 & 11 & 0.493 & 0.114 & 0.467 & 0.132 & 0.561 & 0.128 & 0.296 & 0.091 & 5.833 & 2.609  \\ \hline 
  0.200 & 13 & 0.491 & 0.111 & 0.464 & 0.124 & 0.564 & 0.119 & 0.296 & 0.079 & 5.833 & 2.609  \\ \hline 
  0.200 & 15 & 0.490 & 0.117 & 0.458 & 0.140 & 0.568 & 0.132 & 0.311 & 0.100 & 5.833 & 2.609  \\ \hline 
  0.300 & 5 & 0.478 & 0.096 & 0.450 & 0.123 & 0.575 & 0.121 & 0.410 & 0.091 & 8.667 & 3.771  \\ \hline 
  0.300 & 7 & 0.486 & 0.093 & 0.449 & 0.112 & 0.574 & 0.104 & 0.401 & 0.073 & 8.667 & 3.771  \\ \hline 
  0.300 & 9 & 0.484 & 0.104 & 0.445 & 0.116 & 0.579 & 0.112 & 0.409 & 0.108 & 8.667 & 3.771  \\ \hline 
  0.300 & 11 & 0.474 & 0.090 & 0.439 & 0.119 & 0.581 & 0.109 & 0.412 & 0.095 & 8.667 & 3.771  \\ \hline 
  0.300 & 13 & 0.457 & 0.095 & 0.427 & 0.119 & 0.594 & 0.112 & 0.433 & 0.099 & 8.667 & 3.771  \\ \hline 
  0.300 & 15 & 0.483 & 0.108 & 0.448 & 0.112 & 0.575 & 0.106 & 0.402 & 0.107 & 8.667 & 3.771  \\ \hline 
  0.400 & 5 & 0.484 & 0.077 & 0.447 & 0.120 & 0.571 & 0.108 & 0.477 & 0.096 & 11.917 & 5.251  \\ \hline 
  0.400 & 7 & 0.477 & 0.084 & 0.430 & 0.082 & 0.589 & 0.079 & 0.491 & 0.082 & 11.917 & 5.251  \\ \hline 
  0.400 & 9 & \cellcolor{gray!20} \textbf{0.444} & \cellcolor{gray!20} \textbf{0.084} & 0.408 & 0.093 & \cellcolor{gray!20} \textbf{0.614} & \cellcolor{gray!20} \textbf{0.093} & 0.526 & 0.084 & 11.917 & 5.251  \\ \hline 
  0.400 & 11 & 0.450 & 0.086 & 0.412 & 0.117 & 0.601 & 0.102 & 0.512 & 0.087 & 11.917 & 5.251  \\ \hline 
  0.400 & 13 & 0.462 & 0.089 & 0.422 & 0.131 & 0.589 & 0.112 & 0.499 & 0.103 & 11.917 & 5.251  \\ \hline 
  0.400 & 15 & 0.471 & 0.085 & 0.432 & 0.139 & 0.580 & 0.119 & 0.490 & 0.107 & 11.917 & 5.251  \\ \hline 
  0.500 & 5 & 0.493 & 0.112 & 0.435 & 0.098 & 0.578 & 0.088 & 0.535 & 0.091 & 15.000 & 6.519  \\ \hline 
  0.500 & 7 & 0.481 & 0.106 & 0.428 & 0.099 & 0.587 & 0.093 & 0.546 & 0.093 & 15.000 & 6.519  \\ \hline 
  0.500 & 9 & 0.467 & 0.107 & 0.412 & 0.098 & 0.600 & 0.090 & 0.560 & 0.094 & 15.000 & 6.519  \\ \hline 
  0.500 & 11 & 0.459 & 0.100 & \cellcolor{gray!20} \textbf{0.407} & \cellcolor{gray!20} \textbf{0.098} & 0.603 & 0.087 & 0.563 & 0.088 & 15.000 & 6.519  \\ \hline 
  0.500 & 13 & 0.500 & 0.112 & 0.444 & 0.096 & 0.572 & 0.088 & 0.528 & 0.092 & 15.000 & 6.519  \\ \hline 
  0.500 & 15 & 0.494 & 0.111 & 0.435 & 0.100 & 0.578 & 0.090 & 0.534 & 0.096 & 15.000 & 6.519  \\ \hline 
  0.600 & 5 & 0.520 & 0.140 & 0.449 & 0.077 & 0.564 & 0.073 & 0.559 & 0.096 & 17.917 & 7.719  \\ \hline 
  0.600 & 7 & 0.497 & 0.161 & 0.425 & 0.117 & 0.584 & 0.108 & 0.583 & 0.113 & 17.917 & 7.719  \\ \hline 
  0.600 & 9 & 0.501 & 0.173 & 0.428 & 0.110 & 0.579 & 0.103 & 0.577 & 0.114 & 17.917 & 7.719  \\ \hline 
  0.600 & 11 & 0.511 & 0.173 & 0.438 & 0.116 & 0.570 & 0.109 & 0.567 & 0.125 & 17.917 & 7.719  \\ \hline 
  0.600 & 13 & 0.502 & 0.168 & 0.428 & 0.118 & 0.579 & 0.110 & 0.576 & 0.124 & 17.917 & 7.719  \\ \hline 
  0.600 & 15 & 0.500 & 0.166 & 0.427 & 0.120 & 0.580 & 0.111 & 0.577 & 0.125 & 17.917 & 7.719  \\ \hline 
  0.700 & 5 & 0.528 & 0.219 & 0.438 & 0.122 & 0.567 & 0.120 & \cellcolor{gray!20} \textbf{0.599} & \cellcolor{gray!20} \textbf{0.135} & 21.000 & 9.211  \\ \hline 
  0.700 & 7 & 0.540 & 0.235 & 0.446 & 0.107 & 0.559 & 0.109 & 0.592 & 0.124 & 21.000 & 9.211  \\ \hline 
  0.700 & 9 & 0.567 & 0.218 & 0.473 & 0.094 & 0.535 & 0.093 & 0.570 & 0.117 & 21.000 & 9.211  \\ \hline 
  0.700 & 11 & 0.561 & 0.192 & 0.469 & 0.081 & 0.537 & 0.076 & 0.575 & 0.095 & 21.000 & 9.211  \\ \hline 
  0.700 & 13 & 0.564 & 0.192 & 0.472 & 0.083 & 0.534 & 0.078 & 0.572 & 0.097 & 21.000 & 9.211  \\ \hline 
  0.700 & 15 & 0.551 & 0.197 & 0.459 & 0.080 & 0.546 & 0.077 & 0.583 & 0.097 & 21.000 & 9.211  \\ \hline 
 \caption{Valores das medidas de desempenho para análise do algoritmo \textit{MinCutSeg}, utilizando o texto pré-processado.}
 \end{longtable} 



 \newpage
 \center BayesSeg
\begin{longtable}[c]{|c|c|c|c|c|c|c|c|c|c|c|c|c|c|} 
\hline 
 \#SegsKnown & Seg Rate & Prior & Dispertion & $WinDiff$ & $\sigma$$WinDiff$ & $P_k$ & $\sigma$$P_k$ & Acurácia & $\sigma$Acurácia & $F^1$ & $\sigma$$F^1$ & \#Segs & $\sigma$\#Segs\\ \hline 
 false & Auto & 0.0800 & 0.1000 & 0.395 & 0.084 & 0.377 & 0.105 & 0.640 & 0.092 & 0.528 & 0.087 & 9.667 & 1.748  \\ \hline 
  false & Auto & 0.0900 & 0.1000 & 0.402 & 0.078 & 0.383 & 0.096 & 0.636 & 0.088 & 0.515 & 0.077 & 9.333 & 1.650  \\ \hline 
  false & Auto & 0.1000 & 0.1000 & 0.395 & 0.074 & 0.376 & 0.092 & 0.642 & 0.083 & 0.518 & 0.077 & 9.167 & 1.572  \\ \hline 
  false & Auto & 0.1100 & 0.1000 & 0.402 & 0.081 & 0.383 & 0.099 & 0.636 & 0.090 & 0.508 & 0.075 & 9.000 & 1.414  \\ \hline 
  false & Auto & 0.0800 & 0.3000 & \cellcolor{gray!20} \textbf{0.380} & \cellcolor{gray!20} \textbf{0.086} & \cellcolor{gray!20} \textbf{0.361} & \cellcolor{gray!20} \textbf{0.104} & \cellcolor{gray!20} \textbf{0.655} & \cellcolor{gray!20} \textbf{0.091} & 0.551 & 0.100 & 10.000 & 1.780  \\ \hline 
  false & Auto & 0.0900 & 0.3000 & 0.393 & 0.081 & 0.374 & 0.097 & 0.645 & 0.088 & 0.529 & 0.092 & 9.583 & 1.754  \\ \hline 
  false & Auto & 0.1000 & 0.3000 & 0.393 & 0.071 & 0.374 & 0.089 & 0.644 & 0.081 & 0.520 & 0.083 & 9.167 & 1.404  \\ \hline 
  false & Auto & 0.1100 & 0.3000 & 0.390 & 0.070 & 0.371 & 0.088 & 0.647 & 0.079 & 0.522 & 0.084 & 9.083 & 1.382  \\ \hline 
  false & Auto & 0.0800 & 0.5000 & \cellcolor{gray!20} \textbf{0.380} & \cellcolor{gray!20} \textbf{0.086} & \cellcolor{gray!20} \textbf{0.361} & \cellcolor{gray!20} \textbf{0.104} & \cellcolor{gray!20} \textbf{0.655} & \cellcolor{gray!20} \textbf{0.091} & 0.551 & 0.100 & 10.000 & 1.780  \\ \hline 
  false & Auto & 0.0900 & 0.5000 & 0.398 & 0.082 & 0.379 & 0.099 & 0.640 & 0.090 & 0.523 & 0.095 & 9.583 & 1.656  \\ \hline 
  false & Auto & 0.1000 & 0.5000 & 0.397 & 0.074 & 0.378 & 0.092 & 0.641 & 0.084 & 0.518 & 0.084 & 9.250 & 1.479  \\ \hline 
  false & Auto & 0.1100 & 0.5000 & 0.388 & 0.072 & 0.370 & 0.089 & 0.649 & 0.080 & 0.523 & 0.083 & 9.000 & 1.225  \\ \hline 
  false & Auto & 0.0800 & 0.7000 & 0.385 & 0.073 & 0.366 & 0.089 & 0.652 & 0.081 & 0.546 & 0.095 & 10.000 & 1.683  \\ \hline 
  false & Auto & 0.0900 & 0.7000 & 0.393 & 0.077 & 0.374 & 0.094 & 0.645 & 0.086 & 0.528 & 0.101 & 9.667 & 1.650  \\ \hline 
  false & Auto & 0.1000 & 0.7000 & 0.395 & 0.076 & 0.376 & 0.094 & 0.642 & 0.085 & 0.519 & 0.083 & 9.167 & 1.344  \\ \hline 
  false & Auto & 0.1100 & 0.7000 & 0.388 & 0.072 & 0.370 & 0.089 & 0.649 & 0.080 & 0.523 & 0.083 & 9.000 & 1.225  \\ \hline 
  true & 0.300 & 0.0800 & 0.1000 & 0.428 & 0.150 & 0.398 & 0.171 & 0.617 & 0.154 & 0.491 & 0.122 & 9.250 & 3.961  \\ \hline 
  true & 0.300 & 0.0900 & 0.1000 & 0.428 & 0.150 & 0.398 & 0.171 & 0.617 & 0.154 & 0.491 & 0.122 & 9.250 & 3.961  \\ \hline 
  true & 0.300 & 0.1000 & 0.1000 & 0.428 & 0.150 & 0.399 & 0.170 & 0.614 & 0.151 & 0.485 & 0.121 & 9.250 & 3.961  \\ \hline 
  true & 0.300 & 0.1100 & 0.1000 & 0.427 & 0.150 & 0.398 & 0.174 & 0.615 & 0.155 & 0.487 & 0.129 & 9.250 & 3.961  \\ \hline 
  true & 0.300 & 0.0800 & 0.3000 & 0.428 & 0.150 & 0.398 & 0.171 & 0.617 & 0.154 & 0.491 & 0.122 & 9.250 & 3.961  \\ \hline 
  true & 0.300 & 0.0900 & 0.3000 & 0.428 & 0.150 & 0.399 & 0.170 & 0.614 & 0.151 & 0.485 & 0.121 & 9.250 & 3.961  \\ \hline 
  true & 0.300 & 0.1000 & 0.3000 & 0.428 & 0.150 & 0.399 & 0.170 & 0.614 & 0.151 & 0.485 & 0.121 & 9.250 & 3.961  \\ \hline 
  true & 0.300 & 0.1100 & 0.3000 & 0.424 & 0.152 & 0.395 & 0.176 & 0.618 & 0.156 & 0.492 & 0.130 & 9.250 & 3.961  \\ \hline 
  true & 0.300 & 0.0800 & 0.5000 & 0.428 & 0.150 & 0.399 & 0.170 & 0.614 & 0.151 & 0.485 & 0.121 & 9.250 & 3.961  \\ \hline 
  true & 0.300 & 0.0900 & 0.5000 & 0.428 & 0.150 & 0.399 & 0.170 & 0.614 & 0.151 & 0.485 & 0.121 & 9.250 & 3.961  \\ \hline 
  true & 0.300 & 0.1000 & 0.5000 & 0.428 & 0.150 & 0.399 & 0.170 & 0.614 & 0.151 & 0.485 & 0.121 & 9.250 & 3.961  \\ \hline 
  true & 0.300 & 0.1100 & 0.5000 & 0.428 & 0.150 & 0.399 & 0.170 & 0.614 & 0.151 & 0.485 & 0.121 & 9.250 & 3.961  \\ \hline 
  true & 0.300 & 0.0800 & 0.7000 & 0.428 & 0.150 & 0.399 & 0.170 & 0.614 & 0.151 & 0.485 & 0.121 & 9.250 & 3.961  \\ \hline 
  true & 0.300 & 0.0900 & 0.7000 & 0.428 & 0.150 & 0.399 & 0.170 & 0.614 & 0.151 & 0.485 & 0.121 & 9.250 & 3.961  \\ \hline 
  true & 0.300 & 0.1000 & 0.7000 & 0.428 & 0.150 & 0.399 & 0.170 & 0.614 & 0.151 & 0.485 & 0.121 & 9.250 & 3.961  \\ \hline 
  true & 0.300 & 0.1100 & 0.7000 & 0.428 & 0.150 & 0.399 & 0.170 & 0.614 & 0.151 & 0.485 & 0.121 & 9.250 & 3.961  \\ \hline 
  true & 0.600 & 0.0800 & 0.1000 & 0.480 & 0.133 & 0.416 & 0.056 & 0.598 & 0.052 & 0.601 & 0.079 & 18.417 & 7.794  \\ \hline 
  true & 0.600 & 0.0900 & 0.1000 & 0.473 & 0.137 & 0.410 & 0.057 & 0.605 & 0.054 & 0.607 & 0.083 & 18.417 & 7.794  \\ \hline 
  true & 0.600 & 0.1000 & 0.1000 & 0.467 & 0.139 & 0.404 & 0.056 & 0.611 & 0.052 & 0.613 & 0.079 & 18.417 & 7.794  \\ \hline 
  true & 0.600 & 0.1100 & 0.1000 & 0.462 & 0.141 & 0.399 & 0.055 & 0.615 & 0.051 & \cellcolor{gray!20} \textbf{0.619} & \cellcolor{gray!20} \textbf{0.074} & 18.417 & 7.794  \\ \hline 
  true & 0.600 & 0.0800 & 0.3000 & 0.480 & 0.133 & 0.416 & 0.056 & 0.598 & 0.052 & 0.601 & 0.079 & 18.417 & 7.794  \\ \hline 
  true & 0.600 & 0.0900 & 0.3000 & 0.473 & 0.137 & 0.410 & 0.057 & 0.605 & 0.054 & 0.607 & 0.083 & 18.417 & 7.794  \\ \hline 
  true & 0.600 & 0.1000 & 0.3000 & 0.467 & 0.139 & 0.404 & 0.056 & 0.611 & 0.052 & 0.613 & 0.079 & 18.417 & 7.794  \\ \hline 
  true & 0.600 & 0.1100 & 0.3000 & 0.462 & 0.141 & 0.399 & 0.055 & 0.615 & 0.051 & \cellcolor{gray!20} \textbf{0.619} & \cellcolor{gray!20} \textbf{0.074} & 18.417 & 7.794  \\ \hline 
  true & 0.600 & 0.0800 & 0.5000 & 0.480 & 0.133 & 0.416 & 0.056 & 0.598 & 0.052 & 0.601 & 0.079 & 18.417 & 7.794  \\ \hline 
  true & 0.600 & 0.0900 & 0.5000 & 0.473 & 0.137 & 0.410 & 0.057 & 0.605 & 0.054 & 0.607 & 0.083 & 18.417 & 7.794  \\ \hline 
  true & 0.600 & 0.1000 & 0.5000 & 0.467 & 0.139 & 0.404 & 0.056 & 0.611 & 0.052 & 0.613 & 0.079 & 18.417 & 7.794  \\ \hline 
  true & 0.600 & 0.1100 & 0.5000 & 0.462 & 0.141 & 0.399 & 0.055 & 0.615 & 0.051 & \cellcolor{gray!20} \textbf{0.619} & \cellcolor{gray!20} \textbf{0.074} & 18.417 & 7.794  \\ \hline 
  true & 0.600 & 0.0800 & 0.7000 & 0.480 & 0.133 & 0.416 & 0.056 & 0.598 & 0.052 & 0.601 & 0.079 & 18.417 & 7.794  \\ \hline 
  true & 0.600 & 0.0900 & 0.7000 & 0.480 & 0.133 & 0.416 & 0.056 & 0.598 & 0.052 & 0.601 & 0.079 & 18.417 & 7.794  \\ \hline 
  true & 0.600 & 0.1000 & 0.7000 & 0.467 & 0.139 & 0.404 & 0.056 & 0.611 & 0.052 & 0.613 & 0.079 & 18.417 & 7.794  \\ \hline 
  true & 0.600 & 0.1100 & 0.7000 & 0.462 & 0.141 & 0.399 & 0.055 & 0.615 & 0.051 & \cellcolor{gray!20} \textbf{0.619} & \cellcolor{gray!20} \textbf{0.074} & 18.417 & 7.794  \\ \hline 
  true & 0.900 & 0.0800 & 0.1000 & 0.645 & 0.352 & 0.517 & 0.131 & 0.490 & 0.142 & 0.600 & 0.148 & 27.500 & 11.601  \\ \hline 
  true & 0.900 & 0.0900 & 0.1000 & 0.645 & 0.352 & 0.517 & 0.131 & 0.490 & 0.142 & 0.600 & 0.148 & 27.500 & 11.601  \\ \hline 
  true & 0.900 & 0.1000 & 0.1000 & 0.651 & 0.348 & 0.524 & 0.127 & 0.483 & 0.138 & 0.596 & 0.145 & 27.500 & 11.601  \\ \hline 
  true & 0.900 & 0.1100 & 0.1000 & 0.651 & 0.348 & 0.524 & 0.127 & 0.483 & 0.138 & 0.596 & 0.145 & 27.500 & 11.601  \\ \hline 
  true & 0.900 & 0.0800 & 0.3000 & 0.645 & 0.352 & 0.517 & 0.131 & 0.490 & 0.142 & 0.600 & 0.148 & 27.500 & 11.601  \\ \hline 
  true & 0.900 & 0.0900 & 0.3000 & 0.645 & 0.352 & 0.517 & 0.131 & 0.490 & 0.142 & 0.600 & 0.148 & 27.500 & 11.601  \\ \hline 
  true & 0.900 & 0.1000 & 0.3000 & 0.651 & 0.348 & 0.524 & 0.127 & 0.483 & 0.138 & 0.596 & 0.145 & 27.500 & 11.601  \\ \hline 
  true & 0.900 & 0.1100 & 0.3000 & 0.651 & 0.348 & 0.524 & 0.127 & 0.483 & 0.138 & 0.596 & 0.145 & 27.500 & 11.601  \\ \hline 
  true & 0.900 & 0.0800 & 0.5000 & 0.645 & 0.352 & 0.517 & 0.131 & 0.490 & 0.142 & 0.600 & 0.148 & 27.500 & 11.601  \\ \hline 
  true & 0.900 & 0.0900 & 0.5000 & 0.645 & 0.352 & 0.517 & 0.131 & 0.490 & 0.142 & 0.600 & 0.148 & 27.500 & 11.601  \\ \hline 
  true & 0.900 & 0.1000 & 0.5000 & 0.651 & 0.348 & 0.524 & 0.127 & 0.483 & 0.138 & 0.596 & 0.145 & 27.500 & 11.601  \\ \hline 
  true & 0.900 & 0.1100 & 0.5000 & 0.651 & 0.348 & 0.524 & 0.127 & 0.483 & 0.138 & 0.596 & 0.145 & 27.500 & 11.601  \\ \hline 
  true & 0.900 & 0.0800 & 0.7000 & 0.645 & 0.352 & 0.517 & 0.131 & 0.490 & 0.142 & 0.600 & 0.148 & 27.500 & 11.601  \\ \hline 
  true & 0.900 & 0.0900 & 0.7000 & 0.645 & 0.352 & 0.517 & 0.131 & 0.490 & 0.142 & 0.600 & 0.148 & 27.500 & 11.601  \\ \hline 
  true & 0.900 & 0.1000 & 0.7000 & 0.651 & 0.348 & 0.524 & 0.127 & 0.483 & 0.138 & 0.596 & 0.145 & 27.500 & 11.601  \\ \hline 
  true & 0.900 & 0.1100 & 0.7000 & 0.651 & 0.348 & 0.524 & 0.127 & 0.483 & 0.138 & 0.596 & 0.145 & 27.500 & 11.601  \\ \hline 
 \caption{Valores das medidas de desempenho para análise do algoritmo \textit{BayesSeg}, utilizando o texto pré-processado.}
 \end{longtable} 



 \newpage
\center TextSeg
\begin{longtable}[c]{|c|c|c|c|c|c|c|c|c|c|c|} 
\hline 
 Seg Rate & $WinDiff$ & $\sigma$$WinDiff$ & $P_k$ & $\sigma$$P_k$ & Acurácia & $\sigma$Acurácia & $F^1$ & $\sigma$$F^1$ & \#Segs & $\sigma$\#Segs\\ \hline 
 Auto & \cellcolor{gray!20} \textbf{0.455} & \cellcolor{gray!20} \textbf{0.130} & 0.439 & 0.142 & 0.585 & 0.132 & 0.368 & 0.130 & 6.417 & 0.954  \\ \hline 
  0.100 & 0.502 & 0.146 & 0.486 & 0.160 & 0.548 & 0.158 & 0.163 & 0.122 & 3.167 & 1.344  \\ \hline 
  0.200 & 0.473 & 0.160 & 0.452 & 0.175 & 0.569 & 0.159 & 0.320 & 0.166 & 6.083 & 2.660  \\ \hline 
  0.300 & 0.496 & 0.159 & 0.460 & 0.180 & 0.560 & 0.165 & 0.406 & 0.150 & 9.250 & 3.961  \\ \hline 
  0.400 & 0.484 & 0.119 & 0.444 & 0.142 & 0.575 & 0.125 & 0.487 & 0.111 & 12.083 & 5.123  \\ \hline 
  0.500 & 0.475 & 0.107 & \cellcolor{gray!20} \textbf{0.417} & \cellcolor{gray!20} \textbf{0.108} & \cellcolor{gray!20} \textbf{0.594} & \cellcolor{gray!20} \textbf{0.087} & 0.566 & 0.073 & 15.500 & 6.397  \\ \hline 
  0.600 & 0.504 & 0.124 & 0.439 & 0.087 & 0.571 & 0.067 & 0.582 & 0.054 & 18.417 & 7.794  \\ \hline 
  0.700 & 0.531 & 0.173 & 0.447 & 0.074 & 0.562 & 0.066 & 0.605 & 0.083 & 21.417 & 8.949  \\ \hline 
  0.800 & 0.579 & 0.259 & 0.478 & 0.103 & 0.531 & 0.109 & 0.605 & 0.126 & 24.417 & 10.259  \\ \hline 
  0.900 & 0.604 & 0.340 & 0.484 & 0.130 & 0.524 & 0.140 & \cellcolor{gray!20} \textbf{0.627} & \cellcolor{gray!20} \textbf{0.142} & 27.500 & 11.601  \\ \hline 
 \caption{Valores das medidas de desempenho para análise do algoritmo \textit{TextSeg}, utilizando o texto pré-processado.}
 \end{longtable} 

 % \newpage
% \tiny\begin{longtable}[c]{|c|c|c|c|c|c|c|c|c|c|} 
% \hline 
 % $WinDiff$ & $\sigma$$WinDiff$ & $P_k$ & $\sigma$$P_k$ & Acurácia & $\sigma$Acurácia & $F^1$ & $\sigma$$F^1$ & \#Segs & $\sigma$\#Segs\\ \hline 
 % \cellcolor{gray!20} \textbf{0.640} & \cellcolor{gray!20} \textbf{0.415} & \cellcolor{gray!20} \textbf{0.490} & \cellcolor{gray!20} \textbf{0.149} & \cellcolor{gray!20} \textbf{0.506} & \cellcolor{gray!20} \textbf{0.172} & \cellcolor{gray!20} \textbf{0.638} & \cellcolor{gray!20} \textbf{0.156} & 30.500 & 12.907  \\ \hline 
 % \end{longtable} 

 

\newpage
\tiny
\center TextTiling
\begin{longtable}[c]{|c|c|c|c|c|c|c|c|c|c|c|c|} 
\hline 
 Step & Win Size & $WinDiff$ & $\sigma$$WinDiff$ & $P_k$ & $\sigma$$P_k$ & Acurácia & $\sigma$Acurácia & $F^1$ & $\sigma$$F^1$ & \#Segs & $\sigma$\#Segs\\ \hline 
 20 & 30 & 0.513 & 0.138 & 0.490 & 0.144 & 0.538 & 0.138 & 0.334 & 0.173 & 8.500 & 3.571  \\ \hline 
  20 & 35 & 0.509 & 0.127 & 0.492 & 0.126 & 0.540 & 0.121 & 0.350 & 0.135 & 8.583 & 2.871  \\ \hline 
  20 & 40 & 0.517 & 0.132 & 0.495 & 0.144 & 0.532 & 0.137 & 0.342 & 0.142 & 8.583 & 3.148  \\ \hline 
  20 & 45 & 0.496 & 0.114 & 0.477 & 0.122 & 0.555 & 0.117 & 0.347 & 0.117 & 7.667 & 2.528  \\ \hline 
  20 & 50 & 0.481 & 0.140 & 0.465 & 0.138 & 0.569 & 0.134 & \cellcolor{gray!20} \textbf{0.390} & \cellcolor{gray!20} \textbf{0.178} & 8.750 & 3.467  \\ \hline 
  20 & 55 & 0.512 & 0.133 & 0.493 & 0.135 & 0.542 & 0.132 & 0.337 & 0.156 & 8.250 & 3.295  \\ \hline 
  30 & 30 & 0.511 & 0.130 & 0.494 & 0.130 & 0.538 & 0.128 & 0.284 & 0.145 & 6.667 & 2.173  \\ \hline 
  30 & 35 & 0.517 & 0.100 & 0.500 & 0.109 & 0.536 & 0.113 & 0.285 & 0.099 & 6.583 & 2.019  \\ \hline 
  30 & 40 & 0.512 & 0.128 & 0.491 & 0.131 & 0.543 & 0.121 & 0.299 & 0.082 & 6.750 & 2.586  \\ \hline 
  30 & 45 & 0.502 & 0.112 & 0.483 & 0.108 & 0.555 & 0.106 & 0.320 & 0.087 & 6.917 & 2.499  \\ \hline 
  30 & 50 & 0.510 & 0.107 & 0.493 & 0.117 & 0.539 & 0.117 & 0.313 & 0.112 & 7.333 & 2.560  \\ \hline 
  30 & 55 & 0.498 & 0.146 & 0.480 & 0.162 & 0.543 & 0.146 & 0.328 & 0.115 & 7.250 & 2.350  \\ \hline 
  40 & 30 & 0.493 & 0.132 & 0.477 & 0.141 & 0.555 & 0.134 & 0.248 & 0.071 & 4.917 & 2.060  \\ \hline 
  40 & 35 & 0.482 & 0.121 & 0.465 & 0.132 & 0.558 & 0.123 & 0.267 & 0.106 & 5.417 & 2.178  \\ \hline 
  40 & 40 & 0.476 & 0.112 & 0.459 & 0.120 & 0.565 & 0.114 & 0.275 & 0.120 & 5.500 & 2.566  \\ \hline 
  40 & 45 & 0.501 & 0.134 & 0.482 & 0.144 & 0.549 & 0.143 & 0.260 & 0.120 & 5.333 & 2.095  \\ \hline 
  40 & 50 & 0.498 & 0.123 & 0.481 & 0.135 & 0.551 & 0.134 & 0.266 & 0.087 & 5.333 & 2.285  \\ \hline 
  40 & 55 & 0.505 & 0.116 & 0.487 & 0.131 & 0.544 & 0.131 & 0.243 & 0.077 & 5.083 & 1.706  \\ \hline 
  50 & 30 & \cellcolor{gray!20} \textbf{0.474} & \cellcolor{gray!20} \textbf{0.135} & \cellcolor{gray!20} \textbf{0.455} & \cellcolor{gray!20} \textbf{0.138} & \cellcolor{gray!20} \textbf{0.579} & \cellcolor{gray!20} \textbf{0.132} & 0.295 & 0.106 & 4.917 & 1.552  \\ \hline 
  50 & 35 & 0.528 & 0.126 & 0.511 & 0.137 & 0.531 & 0.146 & 0.202 & 0.088 & 4.583 & 1.706  \\ \hline 
  50 & 40 & 0.501 & 0.103 & 0.488 & 0.121 & 0.539 & 0.122 & 0.234 & 0.108 & 5.000 & 1.683  \\ \hline 
  50 & 45 & 0.489 & 0.112 & 0.476 & 0.125 & 0.558 & 0.135 & 0.275 & 0.092 & 5.167 & 2.034  \\ \hline 
  50 & 50 & 0.498 & 0.158 & 0.483 & 0.171 & 0.545 & 0.162 & 0.304 & 0.100 & 6.083 & 1.891  \\ \hline 
  50 & 55 & 0.490 & 0.151 & 0.470 & 0.167 & 0.556 & 0.157 & 0.303 & 0.123 & 5.583 & 2.178  \\ \hline 
  60 & 30 & 0.499 & 0.092 & 0.486 & 0.103 & 0.557 & 0.123 & 0.234 & 0.098 & 4.417 & 1.754  \\ \hline 
  60 & 35 & 0.509 & 0.143 & 0.494 & 0.164 & 0.537 & 0.159 & 0.243 & 0.111 & 5.000 & 1.472  \\ \hline 
  60 & 40 & 0.501 & 0.113 & 0.486 & 0.128 & 0.545 & 0.129 & 0.182 & 0.108 & 3.833 & 1.462  \\ \hline 
  60 & 45 & 0.493 & 0.118 & 0.478 & 0.129 & 0.558 & 0.136 & 0.227 & 0.136 & 4.167 & 1.462  \\ \hline 
  60 & 50 & 0.495 & 0.110 & 0.478 & 0.118 & 0.562 & 0.127 & 0.225 & 0.081 & 4.083 & 1.656  \\ \hline 
  60 & 55 & 0.500 & 0.104 & 0.485 & 0.114 & 0.550 & 0.120 & 0.198 & 0.075 & 4.000 & 1.155  \\ \hline 
 \caption{Valores das medidas de desempenho para análise do algoritmo \textit{TextTiling}, utilizando o texto o texto integral.}
 \end{longtable} 


 \newpage
\center C99
\begin{longtable}[c]{|c|c|c|c|c|c|c|c|c|c|c|c|c|} 
\hline 
 Seg Rate & Raking Size & Weitght & $WinDiff$ & $\sigma$$WinDiff$ & $P_k$ & $\sigma$$P_k$ & Acurácia & $\sigma$Acurácia & $F^1$ & $\sigma$$F^1$ & \#Segs & $\sigma$\#Segs\\ \hline 
 0.200 & 3 & true & 0.481 & 0.118 & 0.463 & 0.121 & 0.574 & 0.122 & 0.324 & 0.094 & 6.083 & 2.660  \\ \hline 
  0.300 & 3 & true & 0.457 & 0.109 & 0.437 & 0.104 & 0.596 & 0.105 & 0.447 & 0.091 & 9.250 & 3.961  \\ \hline 
  0.400 & 3 & true & 0.450 & 0.153 & 0.425 & 0.142 & 0.602 & 0.123 & 0.513 & 0.143 & 12.083 & 5.123  \\ \hline 
  0.500 & 3 & true & \cellcolor{gray!20} \textbf{0.435} & \cellcolor{gray!20} \textbf{0.155} & \cellcolor{gray!20} \textbf{0.395} & \cellcolor{gray!20} \textbf{0.106} & \cellcolor{gray!20} \textbf{0.629} & \cellcolor{gray!20} \textbf{0.095} & 0.594 & 0.123 & 15.500 & 6.397  \\ \hline 
  0.600 & 3 & true & 0.489 & 0.194 & 0.437 & 0.091 & 0.592 & 0.075 & 0.591 & 0.119 & 18.417 & 7.794  \\ \hline 
  0.700 & 3 & true & 0.482 & 0.232 & 0.420 & 0.111 & 0.602 & 0.107 & \cellcolor{gray!20} \textbf{0.632} & \cellcolor{gray!20} \textbf{0.139} & 21.417 & 8.949  \\ \hline 
  0.200 & 5 & true & 0.488 & 0.122 & 0.469 & 0.133 & 0.565 & 0.135 & 0.313 & 0.106 & 6.083 & 2.660  \\ \hline 
  0.300 & 5 & true & 0.476 & 0.166 & 0.458 & 0.175 & 0.571 & 0.166 & 0.426 & 0.151 & 9.250 & 3.961  \\ \hline 
  0.400 & 5 & true & 0.476 & 0.127 & 0.452 & 0.127 & 0.578 & 0.121 & 0.487 & 0.113 & 12.083 & 5.123  \\ \hline 
  0.500 & 5 & true & 0.463 & 0.142 & 0.425 & 0.095 & 0.605 & 0.087 & 0.566 & 0.119 & 15.500 & 6.397  \\ \hline 
  0.600 & 5 & true & 0.464 & 0.187 & 0.415 & 0.110 & 0.610 & 0.100 & 0.604 & 0.141 & 18.417 & 7.794  \\ \hline 
  0.700 & 5 & true & 0.504 & 0.244 & 0.435 & 0.117 & 0.589 & 0.108 & 0.619 & 0.142 & 21.417 & 8.949  \\ \hline 
  0.200 & 7 & true & 0.478 & 0.124 & 0.459 & 0.133 & 0.574 & 0.135 & 0.328 & 0.108 & 6.083 & 2.660  \\ \hline 
  0.300 & 7 & true & 0.481 & 0.145 & 0.462 & 0.150 & 0.570 & 0.141 & 0.418 & 0.115 & 9.250 & 3.961  \\ \hline 
  0.400 & 7 & true & 0.478 & 0.129 & 0.452 & 0.125 & 0.577 & 0.118 & 0.482 & 0.127 & 12.083 & 5.123  \\ \hline 
  0.500 & 7 & true & 0.471 & 0.171 & 0.427 & 0.108 & 0.604 & 0.093 & 0.563 & 0.131 & 15.500 & 6.397  \\ \hline 
  0.600 & 7 & true & 0.480 & 0.186 & 0.429 & 0.104 & 0.599 & 0.094 & 0.594 & 0.134 & 18.417 & 7.794  \\ \hline 
  0.700 & 7 & true & 0.516 & 0.241 & 0.444 & 0.106 & 0.579 & 0.100 & 0.611 & 0.133 & 21.417 & 8.949  \\ \hline 
  0.200 & 3 & false & 0.469 & 0.119 & 0.453 & 0.129 & 0.579 & 0.130 & 0.335 & 0.107 & 6.083 & 2.660  \\ \hline 
  0.300 & 3 & false & 0.441 & 0.073 & 0.421 & 0.086 & 0.608 & 0.089 & 0.463 & 0.056 & 9.250 & 3.961  \\ \hline 
  0.400 & 3 & false & 0.467 & 0.062 & 0.439 & 0.057 & 0.591 & 0.067 & 0.493 & 0.092 & 12.083 & 5.123  \\ \hline 
  0.500 & 3 & false & 0.483 & 0.137 & 0.442 & 0.082 & 0.593 & 0.078 & 0.554 & 0.108 & 15.500 & 6.397  \\ \hline 
  0.600 & 3 & false & 0.500 & 0.199 & 0.442 & 0.099 & 0.589 & 0.085 & 0.587 & 0.120 & 18.417 & 7.794  \\ \hline 
  0.700 & 3 & false & 0.492 & 0.244 & 0.423 & 0.115 & 0.602 & 0.103 & 0.632 & 0.133 & 21.417 & 8.949  \\ \hline 
  0.200 & 5 & false & 0.495 & 0.161 & 0.476 & 0.170 & 0.555 & 0.160 & 0.300 & 0.128 & 6.083 & 2.660  \\ \hline 
  0.300 & 5 & false & 0.503 & 0.134 & 0.485 & 0.143 & 0.549 & 0.141 & 0.386 & 0.123 & 9.250 & 3.961  \\ \hline 
  0.400 & 5 & false & 0.496 & 0.110 & 0.477 & 0.104 & 0.564 & 0.108 & 0.466 & 0.109 & 12.083 & 5.123  \\ \hline 
  0.500 & 5 & false & 0.488 & 0.114 & 0.452 & 0.072 & 0.574 & 0.067 & 0.533 & 0.104 & 15.500 & 6.397  \\ \hline 
  0.600 & 5 & false & 0.484 & 0.171 & 0.434 & 0.077 & 0.594 & 0.065 & 0.592 & 0.108 & 18.417 & 7.794  \\ \hline 
  0.700 & 5 & false & 0.522 & 0.235 & 0.451 & 0.105 & 0.574 & 0.095 & 0.609 & 0.122 & 21.417 & 8.949  \\ \hline 
  0.200 & 7 & false & 0.489 & 0.162 & 0.471 & 0.170 & 0.560 & 0.159 & 0.307 & 0.132 & 6.083 & 2.660  \\ \hline 
  0.300 & 7 & false & 0.498 & 0.146 & 0.479 & 0.153 & 0.554 & 0.149 & 0.394 & 0.132 & 9.250 & 3.961  \\ \hline 
  0.400 & 7 & false & 0.500 & 0.119 & 0.475 & 0.111 & 0.561 & 0.108 & 0.462 & 0.110 & 12.083 & 5.123  \\ \hline 
  0.500 & 7 & false & 0.479 & 0.145 & 0.441 & 0.089 & 0.592 & 0.080 & 0.551 & 0.115 & 15.500 & 6.397  \\ \hline 
  0.600 & 7 & false & 0.493 & 0.172 & 0.439 & 0.080 & 0.585 & 0.073 & 0.586 & 0.106 & 18.417 & 7.794  \\ \hline 
  0.700 & 7 & false & 0.506 & 0.261 & 0.430 & 0.131 & 0.590 & 0.126 & 0.621 & 0.149 & 21.417 & 8.949  \\ \hline 
 \caption{Valores das medidas de desempenho para análise do algoritmo \textit{C99}, utilizando o texto o texto integral.}
 \end{longtable} 

 \newpage
\center MinCutSeg
\begin{longtable}[c]{|c|c|c|c|c|c|c|c|c|c|c|c|} 
\hline 
 Seg Rate & LenCutoff & $WinDiff$ & $\sigma$$WinDiff$ & $P_k$ & $\sigma$$P_k$ & Acurácia & $\sigma$Acurácia & $F^1$ & $\sigma$$F^1$ & \#Segs & $\sigma$\#Segs\\ \hline 
 0.200 & 5 & 0.513 & 0.132 & 0.489 & 0.143 & 0.539 & 0.137 & 0.257 & 0.118 & 5.833 & 2.609  \\ \hline 
  0.200 & 7 & 0.510 & 0.128 & 0.486 & 0.135 & 0.545 & 0.132 & 0.267 & 0.098 & 5.833 & 2.609  \\ \hline 
  0.200 & 9 & 0.498 & 0.111 & 0.474 & 0.130 & 0.553 & 0.127 & 0.282 & 0.097 & 5.833 & 2.609  \\ \hline 
  0.200 & 11 & 0.487 & 0.115 & 0.459 & 0.135 & 0.566 & 0.128 & 0.302 & 0.103 & 5.833 & 2.609  \\ \hline 
  0.200 & 13 & 0.473 & 0.124 & 0.445 & 0.135 & 0.580 & 0.126 & 0.324 & 0.093 & 5.833 & 2.609  \\ \hline 
  0.200 & 15 & 0.467 & 0.128 & 0.443 & 0.145 & 0.581 & 0.137 & 0.333 & 0.109 & 5.833 & 2.609  \\ \hline 
  0.300 & 5 & 0.483 & 0.082 & 0.451 & 0.110 & 0.573 & 0.104 & 0.402 & 0.062 & 8.667 & 3.771  \\ \hline 
  0.300 & 7 & 0.474 & 0.110 & 0.437 & 0.121 & 0.585 & 0.113 & 0.421 & 0.085 & 8.667 & 3.771  \\ \hline 
  0.300 & 9 & 0.480 & 0.099 & 0.441 & 0.118 & 0.579 & 0.107 & 0.410 & 0.093 & 8.667 & 3.771  \\ \hline 
  0.300 & 11 & 0.454 & 0.098 & 0.418 & 0.119 & 0.601 & 0.109 & 0.442 & 0.092 & 8.667 & 3.771  \\ \hline 
  0.300 & 13 & 0.460 & 0.097 & 0.423 & 0.124 & 0.594 & 0.111 & 0.434 & 0.091 & 8.667 & 3.771  \\ \hline 
  0.300 & 15 & 0.455 & 0.100 & 0.417 & 0.125 & 0.599 & 0.111 & 0.440 & 0.096 & 8.667 & 3.771  \\ \hline 
  0.400 & 5 & 0.444 & 0.082 & 0.407 & 0.117 & 0.609 & 0.107 & 0.523 & 0.104 & 11.917 & 5.251  \\ \hline 
  0.400 & 7 & 0.455 & 0.095 & 0.410 & 0.104 & 0.606 & 0.093 & 0.513 & 0.098 & 11.917 & 5.251  \\ \hline 
  0.400 & 9 & 0.465 & 0.130 & 0.418 & 0.135 & 0.601 & 0.123 & 0.514 & 0.112 & 11.917 & 5.251  \\ \hline 
  0.400 & 11 & 0.442 & 0.137 & 0.404 & 0.156 & 0.613 & 0.142 & 0.533 & 0.136 & 11.917 & 5.251  \\ \hline 
  0.400 & 13 & 0.434 & 0.144 & 0.400 & 0.162 & \cellcolor{gray!20} \textbf{0.620} & \cellcolor{gray!20} \textbf{0.152} & 0.543 & 0.148 & 11.917 & 5.251  \\ \hline 
  0.400 & 15 & \cellcolor{gray!20} \textbf{0.430} & \cellcolor{gray!20} \textbf{0.150} & \cellcolor{gray!20} \textbf{0.397} & \cellcolor{gray!20} \textbf{0.172} & 0.620 & 0.156 & 0.543 & 0.152 & 11.917 & 5.251  \\ \hline 
  0.500 & 5 & 0.484 & 0.128 & 0.426 & 0.112 & 0.587 & 0.099 & 0.550 & 0.085 & 15.000 & 6.519  \\ \hline 
  0.500 & 7 & 0.472 & 0.162 & 0.412 & 0.127 & 0.602 & 0.121 & 0.563 & 0.133 & 15.000 & 6.519  \\ \hline 
  0.500 & 9 & 0.466 & 0.147 & 0.411 & 0.140 & 0.602 & 0.128 & 0.567 & 0.127 & 15.000 & 6.519  \\ \hline 
  0.500 & 11 & 0.465 & 0.141 & 0.413 & 0.141 & 0.598 & 0.127 & 0.564 & 0.122 & 15.000 & 6.519  \\ \hline 
  0.500 & 13 & 0.451 & 0.146 & 0.399 & 0.149 & 0.612 & 0.134 & 0.578 & 0.130 & 15.000 & 6.519  \\ \hline 
  0.500 & 15 & 0.462 & 0.154 & 0.405 & 0.148 & 0.606 & 0.134 & 0.570 & 0.134 & 15.000 & 6.519  \\ \hline 
  0.600 & 5 & 0.500 & 0.154 & 0.431 & 0.099 & 0.581 & 0.088 & 0.581 & 0.091 & 17.917 & 7.719  \\ \hline 
  0.600 & 7 & 0.498 & 0.143 & 0.427 & 0.110 & 0.579 & 0.096 & 0.579 & 0.104 & 17.917 & 7.719  \\ \hline 
  0.600 & 9 & 0.492 & 0.153 & 0.423 & 0.107 & 0.588 & 0.098 & 0.591 & 0.095 & 17.917 & 7.719  \\ \hline 
  0.600 & 11 & 0.482 & 0.161 & 0.412 & 0.112 & 0.598 & 0.102 & 0.600 & 0.102 & 17.917 & 7.719  \\ \hline 
  0.600 & 13 & 0.474 & 0.150 & 0.404 & 0.121 & 0.602 & 0.105 & 0.605 & 0.102 & 17.917 & 7.719  \\ \hline 
  0.600 & 15 & 0.482 & 0.161 & 0.410 & 0.113 & 0.598 & 0.102 & 0.600 & 0.102 & 17.917 & 7.719  \\ \hline 
  0.700 & 5 & 0.512 & 0.193 & 0.424 & 0.076 & 0.579 & 0.076 & \cellcolor{gray!20} \textbf{0.612} & \cellcolor{gray!20} \textbf{0.097} & 21.000 & 9.211  \\ \hline 
  0.700 & 7 & 0.522 & 0.194 & 0.433 & 0.089 & 0.570 & 0.085 & 0.603 & 0.105 & 21.000 & 9.211  \\ \hline 
  0.700 & 9 & 0.528 & 0.205 & 0.438 & 0.098 & 0.565 & 0.091 & 0.602 & 0.097 & 21.000 & 9.211  \\ \hline 
  0.700 & 11 & 0.532 & 0.220 & 0.440 & 0.093 & 0.568 & 0.088 & 0.605 & 0.094 & 21.000 & 9.211  \\ \hline 
  0.700 & 13 & 0.537 & 0.210 & 0.445 & 0.095 & 0.560 & 0.088 & 0.598 & 0.094 & 21.000 & 9.211  \\ \hline 
  0.700 & 15 & 0.530 & 0.208 & 0.438 & 0.085 & 0.567 & 0.080 & 0.604 & 0.087 & 21.000 & 9.211  \\ \hline 
 \caption{Valores das medidas de desempenho para análise do algoritmo \textit{MinCutSeg}, utilizando o texto o texto integral.}
 \end{longtable} 


 \newpage
 \center BayesSeg
\begin{longtable}[c]{|c|c|c|c|c|c|c|c|c|c|c|c|c|c|} 
\hline 
 \#SegsKnown & Seg Rate & Prior & Dispertion & $WinDiff$ & $\sigma$$WinDiff$ & $P_k$ & $\sigma$$P_k$ & Acurácia & $\sigma$Acurácia & $F^1$ & $\sigma$$F^1$ & \#Segs & $\sigma$\#Segs\\ \hline 
 false & Auto & 0.0800 & 0.1000 & 0.399 & 0.087 & 0.380 & 0.108 & 0.637 & 0.095 & 0.526 & 0.088 & 9.750 & 1.785  \\ \hline 
  false & Auto & 0.0900 & 0.1000 & 0.405 & 0.080 & 0.386 & 0.099 & 0.633 & 0.091 & 0.513 & 0.077 & 9.417 & 1.706  \\ \hline 
  false & Auto & 0.1000 & 0.1000 & 0.399 & 0.077 & 0.380 & 0.095 & 0.639 & 0.087 & 0.517 & 0.078 & 9.250 & 1.639  \\ \hline 
  false & Auto & 0.1100 & 0.1000 & 0.405 & 0.083 & 0.387 & 0.102 & 0.633 & 0.093 & 0.506 & 0.075 & 9.083 & 1.498  \\ \hline 
  false & Auto & 0.0800 & 0.3000 & \cellcolor{gray!20} \textbf{0.383} & \cellcolor{gray!20} \textbf{0.089} & \cellcolor{gray!20} \textbf{0.364} & \cellcolor{gray!20} \textbf{0.107} & \cellcolor{gray!20} \textbf{0.652} & \cellcolor{gray!20} \textbf{0.094} & 0.549 & 0.101 & 10.083 & 1.801  \\ \hline 
  false & Auto & 0.0900 & 0.3000 & 0.396 & 0.084 & 0.377 & 0.100 & 0.642 & 0.091 & 0.527 & 0.093 & 9.667 & 1.795  \\ \hline 
  false & Auto & 0.1000 & 0.3000 & 0.397 & 0.074 & 0.378 & 0.092 & 0.641 & 0.084 & 0.518 & 0.084 & 9.250 & 1.479  \\ \hline 
  false & Auto & 0.1100 & 0.3000 & 0.393 & 0.073 & 0.374 & 0.091 & 0.644 & 0.082 & 0.520 & 0.084 & 9.167 & 1.462  \\ \hline 
  false & Auto & 0.0800 & 0.5000 & \cellcolor{gray!20} \textbf{0.383} & \cellcolor{gray!20} \textbf{0.089} & \cellcolor{gray!20} \textbf{0.364} & \cellcolor{gray!20} \textbf{0.107} & \cellcolor{gray!20} \textbf{0.652} & \cellcolor{gray!20} \textbf{0.094} & 0.549 & 0.101 & 10.083 & 1.801  \\ \hline 
  false & Auto & 0.0900 & 0.5000 & 0.401 & 0.084 & 0.382 & 0.102 & 0.637 & 0.093 & 0.521 & 0.096 & 9.667 & 1.700  \\ \hline 
  false & Auto & 0.1000 & 0.5000 & 0.400 & 0.077 & 0.381 & 0.095 & 0.638 & 0.087 & 0.516 & 0.084 & 9.333 & 1.546  \\ \hline 
  false & Auto & 0.1100 & 0.5000 & 0.392 & 0.075 & 0.373 & 0.092 & 0.646 & 0.083 & 0.521 & 0.083 & 9.083 & 1.320  \\ \hline 
  false & Auto & 0.0800 & 0.7000 & 0.388 & 0.077 & 0.369 & 0.093 & 0.649 & 0.085 & 0.545 & 0.096 & 10.083 & 1.706  \\ \hline 
  false & Auto & 0.0900 & 0.7000 & 0.396 & 0.080 & 0.377 & 0.097 & 0.642 & 0.089 & 0.526 & 0.102 & 9.750 & 1.689  \\ \hline 
  false & Auto & 0.1000 & 0.7000 & 0.398 & 0.079 & 0.380 & 0.097 & 0.639 & 0.088 & 0.517 & 0.083 & 9.250 & 1.422  \\ \hline 
  false & Auto & 0.1100 & 0.7000 & 0.392 & 0.075 & 0.373 & 0.092 & 0.646 & 0.083 & 0.521 & 0.083 & 9.083 & 1.320  \\ \hline 
  true & 0.300 & 0.0800 & 0.1000 & 0.421 & 0.144 & 0.391 & 0.165 & 0.624 & 0.147 & 0.499 & 0.110 & 9.250 & 3.961  \\ \hline 
  true & 0.300 & 0.0900 & 0.1000 & 0.421 & 0.144 & 0.391 & 0.165 & 0.624 & 0.147 & 0.499 & 0.110 & 9.250 & 3.961  \\ \hline 
  true & 0.300 & 0.1000 & 0.1000 & 0.421 & 0.144 & 0.393 & 0.163 & 0.620 & 0.145 & 0.493 & 0.110 & 9.250 & 3.961  \\ \hline 
  true & 0.300 & 0.1100 & 0.1000 & 0.420 & 0.143 & 0.392 & 0.168 & 0.621 & 0.148 & 0.495 & 0.119 & 9.250 & 3.961  \\ \hline 
  true & 0.300 & 0.0800 & 0.3000 & 0.421 & 0.144 & 0.391 & 0.165 & 0.624 & 0.147 & 0.499 & 0.110 & 9.250 & 3.961  \\ \hline 
  true & 0.300 & 0.0900 & 0.3000 & 0.421 & 0.144 & 0.393 & 0.163 & 0.620 & 0.145 & 0.493 & 0.110 & 9.250 & 3.961  \\ \hline 
  true & 0.300 & 0.1000 & 0.3000 & 0.421 & 0.144 & 0.393 & 0.163 & 0.620 & 0.145 & 0.493 & 0.110 & 9.250 & 3.961  \\ \hline 
  true & 0.300 & 0.1100 & 0.3000 & 0.417 & 0.146 & 0.389 & 0.169 & 0.624 & 0.150 & 0.500 & 0.120 & 9.250 & 3.961  \\ \hline 
  true & 0.300 & 0.0800 & 0.5000 & 0.421 & 0.144 & 0.393 & 0.163 & 0.620 & 0.145 & 0.493 & 0.110 & 9.250 & 3.961  \\ \hline 
  true & 0.300 & 0.0900 & 0.5000 & 0.421 & 0.144 & 0.393 & 0.163 & 0.620 & 0.145 & 0.493 & 0.110 & 9.250 & 3.961  \\ \hline 
  true & 0.300 & 0.1000 & 0.5000 & 0.421 & 0.144 & 0.393 & 0.163 & 0.620 & 0.145 & 0.493 & 0.110 & 9.250 & 3.961  \\ \hline 
  true & 0.300 & 0.1100 & 0.5000 & 0.421 & 0.144 & 0.393 & 0.163 & 0.620 & 0.145 & 0.493 & 0.110 & 9.250 & 3.961  \\ \hline 
  true & 0.300 & 0.0800 & 0.7000 & 0.421 & 0.144 & 0.393 & 0.163 & 0.620 & 0.145 & 0.493 & 0.110 & 9.250 & 3.961  \\ \hline 
  true & 0.300 & 0.0900 & 0.7000 & 0.421 & 0.144 & 0.393 & 0.163 & 0.620 & 0.145 & 0.493 & 0.110 & 9.250 & 3.961  \\ \hline 
  true & 0.300 & 0.1000 & 0.7000 & 0.421 & 0.144 & 0.393 & 0.163 & 0.620 & 0.145 & 0.493 & 0.110 & 9.250 & 3.961  \\ \hline 
  true & 0.300 & 0.1100 & 0.7000 & 0.421 & 0.144 & 0.393 & 0.163 & 0.620 & 0.145 & 0.493 & 0.110 & 9.250 & 3.961  \\ \hline 
  true & 0.600 & 0.0800 & 0.1000 & 0.473 & 0.137 & 0.410 & 0.057 & 0.605 & 0.054 & 0.607 & 0.083 & 18.417 & 7.794  \\ \hline 
  true & 0.600 & 0.0900 & 0.1000 & 0.473 & 0.137 & 0.410 & 0.057 & 0.605 & 0.054 & 0.607 & 0.083 & 18.417 & 7.794  \\ \hline 
  true & 0.600 & 0.1000 & 0.1000 & 0.467 & 0.139 & 0.404 & 0.056 & 0.611 & 0.052 & 0.613 & 0.079 & 18.417 & 7.794  \\ \hline 
  true & 0.600 & 0.1100 & 0.1000 & 0.462 & 0.141 & 0.399 & 0.055 & 0.615 & 0.051 & \cellcolor{gray!20} \textbf{0.619} & \cellcolor{gray!20} \textbf{0.074} & 18.417 & 7.794  \\ \hline 
  true & 0.600 & 0.0800 & 0.3000 & 0.473 & 0.137 & 0.410 & 0.057 & 0.605 & 0.054 & 0.607 & 0.083 & 18.417 & 7.794  \\ \hline 
  true & 0.600 & 0.0900 & 0.3000 & 0.473 & 0.137 & 0.410 & 0.057 & 0.605 & 0.054 & 0.607 & 0.083 & 18.417 & 7.794  \\ \hline 
  true & 0.600 & 0.1000 & 0.3000 & 0.467 & 0.139 & 0.404 & 0.056 & 0.611 & 0.052 & 0.613 & 0.079 & 18.417 & 7.794  \\ \hline 
  true & 0.600 & 0.1100 & 0.3000 & 0.462 & 0.141 & 0.399 & 0.055 & 0.615 & 0.051 & \cellcolor{gray!20} \textbf{0.619} & \cellcolor{gray!20} \textbf{0.074} & 18.417 & 7.794  \\ \hline 
  true & 0.600 & 0.0800 & 0.5000 & 0.473 & 0.137 & 0.410 & 0.057 & 0.605 & 0.054 & 0.607 & 0.083 & 18.417 & 7.794  \\ \hline 
  true & 0.600 & 0.0900 & 0.5000 & 0.473 & 0.137 & 0.410 & 0.057 & 0.605 & 0.054 & 0.607 & 0.083 & 18.417 & 7.794  \\ \hline 
  true & 0.600 & 0.1000 & 0.5000 & 0.467 & 0.139 & 0.404 & 0.056 & 0.611 & 0.052 & 0.613 & 0.079 & 18.417 & 7.794  \\ \hline 
  true & 0.600 & 0.1100 & 0.5000 & 0.462 & 0.141 & 0.399 & 0.055 & 0.615 & 0.051 & \cellcolor{gray!20} \textbf{0.619} & \cellcolor{gray!20} \textbf{0.074} & 18.417 & 7.794  \\ \hline 
  true & 0.600 & 0.0800 & 0.7000 & 0.473 & 0.137 & 0.410 & 0.057 & 0.605 & 0.054 & 0.607 & 0.083 & 18.417 & 7.794  \\ \hline 
  true & 0.600 & 0.0900 & 0.7000 & 0.473 & 0.137 & 0.410 & 0.057 & 0.605 & 0.054 & 0.607 & 0.083 & 18.417 & 7.794  \\ \hline 
  true & 0.600 & 0.1000 & 0.7000 & 0.467 & 0.139 & 0.404 & 0.056 & 0.611 & 0.052 & 0.613 & 0.079 & 18.417 & 7.794  \\ \hline 
  true & 0.600 & 0.1100 & 0.7000 & 0.462 & 0.141 & 0.399 & 0.055 & 0.615 & 0.051 & \cellcolor{gray!20} \textbf{0.619} & \cellcolor{gray!20} \textbf{0.074} & 18.417 & 7.794  \\ \hline 
  true & 0.900 & 0.0800 & 0.1000 & 0.638 & 0.357 & 0.511 & 0.139 & 0.496 & 0.149 & 0.605 & 0.153 & 27.500 & 11.601  \\ \hline 
  true & 0.900 & 0.0900 & 0.1000 & 0.638 & 0.357 & 0.511 & 0.139 & 0.496 & 0.149 & 0.605 & 0.153 & 27.500 & 11.601  \\ \hline 
  true & 0.900 & 0.1000 & 0.1000 & 0.638 & 0.357 & 0.511 & 0.139 & 0.496 & 0.149 & 0.605 & 0.153 & 27.500 & 11.601  \\ \hline 
  true & 0.900 & 0.1100 & 0.1000 & 0.638 & 0.357 & 0.511 & 0.139 & 0.496 & 0.149 & 0.605 & 0.153 & 27.500 & 11.601  \\ \hline 
  true & 0.900 & 0.0800 & 0.3000 & 0.638 & 0.357 & 0.511 & 0.139 & 0.496 & 0.149 & 0.605 & 0.153 & 27.500 & 11.601  \\ \hline 
  true & 0.900 & 0.0900 & 0.3000 & 0.638 & 0.357 & 0.511 & 0.139 & 0.496 & 0.149 & 0.605 & 0.153 & 27.500 & 11.601  \\ \hline 
  true & 0.900 & 0.1000 & 0.3000 & 0.638 & 0.357 & 0.511 & 0.139 & 0.496 & 0.149 & 0.605 & 0.153 & 27.500 & 11.601  \\ \hline 
  true & 0.900 & 0.1100 & 0.3000 & 0.638 & 0.357 & 0.511 & 0.139 & 0.496 & 0.149 & 0.605 & 0.153 & 27.500 & 11.601  \\ \hline 
  true & 0.900 & 0.0800 & 0.5000 & 0.638 & 0.357 & 0.511 & 0.139 & 0.496 & 0.149 & 0.605 & 0.153 & 27.500 & 11.601  \\ \hline 
  true & 0.900 & 0.0900 & 0.5000 & 0.638 & 0.357 & 0.511 & 0.139 & 0.496 & 0.149 & 0.605 & 0.153 & 27.500 & 11.601  \\ \hline 
  true & 0.900 & 0.1000 & 0.5000 & 0.638 & 0.357 & 0.511 & 0.139 & 0.496 & 0.149 & 0.605 & 0.153 & 27.500 & 11.601  \\ \hline 
  true & 0.900 & 0.1100 & 0.5000 & 0.638 & 0.357 & 0.511 & 0.139 & 0.496 & 0.149 & 0.605 & 0.153 & 27.500 & 11.601  \\ \hline 
  true & 0.900 & 0.0800 & 0.7000 & 0.638 & 0.357 & 0.511 & 0.139 & 0.496 & 0.149 & 0.605 & 0.153 & 27.500 & 11.601  \\ \hline 
  true & 0.900 & 0.0900 & 0.7000 & 0.638 & 0.357 & 0.511 & 0.139 & 0.496 & 0.149 & 0.605 & 0.153 & 27.500 & 11.601  \\ \hline 
  true & 0.900 & 0.1000 & 0.7000 & 0.638 & 0.357 & 0.511 & 0.139 & 0.496 & 0.149 & 0.605 & 0.153 & 27.500 & 11.601  \\ \hline 
  true & 0.900 & 0.1100 & 0.7000 & 0.638 & 0.357 & 0.511 & 0.139 & 0.496 & 0.149 & 0.605 & 0.153 & 27.500 & 11.601  \\ \hline 
 \caption{Valores das medidas de desempenho para análise do algoritmo \textit{BayesSeg}, utilizando o texto o texto integral.}
 \end{longtable} 


 \newpage
\center TextSeg
\begin{longtable}[c]{|c|c|c|c|c|c|c|c|c|c|c|} 
\hline 
 Seg Rate & $WinDiff$ & $\sigma$$WinDiff$ & $P_k$ & $\sigma$$P_k$ & Acurácia & $\sigma$Acurácia & $F^1$ & $\sigma$$F^1$ & \#Segs & $\sigma$\#Segs\\ \hline 
 Auto & \cellcolor{gray!20} \textbf{0.430} & \cellcolor{gray!20} \textbf{0.131} & \cellcolor{gray!20} \textbf{0.413} & \cellcolor{gray!20} \textbf{0.142} & \cellcolor{gray!20} \textbf{0.610} & \cellcolor{gray!20} \textbf{0.131} & 0.397 & 0.133 & 6.083 & 0.862  \\ \hline 
  0.100 & 0.493 & 0.172 & 0.476 & 0.185 & 0.558 & 0.181 & 0.191 & 0.155 & 3.167 & 1.344  \\ \hline 
  0.200 & 0.456 & 0.135 & 0.435 & 0.155 & 0.585 & 0.141 & 0.347 & 0.115 & 6.083 & 2.660  \\ \hline 
  0.300 & 0.483 & 0.135 & 0.451 & 0.168 & 0.567 & 0.151 & 0.419 & 0.125 & 9.250 & 3.961  \\ \hline 
  0.400 & 0.469 & 0.140 & 0.426 & 0.167 & 0.586 & 0.145 & 0.507 & 0.122 & 12.083 & 5.123  \\ \hline 
  0.500 & 0.476 & 0.127 & 0.417 & 0.108 & 0.593 & 0.093 & 0.563 & 0.082 & 15.500 & 6.397  \\ \hline 
  0.600 & 0.496 & 0.150 & 0.425 & 0.071 & 0.587 & 0.058 & 0.593 & 0.070 & 18.417 & 7.794  \\ \hline 
  0.700 & 0.551 & 0.210 & 0.463 & 0.065 & 0.550 & 0.064 & 0.591 & 0.097 & 21.417 & 8.949  \\ \hline 
  0.800 & 0.593 & 0.279 & 0.488 & 0.101 & 0.522 & 0.108 & 0.595 & 0.134 & 24.417 & 10.259  \\ \hline 
  0.900 & 0.620 & 0.342 & 0.495 & 0.115 & 0.511 & 0.130 & \cellcolor{gray!20} \textbf{0.618} & \cellcolor{gray!20} \textbf{0.138} & 27.500 & 11.601  \\ \hline 
 \caption{Valores das medidas de desempenho para análise do algoritmo \textit{TextSeg}, utilizando o texto o texto integral.}
 \end{longtable} 


 \center PseudoSeg
\begin{longtable}[c]{|c|c|c|c|c|c|c|c|c|c|} 
\hline 
 $WinDiff$ & $\sigma$$WinDiff$ & $P_k$ & $\sigma$$P_k$ & Acurácia & $\sigma$Acurácia & $F^1$ & $\sigma$$F^1$ & \#Segs & $\sigma$\#Segs\\ \hline 
 \cellcolor{gray!20} \textbf{0.640} & \cellcolor{gray!20} \textbf{0.415} & \cellcolor{gray!20} \textbf{0.490} & \cellcolor{gray!20} \textbf{0.149} & \cellcolor{gray!20} \textbf{0.506} & \cellcolor{gray!20} \textbf{0.172} & \cellcolor{gray!20} \textbf{0.638} & \cellcolor{gray!20} \textbf{0.156} & 30.500 & 12.907  \\ \hline 
 \caption{Valores das medidas de desempenho para análise do pseudo algoritmo \textit{PseudoSeg}, utilizando o texto o texto integral.}
 \end{longtable} 
 


\end{document} 
 

