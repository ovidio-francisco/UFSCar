
Alguns processos de anotações podem levar longos períodos, criando a necessidade de dividir a tarefa em fases. Nesses casos, frequentemente os anotadores fazem reuniões periódicas a fim de relatar eventuais problemas.  
% Em alguns projetos, abre-se espaço para discussão de pontos com baixa concordância, a qual é chamada de fase de ``reconciliação''. 
Em caso de baixa concordância, pode-se abrir espaço para discussão a fim de que encontrar um ponto de convergência, a qual é chamada de fase de ``reconciliação'' que embora recomendada, em alguns casos pode ocasionar um enviesamento dos resultados. Outra estratégia para baixa concordância é que solicitar que os anotadores marquem o nível de certeza sobre as anotações.
% Essa fase é recomendada para aumentar o nível de concordância, contudo pode ocasionar um enviesamento dos resultados. 





% Recomenda-se ainda, que os anotadores inciem com sub-tarefas relativamente simples, 



mas ser escolhido com vistas a ser utilizado em mais de um propósito


Além disso, 



Devido a diversidade linguística de diferentes domínios e gêneros de textos, a escolha dos documentos de amostra deve procurar ser representativa ao domínio a ser abordado. O corpus é considerado representativo quando o assunto a ser abordado e interpretado na amostra da mesma forma que no público geral desse domínio.


e interpretado na amostra da mesma forma que no público geral desse domínio.











O processo de anotação em corpus pode ser visto como a transformação de um texto puro em texto com marcações que podem ser interpretadas~\cite{Hovy2010}.	




% Produziu-se uma segmentação de referência colocando um marcação 
% questões sobre essa metodologia






% O processo de anotação consiste basicamente 


% produz um corpus anotado

% além de entender aspectos linguísticos envolvidos











