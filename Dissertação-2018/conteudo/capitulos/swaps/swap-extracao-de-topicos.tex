







% A matriz $A$ corresponde a matriz documento-tópico e possui dimensão $k \times n$. $Z$ corresponde a matriz termo-tópico e possui dimensão $m \times k$ sendo $n$ é o número de termos, $m$ é o número de documentos da coleção e $k$ é a quantidade de tópicos a serem extraídos. Uma vez que $k \ll n,m$, então $A$ e $Z$ são menores que a matriz de entrada, o que resulta em uma versão comprimida da matriz original, pois $k \cdot n + m \cdot k \ll n \cdot m$. Ao final, obtém-se uma representação documento-tópico que atribui um peso para cada tópico em cada documento da coleção e uma representação termo-tópico que representa a probabilidade de ocorrência de um termo em um documento dado que o tópico está presente no documento.















