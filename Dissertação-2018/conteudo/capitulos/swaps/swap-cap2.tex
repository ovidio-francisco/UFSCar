


Em seguida, os blocos de texto são representados por vetores que contém as frequências de suas palavras.  Diferente da proposta de Kosima, utiliza \textit{cosine} como medida para a similaridade entre os blocos adjacentes, conforme apresentada na Equação~\ref{equ:cosine}, onde dados dois blocos de texto, $x$ e $y$, $f_{x,j}$ é a frequência do termo $j$ em $x$ e $f_{y,j}$ é a frequência do termo $j$ em $y$.

\begin{equation}
	cosseno(x,y) = \frac
	{\Sigma_j f_{x,j} \times f_{y,j}}
	{\sqrt{\Sigma_j f^2_{x,j} \times \Sigma f^2_{y,j}}}
	\label{equ:cosseno}
\end{equation}



A popularidade dos computares permite a criação e compartilhamento de textos onde a quantidade de informação facilmente extrapola a capacidade de humana de leitura e análise de coleções de documentos, estejam eles disponíveis na Internet ou em computadores pessoais. A necessidade de simplificar e organizar grandes coleções de documentos criou uma demanda por modelos de aprendizado de máquina para extração de conhecimento em bases textuais. Para esse fim, foram desenvolvidas técnicas para descobrir, extrair e agrupar textos de grandes coleções, entre essas, a modelagem de tópicos~\cite{Hofmann1999,Deerwester1990,Lee1999,Blei2012}.  %--> não falar só de modelagem de tópicos. Falar de RI com referências


















