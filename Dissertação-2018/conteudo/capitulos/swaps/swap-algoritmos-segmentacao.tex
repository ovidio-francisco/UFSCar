estatístico 














% ==========  MinCut  ==========



Similar a abordagem do C99, o MinCutSeg vê o texto como uma matrix 
















1 - TextSeg
2 - MinCut
3 - BayesSeg


Lexical cohesion e.g., 
	- Utiyama and Isahara,     2001; {TextSeg}
	- Galley et al.            2003; {LCSeg}
	- Malioutov and Barzilay,  2006  {}








MinCutSeg segmenter (Malioutov & Barzilay 2006)




Assim, cada elemento é comparado com seus vizinhos dentro de uma região denominada máscara.
Nesse capítulo, inicialmente serão introduzidos alguns conceitos comuns 



Embora muitos trabalhos utilizem a coesão léxica do texto, para pequenos segmentos pode não ser confiável,





, em que dado dois blocos de texto $x$ e $y$,

conforme apresentada na , onde dados dois blocos de texto, $x$ e $y$, $f_{x,j}$ é a frequência do termo $j$ em $x$ e $f_{y,j}$ é a frequência do termo $j$ em $y$.

\begin{equation}
	Sim(x,y) = \frac
	{\Sigma_j f_{x,j} \times f_{y,j}}
	{\sqrt{\Sigma_j f^2_{x,j} \times \Sigma f^2_{y,j}}}
	\label{equ:cosine}
\end{equation}




analisa-se o texto circundante a cada candidato a limite.




A partir dos conceitos de coesão léxica, o \textit{TextTiling} foi um dos primeiros algoritmos propostos para segmentação textual. Baseia-se na

A partir desses conceitos, um dos primeiros algoritmos baseados na ideia que um segmento pode ser identificado pela análise das palavras que o compõe foi o \textit{TextTiling}. O \textit{TextTiling} é um algoritmo baseado em janelas deslizantes, em  que, para cada candidato a limite, analisa-se o texto circundante. O \textit{TextTiling} recebe uma lista de candidatos a limite, usualmente finais de parágrafo ou finais de sentenças. Para cada posição candidata são construídos 2 blocos, um contendo sentenças que a precedem e outro com as que a sucedem. O tamanho desses blocos é um parâmetro a ser fornecido ao algoritmo e determina o tamanho mínimo de um segmento. Esse processo é ilustrado na Figura~\ref{fig:TT-slidingwindow}.





% Kosima uniu as tecnicas de SW e CL.
% calcula-se a coesão léxica das palavras contidas na janela. 
% Os algoritmos que se baseiam no cálculo da similaridade entre sentenças, frequentemente o fazem por meio de janelas deslizantes, 

% 

% Os primeiros trabalhos dessa área se apoiam na ideia de que a mudança de assunto em um texto é acompanhada de uma proporcional mudança de vocabulário. Essa ideia, chamada de coesão léxica, sugere que a distribuição das palavras é um forte indicador da estrutura do texto~\cite{Kozima1993}. Demonstrou-se que há uma estreita correlação entre quedas na coesão léxica em janelas de texto e a transição de assuntos. Em seu trabalho, calculou a coesão léxica de uma janela de palavras usando \textit{spreading activation} em uma rede semântica especialmente elaborada para o idioma Inglês. Contudo, a implementação de um algoritmo para outros domínios depende da construção de uma rede adequada. 








% % ==========  Janelas deslizantes  ==========

% % Kosima uniu as tecnicas de SW e CL.
% Para encontrar os segmentos de um texto, alguns dos primeiros algoritmos utilizam a técnica de janelas deslizantes, onde se verifica a frequência dos termos em um fragmento do documento. Inicialmente, estabelece-se a partir do início do texto, um \textit{range} de $w$ termos, chamado janela que em seguida é deslocada em passos de $k$ termos adiante até o final do texto. A cada passo, analisa-se os termos contidos na janela.
% % calcula-se a coesão léxica das palavras contidas na janela. 
% % Os algoritmos que se baseiam no cálculo da similaridade entre sentenças, frequentemente o fazem por meio de janelas deslizantes, 


% %  Coesão léxica como presuposto básico
% % Para encontrar essas posições, 
% Trabalhos anteriores se apoiam na ideia de que a mudança de assunto em um texto é acompanhada de uma proporcional mudança de vocabulário. Essa ideia, chamada de coesão léxica, sugere que a distribuição das palavras é um forte indicador da estrutura do texto~\cite{Kozima1993}. O autor demonstrou que há uma estreita correlação entre quedas na coesão léxica em janelas de texto e a transição de assuntos. Em seu trabalho, calculou a coesão léxica de uma janela de palavras usando \textit{spreading activation} em uma rede semântica especialmente elaborada para o idioma Inglês. Contudo, a implementação de um algoritmo para outros domínios dependia da construção de uma rede adequada. 


