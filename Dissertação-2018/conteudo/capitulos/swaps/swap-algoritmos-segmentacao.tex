





% % ==========  Janelas deslizantes  ==========

% % Kosima uniu as tecnicas de SW e CL.
% Para encontrar os segmentos de um texto, alguns dos primeiros algoritmos utilizam a técnica de janelas deslizantes, onde se verifica a frequência dos termos em um fragmento do documento. Inicialmente, estabelece-se a partir do início do texto, um \textit{range} de $w$ termos, chamado janela que em seguida é deslocada em passos de $k$ termos adiante até o final do texto. A cada passo, analisa-se os termos contidos na janela.
% % calcula-se a coesão léxica das palavras contidas na janela. 
% % Os algoritmos que se baseiam no cálculo da similaridade entre sentenças, frequentemente o fazem por meio de janelas deslizantes, 


% %  Coesão léxica como presuposto básico
% % Para encontrar essas posições, 
% Trabalhos anteriores se apoiam na ideia de que a mudança de assunto em um texto é acompanhada de uma proporcional mudança de vocabulário. Essa ideia, chamada de coesão léxica, sugere que a distribuição das palavras é um forte indicador da estrutura do texto~\cite{Kozima1993}. O autor demonstrou que há uma estreita correlação entre quedas na coesão léxica em janelas de texto e a transição de assuntos. Em seu trabalho, calculou a coesão léxica de uma janela de palavras usando \textit{spreading activation} em uma rede semântica especialmente elaborada para o idioma Inglês. Contudo, a implementação de um algoritmo para outros domínios dependia da construção de uma rede adequada. 


