
\section{Segmentação Textual}


% -- Definição da Tarefa e de Segmentação
A tarefa de segmentação textual consiste em dividir um texto em partes ou segmentos que contenham um significado relativamente independente. Em outras palavras, é identificar as posições nas quais há uma mudança significativa de assuntos. As técnicas de segmentação textual consideram um texto como uma sequência linear de unidades de informação que podem ser, por exemplo, cada termo presente no texto, os parágrafos ou as sentenças. Cada unidade de informação é um elemento do texto que não será dividido no processo de segmentação e cada ponto entre duas unidades é considerado um candidato a limite entre segmentos. Nesse sentido, um segmento pode ser visto como uma sucessão de unidades de informação que compartilham o mesmo assunto.

