
\section{Modelos de Extração de Tópicos}
% Na venda do peixe, citar que a busca é tão difícil quanto a leitura e análise de coleções de documentos.

Os modelos de extração de tópicos são abordagens não-supervisionadas que visam descobrir padrões latentes nas relações entre os documentos e seus termos.  Baseiam-se na premissa de que um documento é produzido a partir de tópicos previamente definidos que determinam os termos a serem utilizados em um documento. Nesse contexto, um documento é uma mistura de tópicos onde cada termo presente no documento pode ser associado a um tópico. Um tópico por sua vez, é uma estrutura com valor semântico que é representada por um conjunto de termos e seus pesos que indicam o quão significante esses termos são para um assunto e pode ser útil para o  entendimento do tema do tópico~\cite{Steyvers2007,Blei2012}.

Para descobrir esses tópicos, algumas técnicas foram propostas. Em termos de metodologia, a maioria dos trabalho enquadram-se em duas principais categorias, os modelos não-probabilísticos e os modelos probabilísticos.


% -- \subsection{Modelos Não Probabilísticos}

Os modelos não-probabilísticas baseiam-se em técnicas de fatoração de matrizes, onde a matrix documento-termo é projetada em um espaço com menor dimensionalidade chamado \textit{Latent Semantic Space}. 
Seja
$d \in D = \{d_1,\cdots,d_n\}$ o vetor que representa a coleção de documentos, 
$t \in T = \{t_1,\cdots,t_m\}$ seus termos distintos e 
$z \in Z = \{z_1,\cdots,z_k\}$ seus tópicos. 
Esses métodos aprendem decompondo a matriz documento-termo $W$, em duas matrizes $Z$ e $A$, tal que a resultante de $ZA$ seja uma aproximação da matriz $W$ original. Mais formalmente tem-se:

\begin{equation}
	Z\cdot A = \hat{W} \approx W
\end{equation}

Sendo $n$ o número de termos, $m$ o número de documentos da coleção, $k$ a quantidade de tópicos a serem extraídos, a matriz $A$ corresponde a matriz documento-tópico e possui dimensão $k \times n$. $Z$ corresponde a matriz termo-tópico e possui dimensão $m \times k$. Uma vez que $k \ll n,m$, então $A$ e $Z$ são menores que a matriz de entrada, o que resulta em uma versão comprimida da matriz original, pois $k \cdot n + m \cdot k \ll n \cdot m$. Ao final, obtém-se uma representação documento-tópico que atribui um peso para cada tópico em cada documento da coleção e uma representação termo-tópico que representa a probabilidade de ocorrência de um termo em um documento dado que o tópico está presente no documento.

Nesse sentido, o \textit{Latente Semantic Indexing} (LSA)~\cite{Deerwester1990} usa a técnica chamada \textit{Singular Value Decomposition} (SVD) para encontrar padrões no relacionamento entre assuntos e termos em uma coleção de texto não estruturada. Entretanto, esse método não fornece uma interpretação para elementos com valores negativos~\cite{Deerwester1990}~\cite{Cheng2013}. % Trocar essa referência do Cheng2013 pela que ele usa na seção 2 do trabalho dele.

% -- NMF
Outro modelo popular é o \textit{Non-Negative Matrix Factorization} (NMF)~\cite{Lee1999}.  Diferente do LSA, no processo de fatoração apenas operações aditivas são permitidas, o que garante que as matrizes resultantes não possuem elementos negativos, permitindo uma interpretação mais intuitiva de seus valores. Além disso, o processo de fatoração proporciona a propriedade de \textit{clustering}, ou seja, agrupar as colunas da matriz $W$, e dessa forma, oferece a característica interessante de agrupar os documentos da coleção.  % -< Onde é utilizado


% -- \subsection{Modelos Probabilísticos}

Os modelos probabilísticos consideram os documentos como uma mistura de tópicos e um tópico como uma distribuição probabilística sobre os termos. O processo de elaboração do documento a partir desses tópicos é chamado de processo generativo ou modelo generativo, o qual é desconhecido porém pode ser estimado com base nos termos presentes no documento, também chamados de variáveis observáveis. Assim, o processo de extração de tópicos consiste em estimar o modelo generativo que deu origem ao documento.
% falar em algum ponto sobre o problema das matrizes esparsas. Principalmente com documentos pequenos
 
% -- PLSA
O PLSA~\cite{Hofmann1999} foi um dos primeiros a estender o modelo LSA e formalizar a extração de tópicos probabilísticos. De maneira similar ao LSA, esse modelo decompõe uma matriz esparsa a fim de reduzir a dimensionalidade. O PLSA cria um modelo estatístico chamado \textit{aspect model} que associa os tópicos às variáveis observáveis atribuindo probabilidades às ligações entre os tópicos e os documentos e entre as palavras e os tópicos. Assim, cada documento pode ser representado como a probabilidade de um tópico estar presente, $P(z|d)$. E a probabilidade de um termo ocorrer dado que um tópico esta presente, $P(t|z)$. Em comparação ao LSA, é considerado um método mais robusto por proporcionar uma interpretação probabilística. Por outro lado, esse modelo apresenta desvantagens como o número de parâmetros do modelo que cresce linearmente com o número de documentos da coleção, o que pode ocasionar \textit{overfitting}.   % - E o Expectation Maximization?

% -- LDA
A fim de contornar esses problemas, o LDA~\cite{Blei2003} estende o modelo PLSA incorporando um modelo generativo onde os cada tópico obedece à distribuição multivariada de \textit{Dirichlet} o que o torna menos propenso ao \textit{overfitting} e capaz de inferir tópicos a documentos ainda não observados. É referenciado na literatura como estado da arte sobre modelos probabilísticos de extração de tópicos e influencia uma grande quantidade de trabalhos, tornando-se base para novos modelos. No modelo LDA, o processo de geração de palavras se dá em duas etapas:

\begin{enumerate}
	\item Atribui-se uma distribuição aleatória sobre os tópicos.
	\item Para cada termo no documento:
		\begin{enumerate}
			\item Atribui-se aleatoriamente a um tópico da distribuição obtida na etapa 1;
			\item Seleciona-se aleatoriamente uma palavra do tópico correspondente.
		\end{enumerate}
\end{enumerate}

Assim cada documento é associado a múltiplos tópicos com proporções distintas (etapa 1). Cada palavra do documento é obtida de um tópico específico (etapa 2.b) que foi anteriormente obtido a partir da distribuição de tópicos do documento (etapa 2.a). Isso permite ao modelo LDA atribuir, para cada documento, múltiplos tópicos com proporções distintas~\cite{Blei2012}.

Os modelos de extração de tópicos foram inicialmente propostos para utilização em mineração texto onde são empregados na redução de dimensionalidade, extração de informações em textos, bem como na organização e recuperação de documentos, sendo utilizados para mensurar a relevância de um termo ou conjunto de termos para determinado assunto ou documento. Visto a popularidade nessas tarefas e flexibilidade dos modelos, logo notou-se sua utilidade em outros tipos de dados com atributos discretos como imagens, grafos e genética. 



