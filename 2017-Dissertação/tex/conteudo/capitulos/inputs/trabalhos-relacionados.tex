
% ========== ========== ========== ========== ========== ========== 

%  Descrição das Atas %  Ausência de marcaçãoes %  ... as lacunas %  Diferenças de entre Línguas 
Uma vez que uma mesma ata pode registrar vários assuntos discutidos na reunião, faz-se necessário dividi-la em segmentos. A aplicação dessas técnicas ao contexto das atas de reunião, requer algumas obervações no que se refere às características desses textos. As atas de reunião diferem dos textos comumente estudados em outros trabalhos em alguns pontos. O estilo de escrita mais sucinto, com poucos detalhes dificulta o processo de segmentação~\cite{Choi2001-LSA}. Embora haja abordagens voltadas para outros idiomas, nota-se um foco maior no idioma inglês, presente na maioria dos artigos publicados e apresenta uma taxa de erro significativamente menor que outros idiomas como o alemão e o espanhol~\cite{Kern2009,Sitbon2004}. Assim, ainda falta uma maior atenção na literatura sobre a língua portuguesa e a documentos com características próprias como as atas de reunião. % não tem um French também?

% ========== ========== ========== ========== ========== ========== 

