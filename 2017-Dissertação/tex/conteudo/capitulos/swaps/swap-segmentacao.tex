
% \subsection{Avaliação}
% \subsubsection{Medidas de Avaliação}

% 
% 
% 
% 
%%%%%%%%%%%%%%%
% Pk 
%%%%%%%%%%%%%%% 


% Considerando o conceito de \textit{near misses}, algumas soluções foram propostas sendo as mais utilizadas a P$_k$ e \textit{WindowDiff}. Proposta por~\cite{Beeferman1999}, P$_k$ atribui valores parciais a \textit{near misses}, ou seja, limites sempre receberão um peso proporcional à sua proximidade, desde que dentro de um janela de tamanho~$k$.  Para isso, esse método move uma janela de tamanho $k$ ao longo do texto.  A cada passo verifica, na referência e na hipótese, se o início e o final da janela estão ou não dentro do mesmo segmento, então, penaliza o algoritmo caso não concorde com a referência. Ou seja, dado duas palavras de distância $k$, o algoritmo é penalizado quando não concordar com a segmentação de referência se as palavras estão ou não no mesmo segmento.  O valor de $k$ é calculado como a metade da média dos comprimentos dos segmentos reais. Como resultado, é retornado a contagem de discrepâncias divida pelo quantidade de segmentações analisadas.  P$_k$ é uma medida de dissimilaridade entre as segmentações e pode ser interpretada como a probabilidade de duas sentenças extraídas aleatoriamente pertencerem ao mesmo segmento.  % TODO: Rafael: Não dá pra fazer uma fórmula matemática pra deixar o funcionamento dessa medida mais claro?


%\item Em~\cite{Beeferman1999} é apresentada uma medida denominada P$_k$, 
% Valores Parciais 
% Funcionamento 
% Compara hipotese e referência		
% Penalisa em caso de discrepânica 
% Calculo de k 
% Medida de dissimilaridade e interpretação 
%%%%%%%%%%%%%%%
%  WindowDiff
%%%%%%%%%%%%%%% 
%Em \cite{Pevzner2002} 
% Medida alternativa que considera outros aspectos 
% Funcionamento do Windiff 
% Solução para considerar o tamanho das sentenças 
% Solução para equilibrar falsos positivos e near missses 



Precisão é a fração de instâncias recuperadas que são relevantes, 
Revocação é a fração de instâncias relevantes que são recuperadas.

% As medidas de avaliação tradicionais como precisão e revocação são usadas em recuperação de informação e classificação automática para medir o desempenho de modelos de classificação e predição. São baseadas na comparação dos valores produzidos por uma hipótese com os valores reais. 

% Esses valores são apresentados em uma tabela que permite a visualização do desempenho de um algoritmo, a qual é chamada de matriz de confusão. Na Tabela~\ref{tab:matrizconfusao} é apresentada a matriz de confusão para duas classes (Positivo e Negativo). Uma matriz de confusão é uma tabela que permite a visualização do desempenho de um algoritmo. 



As medidas de avaliação tradicionais como precisão e revocação são usadas nos campos de recuperação de informação e aprendizado de máquina para medir o desempenho de seus modelos usando para isso a comparação dos valores produzidos por um algoritmo com os valores esperados.



As medidas de avaliação tradicionais como precisão e revocação comparam os valores produzidos por um algoritmo com os valores reais 
% --> Falar só de RI msmo



. São baseadas na comparação dos valores produzidos por uma hipótese com os valores reais. 












% --> Explicar melhor isso e devolver ao 2º parágrafo (na parte 'onde é usado')
A navegação pelo documento pode ser aprimorada, em especial na utilização por usuários com deficiência visual, os quais utilizam  sintetizadores de texto como ferramenta de acessibilidade~\cite{Choi2000}. 








O interesse por segmentação textual tem crescido em em aplicações voltadas a recuperação de informação %citar o [15] ...
e sumarização de textos~\cite{Maziero2016}. %... e [2] do "Efficient Linear T S"
Essa técnica pode ser usada para melhorar o acesso a informação quando essa é solicitada por meio de uma consulta, onde é possível oferecer porções menores de texto mais relevantes ao invés de exibir um documento grande que pode conter informações menos pertinentes.  Além disso, encontrar pontos onde o texto muda de assunto, pode ser útil como etapa de pré-processamento em aplicações voltadas ao entendimento do texto, principalmente em textos longos. 
A sumarização de texto pode ser melhorada ao processar segmentos separados por tópicos ao invés de documentos inteiros~\cite{Bhatia2016, Maziero2016, Bokaei2016}. 

% A tarefa de segmentação textual consiste em dividir um texto em partes ou segmentos que contenham um significado relativamente independente. Em outras palavras, é identificar as posições nas quais há uma mudança significativa de assuntos. Nesse sentido, um segmento pode ser visto como uma sucessão de unidades de informação que compartilham o mesmo assunto e cada ponto entre duas unidades é considerado um candidato a limite entre segmentos. 



% Na Figura~\ref{fig:exemplomatrixrank} é apresentado um quadro de dimensões 3~x~3 destacado na matriz de similaridades, que contém os valores $\{0,3; 0,4; 0,4; 0,6; 0,5; 0,2; 0,9; 0,5; 0,7\}$, onde cada elemento da matriz é a similaridade entre duas unidades de informação. Tomando como exemplo o elemento com valor $0,5$, a mesma posição na matriz de \textit{ranks} terá o valor $4$, pois esse é o número de vizinhos com valores inferiores a $0,5$ dentro do quadro analisado na matriz de similaridades. 


% As medidas de avaliação tradicionais como precisão e revocação computam os erros do algoritmo, isto é, falsos positivos e falsos negativos, a fim de calcular seu desempenho. Além dessas medidas, que consideram apenas se um segmento foi corretamente definido, pode-se também considerar a distância entre o segmento extraído automaticamente e o segmento de referência~\cite{Kern2009}. Chama-se \textit{near misses} o caso em que um limite identificado automaticamente não coincide exatamente com a referência, mas é necessário considerar a proximidade entre eles.




% Observa-se  melhores resultados de \textit{WindowDiff} quando os algoritmos aproximam a quantidade de segmentos automáticos da quantidade de segmentos da referência. %em torno de 10. 
% A configuração do tamanho do passo (P) e da proporção de segmentos em relação ao número de candidatos (S), influenciam os algoritmos na quantidade de segmentos extraídos. 








% -< Explicar que essas unidades que antecedem e antecedem não são 2 mas sim uma janela --> dá pra fazer uma figura também.


% Diferenças de performance podem ser vistas no mesmo algoritmo quando aplicado em documentos de diferentes idiomas, onde a aplicação em textos em inglês apresenta um taxa de erro significativamente menor que o alemão e o espanhol~\cite{Kern2009,Sitbon2004}.
% falta ainda uma maior atenção na literatura sobre língua portuguesa e a documentos com características próprias como as atas de reunião. 

% Há também um maior foco no idioma inglês, presente na maioria dos artigos publicados. Embora haja abordagens voltadas para outros idiomas, uma vez que

% De maneira geral, o algoritmo \textit{C99} apresentou melhores resultados em relação ao \textit{TextTiling}, contudo, testes estatísticos realizados indicaram que não houve diferença significativa entre os métodos. A etapa de pré-processamento proporciona melhora de desempenho quando aplicada, porém o seu maior benefício é a diminuição do custo computacional, uma vez que não prejudica a qualidade dos resultados.





