\chapter{Introdução}\label{chap:introducao}

% -- Motivação
% -< Classificação
% -- Tópicos
% -- Gaps da área
% -- Objetivos




%%%%%%%%%%
% Descrição das Reuniões
%%%%%%%%%%
Reuniões são tarefas presentes em atividades corporativas, ambientes de gestão e organizações de um modo geral. Seu conteúdo é frequentemente registrado em texto na forma de atas para fins de documentação e consulta posterior a qual devida a manual desses arquivos torna-se uma tarefa custosa, especialmente considerando o seu crescimento em uma instituição~\cite{Lee2011, Masakazu2013,Miriam2013}. 


%        ========== ==========   Descrição das Atas   ========== ==========



%%%%%%%%%%
% Descrição das Atas
%%%%%%%%%%
%  
%  Ausência de marcaçãoes 
% 
%
As atas de reunião possuem características particulares, frequentemente têm a característica de apresentar um texto com poucas quebras de parágrafo e sem marcações de estrutura, como capítulos, seções ou quaisquer indicações sobre o tema do texto. 
%
% Estilo compacto e formal desfavorece
%%% 
O estilo contribui para a escrita de textos sucintos com poucos detalhes, pois o ambiente dá preferência a textos curtos. 






    
 Os documentos apresentam um texto com poucas quebras de parágrafo e sem marcações de estrutura, como capítulos, seções ou quaisquer indicações sobre o assunto do texto. É comum a presença de cabeçalhos, rodapés e numeração de páginas e linhas o que pode prejudicar tanto similaridade entre sentenças como a apresentação dos segmentos ao usuário. %Esses elementos podem ser reduzidos ou eliminados como mostrado na Subseção~\ref{subsec:preprocessamento}, sobre preprocessamento.



%O estilo de escrita favorece a escrita de textos sucintos com poucos detalhes, pois o ambiente dá preferência a textos curtos. Ao redigir o documento tem-se o cuidado de não repetir ideias e palavras. Tal característica enfraquece a coesão léxica e portanto o cálculo da similaridade é prejudicado. Por exemplo, duas sentenças diferem se uma contiver a palavra \textit{computadores} e na seguinte \textit{equipamentos}, mesmo que se refiram à mesma ideia. Além disso, o documento compartilha um certo vocabulário próprio do ambiente onde os assuntos são discutidos e com isso nota-se que os segmentos, embora tratem de assuntos diferentes, são semelhantes em vocabulário.


% Para evitar um texto monótono à leitura, ao redigir o documento tem-se o cuidado de não repetir ideias e palavras em favor da elegância do documento. Tal característica enfraquece a coesão léxica e portanto o cálculo da similaridade é prejudicado. Por exemplo, duas sentenças diferem se uma contiver a palavra \textit{computadores} e na seguinte \textit{equipamentos}, mesmo que se refiram à mesma ideia. Além disso, o documento compartilha um certo vocabulário próprio do ambiente onde os assuntos são discutidos e com isso nota-se que os segmentos, embora tratem de assuntos diferentes, são semelhantes em vocabulário.










% As atas, apresentam-se como uma sucessão de tópicos. Assim, o objetivo desse trabalho é identificar, automaticamente, onde há a mudança de um tópico para seus adjacentes.






%        ========== ==========   Motivação   ========== ==========



%%%%%%%%%%
% Atas são documentos não estruturados e
% diferencial do sistema
%%%%%%%%%%
As atas são documentos textuais que em geral descrevem dados não estruturados. Assim, um sistema que responde a consultas do usuário ao conteúdo das atas, retornando trechos de textos relevantes à sua intenção, é um desafio que envolve a compreensão de seu conteúdo~\cite{Bokaei2015}. 






%        ========== ==========   Busca manual   ========== ==========




%%%%%%%%%%%%%%%
% Busca manual por Assuntos
%%%%%%%%%%%%%%% 
%  É custosa 
%  Memória e Buscas por KeyWords
%  Problemas na busca por palavras-chave
%    O usuário deve inserir palavras acertadas
%    O texto continua longo, apenas é apresentado com as palavras destacadas
%    Não é possível rankear os resultados
%  
Devido a fatores como a não estruturação e volume dos textos, a localização de assunto em uma ata é uma tarefa custosa.  Usualmente, o que se faz são buscas manuais guiadas pela memória ou com uso de ferramentas computacionais baseadas em localização de palavras-chave.  Normalmente esse tipo de busca exige a inserção de termos exatos e 
% de termos exatos que devem necessariamente estar contidos no trecho onde está o assunto.
apresentam ao usuário um documento com as palavras buscadas em destaque, mantendo o texto longo, o que dificulta por exemplo o ranqueamento por relevância. 

% Não evita que o texto circundante seja exibido, caso pertença a outro assunto
% Além disso, não evita que textos próximos sejam retornados.




%           ========== ==========  Segmentação  ========== ==========
%  ==============================================================================

%%%%%%%%%%
% Segmentar e extrair tópicos
%%%%%%%%%%
Uma vez que a ata registra a sucessão de assuntos discutidos na reunião, há interesse em um sistema que aponte trechos de uma ata que tratam de um assunto específico. Tal sistema tem duas principais tarefas: 1) Descobrir quando há uma mudança de assunto. 2) Descobrir quais são esses assuntos. Este trabalho tem como foco principal, a detecção de mudança de assuntos, que pode ser atendida pela segmentação automática de textos~\cite{Chen2017,Naili2016,Cardoso2017}. E a extração de automática de tópicos. 



%%%%%%%%%%
% Definição da Tarefa de Segmentação
%%%%%%%%%%
A tarefa de segmentação textual consiste em dividir um texto em partes que contenham um significado relativamente independente. Em outras palavras, é identificar as posições nas quais há uma mudança significativa de assuntos. 

%%%%%%%%%%
% Utilidade da segmentação (domínios onde é aplicada)
%%%%%%%%%%
A segmentação de textos é útil em aplicações que trabalham com textos sem indicações de quebras de assunto, ou seja, não apresentam seções ou capítulos, como transcrições automáticas de áudio, vídeos e grandes documentos que contêm vários assuntos como atas de reunião e notícias.


Pode ser usada para melhorar o acesso a informação solicitada por meio de uma consulta, onde é possível oferecer porções menores de texto mais relevantes ao invés de exibir um documento grande que pode conter informações menos pertinentes. 
%
%
A navegação pelo documento pode ser aprimorada, em especial na utilização por usuários com deficiência visual, os quais utilizam  sintetizadores de texto como ferramenta de acessibilidade~\cite{Choi2000}. 
%
Além disso, encontrar pontos onde o texto muda de assunto, pode ser útil como etapa de pré-processamento em aplicações voltadas ao entendimento do texto, principalmente em textos longos.




%A tarefa de segmentação textual consiste em encontrar pontos onde há mudança de tópicos no texto.  %Em outras palavras é identificar divisões entre unidade de informação sucessivas

%[ The task of linear text segmentation is to split a large text document into shorter fragments, usually blocks of consecutive sentences. ]

% **Segmentação é identificar divisiões entre unidades de informação sucessivas (Beeferman, Berger, and Lafferty (1997))**
 
% [ Text segmentation is the task of determining the positions at which topics change in a stream of text ]




