\section{Conclusão}
	\label{sec:conclusao}

As atas de reunião, objeto de estudo desse artigo, apresentam características peculiares em relação à discursos e textos em geral. Características como segmentos curtos e coesão mais fraca devida ao estilo que evita repetição de palavras e ideias em benefício da leitura por humanos, dificultam o processamento por computadores.

%%%%%%%%%%
% Benefícios 
%%%%%%%%%%


Os algoritmos \textit{TextTiling} e \textit{C99} foram testados em um conjunto de atas coletadas do Departamento de Computação da UFSCar-Sorocaba. Por meio da análise dos dados chegou-se a uma configuração cujos segmentos melhor se aproximaram das amostras segmentadas por participantes das reuniões. 


Na maioria das medidas, o \textit{C99} sobressaiu-se em relação ao \textit{TextTiling}, contudo, os testes estatísticos, não apresentam diferença significativa. 


%%%%%%%%%%%%%%%
% O Impacto do Preprocessamento 
%%%%%%%%%%%%%%%

Da mesma forma, a etapa de preprocessamento proporciona melhora de performance quando aplicada, porém o seu maior benefício é a diminuição do custo computacional, uma vez que não prejudica a qualidade dos resultados.



A segmentação de atas de reunião pode ajudar na organização, busca e compreensão dos conteúdos nelas contidos. Também outros domínios e aplicações diferentes podem se beneficiar dos resultados apresentados, como aplicações voltadas a resgate de informação, sumarização e acessibilidade. Assim, espera-se que outros trabalhos possam aproveitar deste.
	
	
Em trabalhos futuros, serão investigadas técnicas de extração de tópicos para descrever os segmentos e com isso aprimorar o acesso ao conteúdo das atas de reunião.




