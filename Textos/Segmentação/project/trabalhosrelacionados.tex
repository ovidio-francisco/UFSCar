% Ideia básica dos algorítmos (Coesão léxica ) como **presuposto básico**

Os principais algoritmos de segmentação textual baseiam-se na ideia de coesão léxica entre assuntos. Isto é, a mudança de tópicos é acompanhada de uma proporcional mudança de vocabulário. A partir disso, vários algoritmos foram propostos. Dessa forma, assumem o pressuposto que um segmento pode ser identificado e delimitado pela análise das palavras que o compõe.

% A coesão léxica é um termômetro para as mudanças de tópicos, e portanto, um indicador para quebras de segmento.

 
 
% Nesse artigo, os principais serão analisados na perspectiva de atas de reunião.


%Os principais algoritmos de segmentação textual assumem o pressuposto que um segmento pode ser identificado e delimitado pela análise de seu vocabulário





%Os entre os principais trabalhos relacionados a segmentação textual estão o \textit{TextTiling} e o \textit{C99}



%\subsubsection{TextTiling}
%	O algoritmo TextTiling, proposto por 
	
%
%\subsubsection{C99}



Entre os mais influentes podemos citar o \textit{TextTiling}~\cite{Hearst1994} 




Semelhante a esse trabalho, outras abordagens foram propostas como ...



%\cite{Pevzner200219}
%
%\cite{Banerjee200657}
%
%\cite{Salton199653}
%
%\cite{Kern2009167}
%
%\cite{Beeferman1999}
%
%\cite{CHAIBI2014437}
%\cite{Bokaei2015}
%\cite{Galley2003}


















