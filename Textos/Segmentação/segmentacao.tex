
\documentclass{sig-alternate-05-2015}
\usepackage[portuguese]{babel}
%\usepackage{multirow}
%\usepackage{adjustbox}
%\usepackage{graphicx}
%\usepackage{array}
%\usepackage{tabulary}
% \usepackage{pgfplotstable}
% recommended:
%\usepackage{booktabs}
%\usepackage{array}
%\usepackage{colortbl}
%\usepackage{emptypage}
%\newcolumntype{l}[1]{>{\centering\arraybackslash}p{#1}}
% \usepackage{showframe}   %% just for demo
% \usepackage[brazilian,hyperpageref]{backref}	 % Paginas com as citações na bibl
% \usepackage[alf]{abntex2cite}	% Citações padrão ABNT
\usepackage[utf8]{inputenc}		% Codificacao do documento (conversão automática dos acentos)

\usepackage{rotating}
\usepackage{tabularx}
\usepackage{multicol}
\usepackage{amsmath}
\usepackage[linesnumbered,ruled]{algorithm2e}

\usepackage{caption}
\usepackage{subcaption}




% Define o caminho das figuras
\graphicspath{{images/}}



\begin{document}

\title{Segmentação topical automática de atas de reunião}


\numberofauthors{1} 

\author{
\alignauthor Ovídio José Francisco\\
       \email{ovidiojf@gmail.com}
}

\maketitle

%\begin{abstract}   
%
%\end{abstract}

\section*{RESUMO}



\keywords{}

\begingroup
\let\clearpage\relax

	Frequentemente atas de reunião tem a característica de apresentar um texto com poucas quebras de parágrafo e sem marcações de estrutura, como capítulos, seções ou quaisquer indicações sobre o tema do texto. 


% Definição 

A tarefa de segmentação textual consiste dividir um texto em partes que contenham um significado relativamente independente. Em outras palavras, é identificar as posições onde há uma mudança significativa de tópicos.

% Usos
É útil em aplicações que trabalham com textos sem quebras de assunto, ou seja, não apresentam parágrafos, seções ou capítulos, como transcrições automáticas de áudio e grandes documentos que contêm assuntos não idênticos como atas de reunião e noticias.


% Interesses
O interesse por segmentação textual tem crescido em em aplicações voltadas a recuperação de informação %citar o [15] ...
e sumarização de textos. % ... e [2] do "Efficient Linear T S"
Essa técnica pode ser usada para aprimorar o acesso a informação quando essa é solicitada por um usuário por meio de uma consulta, onde é possível oferecer porções menores de texto mais relevante ao invés de exibir um documento maior que pode conter informações menos pertinente. A sumarização de texto também pode ser aprimorada ao processar segmentos separados por tópicos ao invés de documentos inteiros.




% As Atas
Assim, esse trabalho trata da adaptação e avaliação de algoritmos tradicionais ao contexto de documentos em português do Brasil, com ênfase especial nas atas de reuniões.


%As atas, como frequentemente são, apresentam-se como uma sucessão de tópicos. Assim, o objetivo desse trabalho é identificar, automaticamente, onde há a mudança de um tópico para seus adjacentes.


% Diversas aplicações fazem uso de segmentação textual, incluindo 

% Entre as principais mais frequentes de segmentação textual estão a tra



%É principalmente utilizada em aplicações que processam textos longos como transcrições de áudio e documentos longos, além de aprimorar técnicas de sumarização e information retrievel.



% Usos:
%	* quando não há identificações
%	* em transcrições de áudio
%	* em documentos longos
% 	* text summarization (ver a referencia [2] de Efficient Linear Text...)




%Isto é, dado um texto, identificar onde há mudança de tópicos.


% Interest in automatic text segmentation has blossomed over the last few years, with applications ranging from information retrieval to text summariza-tion to story segmentation of video feeds. [A Critique and Improvement of an Evaluation Metric for Text Segmentation]



%Em outras palavras é identificar divisões entre unidade de informação sucessivas

%A tarefa de segmentação textual consiste em encontrar pontos onde há mudança de tópicos no texto.



%[ The task of linear text segmentation is to split a large text document into shorter fragments, usually blocks of consecutive sentences. ]


% **Segmentação é identificar divisiões entre unidades de informação sucessivas (Beeferman, Berger, and Lafferty (1997))**

%   [Text segmentation is the task of determining the positions at which topics change in a stream of text]





	% Ideia básica dos algorítmos (Coesão léxica ) como **presuposto básico**

Os principais algoritmos de segmentação textual baseiam-se na ideia de coesão léxica entre assuntos. Isto é, a mudança de tópicos é acompanhada de uma proporcional mudança de vocabulário. A partir disso, vários algoritmos foram propostos. Dessa forma, assumem o pressuposto que um segmento pode ser identificado e delimitado pela análise das palavras que o compõe.







% A coesão léxica é um termômetro para as mudanças de tópicos, e portanto, um indicador para quebras de segmento.

 
 
% Nesse artigo, os principais serão analisados na perspectiva de atas de reunião.


%Os principais algoritmos de segmentação textual assumem o pressuposto que um segmento pode ser identificado e delimitado pela análise de seu vocabulário





%Os entre os principais trabalhos relacionados a segmentação textual estão o \textit{TextTiling} e o \textit{C99}



%\subsubsection{TextTiling}
%	O algoritmo TextTiling, proposto por 
	
%
%\subsubsection{C99}



Entre os mais influentes podemos citar o \textit{TextTiling}~\cite{Hearst1994} 




Semelhante a esse trabalho, outras abordagens foram propostas como ...

\cite{Banerjee200657} faz uma adaptação do \textit{TextTiling} ao contexto das conversas em reuniões com múltiplos participantes.  



%%%%%%%
% C99 %
%%%%%%%

Choi \cite{Choi2000} apresenta um trabalho que usa \textit{cosine} como medida similaridade e apresenta um esquema de ranking em seu algoritmo, o C99.
%
Embora muitos dos melhores trabalho utilizarem matrizes de similaridades, o autor traz obervações.
%
Ele aponta que para pequenos segmentos, o cálculo de suas similaridades não é confiável. Pois uma ocorrência adicional de uma palavra causa um impacto desproporcional no cálculo.
%
Além disso, o estilo da escrita pode não ser constante em todo o texto. Choi sugere que, por exemplo, textos iniciais dedicados a introdução costumam apresentar menor coesão do que trechos dedicados a um tópico específico. Portanto comparar a similaridade entre trechos de diferentes regiões, não é apropriado.
% Complexidade O(n²)
Devido a isso, as similaridades não podem ser comparadas em valores absolutos. O autor apresenta um esquema de ranking para contornar esse problema.

Cada valor na matriz similaridade é substituída por seu ranking local. O ranking é o número de elementos vizinhos com similaridade menor, conforme a imagem abaixo.


\begin{equation}
Sim(x,y) = \frac
{\Sigma_j f_{x,j} \times f_{y,j}}
{\sqrt{\Sigma_j f^2_{x,j} \times \Sigma f^2_{x,j}}}
\end{equation}



\begin{equation}
r(x,y) = \frac
{Numero\ de\ elementos\ com\ similaridade\ menor}
{Numero\ de\ elementos\ examinados}
\end{equation}













	\section{Proposta: Segmentação Linear Automática de Atas de Reunião}
	\label{sec:proposta}






%%%%%%%%%%
% TextTiling e C99 criados para inglês e independente de domínio
%%%%%%%%%%
Os algoritmos \textit{TextTiling} e \textit{C99} foram propostos para o inglês, independente de domínio, ou seja, a proposta inicial dos autores é trabalhar em qualquer texto nessa língua.
%%%%%%%%%%
% Adapatar para Atas em português
%%%%%%%%%%
Assim, propõe-se adaptá-los ao contexto das atas de reunião em português do Brasil, ou seja, em uma língua diferente e dentro de um contexto específico. As subseções seguintes tratam das adaptações para esse nicho mais específico. A Seção~\ref{sec:avaliacao} mostra a análise dos algoritmos adaptados.

%%%%%%%%%%
% Dificuldade: Coesão léxica não tão bem definida
%%%%%%%%%%
O vocabulário das reuniões, ainda que em tópicos diferentes, compartilham certo vocabulário pertencente ao ambiente onde se deram as reuniões. Isso é um fator que diminui a coesão léxica entre os segmentos.
%%%%%%%%%%
% Dificuldade: estilo da escrita
% - Paragrafo único
% - Cabeçalhos e rodapés
% - Pontuação --> ; encerrando sentenças
% - Insersão de espaços que não são quebra de sentença
% - Ruídos
%%%%%%%%%%
As atas de reunião costumam ter um estilo de escrita que deve ser levado em conta na adaptação do algoritmos, como a identificação de finais de sentença na ausência de quebras de parágrafo, inserção de linhas que não separam assuntos, utilização de pontuação para transição de tópicos, e cabeçalhos e numerais ruidosos. 

Nas subseções a seguir serão apresentados o pré-processamento e a identificação de segmentos candidatos considerados para a segmentação de atas.





\subsection{Pré-processamento}
	\label{subsec:preprocessamento}




	O texto a ser segmentado frequentemente é extraído de documentos em formatos como \textit{pdf}, \textit{doc}, \textit{docx} ou \textit{odt}. Após a extração do texto, esse pode passar por processos de transformação os quais serão apresentados a seguir.
	
	A etapa de pré-processamento, em um documento contendo texto puro, acontece em quatro passos principais. 

	
\begin{enumerate}


%
% 1) Identificação de sentenças
\item \textit{Identificação de sentenças}: 
Cada final de sentença e identificado e marcado com uma \textit{string} especial, esse processo é melhor descrito na Subseção~\ref{subsec:indentificacaosentencas}.
%
%conforme mostrado no Algoritmo~\ref{alg:identificacaofinaisdesent}.

%	
% 2) Eliminação de ruídos
\item \textit{Remoção de ruídos}:

%%%%%%%%%%
% Cabeçalhos e rodapés
%%%%%%%%%%
As atas frequentemente contém trechos que podem ser considerados pouco informativos e descartados durante o pré-processamento, como cabeçalhos e rodapés que se misturam aos tópicos tratados na reunião, podendo ser  inseridos no meio de um tópico e criando uma quebra que prejudica tanto o algoritmo de segmentação, quanto a leitura do texto pelo usuário.
%%%%%%%%%%
% Numerais
%%%%%%%%%%
Também é comum o uso de numerais para marcação de páginas e linhas, da mesma forma, são pouco informativos e podem ser descartados.


%%%%%%%%%%
% Remoção de ruídos
%%%%%%%%%%
Nesse trabalho, esses elementos são removidos, uma vez que, o descarte não causa perca de informação e pode facilitar a identificação dos segmentos, pois melhora a coesão do texto. Outro benefício é manter o texto livre de trechos que fogem do assunto circundante.
%%%
% TODO: Como são removidos?
% por meio de heurísticas simples
%%%

Elimina-se também nesse passo a acentuação, sinais de pontuação, restando apenas palavras.  



%	
% 3) Stop Words
\item \textit{Stop Words}: 
Remove-se as palavras consideradas menos informativas, as quais são chamadas de \textit{stop words}, para isso, utiliza-se uma lista contendo 438 palavras. 

%
% 4) Stemming
\item \textit{Stemming}:
Extrai-se o radical de cada palavra, para isso, as letras são convertidas em caixa baixa e aplica-se o algoritmo \textit{Orengo}\footnote{\urlorengo} para remoção de sufixos.


\end{enumerate}
	
	



Tem-se então, uma lista com os elementos considerados mais significativos do texto. A Figura~\ref{fig:exemplopreprocessamento} mostra a etapa de pré-processamento em uma sentença em português.
	



  \begin{figure}[!h]
	\centering
	\includegraphics[width=0.45\textwidth]{exemplo-preprocessamento.jpg}
	\caption{Exemplo de pré-processamento.}
	\label{fig:exemplopreprocessamento}
  \end{figure}






\subsection{Identificação de candidatos}
	\label{subsec:indentificacaosentencas}
	
%%%%%%%%%%	
% Indicar unidade mínima de Segmento
%%%%%%%%%%
	
	
%	Como entrada para os 
%	Os algoritmos de segmentação devem ser
	É preciso fornecer aos algoritmos os candidatos iniciais a limites de segmento. Antes disso, é necessário escolher qual será a unidade de informação mínima que constitui um segmento. Baseando-se no estilo de escrita e considerando as pontuações dos textos, é possível indicar quebras de parágrafo, finais de sentenças ou mesmo palavras como elementos que encerram um segmento. 

	Ocorre que em atas de reunião é uma prática comum redigi-las de forma que o conteúdo discutido fica em parágrafo único, além disso, as quebras de parágrafo são usados para formatação de outros elementos como espaço para assinaturas. 

	Indicar todo ponto entre \textit{token} como candidato obriga a ajustar posteriormente os segmentos de maneira a não quebrar uma ideia ou frase. Assim, neste trabalho, os finais de sentença são considerados unidades de informação e portanto, passíveis a limite entre segmentos. 
	
	Devido ao estilo de pontuação desses documentos, como encerrar sentenças usando um \textit{";"} e inserção de linhas extras, foram usadas as regras apresentadas no Algoritmo 1 para identificar os finais de sentenças.  


\begin{algorithm}
	\SetKwInOut{Input}{Entrada}
	\SetKwInOut{Output}{Saída}
	\SetKwBlock{Inicio}{início}{fim}
	\SetKwFor{ParaTodo}{para todo}{}{fim para todo}
	\SetKwIF{Se}{SenaoSe}{Senao}{}{}{senao se}{senao}{fim se}
	\SetKwFor{Para}{}{}{}
%	\SetKwAlgorithm{Algorithm}{Algoritmo}{}

	
	\Input{Texto}
	\Output{Texto com identificações de finais de sentença}
	
	\ParaTodo {token, marcá-lo como final de sentença se:} {	

	Terminar com um \texttt{!}\\
	Terminar com um \texttt{.} e não for uma abreviação\\
	Terminar em \texttt{.?;} e:
		\Para{}{
			For seguido de uma quebra de parágrafo ou tabulação\\
			O próximo \textit{token} iniciar com  \texttt{(\{["'}\\
			O próximo \textit{token} iniciar com letra maiúscula\\
			O penúltimo caracter  for \texttt{)\}]"'}\\
		}
	}
	
	\caption{Identificação de finais de sentença}
	\label{alg:identificacaofinaisdesent}
\end{algorithm}









%Como forma de padronização, as instituições acrescentam ao documento

%		passos menores
%		1 - heurística simples para remover cabeçalho e rodapé.
%		2 - remoção de numerais
%		3 - remoção de acentos, transformações de caixa, remoção de pontuação.
		
	% Esses passos são realizados internamente em cada algorímo, para que a saida seja legível ao usuário final.





%A qualidade do algoritmo é sempre dependente da escrita correta! Ausência de emoticons, códigos de computador e gírias.





%Nas subseções a seguir serão expostas as alterações para aumentar a eficiência dos algoritmos e encontrar o melhor modelo para a tarefa de segmentar o texto das atas em tópicos.



%Tais aspectos não se aplicam ao contexto das atas, onde o estilo de escrita em forma de narrativa, prefere poupar o leitor de diálogos secundários durante transições de tópicos. 







	
	
%Elimina-se textos considerados de pouca relevância como a numeração de páginas e linhas, cabeçalhos e rodapés, também elemina-se acentuação, sinais de pontuação, restando apenas palavras.  


	
	
%%
%% 1) Identificação de sentenças
%Primeiro cada final de sentença e identificado e marcado com uma \textit{string} especial, esse processo é melhor descrito na Subseção~\ref{subsec:indentificacaosentencas}.
%%
%%conforme mostrado no Algoritmo~\ref{alg:identificacaofinaisdesent}.
%
%%	
%% 2) Eliminação de ruídos
%Depois, elimina-se textos considerados de pouca relevância como a numeração de páginas e linhas, cabeçalhos e rodapés, também elemina-se acentuação, sinais de pontuação, restando apenas palavras.  
%
%%	
%% 3) Stop Words
%Em seguida, remove-se as palavras consideradas menos informativas, as quais são chamadas de \textit{stop words}, para isso, utiliza-se uma lista contendo 438 palavras. 
%
%%
%% 4) Stemming
%Por fim, extrai-se o radical de cada palavra, para isso, as letras são convertidas em caixa baixa e aplica-se o algoritmo \textit{Orengo}\footnote{\urlorengo} para remoção de sufixos.
%
%
%Tem-se então, uma lista com os elementos considerados mais significativos do texto. A Figura~\ref{fig:exemplopreprocessamento} mostra a etapa de pré-processamento em uma sentença em português.
%	





	
	
\section{Avaliação}
	\label{sec:avaliacao}



%%%%%%%%%% 
% Necessidade de uma referência
%%%%%%%%%%
Para que se possa avaliar um segmentador automático de textos, é preciso uma referência, isto é, um texto com os limites entre os segmento conhecidos. Essa referência, deve ser confiável, sendo uma segmentação legítima que é capaz de dividir o texto em porções relativamente independentes, mantendo um conteúdo legível, ou seja, uma segmentação ideal.
%

Entre as abordagens mais comuns para se conseguir essas referências, encontramos: A concatenação aleatória de documentos distintos, onde o ponto entre o final de um texto e o inicio do seguinte é um limite entre eles. A segmentação manual dos documentos, nesse caso, pessoas capacitadas, também chamadas de juízes, ou mesmo o autor do texto, são consultadas e indicam manualmente onde há uma quebra de segmento. Em transcrição de conversas faladas em reuniões com múltiplos participantes, um mediador é responsável por encerrar um assunto e iniciar um novo, nesse caso o mediador anota manualmente o tempo onde há uma transição de tópico. Em aplicações onde a segmentação é tarefa secundária, é possível, ao invés de avaliar o segmentador, analisar seu impacto na aplicação final.


%%%%%%%%%%
% As 2 principais dificuldades na avaliação
%%%%%%%%%%
De acordo com \cite{Pevzner2002} há duas principais dificuldades na avaliação de segmentadores automáticos. A primeira é conseguir um referência, já que juízes humanos costumam não concordar entre si, sobre onde os limites estão e outras abordagens podem não se aplicar ao contexto. A segunda é que tipos diferentes de erros devem ter pesos diferentes de acordo com a aplicação. Há casos onde certa imprecisão é tolerável e outras, como a segmentação de notícias, onde a precisão é mais importante.


%%%%%%%%%%
% Definição do que é um bom algoritmo de segmentação
%%%%%%%%%%
Para fins de avaliação desse trabalho, um bom método de segmentação é aquele cujo resultado melhor se aproxima do ideal, sem a obrigatoriedade de estar perfeitamente alinhado com tal. Ou seja, visto o contexto das atas de reunião, e a subjetividade da tarefa, não é necessário que os limites entre os segmentos (real e hipótese) sejam idênticos, mas que se assemelhem em localização e quantidade.


%Para quantificar a eficiência dos algoritmos, segue uma revisão das principais métricas aplicáveis.

As próximas subseções mostram o conjunto de atas e a segmentação usada como referência, uma revisão das principais métricas aplicáveis à segmentação e os testes realizados para avaliar os métodos.

\subsection{Conjunto de documentos}
	A fim de obter um conjunto de documentos segmentados que possam servir como referência na avaliação, seis atas de reunião foram coletadas junto ao departamento de computação da UFSCar-Sorocaba. Os documentos foram oferecidos à profissionais que participam de reuniões desse departamento e por meio de um \textit{software} segmentaram o texto das atas conforme o julgamento de cada um. Os segmentos gerados manualmente foram comparados à segmentação automática conforme os critérios descritos a seguir.
	
	As atas de reunião diferem dos textos comumente estudados em outros trabalhos em alguns pontos. O estilo de escrita favorece textos sucintos com poucos detalhes de maneira que o ambiente dá preferência a textos curtos. Segundo Choi~\cite{Choi2001-LSA}, o segmentador tem a acurácia reduzida em segmentos curtos (em torno de 3 a 5 sentenças).
	
	Para evitar um texto monótono à leitura, a redação do documento tem o cuidado de não repetir ideias e palavras em favor da elegância do texto. Tal característica enfraquece a coesão léxica e portanto o cálculo da similaridade é prejudicado. Por exemplo, duas sentenças diferem se uma contiver a palavra \textit{computadores} e na seguinte \textit{equipamentos}, mesmo que se refiram à mesma ideia.
	
	Além disso, o documento compartilha um certo vocabulário próprio do ambiente onde os assuntos são discutidos e com isso nota-se que os segmentos, embora tratem de assuntos diferentes, são semelhantes em vocabulário.
	
A presença de ruídos como cabeçalhos, rodapés e numeração de páginas e linhas prejudicam tanto similaridade entre sentenças como a apresentação final ao usuário. Porém, esses ruídos podem ser reduzidos ou eliminados como mostrado na Subseção~\ref{subsec:preprocessamento}, sobre preprocessamento.


\subsection{Medidas de Avaliação}


	As medidas de avaliação tradicionalmente utilizadas em \textit{information retrieval} como precisão e revocação trazem alguns problemas na avalização de segmentadores automáticos.  
Conforme o algoritmo aponta mais segmentos no texto, tende a melhorar a revocação e ao mesmo tempo, reduzir a precisão, um problema que pode ser contornado usando \textit{F-measure} que faz uma combinação da duas levando em conta seus pesos, o que por outro lado é mais difícil de interpretar. 
Essas medidas falham ao não serem sensíveis a \textit{near misses}, ou seja, quando um limite não coincide exatamente com o esperado, mas fica próximo a ele~\cite{Kern2009}.

A Figura~\ref{fig:exemplosegmentacaozoom} mostra um exemplo com duas segmentações hipotéticas e uma referência. Na Figura~\ref{fig:exemplosegmentacao}, em ambos os casos não há nenhum verdadeiro positivo, o que implica em zero para os valores de precisão, acurácia, e revocação, embora a segunda hipótese possa ser considerada superior à primeira se levado em conta a proximidade dos limites.



  \begin{figure}[!h]

	\centering
	\includegraphics[width=0.47\textwidth]{windiffzoom.jpg}
	\caption{Exemplos de \textit{near missing} e falso positivo puro. Os blocos indicam uma unidade de informação e as linha verticais representam os limites entre segmentos. }
	\label{fig:exemplosegmentacaozoom}

  \end{figure}
  
  \begin{figure}[!h]

	\centering
	\includegraphics[width=0.47\textwidth]{windiff.jpg}
	\caption{
	Exemplo de duas segmentações hipotéticas em comparação a uma ideal. 
	}
	\label{fig:exemplosegmentacao}

  \end{figure}
  
  
Entre as medidas mais utilizadas para avaliar segmentadores estão:

\subsubsection{P$_k$}
A fim de resolver o problema de \textit{near misses}, Beeferman \textit{et. al.}~\cite{Beeferman1999} apresentam uma nova medida chamada P$_k$ que atribui valores parciais a \textit{near misses}. Esse método move uma janela de tamanho $k$ e a cada posição e verifica se o início e o final da janela estão ou não dentro do mesmo segmento e penaliza o algoritmo em caso de discrepância. 

Ou seja, dado duas palavras de distancia $k$, uma discrepância é computada quando o algoritmo e a referência não concordam se as palavras estão ou não no mesmo segmento.

O valor de $k$ é calculado como a metade da média dos comprimentos dos segmentos reais. Como resultado, é retornado a contagem de discrepâncias divido pelo quantidade de segmentações analisadas. Esse valor serve como medida de dissimilaridade entre as segmentações e pode ser interpretada como a probabilidade de duas sentenças extraídas aleatoriamente pertencerem ao mesmo segmento.



\subsubsection{WindowDiff}

Pevzner~\cite{Pevzner2002} aponta problemas na avaliação mais tradicional P$_k$~\cite{Beeferman1999}. Eles apontam que esse método penaliza demasiadamente os falsos negativos em relação aos falsos positivos e a \textit{near misses}, além disso, desconsidera o tamanho e a quantidade de segmentos, entre outros problemas.

Como solução, propõem um novo método, o qual chamam de \textit{WindowDiff} que traz duas diferenças principais: a dobra a penalidade para os falsos positivos a fim de diminuir o problema da subestimação dessa medida e, diferente de P$_k$, ao mover a janela pelo texto, penaliza o algoritmo sempre que o número de limites proposto pelo algoritmo não coincidir com o número de limites esperados para aquela janela de texto. 

Com isso, demonstram em seu trabalho que, em relação a P$_k$, consegue resolver seus principais problemas e mantém sua proposta inicial de sensibilidade a \textit{near misses}, penalizando-os menos que os falsos positivos puros.


  

%Falar do software para segmentação manual????


\subsection{Avaliação dos segmentadores}


%%%%%%%%%%
% Parâmetros
%%%%%%%%%%
As implementações dos algoritmos permitem ao usuário a configuração de seus parâmetros. 
%
O \textit{TextTiling} permite ajustarmos dois parâmetros, sendo, o tamanho da janela (distância entre a primeira e a última sentença) para o qual atribuiu-se os valores 20, 40 e 60. Para o segundo parâmetro, o passo (distância que a janela desliza), atribuiu-se os valores 3, 6, 9 e 12. Gerando ao final 20 modelos.
%

O \textit{C99} permite ajustarmos três parâmetros, sendo, a quantidade segmentos desejados, o qual é calculado como uma proporção dos candidatos a limite. Para isso atribuiu-se as proporções de 0,2 a 1,0 em intervalos de 0,2 O segundo parâmetro, o tamanho da máscara utilizada para gerar a matriz de ranking, atribuiu-se os valores 9 e 11. Permite ainda, definirmos se as sentenças serão representados por vetores contendo a frequência ou o peso de cada termo, onde ambas as representações foram utilizadas. Gerando ao final 20 modelos.



%%%%%%%%%%
% Cálculo das medidas para cada modelo
%%%%%%%%%%
Pela comparação dos resultados com a segmentação fornecida pelos especialistas, calculou-se para cada modelo as medidas tradicionais acurácia, precisão, revocação, F-medida. Além dessas, computou-se também as métricas mais aplicadas à segmentação textual P$_k$ e \textit{WindowDiff}.



%%%%%%%%%%
% Teste de Fiedman e CD
% 1ª Etapa
%%%%%%%%%%
Em seguida aplicou-se o teste de Friedman a fim de saber se há diferenças significativas entre a eficácia dos modelos. O pós-teste de Nemenyi foi aplicado para descobrir quais diferenças são significativas. 
%
Exite diferença quando seus \textit{rankings} médios diferirem em um valor mínimo, chamado de diferença critica (CD). 
%

%%%%%%%%%%
% Dados Obtidos
%%%%%%%%%%
Com isso foi possível, pela análise do diagrama de diferença crítica, verificar qual é o melhor modelo para cada medida
% e quão significativamente 
em relação aos demais. 


A tabela~\ref{tab:mediasC99} mostra os dados obtidos com o \textit{C99}, onde \texttt{S} é a proporção de segmentos em relação a quantidade de candidatos, \texttt{M} é o tamanho da máscara utilizada para criar a matriz de \textit{ranking} e \texttt{W} indica se os segmentos são representados por vetores contendo a frequência ou um peso das palavras. 



\begin{table}[!h]
	\centering

	\begin{tabular}{|c|c|c|c|c|}
	
		\hline
		Medida & \texttt{S} & \texttt{M} & \texttt{W} & \textbf{Média}\\		
		\hline

		Acuracy		& 40	& 11 & Sim & 0.6199	\\ \hline	
		F1			& 60	& 9	 & Sim & 0.6167	\\ \hline	
		Precision	& 40	& 11 & Sim & 0.7106	\\ \hline			
		Recall		& 100	& 9	 & Não & 0.8516	\\ \hline		
		Pk			& 40	& 11 & Sim & 0.1163	\\ \hline	
		Windiff		& 40	& 11 & Sim & 0.3800	\\ \hline		

		
	\end{tabular}
	
	\caption{Médias das medidas obtidas com \textit{C99}}
	\label{tab:mediasC99}
\end{table}


A tabelas~\ref{tab:mediasTextTiling} mostra os dados obtidos com o \textit{TextTiling}, onde \texttt{J} é o tamanho da janela e \texttt{P} é o passo.

\begin{table}[!h]
	\centering

	\begin{tabular}{|c|c|c|c|}
	
		\hline
		Medida & \texttt{J} & \texttt{P} & \textbf{Média}\\		
		\hline

		Acuracy		& 50 & 9 	& 0.5510 \\ \hline	
		F1			& 50 & 3 	& 0.5898 \\ \hline	
		Precision	& 60 & 12 	& 0.5746 \\ \hline			
		Recall		& 50 & 3 	& 0.7717 \\ \hline		
		Pk			& 30 & 9 	& 0.1572 \\ \hline	
		Windiff		& 50 & 9 	& 0.4489 \\ \hline		

		
	\end{tabular}
	
	\caption{Médias das medidas obtidas com o \textit{TextTiling}}
	\label{tab:mediasTextTiling}
\end{table}


Uma vez sabendo quais valores de parâmetros melhor configuram um algoritmo para uma medida, resta então saber qual dos dois algoritmos é mais eficiente segundo essa medida. Para isso aplicou-se novamente o teste de Friedman com pós-teste de Nemenyi, dessa vez, com os melhores modelos dos dois algoritmos para cada medida. O resultado segue na Tabela~\ref{tab:melhoresmodelos}

\begin{table}[!h]
	\centering
	
	\begin{tabular}{|c|c|c|c|c|}

		\hline
		Medida & Algoritmo & \texttt{S} & \texttt{M} & \texttt{W}\\		
		\hline
		
	
		Acuracy		& C99 & 40 	& 11	& Sim \\ \hline
		Precision	& C99 & 40 	& 11	& Sim \\ \hline
		Pk			& C99 & 40 	& 11	& Sim \\ \hline
		Windiff		& C99 & 40 	& 11	& Sim \\ \hline
		F1			& C99 & 60 	& 9		& Sim \\ \hline
		Recall		& C99 & 100 & 9		& Não \\ \hline
 	
	
	\end{tabular}

	\caption{Melhores modelos para cada medida segundo diagramas de diferença crítica}
	\label{tab:melhoresmodelos}	
	
\end{table}


Na análise do diagrama de diferença crítica verificou-se que o algoritmo \textit{C99} apresenta melhor eficiência em todas as medidas e os valores das quatro primeiras os valores de \texttt{S}, \texttt{M} e \texttt{W} se repetiram, sugerindo uma configuração otimizada para o problema da segmentação de atas de reunião.







	\section{Análise dos Resultados}
	\label{sec:resultados}





	\section{Conclusão}
	\label{sec:conclusao}
	
	As atas de reunião, objeto de estudo desse artigo, apresentam características peculiares em relação à discursos em reuniões e textos em geral. Características como segmentos curtos e coesão mais fraca devida ao estilo que evita repetição de palavras e ideias em benefício da leitura por humanos, dificulta o processamento por computadores.

	Os algoritmos \textit{TextTiling} e \textit{C99} foram testados em um conjunto de atas reais coletadas do departamento de computação da UFSCar-Sorocaba. Por meio da análise dos dados chegou-se a um modelo cujos segmentos melhor se aproximaram as amostras de participantes das reuniões. Obteve-se resultados comparáveis aos vistos em discursos longos, o que pode ser justificado pelo estilo peculiar de escrita.	
	
	Em trabalhos futuros, serão investigadas técnicas para descrever os segmentos e com isso aprimorar o acesso ao conteúdo das atas de reunião.


\endgroup



\bibliographystyle{abbrv}
\bibliography{sigproc}  % sigproc.bib is the name of the Bibliography in this case
	
\pagestyle{empty}
 	\label{sec:anexo}
 	\include{project/anexo}
\end{document}
