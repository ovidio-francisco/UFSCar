
\documentclass{sig-alternate-05-2015}
\usepackage[portuguese]{babel}
%\usepackage{multirow}
%\usepackage{adjustbox}
%\usepackage{graphicx}
%\usepackage{array}
%\usepackage{tabulary}
% \usepackage{pgfplotstable}
% recommended:
%\usepackage{booktabs}
%\usepackage{array}
%\usepackage{colortbl}
%\usepackage{emptypage}
%\newcolumntype{l}[1]{>{\centering\arraybackslash}p{#1}}
% \usepackage{showframe}   %% just for demo
% \usepackage[brazilian,hyperpageref]{backref}	 % Paginas com as citações na bibl
% \usepackage[alf]{abntex2cite}	% Citações padrão ABNT
\usepackage[utf8]{inputenc}		% Codificacao do documento (conversão automática dos acentos)

\usepackage{rotating}
\usepackage{tabularx}
\usepackage{multicol}

\begin{document}

\title{Segmentação topical automática de atas de reunião}


\numberofauthors{1} 

\author{
\alignauthor Ovídio José Francisco\\
       \email{ovidiojf@gmail.com}
}

\maketitle

%\begin{abstract}   
%
%\end{abstract}

\section*{RESUMO}



\keywords{}

\begingroup
\let\clearpage\relax

\section{Introdução}
	\label{sec:introducao}
	Frequentemente atas de reunião tem a característica de apresentar um texto com poucas quebras de parágrafo e sem marcações de estrutura, como capítulos, seções ou quaisquer indicações sobre o tema do texto. 


% Definição 

A tarefa de segmentação textual consiste dividir um texto em partes que contenham um significado relativamente independente. Em outras palavras, é identificar as posições onde há uma mudança significativa de tópicos.

% Usos
É útil em aplicações que trabalham com textos sem quebras de assunto, ou seja, não apresentam parágrafos, seções ou capítulos, como transcrições automáticas de áudio e grandes documentos que contêm assuntos não idênticos como atas de reunião e noticias.


% Interesses
O interesse por segmentação textual tem crescido em em aplicações voltadas a recuperação de informação %citar o [15] ...
e sumarização de textos. % ... e [2] do "Efficient Linear T S"
Essa técnica pode ser usada para aprimorar o acessao a informação quando essa é solicitada por um usário por meio de uma consulta, onde é possível oferecer porções menores de texto mais relevante ao invés de exibir um documento maior que pode conter informações menos pertinente. A sumarização de texto também pode ser aprimorada ao processar segmentos separados por tópicos ao invés de documentos inteiros.


% Coesão Léxica
Os algoritmos avaliados baseiam-se na ideia de coesão léxica entre assuntos. Isto é, a mudança de tópicos é acompanhada % é relacionada à
 de uma proporcional mudança de vocabulário. A partir disso, vários algorítmos foram propostos. Nesse artigo, os principais serão analizados na prespectiva de atas de reunião.





% Diversas aplicações fazem uso de segmentação textual, incluindo 

% Entre as principais mais frequentes de segmentação textual estão a tra



%É principalmente utilizada em aplicações que processam textos longos como transcrições de áudio e documentos longos, além de aprimorar técnicas de sumarização e information retrievel.



% Usos:
%	* quando não há identificações
%	* em transcrições de áudio
%	* em documentos longos
% 	* text summarization (ver a referencia [2] de Efficient Linear Text...)




%Isto é, dado um texto, identificar onde há mudança de tópicos.


% Interest in automatic text segmentation has blossomed over the last few years, with applications ranging from information retrieval to text summariza-tion to story segmentation of video feeds. [A Critique and Improvement of an Evaluation Metric for Text Segmentation]



%Em outras palavras é identificar divisões entre unidade de informação sucessivas

%A tarefa de segmentação textual consiste em encontrar pontos onde há mudança de tópicos no texto.



%[ The task of linear text segmentation is to split a large text document into shorter fragments, usually blocks of consecutive sentences. ]


% **Segmentação é identificar divisiões entre unidades de informação sucessivas (Beeferman, Berger, and Lafferty (1997))**

%   [Text segmentation is the task of determining the positions at which topics change in a stream of text]






\section{Trabalhos Relacionados}
	\label{sec:trabalhos}
	\section{Trabalhos Relacionados}
	\label{sec:trabalhos}


Semelhante a esse trabalho, outras abordagens foram propostas como em~\cite{CHAIBI2014} onde os autores adaptam os \textit{TextTiling} e \textit{C99} ao idioma árabe. Apresentam os resultados de experimentos no qual avaliam a performance em jornais de diferentes países em árabe. As adaptações consistem basicamente na etapa de pré-processamento e apontam que diferenças no dialeto de cada pais devem ser consideradas no processo de segmentação e que a adaptação depende da escolha de um \textit{stemmer} adequado.

É recorrente a aplicação de segmentadores à reuniões com múltiplos participantes onde se estuda os discursos extraídos de reuniões, ou seja, o texto a ser segmentado é uma transcrição das falas dos participantes durante a reunião.
%
Banerjee~\cite{Banerjee2006} apresenta um segmentador baseado no  \textit{TextTiling} ao contexto das reuniões, com múltiplos participantes. Utiliza como \textit{corpus} a transcrição da fala dos participantes durante a reunião a qual foi conduzida por um mediador que propunha os tópicos e anotava o tempo onde os participantes mudavam o assunto. 
% outros exemplos são

Ainda no contexto de reuniões com múltiplos participantes, alguns elementos da fala são utilizados para encontrar melhores segmentos.
%
Bokaei~\cite{Bokaei2015}, traz um trabalho voltado à segmentação funcional do texto, que foca na atividade dos participantes, as quais categoriza em diálogos e monólogos e sugere que alguns comportamentos podem dar pistas de mudança de tópico, como quando um participante toma a palavra por um tempo prolongado. 
Galley~\cite{Galley2003} por sua vez considera de elementos como pausas, trocas de falantes e entonação para encontrar melhores segmentos.

Kern aponta em sua pesquisa~\cite{Kern2009} que algoritmos que apresentam melhor performance o fazem ao custo de maior complexidade computacional, que se deve à construção de matrizes de similaridade entre todas as sentenças como em~\cite{Choi2000}. Ele apresenta uma abordagem que otimiza o cálculo ao computar as médias das similaridades de cada bloco, a qual chama de \textit{inner similarity} e em seguida usa esses valores para calcular as medias das similaridades entre todos os blocos a qual chama de \textit{outter similarity}. Dessa forma não é criada uma matriz que contem as similaridades de todas as sentenças, mas apenas daquelas mais próximas. Os autores reportam uma eficiência superior e uma eficiência comparável aos algoritmos mais complexos.
% usa vetores contendo o peso da palavra ao inves da frequencia.




\section{Análise dos Resultados}
	\label{sec:resultados}
	\section{Resultados}
	\label{sec:resultados}

%%%%%%%%%%
% Objetivos
%%%%%%%%%%
A fim de encontrar o melhor método que divida uma ata em segmentos coerentes, realizou-se experimentos com o \textit{TextTiling} e \textit{C99} a fim de encontrar os melhores parâmetros para esses documentos.

%tópicos retornando segmentos  que retorne segmentos coerentes 

%%%%%%%%%%
% Parâmetros
%%%%%%%%%%
As implementações dos algoritmos permitem ao usuário a configuração de seus parâmetros. 
%
O \textit{TextTiling} permite ajustarmos dois parâmetros, sendo, o tamanho da janela (distância entre a primeira e a última sentença) para o qual atribuiu-se os valores 20, 40 e 60. Para o segundo parâmetro, o passo (distância que a janela desliza), atribuiu-se os valores 3, 6, 9 e 12. Gerando ao final 20 modelos.
%

O \textit{C99} permite ajustarmos três parâmetros, sendo, a quantidade segmentos desejados, o qual é calculado como uma proporção dos candidatos a limite. Para isso atribuiu-se as proporções {0,2; 0,4; 0,6; 0,8; 1,0}. O segundo parâmetro, o tamanho da máscara utilizada para gerar a matriz de ranking, atribuiu-se os valores 9 e 11. Permite ainda, definirmos se as sentenças serão representados por vetores contendo a frequência ou o peso de cada termo, onde ambas as representações foram utilizadas.
%
 Considerando todos os parâmetros, foram gerados 20 modelos para o algoritmo C99.% By Rafael 



%%%%%%%%%%
% Cálculo das medidas para cada modelo
%%%%%%%%%%

% --> Isso já foi falado no texto
%Pela comparação dos resultados com a segmentação fornecida pelos especialistas, calculou-se para cada modelo as medidas tradicionais acurácia, precisão, revocação, F-medida. Além dessas, computou-se também as métricas mais aplicadas à segmentação textual P$_k$ e \textit{WindowDiff}.



%%%%%%%%%%
% Teste de Fiedman e CD
% 1ª Etapa
%%%%%%%%%%
Em seguida aplicou-se o teste de Friedman a fim de saber se há diferenças significativas entre a eficácia dos modelos. O pós-teste de Nemenyi foi aplicado para descobrir quais diferenças são significativas. 
%
Exite diferença quando seus \textit{rankings} médios diferirem em um valor mínimo, chamado de diferença critica (CD). 
%

%%%%%%%%%%
% Dados Obtidos
%%%%%%%%%%
Com isso foi possível, pela análise do diagrama de diferença crítica, verificar qual é o melhor modelo para cada medida
% e quão significativamente 
em relação aos demais. 


A Tabela~\ref{tab:mediasC99} mostra os dados obtidos com o \textit{C99}, onde \texttt{S} é a proporção de segmentos em relação a quantidade de candidatos, \texttt{M} é o tamanho da máscara utilizada para criar a matriz de \textit{ranking} e \texttt{W} indica se os segmentos são representados por vetores contendo a frequência ou um peso das palavras. 



\begin{table}[!h]
	\centering

	\begin{tabular}{|c|c|c|c|c|}
	
		\hline
		Medida & \texttt{S} & \texttt{M} & \texttt{W} & \textbf{Média}\\		
		\hline

		Acuracy		& 40	& 11 & Sim & 0.6199	\\ \hline	
		F1			& 60	& 9	 & Sim & 0.6167	\\ \hline	
		Precision	& 40	& 11 & Sim & 0.7106	\\ \hline			
		Recall		& 100	& 9	 & Não & 0.8516	\\ \hline		
		Pk			& 40	& 11 & Sim & 0.1163	\\ \hline	
		Windiff		& 40	& 11 & Sim & 0.3800	\\ \hline		

		
	\end{tabular}
	
	\caption{Médias das medidas obtidas com \textit{C99}}
	\label{tab:mediasC99}
\end{table}


A tabelas~\ref{tab:mediasTextTiling} mostra os dados obtidos com o \textit{TextTiling}, onde \texttt{J} é o tamanho da janela e \texttt{P} é o passo.

\begin{table}[!h]
	\centering

	\begin{tabular}{|c|c|c|c|}
	
		\hline
		Medida & \texttt{J} & \texttt{P} & \textbf{Média}\\		
		\hline

		Acuracy		& 50 & 9 	& 0.5510 \\ \hline	
		F1			& 50 & 3 	& 0.5898 \\ \hline	
		Precision	& 60 & 12 	& 0.5746 \\ \hline			
		Recall		& 50 & 3 	& 0.7717 \\ \hline		
		Pk			& 30 & 9 	& 0.1572 \\ \hline	
		Windiff		& 50 & 9 	& 0.4489 \\ \hline		

		
	\end{tabular}
	
	\caption{Médias das medidas obtidas com o \textit{TextTiling}.}
	\label{tab:mediasTextTiling}
\end{table}


Uma vez sabendo quais valores de parâmetros melhor configuram um algoritmo para uma medida, resta então saber qual dos dois algoritmos é mais eficiente segundo essa medida. Para isso aplicou-se novamente o teste de Friedman com pós-teste de Nemenyi, dessa vez, com os melhores modelos dos dois algoritmos para cada medida. O resultado segue na Tabela~\ref{tab:melhoresmodelos}

\begin{table}[!h]
	\centering
	
	\begin{tabular}{|c|c|c|c|c|}

		\hline
		Medida & Algoritmo & \texttt{S} & \texttt{M} & \texttt{W}\\		
		\hline
		
	
		Acuracy		& C99 & 40 	& 11	& Sim \\ \hline
		Precision	& C99 & 40 	& 11	& Sim \\ \hline
		Pk			& C99 & 40 	& 11	& Sim \\ \hline
		Windiff		& C99 & 40 	& 11	& Sim \\ \hline
		F1			& C99 & 60 	& 9		& Sim \\ \hline
		Recall		& C99 & 100 & 9		& Não \\ \hline
 	
	
	\end{tabular}

	\caption{Melhores modelos para cada medida segundo diagramas de diferença crítica.}
	\label{tab:melhoresmodelos}	
	
\end{table}


Na análise do diagrama de diferença crítica verificou-se que o algoritmo \textit{C99} apresenta melhor eficiência em todas as medidas e os valores das quatro primeiras os valores de \texttt{S}, \texttt{M} e \texttt{W} se repetiram, sugerindo uma configuração otimizada para o problema da segmentação de atas de reunião.






	
\section{TextTilingBR}

Adaptações nos algoritmos originais para o contexto das atas


\section{Conclusão}
	\label{sec:conclusao}
	\section{Conclus�o}
	\label{sec:conclusao}

As atas de reuni�o, objeto de estudo desse artigo, apresentam caracter�sticas peculiares em rela��o � discursos e textos em geral. Caracter�sticas como segmentos curtos e coes�o mais fraca devida ao estilo que evita repeti��o de palavras e ideias em benef�cio da leitura por humanos, dificultam o processamento por computadores.

%%%%%%%%%%
% Benef�cios 
%%%%%%%%%%


Os algoritmos \textit{TextTiling} e \textit{C99} foram testados em um conjunto de atas coletadas do Departamento de Computa��o da UFSCar-Sorocaba. Por meio da an�lise dos dados chegou-se a uma configura��o cujos segmentos melhor se aproximaram das amostras segmentadas por participantes das reuni�es. 


Na maioria das medidas, o algoritmo \textit{C99} sobressaiu-se em rela��o ao \textit{TextTiling}, contudo, os testes estat�sticos, n�o apresentam diferen�a significativa. 


%%%%%%%%%%%%%%%
% O Impacto do Preprocessamento 
%%%%%%%%%%%%%%%

Da mesma forma, a etapa de preprocessamento proporciona melhora de performance quando aplicada, por�m o seu maior benef�cio � a diminui��o do custo computacional, uma vez que n�o prejudica a qualidade dos resultados.



A segmenta��o de atas de reuni�o pode ajudar na organiza��o, busca e compreens�o dos conte�dos nelas contidos. Tamb�m outros dom�nios e aplica��es diferentes podem se beneficiar dos resultados apresentados, como aplica��es voltadas a resgate de informa��o, sumariza��o e acessibilidade. Assim, espera-se que outros trabalhos possam aproveitar deste.
	
	
Em trabalhos futuros, ser�o investigadas t�cnicas de extra��o de t�picos para descrever os segmentos e com isso aprimorar o acesso ao conte�do das atas de reuni�o.





\endgroup



\bibliographystyle{abbrv}
\bibliography{sigproc}  % sigproc.bib is the name of the Bibliography in this case
	
\pagestyle{empty}
 	\label{sec:anexo}
 	\include{project/anexo}
\end{document}
