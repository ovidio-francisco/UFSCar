\section{Conclusão}
	\label{sec:conclusao}
	
	As atas de reunião, objeto de estudo desse artigo, apresentam características peculiares em relação à discursos em reuniões e textos em geral. Características como segmentos curtos e coesão mais fraca devida ao estilo que evita repetição de palavras e ideias em benefício da leitura por humanos, dificulta o processamento por computadores.

%%%%%%%%%%
% Benefícios 
%%%%%%%%%%


	Os algoritmos \textit{TextTiling} e \textit{C99} foram testados em um conjunto de atas reais coletadas do departamento de computação da UFSCar-Sorocaba. Por meio da análise dos dados chegou-se a um modelo cujos segmentos melhor se aproximaram as amostras de participantes das reuniões. Obteve-se resultados comparáveis aos vistos em discursos longos, porém um pouco inferiores, o que pode ser justificado pelo estilo peculiar de escrita das atas.	


	A segmentação de atas de reunião pode ajudar na organização, busca e compreensão dos conteúdos nelas contidos. Também outros domínios e aplicações diferentes podem se beneficiar dos resultados apresentados, como aplicações voltadas a resgate de informação, sumarização e acessibilidade. Assim, espera-se que outros trabalhos possam aproveitar deste.
	
	
	Em trabalhos futuros, serão investigadas técnicas para descrever os segmentos e com isso aprimorar o acesso ao conteúdo das atas de reunião.



