\section{Trabalhos Relacionados}
	\label{sec:trabalhos}

%%%%%%%%%%
% Retomada da proposta
%%%%%%%%%%
A proposta é adaptar os algoritmos tradicionais à segmentação de atas de reunião em português, bem como encontrar um modelo que melhor se ajuste a esse contexto. Semelhante a este trabalho, outras abordagens foram propostas no sentido como mostradas a seguir.



%%%%%%%%%%
% Topic segmentation for textual document written in arabic language
%%%%%%%%%%
A fim de adaptar os algoritmos \textit{TextTiling} e \textit{C99} ao idioma árabe de vários países, foram propostos ajustes na etapa de preprocessamento onde verificou-se que diferenças diferenças no dialeto de cada
pais devem ser consideradas no processo de segmentação e que a adaptação depende da escolha de um \textit{stemmer} adequado à cada dialeto~\cite{CHAIBI2014}.


Semelhante a esse trabalho, a principal motivação dos autores é adaptar os algoritmos mais influentes ao idioma árabe, devido a falta de trabalhos consistentes nesse sentido. 

Por outro lado, na avaliação, os autores utilizaram como \textit{corpus} a concatenação de textos extraídos de notícias de diferentes países como Tunísia, Egito e Algeria, que tratam de assuntos distintos como política, esporte, cultura, história, tecnologia e artes, e que trazem particularidades nos estilos de escrita e até diferenças entre dialetos locais. Ocorre que essas características favorecem o processo de segmentação uma vez que os documentos produzidos pela concatenação contém tópicos obtidos de domínios distintos que compartilham pouco vocabulário. Por outro lado, essas características não estão presentes no contexto das atas, onde um documento é redigido por um mesmo autor, e todos tópicos foram produzidos no mesmo domínio, e dessa forma, têm estilo de escrita e vocabulário similares.


%%%%%%%%%%
%%%%%%%%%%


É recorrente a aplicação de segmentadores à reuniões com múltiplos participantes onde se estuda os discursos extraídos de reuniões, ou seja, o texto a ser segmentado é uma transcrição das falas dos participantes durante a reunião.
%

%%%%%%%%%%
% A texttiling based apporach to topic boundary detection in meetings
%%%%%%%%%%
Nesse sentido, é apresentado um segmentador baseado no  \textit{TextTiling}, o qual foi aplicado em reuniões com múltiplos participantes. Utiliza como \textit{corpus} a transcrição da fala dos participantes durante as reuniões as quais foram conduzidas por um mediador que propunha os tópicos e anotava o tempo onde os participantes mudavam o assunto~\cite{Banerjee2006}. 
%%%%%%%%%%
% Discourse segmentation of multi-party conversation
%%%%%%%%%%
Os autores também se apoiaram em trabalhos anteriores que consideram elementos da fala como pausas, trocas de falantes e entonação para encontrar melhores segmentos~\cite{Galley2003}.
% outros exemplos são


%%%%%%%%%%
% Diferenças
%%%%%%%%%%


O estudo da análise do discurso em reuniões, como mostrado acima, difere desse trabalho em algumas características que são típicas de um estilo formal como o texto sucinto, onde o autor se limita a relatar apenas o essencial de uma discussão e omite detalhes da discussão, o que resulta em segmentos mais curtos os quais desfavorecem algoritmos baseados em coesão léxica~\cite{Choi2000}. Mais detalhes referentes às diferenças entre as atas e os textos analisados na maioria dos trabalhos são mostrados na Subseção~\ref{subsec:conjunto-de-documentos}.


%%%%%%%%%%
% Linear discourse segmentation of multi-party meetings bases ond local and global information
%%%%%%%%%%
Ainda no contexto de reuniões com múltiplos participantes temos a segmentação funcional dos discursos, onde outros aspectos podem ser analisados, como a participação dos presentes na reunião. Nesse sentido, estuda-se a contribuição das pessoas as quais podem ser categorizadas,  por exemplo, em diálogos, discussões e monólogos. Sugere-se que alguns comportamentos podem dar pistas de mudança de discurso, como quando um participante toma a palavra por um tempo prolongado~\cite{Bokaei2015}. 

Tais aspectos não se aplicam ao contexto das atas, onde o estilo de escrita em forma de narrativa, prefere poupar o leitor de diálogos secundários durante transições de tópicos.

%

%%%%%%%%%%
% Palavras Pista
%%%%%%%%%%
A fim de aprimorar a detecção de limites entre segmentos, a presença de \textit{pistas} pode ser um indicativo de finais ou inícios de segmentos, uma vez que alguns elementos são frequentes na transição de tópicos. Essas \textit{pistas} podem ser palavras como \textit{``Ok''}, \textit{``continuando''}, frases como \textit{``Boa noite''}, \textit{``Dando prosseguimento''} ou pausas prolongadas. Essas palavras podem ser detectadas por meio de algoritmos de aprendizado de máquina ou anotadas manualmente
\cite{Hsueh2006} % Faz combinações com Lexical e Cue LC-CUE
\cite{Galley2003} 
%\cite{Boguraev2000}%\textit{por outro lado}
\cite{Beeferman1999}
%\cite{Reynar1999}
.


%%%%%%%%%%
% Efficient linear text segmentation based on information retrieval techniques
%%%%%%%%%%
Algoritmos que apresentam melhor performance o fazem ao custo de maior complexidade computacional, que se deve, muitas vezes, à construção de matrizes de similaridade entre todas as sentenças como em~\cite{Choi2000} onde também se vê que, calcular similaridades entre blocos de sentenças muito distantes no documento, pode representar um custo computacional dispensável. %Pode citar o mesmo autor no mesmo parágrafo quando ele fala de dois principios diferentes? 

Nesse sentido, é apresentada uma abordagem que otimiza o processo ao computar as médias das similaridades entre sentenças de cada bloco, a qual chama de \textit{inner similarity} e em seguida usa esses valores para calcular as medias das similaridades entre todos os blocos a qual chama de \textit{outter similarity}. Dessa forma não é criada uma matriz que contem as similaridades de todas as sentenças, mas apenas daquelas mais próximas. Os autores reportam uma eficiência superior e uma eficácia comparável aos algoritmos mais complexos~\cite{Kern2009}.

Nesse trabalho, não há preocupação com o tempo de processamento, uma vez que as atas são documentos relativamente curtos (com cerca de três páginas) e não chega a impactar na eficácia do sistema, estando o foco principal na eficácia em produzir boas segmentações.

% 
% usa vetores contendo o peso da palavra ao inves da frequencia.








%como em~\cite{CHAIBI2014} onde os autores adaptam os algoritmos \textit{TextTiling} e \textit{C99} ao idioma árabe. Eles apresentam os resultados de experimentos no qual avaliaram a performance em notícias de diferentes países em árabe. As adaptações consistem basicamente na etapa de pré-processamento e apontam que diferenças no dialeto de cada pais devem ser consideradas no processo de segmentação e que a adaptação depende da escolha de um \textit{stemmer} adequado.

%Galley~\cite{Galley2003} por sua vez considera de elementos como pausas, trocas de falantes e entonação para encontrar melhores segmentos.

%Bokaei~\cite{Bokaei2015}, traz um trabalho voltado à segmentação funcional do texto, que foca na atividade dos participantes, as quais categoriza, por exemplo em diálogos, discussões e monólogos e sugere que alguns comportamentos podem dar pistas de mudança de tópico, como quando um participante toma a palavra por um tempo prolongado. 
