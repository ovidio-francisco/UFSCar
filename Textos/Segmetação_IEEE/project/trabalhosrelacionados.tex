\section{Trabalhos Relacionados}
	\label{sec:trabalhos}


Semelhante a este trabalho, outras abordagens foram propostas como em~\cite{CHAIBI2014} onde os autores adaptam os algoritmos \textit{TextTiling} e \textit{C99} ao idioma árabe. Eles apresentam os resultados de experimentos no qual avaliaram a performance em jornais de diferentes países em árabe. As adaptações consistem basicamente na etapa de pré-processamento e apontam que diferenças no dialeto de cada pais devem ser consideradas no processo de segmentação e que a adaptação depende da escolha de um \textit{stemmer} adequado.

É recorrente a aplicação de segmentadores à reuniões com múltiplos participantes onde se estuda os discursos extraídos de reuniões, ou seja, o texto a ser segmentado é uma transcrição das falas dos participantes durante a reunião.
%


Banerjee~\cite{Banerjee2006} apresenta um segmentador, baseado no  \textit{TextTiling}, ao contexto das reuniões com múltiplos participantes. Utiliza como \textit{corpus} a transcrição da fala dos participantes durante a reunião a qual foi conduzida por um mediador que propunha os tópicos e anotava o tempo onde os participantes mudavam o assunto. 
% outros exemplos são

Ainda no contexto de reuniões com múltiplos participantes, alguns elementos da fala são utilizados para encontrar melhores segmentos.
%
Bokaei~\cite{Bokaei2015}, traz um trabalho voltado à segmentação funcional do texto, que foca na atividade dos participantes, as quais categoriza, por exemplo em diálogos, discussões e monólogos e sugere que alguns comportamentos podem dar pistas de mudança de tópico, como quando um participante toma a palavra por um tempo prolongado. 
Galley~\cite{Galley2003} por sua vez considera de elementos como pausas, trocas de falantes e entonação para encontrar melhores segmentos.

Kern aponta em sua pesquisa~\cite{Kern2009} que algoritmos que apresentam melhor performance o fazem ao custo de maior complexidade computacional, que se deve à construção de matrizes de similaridade entre todas as sentenças como em~\cite{Choi2000}. Ele apresenta uma abordagem que otimiza o cálculo ao computar as médias das similaridades de cada bloco, a qual chama de \textit{inner similarity} e em seguida usa esses valores para calcular as medias das similaridades entre todos os blocos a qual chama de \textit{outter similarity}. Dessa forma não é criada uma matriz que contem as similaridades de todas as sentenças, mas apenas daquelas mais próximas. Os autores reportam uma eficiência superior e uma eficácia comparável aos algoritmos mais complexos.
% usa vetores contendo o peso da palavra ao inves da frequencia.


