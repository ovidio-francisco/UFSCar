
\section{Conjunto de documentos}
	\label{sec:conjutodedocumentos} 


	
Para a condução desse trabalho é necessário que hajam documentos a serem segmentados automaticamente e que se conheça uma segmentação de referência para compará-los e avaliar a proposta. A seguir, são apresentadas as caraterísticas particulares das atas em relação a outros tipos de textos e a segmentação de referência a ser utilizada.


\subsection{Características das atas}
	
%%%%%%%%%%
% Características particulares
%%%%%%%%%%
As atas de reunião diferem dos textos comumente estudados em outros trabalhos em alguns pontos. O estilo contribui para a escrita de textos sucintos com poucos detalhes, pois o ambiente dá preferência a textos curtos. Segundo Choi~\cite{Choi2001-LSA}, o segmentador tem a acurácia reduzida em segmentos curtos (em torno de 3 a 5 sentenças). % Já te vi 
	
	Para evitar um texto monótono à leitura, ao redigir o documento tem-se o cuidado de não repetir ideias e palavras em favor da elegância do documento. Tal característica enfraquece a coesão léxica e portanto o cálculo da similaridade é prejudicado. Por exemplo, duas sentenças diferem se uma contiver a palavra \textit{computadores} e na seguinte \textit{equipamentos}, mesmo que se refiram à mesma ideia. Além disso, o documento compartilha um certo vocabulário próprio do ambiente onde os assuntos são discutidos e com isso nota-se que os segmentos, embora tratem de assuntos diferentes, são semelhantes em vocabulário.
	
A presença de ruídos como cabeçalhos, rodapés e numeração de páginas e linhas prejudicam tanto similaridade entre sentenças como a apresentação final ao usuário. Porém, esses ruídos podem ser reduzidos ou eliminados como mostrado na Subseção~\ref{subsec:preprocessamento}, sobre preprocessamento.



 
%%%%%%%%%%
% Há duas diferenças entre os tipos de documentos mais estudados
%%%%%%%%%%
A maioria dos trabalhos enquadram-se em duas categorias no que se refere ao \textit{corpus} estudado.  
%%%%%%%%%%
% A concatenação de textos favorece a segmentação
% E diferenças
% 
%   1 - Mesmo autor e mesmo domínio --> mesmo estilo e vocabulário
%
%%%%%%%%%%
A primeira utiliza a concatenação de textos de diferentes fontes e assuntos. 
Ocorre que essas características favorecem o processo de segmentação uma vez que os documentos produzidos pela concatenação contém tópicos obtidos de domínios distintos que compartilham pouco vocabulário. Por outro lado, essas características não estão presentes no contexto das atas, onde um documento é redigido por um mesmo autor, e todos tópicos foram produzidos no mesmo domínio, e dessa forma, têm estilo de escrita e vocabulário similares.


%%%%%%%%%%
% Discursos falados
% E difenças
%
%   1 - texto sucinto, sem detalhes e reptições --> segmentos mais curtos
%
%%%%%%%%%%
A segunda estuda a segmentação automática de textos obtidos da transcrição de áudios como de vídeos e de gravações de reuniões com grupos de pessoas. 
A transcrição da fala de pessoas, difere desse trabalho, pois este possui características típicas de um estilo formal, por exemplo, o texto sucinto, onde o autor se limita a relatar apenas o essencial de uma discussão e omite detalhes da discussão, poupando o leitor de diálogos repetitivos ou falas durante a transição de tópicos, o que resulta em segmentos mais curtos os quais desfavorecem algoritmos baseados em coesão léxica~\cite{Choi2000}. 




\subsection{Segmentação de Referência}


%%%%%%%%%% 
% Necessidade de uma referência
%%%%%%%%%%
Para que se possa avaliar um segmentador automático de textos é preciso uma referência, isto é, um texto com os limites entre os segmento conhecidos. Essa referência, deve ser confiável, sendo uma segmentação legítima que é capaz de dividir o texto em porções relativamente independentes, ou seja, uma segmentação ideal.



%%%%%%%%%%
% Formas de conseguir a Segentançaõ de Referência
%%%%%%%%%%
Entre as abordagens mais comuns para se conseguir essas referências, encontramos: 

\begin{itemize}

% Concatenação
\item A concatenação aleatória de documentos distintos, onde o ponto entre o final de um texto e o início do seguinte é um limite entre eles. 

% Segmentação manual
\item A segmentação manual dos documentos, nesse caso, pessoas capacitadas, também chamadas de juízes, ou mesmo o autor do texto, são consultadas e indicam manualmente onde há uma quebra de segmento. 
% Mediador em conversas
\item Em transcrição de conversas faladas em reuniões com múltiplos participantes, um mediador é responsável por encerrar um assunto e iniciar um novo, nesse caso o mediador anota manualmente o tempo onde há uma transição de tópico. 

\end{itemize}

% Não avaliar
Em aplicações onde a segmentação é uma tarefa secundária e quando essas abordagens são custosas ou não se aplicam, é possível, ao invés de avaliar o segmentador, analisar seu impacto na aplicação final.













% Em outras palavras, uma ata é a narração de uma reunição que poupa




% Mais detalhes referentes às diferenças entre as atas e os textos analisados na maioria dos trabalhos são mostrados na Subseção~\ref{subsec:conjunto-de-documentos}.

