% ==============================================================================
% ==  Rascunho                                                                ==
% ==============================================================================









diminuir o n. de atributos

manter a qualidade dos resultados

mantem a qualidade original e diminui a complexidade 

Não teve diferença significativa





%  Obteve-se resultados comparáveis aos vistos em discursos longos, porém um pouco inferiores, o que pode ser justificado pelo estilo peculiar de escrita das atas.	












%
% Teste de Friedman em múltilas atas  
%%%
O teste de Friedman foi utilizado para avaliar o desempenho dos algoritmos considerando a base de dados de atas. 
%
% Testa o rankings médios e testa hip. nula 
%%%
O teste comparou os ranks médios dos algoritmos em cada ata e testou a hipótese nula de que todos os algoritmos são são equivalentes. 
%
% Se a hip. nula for rejeitada, executa Nemenyi  
%%%
Quando a hipótese nula é rejeitada, executa-se o pos-teste de Nemenyi...

Exite diferença significativa quando seus \textit{rankings} médios diferem em um valor mínimo, chamado de diferença critica (CD). 

O teste de Friedman é utilizado para avaliar o desempenho de algoritmos em múltiplos datasets. O teste compara os ranks médios dos algoritmos em cada dataset e testa a hipótese nula de que todos os algoritmos são equivalentes. Quando a hipótese nula é rejeitada, executa-se o pos-teste de Nemenyi a fim de saber se há diferença significativa entre os algotimos.


Em seguida o Teste de Firedman foi aplicado para saber se há









A fim de encontrar a melhor configuração que divida uma ata em segmentos coerentes, 

%%%%%%%%%%%%%%%
% Medidas calculadas pela comparação com a referência 
%%%%%%%%%%%%%%% 
Os algoritmos \textit{TextTiling} e \textit{C99} foram comparados com a segmentação fornecida pelos especialistas e calculou-se as medidas tradicionais acurácia, precisão, revocação, $F^1$. Além dessas, computou-se também as medidas mais aplicadas à segmentação textual, P$_k$ e \textit{WindowDiff}.


computou-se para cada configuração 


- cálculo das medidas tradicionais, Pk e Windiff
- 









% ==============================================================================
% ==  Critérios de Avaliação                                                  ==
% ==============================================================================
\subsection{Critérios de avaliação}
% Os critérios vão aqui 
% desde as metricas ate a configuração dos resultados
% quais são as métricas e como eu comparo as métricas
% comparo as metricas utilizando teste de friedman


Para verificar se há diferença significativa entre as configurações, o teste de Friedman foi utilizado com cada medida calculada. 





Em seguida aplicou-se o teste de Friedman para saber se há difirenças significativas entre os resultados. O te

Em seguida aplicou-se o teste de Friedman para saber se há diferenças significativas entre a eficácia as configurações. O pós-teste de Nemenyi foi aplicado para descobrir quais diferenças são significativas. 
%
Exite diferença quando seus \textit{rankings} médios diferirem em um valor mínimo, chamado de diferença critica (CD). 
%

Com isso foi possível, pela análise do diagrama de diferença crítica, verificar 
se há diferença significativa entre entre as configurações.
% e quão significativamente 
% qual é o melhor modelo para cada medida
%em relação aos demais. 


%
% Statistical Comparisons of Classifiers over Multiple Data Sets
%%% 
% The Friedman test (Friedman, 1937, 1940) is a non-parametric equivalen t of the repeated-measures ANOVA. It ranks the algorithms for each data set separately, the best pe rforming algorithm getting the rank of 1, the second best rank 2. . . , as shown in Table 6. In case of ties (like in iris, lung cancer, mushroom and primary tumor), average ranks are assigned.



% Aplicou-se o teste de Friedman para saber se há difirença entre as configurações
% Aplicou-se o pos-teste de Friedman para saber quais são as diferenças significativas.
% O Teste de Frieman é utilizado
% O teste de Friedman testa a hipótese nula de que todos são iguais?

% -------------------------------------------------
% 1 - teste de Friedman com pós teste de Nemeny
	% - Gera um ranking
	
	não entrar em detalhes disso !!!!

% 2 - Cria o diagrama de Diferença crítica 
	% - Pelo diagrama sabe-se qual o melhor alg.








