
\section{Avaliação Experimental}
	\label{sec:avaliacao-experimental}

%%%%%%%%%% 
% Necessidade de uma referência
%%%%%%%%%%
Para que se possa avaliar um segmentador automático de textos, é preciso uma referência, isto é, um texto com os limites entre os segmento conhecidos. Essa referência, deve ser confiável, sendo uma segmentação legítima que é capaz de dividir o texto em porções relativamente independentes, mantendo um conteúdo legível, ou seja, uma segmentação ideal.



%%%%%%%%%%
% Formas de conseguir a Segentançaõ de Referência
%%%%%%%%%%
Entre as abordagens mais comuns para se conseguir essas referências, encontramos: 

\begin{itemize}

% Concatenação
\item A concatenação aleatória de documentos distintos, onde o ponto entre o final de um texto e o início do seguinte é um limite entre eles. 

% Segmentação manual
\item A segmentação manual dos documentos, nesse caso, pessoas capacitadas, também chamadas de juízes, ou mesmo o autor do texto, são consultadas e indicam manualmente onde há uma quebra de segmento. 
% Mediador em conversas
\item Em transcrição de conversas faladas em reuniões com múltiplos participantes, um mediador é responsável por encerrar um assunto e iniciar um novo, nesse caso o mediador anota manualmente o tempo onde há uma transição de tópico. 

\end{itemize}

% Não avaliar
Em aplicações onde a segmentação é uma tarefa secundária e quando essas abordagens são custosas ou não se aplicam, é possível, ao invés de avaliar o segmentador, analisar seu impacto na aplicação final.

%%%%%%%%%%
% As 2 principais dificuldades na avaliação
%%%%%%%%%%
De acordo com \cite{Pevzner2002} há duas principais dificuldades na avaliação de segmentadores automáticos. A primeira é conseguir um referência, já que juízes humanos costumam não concordar entre si, sobre onde os limites estão e outras abordagens podem não se aplicar ao contexto. A segunda é que tipos diferentes de erros devem ter pesos diferentes de acordo com a aplicação. Há casos onde certa imprecisão é tolerável e outras, como a segmentação de notícias, onde a precisão é mais importante.


%%%%%%%%%%
% Definição do que é um bom algoritmo de segmentação
%%%%%%%%%%
Para fins de avaliação desse trabalho, um bom método de segmentação é aquele cujo resultado melhor se aproxima do ideal, sem a obrigatoriedade de estar perfeitamente alinhado com tal. Ou seja, visto o contexto das atas de reunião, e a subjetividade da tarefa, não é necessário que os limites entre os segmentos (real e hipótese) sejam idênticos, mas que se assemelhem em localização e quantidade.


%%%%%%%%%%
% Tratamento das atas
%%%%%%%%%%


As próximas subseções mostram o conjunto de atas e a segmentação usada como referência em seguida são apresentadas as métricas de avaliação utilizadas neste trabalho e os testes realizados para avaliar os métodos.


\subsection{Conjunto de documentos}
	\label{subsec:conjunto-de-documentos}
	
% Como foram obtidas
	% Software
% Como foram tratadas	
	


\subsection{Medidas de Avaliação}

%%%%%%%%%%
%	Avaliações baseadas em hits 
%%%%%%%%%%

Medidas de avaliação tradicionais, baseiam se na contagem de acertos. No contexto da segmentação de textos um acerto é quando um o limite hipotético coincide com um limite de referência.

Nesse sentido, 
%
a precisão, que é a proporção de limites corretamente identificados pelo algoritmo, e 
%
a revocação, que é a proporção de limites verdadeiros que foram identificados pelo algoritmo,
%
trazem alguns problemas na avaliação de segmentadores automáticos.
 	
	
Conforme o algoritmo aponta mais segmentos no texto, este tende a melhorar a revocação e ao mesmo tempo, reduzir a precisão. Esse problema de avaliação pode ser contornado utilizado a medida F-1 que é uma média harmônica entre precisão e revocação onde ambas tem a o mesmo peso. Por por outro lado, tem a desvantagem de ser mais difícil de interpretar. 

As medidas apresentadas acima falham ao não serem sensíveis a \textit{near misses}, ou seja, quando um limite não coincide exatamente com o esperado, mas está próximo a ele~\cite{Kern2009}.

Na Figura~\ref{fig:exemplosegmentacaozoom} é apresentado um exemplo com duas segmentações hipotéticas e uma referência. Na Figura~\ref{fig:exemplosegmentacao}, em ambos os casos não há nenhum verdadeiro positivo, o que implica em zero para os valores de precisão, acurácia, e revocação, embora a segunda hipótese possa ser considerada superior à primeira se levado em conta a proximidade dos limites.



  \begin{figure}[!h]

	\centering
	\includegraphics[width=0.47\textwidth]{windiffzoom.jpg}
	\caption{Exemplos de \textit{near missing} e falso positivo puro. Os blocos indicam uma unidade de informação e as linha verticais representam os limites entre segmentos de texto representando um tópico do texto. }
	\label{fig:exemplosegmentacaozoom}

  \end{figure}
  
  \begin{figure}[!h]

	\centering
	\includegraphics[width=0.47\textwidth]{windiff.jpg}
	\caption{
	Exemplo de duas segmentações hipotéticas em comparação a uma ideal. 
	}
	\label{fig:exemplosegmentacao}

  \end{figure}
  
  
Entre as medidas mais utilizadas para avaliar segmentadores estão:

\begin{enumerate}
	\item P$_k$. A fim de resolver o problema de \textit{near misses}, Beeferman \textit{et. al.}~\cite{Beeferman1999} apresentam uma medida chamada P$_k$ que atribui valores parciais a \textit{near misses}. Esse método move uma janela de tamanho $k$ e a cada posição e verifica se o início e o final da janela estão ou não dentro do mesmo segmento e penaliza o algoritmo em caso de discrepância. Ou seja, dado duas palavras de distancia $k$, uma discrepância é computada quando o algoritmo e a referência não concordam se as palavras estão ou não no mesmo segmento.

O valor de $k$ é calculado como a metade da média dos comprimentos dos segmentos reais. Como resultado, é retornado a contagem de discrepâncias divido pelo quantidade de segmentações analisadas. Esse valor serve como medida de dissimilaridade entre as segmentações e pode ser interpretada como a probabilidade de duas sentenças extraídas aleatoriamente pertencerem ao mesmo segmento.

\item WindowDiff. Pevzner~\cite{Pevzner2002} aponta problemas na avaliação mais tradicional P$_k$~\cite{Beeferman1999}. Eles apontam que esse método penaliza demasiadamente os falsos negativos em relação aos falsos positivos e a \textit{near misses}, além disso, desconsidera o tamanho e a quantidade de segmentos. Como solução, propõem um novo método, o qual chamam de \textit{WindowDiff} que traz duas diferenças principais: a dobra a penalidade para os falsos positivos a fim de diminuir o problema da subestimação dessa medida e, diferente de P$_k$, ao mover a janela pelo texto, penaliza-se o algoritmo sempre que o número de limites proposto pelo algoritmo não coincidir com o número de limites esperados para aquela janela de texto. 

Com isso, demonstram em seu trabalho que, em relação a P$_k$, consegue resolver seus principais problemas e mantém sua proposta inicial de sensibilidade a \textit{near misses}, penalizando-os menos que os falsos positivos puros.


\end{enumerate}

% Baseando-se em medidas de avaliação tradicionais que utilizam a quantidade de acertos 


%	As medidas de avaliação tradicionalmente utilizadas na área de recuperação de informação como precisão e revocação trazem alguns problemas na avalização de segmentadores automáticos. 
%	
%	se considerarmos 

%	As medidas de avaliação tradicionalmente utilizadas na área de recuperação de informação como  
%
%Precisão é a proporção de limites identificados pelo algoritmo que são verdadeiros
%
%Revocação e a proporção limites verdadeiros que foram identificados pelo algoritmo

%corretamente identificados pelo algorítmo	
	 



%Para quantificar a eficiência dos algoritmos, segue uma revisão das principais métricas aplicáveis.



