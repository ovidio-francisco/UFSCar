\section{Resultados}
	\label{sec:resultados}

%%%%%%%%%%
% Objetivos
%%%%%%%%%%
A fim de encontrar o melhor método que divida uma ata em segmentos coerentes, realizou-se experimentos com o \textit{TextTiling} e \textit{C99} a fim de encontrar os melhores parâmetros para esses documentos.

%tópicos retornando segmentos  que retorne segmentos coerentes 

%%%%%%%%%%
% Parâmetros
%%%%%%%%%%
As implementações dos algoritmos permitem ao usuário a configuração de seus parâmetros. 
%
O \textit{TextTiling} permite ajustarmos dois parâmetros, sendo, o tamanho da janela (distância entre a primeira e a última sentença) para o qual atribuiu-se os valores 20, 40 e 60. Para o segundo parâmetro, o passo (distância que a janela desliza), atribuiu-se os valores 3, 6, 9 e 12. Gerando ao final 20 modelos.
%

O \textit{C99} permite ajustarmos três parâmetros, sendo, a quantidade segmentos desejados, o qual é calculado como uma proporção dos candidatos a limite. Para isso atribuiu-se as proporções {0,2; 0,4; 0,6; 0,8; 1,0}. O segundo parâmetro, o tamanho da máscara utilizada para gerar a matriz de ranking, atribuiu-se os valores 9 e 11. Permite ainda, definirmos se as sentenças serão representados por vetores contendo a frequência ou o peso de cada termo, onde ambas as representações foram utilizadas.
%
 Considerando todos os parâmetros, foram gerados 20 modelos para o algoritmo C99.% By Rafael 



%%%%%%%%%%
% Cálculo das medidas para cada modelo
%%%%%%%%%%

% --> Isso já foi falado no texto
%Pela comparação dos resultados com a segmentação fornecida pelos especialistas, calculou-se para cada modelo as medidas tradicionais acurácia, precisão, revocação, F-medida. Além dessas, computou-se também as métricas mais aplicadas à segmentação textual P$_k$ e \textit{WindowDiff}.



%%%%%%%%%%
% Teste de Fiedman e CD
% 1ª Etapa
%%%%%%%%%%
Em seguida aplicou-se o teste de Friedman a fim de saber se há diferenças significativas entre a eficácia dos modelos. O pós-teste de Nemenyi foi aplicado para descobrir quais diferenças são significativas. 
%
Exite diferença quando seus \textit{rankings} médios diferirem em um valor mínimo, chamado de diferença critica (CD). 
%

%%%%%%%%%%
% Dados Obtidos
%%%%%%%%%%
Com isso foi possível, pela análise do diagrama de diferença crítica, verificar qual é o melhor modelo para cada medida
% e quão significativamente 
em relação aos demais. 


A Tabela~\ref{tab:mediasC99} mostra os dados obtidos com o \textit{C99}, onde \texttt{S} é a proporção de segmentos em relação a quantidade de candidatos, \texttt{M} é o tamanho da máscara utilizada para criar a matriz de \textit{ranking} e \texttt{W} indica se os segmentos são representados por vetores contendo a frequência ou um peso das palavras. 



\begin{table}[!h]
	\centering

	\begin{tabular}{|c|c|c|c|c|}
	
		\hline
		Medida & \texttt{S} & \texttt{M} & \texttt{W} & \textbf{Média}\\		
		\hline

		Acuracy		& 40	& 11 & Sim & 0.6199	\\ \hline	
		F1			& 60	& 9	 & Sim & 0.6167	\\ \hline	
		Precision	& 40	& 11 & Sim & 0.7106	\\ \hline			
		Recall		& 100	& 9	 & Não & 0.8516	\\ \hline		
		Pk			& 40	& 11 & Sim & 0.1163	\\ \hline	
		Windiff		& 40	& 11 & Sim & 0.3800	\\ \hline		

		
	\end{tabular}
	
	\caption{Médias das medidas obtidas com \textit{C99}}
	\label{tab:mediasC99}
\end{table}


A tabelas~\ref{tab:mediasTextTiling} mostra os dados obtidos com o \textit{TextTiling}, onde \texttt{J} é o tamanho da janela e \texttt{P} é o passo.

\begin{table}[!h]
	\centering

	\begin{tabular}{|c|c|c|c|}
	
		\hline
		Medida & \texttt{J} & \texttt{P} & \textbf{Média}\\		
		\hline

		Acuracy		& 50 & 9 	& 0.5510 \\ \hline	
		F1			& 50 & 3 	& 0.5898 \\ \hline	
		Precision	& 60 & 12 	& 0.5746 \\ \hline			
		Recall		& 50 & 3 	& 0.7717 \\ \hline		
		Pk			& 30 & 9 	& 0.1572 \\ \hline	
		Windiff		& 50 & 9 	& 0.4489 \\ \hline		

		
	\end{tabular}
	
	\caption{Médias das medidas obtidas com o \textit{TextTiling}.}
	\label{tab:mediasTextTiling}
\end{table}


Uma vez sabendo quais valores de parâmetros melhor configuram um algoritmo para uma medida, resta então saber qual dos dois algoritmos é mais eficiente segundo essa medida. Para isso aplicou-se novamente o teste de Friedman com pós-teste de Nemenyi, dessa vez, com os melhores modelos dos dois algoritmos para cada medida. O resultado segue na Tabela~\ref{tab:melhoresmodelos}

\begin{table}[!h]
	\centering
	
	\begin{tabular}{|c|c|c|c|c|}

		\hline
		Medida & Algoritmo & \texttt{S} & \texttt{M} & \texttt{W}\\		
		\hline
		
	
		Acuracy		& C99 & 40 	& 11	& Sim \\ \hline
		Precision	& C99 & 40 	& 11	& Sim \\ \hline
		Pk			& C99 & 40 	& 11	& Sim \\ \hline
		Windiff		& C99 & 40 	& 11	& Sim \\ \hline
		F1			& C99 & 60 	& 9		& Sim \\ \hline
		Recall		& C99 & 100 & 9		& Não \\ \hline
 	
	
	\end{tabular}

	\caption{Melhores modelos para cada medida segundo diagramas de diferença crítica.}
	\label{tab:melhoresmodelos}	
	
\end{table}


Na análise do diagrama de diferença crítica verificou-se que o algoritmo \textit{C99} apresenta melhor eficiência em todas as medidas e os valores das quatro primeiras os valores de \texttt{S}, \texttt{M} e \texttt{W} se repetiram, sugerindo uma configuração otimizada para o problema da segmentação de atas de reunião.





