

\section{Introdução}
	\label{sec:introducao}

%%%%%%%%%%
% Reuniões
%%%%%%%%%%
Reuniões são tarefas presentes em atividades corporativas, ambientes de gestão e nas organizações de um modo geral. São frequentemente registradas em texto na forma de atas de reunião para fins de documentação e consulta posterior. A organização e consulta manual desses arquivos torna-se uma tarefa custosa, especialmente considerando o seu crescimento em uma instituição. 

%%%%%%%%%%
% Atas são documentos não estruturados
%%%%%%%%%%
As atas são documentos textuais e portanto constituem dados não estruturados, assim, um sistema que automaticamente responde a consultas do usuário ao conteúdo das atas é um desafio que envolve a sua compreensão~\cite{Bokaei2015}. 
%%%%%%%%%%
% Segmentar e extrair tópicos
%%%%%%%%%%
Uma vez que a ata registra a sucessão de assuntos discutidos na reunião, tal sistema tem duas principais tarefas: descobrir quando há uma mudança de assunto, e quais são esses assuntos~\cite{Banerjee2006}. Esse trabalho tem foco principal na primeira tarefa, a segmentação topical automática.


%%%%%%%%%%
% Descrição das Atas
%%%%%%%%%%
Frequentemente atas de reunião tem a característica de apresentar um texto com poucas quebras de parágrafo e sem marcações de estrutura, como capítulos, seções ou quaisquer indicações sobre o tema do texto. 

%%%%%%%%%%
% Definição da Tarefa
%%%%%%%%%%
A tarefa de segmentação textual consiste dividir um texto em partes que contenham um significado relativamente independente. Em outras palavras, é identificar as posições onde há uma mudança significativa de tópicos.

%%%%%%%%%%
% Utilidade da segmentação (domínios onde é aplicada)
%%%%%%%%%%
É útil em aplicações que trabalham com textos sem indicações de quebras de assunto, ou seja, não apresentam parágrafos, seções ou capítulos, como transcrições automáticas de áudio, vídeos e grandes documentos que contêm vários assuntos como atas de reunião e noticias.


%%%%%%%%%%
% Utilidade da segmentação
%%%%%%%%%%
O interesse por segmentação textual tem crescido em em aplicações voltadas a recuperação de informação~\cite{Reynar1999} %citar o [15] ...
e sumarização de textos~\cite{Boguraev2000}~\cite{Boguraev2000b}. % ... e [2] do "Efficient Linear T S"
Essa técnica pode ser usada para aprimorar o acesso a informação quando essa é solicitada por meio de uma consulta, onde é possível oferecer porções menores de texto mais relevantes ao invés de exibir um documento maior que pode conter informações menos pertinentes. 
A sumarização de texto pode ser aprimorada ao processar segmentos separados por tópicos ao invés de documentos inteiros. A navegação pelo documento pode ser aprimorada, em especial na utilização por usuários com deficiência visual~\cite{Choi2000}.



%%%%%%%%%%
% As Atas
%%%%%%%%%%
Assim, esse trabalho trata da adaptação e avaliação de algoritmos tradicionais ao contexto de documentos em português do Brasil, com ênfase especial nas atas de reuniões.


%As atas, como frequentemente são, apresentam-se como uma sucessão de tópicos. Assim, o objetivo desse trabalho é identificar, automaticamente, onde há a mudança de um tópico para seus adjacentes.


% Diversas aplicações fazem uso de segmentação textual, incluindo 

% Entre as principais mais frequentes de segmentação textual estão a tra



%É principalmente utilizada em aplicações que processam textos longos como transcrições de áudio e documentos longos, além de aprimorar técnicas de sumarização e information retrievel.



% Usos:
%	* quando não há identificações
%	* em transcrições de áudio
%	* em documentos longos
% 	* text summarization (ver a referencia [2] de Efficient Linear Text...)




%Isto é, dado um texto, identificar onde há mudança de tópicos.


% Interest in automatic text segmentation has blossomed over the last few years, with applications ranging from information retrieval to text summariza-tion to story segmentation of video feeds. [A Critique and Improvement of an Evaluation Metric for Text Segmentation]



%Em outras palavras é identificar divisões entre unidade de informação sucessivas

%A tarefa de segmentação textual consiste em encontrar pontos onde há mudança de tópicos no texto.



%[ The task of linear text segmentation is to split a large text document into shorter fragments, usually blocks of consecutive sentences. ]


% **Segmentação é identificar divisiões entre unidades de informação sucessivas (Beeferman, Berger, and Lafferty (1997))**

%   [Text segmentation is the task of determining the positions at which topics change in a stream of text]




