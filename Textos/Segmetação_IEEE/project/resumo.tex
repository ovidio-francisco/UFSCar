\begin{abstract}

%%%%%%%%%%
% Breve definição de segmentação
%%%%%%%%%%
A tarefa de segmentação topical consiste em dividir um texto em porções com significado relativamente independente, de maneira que cada segmento contenha um assunto ou tópico. 
%
%
%%%%%%%%%
% Literatura focada em documentos longos e em inglês
%%%%%%%%%
Muitas pesquisas avaliam técnicas usando textos longos como a concatenação de notícias de vários assuntos ou obtidos pela transcrição de discursos ou reuniões com múltiplos participantes.  
%Por conta disso, muitos segementadores apresentam  melhor performance quando apclicados nessas condições, 
%
%%%%%%%%%%
% Maior foco e perfomance no inglês
%%%%%%%%%%
Nota-se também uma maior atenção ao idioma inglês. Por conta disso, muitos segmentadores apresentam melhor performance nessas condições.
%
%
%%%%%%%%%%
% Peculiaridades das atas
%%%%%%%%%%
As atas de reunião 
%frequentemente são escritas em parágrafo único  e não apresentam marcações de estrutura como capítulos ou seções. Além disso, 
possuem estilo de redação mais formal e sucinto, o que desfavorece técnicas baseadas em frequência de palavras.
%%%%%%%%%%
% Utilidade
%%%%%%%%%%%
A segmentação desses documentos facilita a sua organização e acesso, oferecendo vantagens no processo de recuperação de informação em relação à busca por palavras chave pois é possível retornar porções de texto menores que contenham tópicos relevantes à intenção do usuário, ao invés de destacar as palavras chave em um texto longo que pode conter informações menos pertinentes. Além disso, pode aprimorar a navegação pelo documento, sobre tudo por usuários com deficiência visual.
% Sumarização  ??????
% Etapa de pre-processamento ???
%apresenta vantagens em relação a organização manual 
%%%%%%%%%%
% Objetivo: Adaptar algs e encontar modelo
%%%%%%%%%%%
Este artigo se concentra nos principais algoritmos da literatura adaptando-os ao contexto das atas de reunião em português, bem como encontrar um modelo que ofereça segmentos coesos e contribua para aprimorar sistemas de recuperação de informação para que respondam melhor à buscas do usuário.
%%%%%%%%%%
% Modelo melhor avaliado
%%%%%%%%%%
Para a avaliação, um conjunto de atas  do departamento de pós-graduação da UFSCar-Sorocaba foi manualmente segmentado por participantes das reuniões, então comparou-se as divisões manuais com as geradas automaticamente a fim de encontrar um método que melhor retorne segmentos coesos de atas de reunião. Ao final, chegou-se a um modelo capaz de segmentar as atas com eficiência similar à trabalhos anteriores, entregando segmentos com significado relativamente independente do documento. 
%
%


% se levado em conta a peculiaridade desses documentos.



%Os testes registraram precisão de 0.7106, revocação de 0.8516, as medidas P$_k$ e \textit{WindowDiff} mostram respectivamente 0.1163 e 0.3800 de dissimilaridade.





% na adaptação dos algoritmos mais influentes ao contexto das atas de 

%a fim de melhor responder a buscas do usuário em sistemas de recuperação de informação.

%divida um documento em partes .



% os influentes tratam muito de discursos e textos longos
% as atas apresentam texto mais enxuto/compacto e com estilo próprio.
% as atas são uma paráfrase da reunião

\end{abstract}
