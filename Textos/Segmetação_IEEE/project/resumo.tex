\begin{abstract}


A tarefa de segmentação topical consiste em dividir um texto em porções com significado relativamente independente, de maneira que cada segmento contenha um assunto. 
%
%
%A segmentação de atas de reuniões é útil na organização desses documentos e para facilitar o seu acesso.
%
%
Este artigo se concentra nos algoritmos mais influentes adaptando-os ao contexto das atas de reunião em português, bem como a encontrar um modelo que ofereça segmentos coesos e contribua a sistemas de recuperação de informação para que respondam melhor à buscas do usuário.
%
%
Para a avaliação, um conjunto de atas foi manualmente segmentado por participantes das reuniões, então comparou-se as divisões manuais com as geradas automaticamente. Os testes registraram precisão de 0.7106, revocação de 0.8516, as medidas P$_k$ e \textit{WindowDiff} mostram respectivamente 0.1163 e 0.3800 de dissimilaridade.





% na adaptação dos algoritmos mais influentes ao contexto das atas de 

%a fim de melhor responder a buscas do usuário em sistemas de recuperação de informação.

%divida um documento em partes .



% os influentes tratam muito de discursos e textos longos
% as atas apresentam texto mais enxuto/compacto e com estilo próprio.
% as atas são uma paráfrase da reunião

\end{abstract}
