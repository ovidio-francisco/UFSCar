\begin{abstract}
%
% Breve definição de segmentação
%%%
A tarefa de segmentação textual consiste em dividir um texto em porções com significado relativamente independente, de maneira que cada segmento contenha um assunto. 
%
% Literatura focada em documentos longos
%%%
Muitas pesquisas avaliam técnicas de segmentação texutal usando textos longos com a concatenação de notícias de vários assuntos ou com a transcrição de discursos ou reuniões com múltiplos participantes.  
%
% Maior foco e perfomance no inglês
%%%
Além disso, a maior parte das pesquisas é realizada considerando o idioma inglês.
%
% Peculiaridades das atas
%%%
Esse artigo tem como objeto de estudo atas de reunião escritas em português, as quais as além do idioma menos estudado sob essas técnicas, possuem estilo de redação mais formal e sucinto, o que desfavorece técnicas baseadas em frequência de palavras. 
%
% Utilidade
%%%
A segmentação desses documentos facilita a sua organização e acesso, oferecendo vantagens no processo de recuperação de informação em relação à busca por palavras-chave, pois é possível retornar porções de texto menores que contenham assuntos relevantes à intenção do usuário, ao invés de destacar as palavras-chave em um texto longo que pode conter informações menos pertinentes. 
%
% Objetivo: Adaptar algs e encontar melhor técnica
%%%
Este artigo concentra-se nos principais algoritmos da literatura aplicando-os ao contexto das atas de reunião em português, bem como encontrar uma técnica que ofereça segmentos coesos e contribua para aprimorar sistemas de recuperação de informação para que respondam melhor à buscas do usuário.
%
% Configuração melhor avaliada
%%%
Para a avaliação experimental, um conjunto de atas do departamento de pós-graduação da UFSCar-Sorocaba foi manualmente segmentado por participantes das reuniões. Então, as divisões manuais foram comparadas com as divisões geradas automaticamente a fim de encontrar uma configuração que melhor retorne segmentos coesos de atas de reunião. Ao final, chegou-se a uma técnica capaz de segmentar as atas com performance similar a outros trabalhos da literatura. 
%
%
\end{abstract}
	





% As atas de reunião possuem estilo de redação mais formal e sucinto, o que desfavorece técnicas baseadas em frequência de palavras.

% Por conta disso, muitos segmentadores apresentam melhor performance nessas condições.

% Sumarização  ??????
% Etapa de pre-processamento ???
%apresenta vantagens em relação a organização manual 

%frequentemente são escritas em parágrafo único  e não apresentam marcações de estrutura como capítulos ou seções. Além disso, 

% Além disso, a segmentação de textos longos pode aprimorar a navegação pelo documento, sobre tudo por usuários com deficiência visual.

% entregando segmentos com significado relativamente independente do documento. 

% se levado em conta a peculiaridade desses documentos.

%Os testes registraram precisão de 0.7106, revocação de 0.8516, as medidas P$_k$ e \textit{WindowDiff} mostram respectivamente 0.1163 e 0.3800 de dissimilaridade.

% na adaptação dos algoritmos mais influentes ao contexto das atas de 

%a fim de melhor responder a buscas do usuário em sistemas de recuperação de informação.

%divida um documento em partes .



% os influentes tratam muito de discursos e textos longos
% as atas apresentam texto mais enxuto/compacto e com estilo próprio.
% as atas são uma paráfrase da reunião

