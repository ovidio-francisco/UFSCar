
\section{Conjunto de documentos}
	\label{sec:conjutodedocumentos} 




Caracteristicas das atas
  Tamanhos diferentes
  # paginas
  # palavras
  # sentenças
  # media de segmentos pelos especialistas












%
% 1 Atas coletadas para segemntação automática 
%%% 

%
% 2 Atas segmentadas manualmente para referência 
%%% 
% A fim de obter referências para os algoritmos de segmentação, extração de tópicos e classificação, foram coletadas as atas de reuniões do Conselhos de Pós Graduação, Conselho de Cursos e Conselho de Departamento de da UFSCar-Sorocaba. Os documentos foram oferecidos à profissionais que participam de reuniões desse departamento para segmentá-las. Para isso, utilizou-se um \textit{software} que permitiu aos voluntários visualizar um documento, e indicar livremente as divisões entre segmentos com um assunto relativamente independente. Ao final, o software coletou os dados da segmentação de doze das os quais serviram como referência para a avaliação dos algoritmos.

% Os arquivos gerados foram tratados para remoção de ruídos e ajustes para que os segmentos sempre terminem em uma sentença reconhecida pelo algoritmo, uma vez que as sentenças são a unidade mínima de informação nesse trabalho.














% ------------------------------------------------------------------------------

% Para a condução desse trabalho é necessário que hajam documentos a serem segmentados automaticamente e que se conheça uma segmentação de referência para compará-los e avaliar a proposta. A seguir, é apresentada a segmentação de referência a ser utilizada.
% A seguir, são apresentadas as caraterísticas particulares das atas que foram coletadas para segmentação automática em relação a outros tipos de textos. É apresentada também a segmentação de referência a ser utilizada.

% Em outras palavras, uma ata é a narração de uma reunição que poupa




% Mais detalhes referentes às diferenças entre as atas e os textos analisados na maioria dos trabalhos são mostrados na Subseção~\ref{subsec:conjunto-de-documentos}.


%%%%%%%%%%
% Características particulares
%%%%%%%%%%
	
% \subsection{Segmentação de Referência}

% As atas de reunião diferem dos textos comumente estudados em outros trabalhos em alguns pontos. O estilo contribui para a escrita de textos sucintos com poucos detalhes, pois o ambiente dá preferência a textos curtos. Segundo Choi~\cite{Choi2001-LSA}, o segmentador tem a acurácia reduzida em segmentos curtos (em torno de 3 a 5 sentenças). % Já te vi 
	
% Para evitar um texto monótono à leitura, ao redigir o documento tem-se o cuidado de não repetir ideias e palavras em favor da elegância do documento. Tal característica enfraquece a coesão léxica e portanto o cálculo da similaridade é prejudicado. Por exemplo, duas sentenças diferem se uma contiver a palavra \textit{computadores} e na seguinte \textit{equipamentos}, mesmo que se refiram à mesma ideia. Além disso, o documento compartilha um certo vocabulário próprio do ambiente onde os assuntos são discutidos e com isso nota-se que os segmentos, embora tratem de assuntos diferentes, são semelhantes em vocabulário.


% A presença de ruídos como cabeçalhos, rodapés e numeração de páginas e linhas prejudicam tanto similaridade entre sentenças como a apresentação final ao usuário. Porém, esses ruídos podem ser reduzidos ou eliminados como mostrado na Subseção~\ref{subsec:preprocessamento}, sobre preprocessamento.



%%%%%%%%%%
% Há duas diferenças entre os tipos de documentos mais estudados
%%%%%%%%%%
% A maioria dos trabalhos enquadram-se em duas categorias no que se refere ao \textit{corpus} estudado.  
%%%%%%%%%%
% A concatenação de textos favorece a segmentação
% E diferenças
% 
%   1 - Mesmo autor e mesmo domínio --> mesmo estilo e vocabulário
%
%%%%%%%%%%
% A primeira utiliza a concatenação de textos de diferentes fontes e assuntos. 
% Ocorre que essas características favorecem o processo de segmentação uma vez que os documentos produzidos pela concatenação contém tópicos obtidos de domínios distintos que compartilham pouco vocabulário. Por outro lado, essas características não estão presentes no contexto das atas, onde um documento é redigido por um mesmo autor, e todos tópicos foram produzidos no mesmo domínio, e dessa forma, têm estilo de escrita e vocabulário similares.


%%%%%%%%%%
% Discursos falados
% E difenças
%
%   1 - texto sucinto, sem detalhes e reptições --> segmentos mais curtos
%
%%%%%%%%%%
% A segunda estuda a segmentação automática de textos obtidos da transcrição de áudios como de vídeos e de gravações de reuniões com grupos de pessoas. 
% A transcrição da fala de pessoas, difere desse trabalho, pois este possui características típicas de um estilo formal, por exemplo, o texto sucinto, onde o autor se limita a relatar apenas o essencial de uma discussão e omite detalhes da discussão, poupando o leitor de diálogos repetitivos ou falas durante a transição de tópicos, o que resulta em segmentos mais curtos os quais desfavorecem algoritmos baseados em coesão léxica~\cite{Choi2000}. 




