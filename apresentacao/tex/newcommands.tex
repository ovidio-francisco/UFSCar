\newcommand{\sumario}{
\begin{frame}[plain]
   \frametitle{Plano de Aula}
   \tableofcontents
\end{frame}
}

\newcommand{\frameTalkIsCheap}{
\begin{frame}{Exercícios}

	\addimage{images/talkischeap3}{0.65}{}

	\begin{flushright}
		Linus Torvalds
	\end{flushright}
\end{frame}		
}

\newcommand{\frameDuvidas}{
\begin{frame}{Dúvidas?}
	\addimage{images/questions.png}{0.65}{}
	 	
\end{frame}		
}


\newcommand{\addimage}[2]{
  \begin{figure}[!h]
  \centering
  \includegraphics[width=#2\paperwidth]{#1}
%  \caption{#3}
%  \label{fig:#4}
  \end{figure}
}


\newcommand{\addpdfpageh}[4]{
\includegraphics[trim={#1cm #2cm #1cm #2cm},clip,page=#4,height=0.8\paperheight]{#3}
}
\newcommand{\addpdfpagew}[4]{
\includegraphics[trim={#1cm #2cm #1cm #2cm},clip,page=#4,width=0.8\paperwidth]{#3}
}


%%%%%%%%%%%%%%%%%%%%%%%%%%
% blocks
%%%%%%%%%%%%%%%%%%%%%%%%%%

\newcommand{\popc}[2]{
\begin{block}{#1}
	\pause	
	#2
\end{block}
}

\newcommand{\pop}[2]{
\pause	
\begin{block}{#1}
	#2
\end{block}
}

\newcommand{\nblock}[2]{
\begin{block}{#1}
	#2
\end{block}
}

\newcommand{\eblock}[2]{
\begin{exampleblock}{#1}
	#2
\end{exampleblock}
}

\newcommand{\ablock}[2]{
\begin{alertblock}{#1}
	#2
\end{alertblock}
}



% Uma macro que cria \newenvironments
\newcommand{\coloredblock}[3]{
	\newenvironment{#1}[1]
	{
		\setbeamercolor{block title}{fg=white,bg=#2!75!black}%
		\begin{block}{#3}
	}
	{
		\end{block}
	}	
}

% Criaçao de \newenvironments
\coloredblock{blackblockenv}{black}{#1}
\coloredblock{blueblockenv}{blue}{#1}
\coloredblock{redblockenv}{red}{#1}

% Macro para chamar as \newenvironments criadas
\newcommand{\blackblock}[2]{
	\begin{blackblockenv}{#1}
		#2
	\end{blackblockenv}
}
\newcommand{\redblock}[2]{
	\begin{redblockenv}{#1}
		#2
	\end{redblockenv}
}
\newcommand{\blueblock}[2]{
	\begin{blueblockenv}{#1}
		#2
	\end{blueblockenv}
}









%\newcommand{\coloredblock}[3]{
%	\setbeamercolor{block title}{fg=white,bg=#1!75!black}%
%	\begin{block}{#2}
%		#3
%	\end{block}
%}
%
%\newcommand{\redblock}[2]{
%	\coloredblock{red}{#1}{#2}
%}
%
%\newcommand{\yblock}[2]{
%	\setbeamercolor{block title}{fg=white,bg=yellow!75!black}%
%	\begin{block}{#1}
%		#2
%	\end{block}
%}



%\newenvironment<>{myblock}[1]
%{%
%	\setbeamercolor{block title}{fg=white,bg=red!75!black}%
%	\begin{block}#2{#1}
%}
%{
%	\end{block}
%}
%  
%
%
%\newenvironment{blueblock}[1]
%{
%	\setbeamercolor{block title}{fg=white,bg=blue!75!black}%
%	\begin{block}{#1}
%}
%{
%	\end{block}
%}
%
%
%
%
%\newcommand{\mblock}[2]{
%\begin{block}{#1}
%	#2
%\end{block}
%}

%
%
%% Uma macro que cria \newenvironments
%\newcommand{\coloredblock}[3]{
%
%	\newenvironment{#1}[1]
%	{
%		\setbeamercolor{block title}{fg=white,bg=#2!75!black}%
%		\begin{block}{#3}
%	}
%	{
%		\end{block}
%	}
%	
%}
%
%% Criaçao de \newenvironment 
%\coloredblock{preto}{black}{#1}
%
%% Macro para chamar as \newenvironment criadas
%\newcommand{\blackblock}[2]{
%	\begin{preto}{#1}
%		#2
%	\end{preto}
%}