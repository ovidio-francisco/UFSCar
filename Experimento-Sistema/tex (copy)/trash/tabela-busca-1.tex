
\newpage


\subsection*{Resultado de uma busca}
%\subsection*{Resultado de um agrupamento}

Segue abaixo o resultado de uma pesquisa no sistema à um conjunto de atas públicas obtidas pelo Departamento de Ciência da Computada da UFSCar campus Sorocaba. O resultado a ser analisado são os 10 primeiros trechos apresentados quando o usuário pesquisou pelos termos ``\textit{verba para compra de equipamentos}''.

\vspace{0.2 cm}


%\begin{longtable}{|p{17.6cm}|}
%\hline 
%\textbf{Descritores:}  planos, ensino, sistema, disciplinas, algoritmos\\ 
%\hline 
%\end{longtable} 

\begin{description}
\item[Pesquisa por: ] ``\textit{verba para compra de equipamentos}''
\end{description}

\noindent
\textbf{Trechos apresentados pelo sistema:}
\begin{longtable}{|p{0.2cm}|p{17cm}|}
\hline 
1 & 
(III.VIII) Foi deliberado que será elaborado um ofício, onde será relatada a realidade desse semestre, informando também que esse cenário será agravado nos próximos semestres (III.IX) A profª Katti solicita que seja evidenciado que para o próximo semestre, será necessário que um laboratório seja disponibilizado para os alunos da computação. \\ \hline 
2 &
Informes: A professora Sahudy informou que o ofício a respeito do processo de seleção do PIBIC será enviado, no entanto o tom do ofício foi suavizado, informou também que tem um valor em auxílio estudante que precisará liquidar ainda este ano, pois provavelmente ele não virará o ano por ser de 2015, sendo assim está aceitando sugestões para gastos do referido dinheiro com apoio a projetos de disciplinas no valor máximo de até 800 reais e o que sobrar será enviado como ajuda custos para SeCoT 2017.
 \\ \hline 
3 &
Informes:. (I) RESPOSTA AO OFÍCIO DE COMPRAS. (I.I) A professora Sahudy informou que somente o setor de compras/Sorocaba respondeu ao ofício 038/2015, os demais não se manifestaram, e ainda devolveram os ofícios, pois entendem que a responsabilidade da resposta do ofício em questão é de competência do setor de compras Sorocaba. Os professores não aceitaram tal justificativa e solicitaram que os ofícios sejam enviados novamente. O pedido foi aceito. (II) INFORME SOBRE COMPRAS DE MATERIAL PERMANENTE. (II.I) Foi informado que serão atendidos as prioridades 1 e 2 da tabela de preferência.
 \\ \hline 
4 &
Informes:. A professora Sahudy informou que no semestre anterior foi aprovado em reunião um dia específico para a manutenção dos laboratórios e pediu para o técnico não marcar médico e respeitar os horários nos dias da manutenção e também fazer um relatório dos problemas encontrados.
 \\ \hline 
5 &
(II.I) A professora Sahudy informou que gastou R\$ 3584,65 com material de informática, e que a cópia de prestação de contas está na secretaria do DComp, e informou também que a verba RTI de 2015 também já foi liberada, porém só será utilizada quando terminar a prestação de contas do CCTS referente a 2014. (III) REDISTRIBUIÇÃO DO TÉCNICO THIAGO. (III.I) A professora Sahudy informou que o técnico Thiago André Pereira Leite, solicitou redistribuição para o Instituto Federal de São Paulo e que ela aprovou ad referendum o pedido do mesmo, não esquecendo que tal redistribuição tem como condição a permuta de vagas.
 \\ \hline 
6 &
Informes: O Prof. Gustavo informou que os pedidos de compra estão sendo realizados no novo sistema como planejado. Também informou que o ar-condicionado do LEC foi instalado. 
\\ \hline 
7 &
(II) Informou também sobre o atraso da verba 18 PROAP para 2015, que segundo informações da Pró-Reitora deverá ser liberado ao final 19 de fevereiro;
Informes:.A professora Cândida informou que a licitação de compra dos aparelhos de ar condicionado deu certo e que em breve os mesmos serão instalados; também informou que a atividade de extensão (Desenvolvimento em Nuvem;
 \\ \hline 
8 &
3.4 Verba do curso: foi deliberado que a verba da coordenação do curso será destinada para compra de ar condicionado e para a organização do Workshop de Intercâmbio de Experiências em Computação que está sob responsabilidade do prof. José de Oliveira Guimarães.
 \\ \hline 
9 &
(IV) PATRIMÔNIOS (TERMOS DE RESPONSABILIDADE). (IV.I) A professora Sahudy esclareceu que foi enviado um e-mail ontem com as duas opções que seriam discutidas hoje e colocada em votação. 1) O chefe do departamento ficaria responsável pelos bens comuns e cada um seria responsável pelo que tem na sala. 2) O chefe ficaria responsável por todos os patrimônios. Foi colocado em votação, sendo 5 votos para primeira opção e 2 para a segunda. Sobre os equipamentos de doações vamos verificar em que nome tem saído os termos de responsabilidade.
 \\ \hline 
10 &
Informes: Compras: foi enviada a requisição da compra de aparelhos de ar condicionado para todas as salas do departamento e secretarias, utilizando toda verba de material permanente do departamento, com a verba do curso foram comprados dois aparelhos e um com repasse FAI e ainda ficamos com uma dívida de cerca de R\$ 878,77 com o centro a qual será paga o ano que vem, ou pode vir a ser compensada após a licitação acontecer, caso consigamos comprar por um valor menor que o empenhado. As compras envolvendo mobiliário e computadores serão licitados de formas diferentes, será realizado um pedido para toda a universidade e serão comprados via ata, maiores informações de gastos verificar planilha anexa.

 \\ \hline 

\end{longtable} 


