\newpage

\subsection*{Resultado de uma consulta (PLSA)}

Segue abaixo o resultado de uma pesquisa no sistema à um conjunto de atas públicas obtidas pelo Departamento de Ciência da Computação da UFSCar campus Sorocaba. O resultado a ser analisado são 5 trechos apresentados quando o usuário pesquisou pelos termos ``\textit{compra de equipamentos}''.

\vspace{0.5 cm}

\begin{description}
	\item[Palavras descritoras: ] \textit{verba, compra, pagamento, valor e realizada}
\end{description}


\noindent
\textbf{Trechos apresentados pelo sistema:}
\begin{longtable}{|p{17.5cm}|}
\hline 
%1 & 
Informes: Compras: foi enviada a requisição da compra de aparelhos de ar condicionado para todas as salas do departamento e secretarias, utilizando toda verba de material permanente do departamento, com a verba do curso foram comprados dois aparelhos e um com repasse FAI e ainda ficamos com uma dívida de cerca de R\$ 878,77 com o centro a qual será paga o ano que vem, ou pode vir a ser compensada após a licitação acontecer, caso consigamos comprar por um valor menor que o empenhado. As compras envolvendo mobiliário e computadores serão licitados de formas diferentes, será realizado um pedido para toda a universidade e serão comprados via ata, maiores informações de gastos verificar planilha anexa.

 \\ \hline 
%2 &
No momento o saldo que o departamento tem na FAI é de apenas R\$ 264,91. Em relação aos ofícios que o conselho decidiu enviar a vários órgãos cobrando explicações sobre os cancelamentos das compras, a professora Cândida informou que tal ofício já foi enviado ao CCGT, Setor de Compras Sorocaba, Reitoria, ProAd, ProGrad e Apoio ProAd.; no entanto só recebemos até agora a resposta do Setor de Compras Sorocaba. O ofício foi lido para todos os membros e diante da resposta do mesmo a sugestão é que para o ano que vem, 2016, todos os pedidos de compras sejam colocados no sistema o mais rápido possível. Foi informado que a impressora do prédio já foi instalada e como o custo da impressão por esta impressora é bem mais baixo foi pedido a todos que as impressões sejam feitas na impressora em questão; porém a conexão por rede ainda não está funcionando, então para imprimir ainda é necessário o uso do pen drive. O professor Mário colocou que quando é cobrado do departamento o valor é referente ao papel e a impressão em si, no entanto não tem papel disponível e cada professor está levando seu próprio papel, sendo assim o departamento irá pagar duas vezes pelo mesmo papel. Diante deste problema iremos contatar o centro para que o mesmo nos apresente uma solução. Sobre os pagamentos de pró-labores para membros de banca da pós, foi informado que para este ano temos apenas mais dois, o qual está sendo pago com saldo de empenho de 2014 e como provavelmente este empenho não irá virar para 2016 a solução encontrada para não perder dinheiro foi elevar um pouco o valor do pagamento dos pró-labores. Foi realizado um pedido de inclusão de pauta pela professora Tiemi, a qual pede ajuda financeira do departamento para divulgação da calourada 2016, o pedido foi aceito. 
% Ordem do Dia: (I) APROVAÇÃO DAS ATAS DAS REUNIÕES ANTERIORES. (I.I) As Atas foram lidas aprovadas e assinadas pelos membros. (II) APROVAÇÃO DE RELATÓRIO ATIVIDADE DE EXTENSÃO, “ENCONTRO REGIONAL SOROCABAJS”, COORDENADORA PROFA. SAHUDY. (II.I) O relatório já foi aprovado ad referendum e trazido para conselho para tomar ciência.(III) DELIBERAÇÃO SOBRE A REOFERTA DA ATIVIDADE DE EXTENSÃO, “MBA ECONOMIA E NEGÓCIOS”, PARTICIPAÇÃO DO PROFESSOR ALEXANDRE ÁLVARO. (III.I) O reoferta já foi aprovado ad referendum e trazida para conselho para tomar ciência. (IV) CONSULTA SOBRE A INSTALAÇÃO DE DATA-SHOW NAS SALAS DO CENTRO, COM RESSARCIMENTO DO CCGT AO DEPARTAMENTO NO ANO QUE VEM. (IV.I) O centro (CCGT) sugeriu que cada departamento emprestasse um data-show para que os mesmos fossem instalados nas salas de aula do referido centro, pois no momento este não dispõe de verba de capital para compra de tais aparelhos. Foi observado que faltavam alguns esclarecimentos de detalhes de como vai ser esse ressarcimento. O Conselho aprova o empréstimo, porém decidiu que se a proposta for devolver o mesmo aparelho será pedida uma lâmpada nova para deixar de reserva. (V) CONSULTA SOBRE O USO DE PRÓ-LABORE PARA PAGAR INSTALAÇÃO DE INSULFILM NAS SALAS DE AULA DO CCGT. (V.I) Na última reunião do centro foi pedido que os chefes fizessem uma consulta aos membros de seus conselhos para saber se há disponibilidade de verba e se eles aprovam que cada departamento financie a instalação de insulfilm em uma sala de aula. Os membros do conselho aprovam a ajuda, porém quer saber de mais detalhes sobre como será realizado o pagamento, pois acreditamos que um montante tão alto assim não seja liberado com pró-labore e também questionou a eficiência desse produto, acreditam que para resolver o problema de luz e calor seria melhor que fossem instaladas cortinas. (VI) PEDIDO DA COORDENADORA DO BCCS PARA FINANCIAMENTO DE PREPARAÇÃO DE MATERIAL DE DIVULGAÇÃO DA CALOURADA. (VI.I) A Coordenadora do curso pediu ajuda para o departamento para financiar a preparação do material de divulgação da calourada que os alunos irão produzir. Tal ajuda seria entre 250,00 a 300,00 reais. (VI) Os membros do conselho aprovaram a solicitação no valor de R$ 300,00. Encerramento: Estando todos de acordo e nada mais havendo a deliberar, lavra-se, leia-se, aprova-se e assina-se esta Ata por todos os membros do conselho do Departamento de Computação de Sorocaba, Campus Sorocaba, participantes desta reunião que, em 02 (duas) vias, será levada à registro e arquivamento junto à chefia do Departamento de Computação de Sorocaba, ficando ali à disposição para consulta. Nada mais.

 \\ \hline 
%3 &
A estagiária foi contratada em 11 de junho e desde então tem auxiliado nas tarefas da secretaria dos cursos Ciência da Computação e Ciências Econômicas. 1.2.Verba: Os empenhos em Compra de materiais e pagamento estudantil estão em andamento, conforme definido na 30ª Reunião Ordinária do CoCCCS.

 \\ \hline 
%4 &
(II) VERBA DO CURSO: a Profa. Luciana sugeriu que fosse destinada parte da verba para custear a participação dos alunos nas maratonas da computação. O aluno Lúcio (2009) informou que está sem patrocinadores para a realização da SECOT 2012, com isso, a Profa. Luciana sugeriu destinar três empenhos de pró-labore, no valor de R$ 180,00 cada, com um total de R$ 540,00 para a SECOT, assim como o Prof. Fábio se comprometeu em contatar algumas empresas para negociar patrocínio.

 \\ \hline 
%5 &
Os membros do conselho decidiu que esse dinheiro será utilizado para compra de material para infraestrutura dos laboratórios e departamento. Os membros apresentaram as seguintes sugestões: braçadeiras, alicates, expositor etiquetas, conectores de rede (fêmea), extensões para as bancadas do LEC e tomadas nos lugares que estão faltando para o LabRedes.

 \\ \hline 

\end{longtable} 




