
\begin{longtable}{|p{17.5cm}|}
\hline 
%1 & 
Informes: Compras: foi enviada a requisição da compra de aparelhos de ar condicionado para todas as salas do departamento e secretarias, utilizando toda verba de material permanente do departamento, com a verba do curso foram comprados dois aparelhos e um com repasse FAI e ainda ficamos com uma dívida de cerca de R\$ 878,77 com o centro a qual será paga o ano que vem, ou pode vir a ser compensada após a licitação acontecer, caso consigamos comprar por um valor menor que o empenhado. As compras envolvendo mobiliário e computadores serão licitados de formas diferentes, será realizado um pedido para toda a universidade e serão comprados via ata, maiores informações de gastos verificar planilha anexa.

 \\ \hline 
%2 &
No momento o saldo que o departamento tem na FAI é de apenas R\$ 264,91. Em relação aos ofícios que o conselho decidiu enviar a vários órgãos cobrando explicações sobre os cancelamentos das compras, a professora AAA informou que tal ofício já foi enviado ao CCGT, Setor de Compras Sorocaba, Reitoria, ProAd, ProGrad e Apoio ProAd.; no entanto só recebemos até agora a resposta do Setor de Compras Sorocaba. O ofício foi lido para todos os membros e diante da resposta do mesmo a sugestão é que para o ano que vem, 2016, todos os pedidos de compras sejam colocados no sistema o mais rápido possível. Foi informado que a impressora do prédio já foi instalada e como o custo da impressão por esta impressora é bem mais baixo foi pedido a todos que as impressões sejam feitas na impressora em questão; porém a conexão por rede ainda não está funcionando, então para imprimir ainda é necessário o uso do pen drive. O professor BBB colocou que quando é cobrado do departamento o valor é referente ao papel e a impressão em si, no entanto não tem papel disponível e cada professor está levando seu próprio papel, sendo assim o departamento irá pagar duas vezes pelo mesmo papel. Diante deste problema iremos contatar o centro para que o mesmo nos apresente uma solução. Sobre os pagamentos de pró-labores para membros de banca da pós, foi informado que para este ano temos apenas mais dois, o qual está sendo pago com saldo de empenho de 2014 e como provavelmente este empenho não irá virar para 2016 a solução encontrada para não perder dinheiro foi elevar um pouco o valor do pagamento dos pró-labores. Foi realizado um pedido de inclusão de pauta pela professora CCC, a qual pede ajuda financeira do departamento para divulgação da calourada 2016, o pedido foi aceito. 

 \\ \hline 
%3 &
A estagiária foi contratada em 11 de junho e desde então tem auxiliado nas tarefas da secretaria dos cursos Ciência da Computação e Ciências Econômicas. 1.2.Verba: Os empenhos em Compra de materiais e pagamento estudantil estão em andamento, conforme definido na 30ª Reunião Ordinária do CoCCCS.

 \\ \hline 
%4 &
(II) VERBA DO CURSO: a Profa. DDD sugeriu que fosse destinada parte da verba para custear a participação dos alunos nas maratonas da computação. O aluno EEE (2009) informou que está sem patrocinadores para a realização da SECOT 2012, com isso, a Profa. DDD sugeriu destinar três empenhos de pró-labore, no valor de R$ 180,00 cada, com um total de R$ 540,00 para a SECOT, assim como o Prof. FFF se comprometeu em contatar algumas empresas para negociar patrocínio.

 \\ \hline 
%5 &
Os membros do conselho decidiu que esse dinheiro será utilizado para compra de material para infraestrutura dos laboratórios e departamento. Os membros apresentaram as seguintes sugestões: braçadeiras, alicates, expositor etiquetas, conectores de rede (fêmea), extensões para as bancadas do LEC e tomadas nos lugares que estão faltando para o LabRedes.

 \\ \hline 

\end{longtable} 




