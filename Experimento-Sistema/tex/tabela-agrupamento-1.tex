
\newpage


\subsection*{Resultado de um agrupamento}
%\subsection*{Resultado de um agrupamento}

Após realizar a busca por palavras chave, o usuário selecionou o grupo ao qual pertence o trecho 5 dos resultados anteriores.
Abaixo estão relacionados os 10 primeiros trechos que o sistema apresentou após essa ação bem como os descritores desse grupo.

\vspace{0.5 cm}


% é possível analisar o grupo ao qual cada trecho pertence para encontrar outros resultados similares. 



%\begin{longtable}{|p{17.6cm}|}
%\hline 
%\textbf{Descritores:}  planos, ensino, sistema, disciplinas, algoritmos\\ 
%\hline 
%\end{longtable} 

\begin{description}
\item[Descritores: ] planos, ensino, sistema, disciplinas, algoritmos
\end{description}


\noindent
\textbf{Trechos apresentados pelo sistema:}
\begin{longtable}{|p{0.2cm}|p{17cm}|}
\hline 
1 & 
Ordem do Dia: (I) APRECIAÇÃO DOS PLANOS DE ENSINO REFERENTE AO SEGUNDO PERÍODO LETIVO DE 2008: (I.I) O Prof. Dr. Siovani Cintra Felipussi solicita que cada consultor apresente o plano de ensino da disciplina que avaliou.
 \\ \hline 
2 &
(I.II) Após discussão todos os planos de ensino foram aprovados.
 \\ \hline 
3 &
(V.VIII) Após discussão, a oferta de disciplinas para os alunos do perfil três no primeiro semestre de dois mil e nove e respectivos professores que deverão confeccionar as fichas de caracterização ficou assim decidida: Disciplina Professor Introdução aos Sistemas de Informação José de Oliveira Guimarães Algoritmos e Complexidade José de Oliveira Guimarães Inteligência Artificial Katti Faceli Arquitetura e Organização de Computadores Yeda Regina Venturini Estrutura de Dados 2 Tiemi Christine Sakata Noções Básicas de Economia Danilo Rolim Gestão de Pequenas Empresas Danilo Rolim Probabilidade e Estatística Magda Peixoto Geometria Analítica e Álgebra Linear Lynnyngs Saraiva de Paiva Encerramento: Estando todos de acordo e nada mais havendo a deliberar, lavra-se, leia-se, aprova-se e assina-se esta Ata por todos os membros do Conselho do Curso de Bacharelado em Ciência da Computação, Campus Sorocaba, participantes desta reunião que, em 02 (duas) vias, será levada a registro e arquivamento junto à Coordenação do Conselho do Curso de Bacharelado em Ciência da Computação - Sorocaba, ficando ali à disposição para consulta restrita aos professores da UFSCar - Sorocaba.
 \\ \hline 
4 &
(I.II) Após discussão, decidiu-se os ministrantes das disciplinas a serem ofertadas no primeiro semestre de dois mil e nove como segue: Disciplina Professor Introdução aos Sistemas de Informação A contratar Algoritmos e Complexidade Tiemi Christine Sakata Algoritmos e Programação de Computadores Siovani Cintra Felipussi / A contratar Algoritmos e Programação de Computadores Yeda Regina Venturini / A contratar Inteligência Artificial Katti Faceli Arquitetura e Organização de Computadores A contratar Estrutura de Dados 2 A contratar Noções Básicas de Economia A definir pelo Prof. Danilo Rolim Gestão de Pequenas Empresas A definir pelo Prof. Jorge Meireles Probabilidade e Estatística A definir pela Profa. Magda Peixoto Geometria Analítica e Álgebra Linear A definir pela Profa. Magda Peixoto Física para Computação A definir pelo Prof. Antonio Riul Lógica para Computação José de Oliveira Guimarães Cálculo Diferencial e Integral 1 Laércio José dos Santos (II) DEFINIÇÃO DA GRADE HORÁRIA E TIPOS DE SALA.
 \\ \hline 
5 &
Ele solicita que cada professor ou professora apresente sua ficha de caracterização. 
(I.II) O Prof. Dr. José de Oliveira Guimarães apresenta as disciplinas Introdução aos Sistemas de Informação e Algoritmos e Complexidade.
 \\ \hline 
6 &
(I.IV) A Profa. Dra. Katti Faceli apresenta a disciplina Inteligência Artificial.
 \\ \hline 
7 &
(I.VII) Prof. Dr. Siovani Cintra Felipussi apresenta as disciplinas Noções Básicas de Economia, Gestão de Pequenas Empresas, Probabilidade e Estatística e Geometria Analítica e Álgebra Linear.
 \\ \hline 
8 &
(I.VIII) Após discussão sobre os objetivos, ementas, distribuição de créditos e requisitos, as fichas de caracterização para o perfil 3, a serem oferecidas no primeiro período letivo de 2009, foram aprovadas como segue: Disciplina CR Teo Lab CH Introdução aos Sistemas de Informação 04 04 00 60 Algoritmos e Complexidade 04 03 01 60 Inteligência Artificial 04 03 01 60 Arquitetura e Organização de Computadores 04 04 00 60 Estrutura de Dados 2 04 03 01 60 Noções Básicas de Economia 02 02 00 30 Gestão de Pequenas Empresas 02 02 00 30 Probabilidade e Estatística 04 04 00 60 Geometria Analítica e Álgebra Linear 04 04 00 60 A disciplina Geometria Analítica e Álgebra Linear teve sua ficha de caracterização aprovada com ressalva.
 \\ \hline 
9 &
Após discussão o espelhamento é aprovado por unanimidade como segue: Disciplina CR Teo Lab CH Algoritmos e Programação de Computadores 08 04 04 120 Cálculo Diferencial e Integral 1 04 03 01 60 Física para Computação 04 03 01 60 Geometria Analítica e Álgebra Linear 04 04 00 60 Lógica para Computação 04 03 01 60 (III) DISCUSSÃO SOBRE O REAPROVEITAMENTO DE VAGAS: (II.I) O presidente discorre brevemente sobre a distribuição de vagas já discutida em reunião anterior.
 \\ \hline 
10 &
A avaliação dos planos de ensino de cada disciplina oferecida no curso de Ciência da Computação foi atribuída para cada professor, as avaliações foram devidamente realizadas e os pareceres efetuados pelos respectivos professores bem como os planos de ensino na íntegra foram submetidos à avaliação do conselho que decidiu por unanimidade.
 \\ \hline 

\end{longtable} 


