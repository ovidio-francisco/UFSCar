
\newpage


\subsection*{Resultado de uma consulta}
%\subsection*{Resultado de um agrupamento}

%Após realizar a busca por palavras chave, o usuário selecionou o grupo ao qual pertence o trecho 5 dos resultados anteriores.
%Abaixo estão relacionados os 10 primeiros trechos que o sistema apresentou após essa ação bem como os descritores desse grupo.


Segue abaixo o resultado de uma pesquisa no sistema à um conjunto de atas públicas obtidas pelo Departamento de Ciência da Computação da UFSCar campus Sorocaba. O resultado a ser analisado são os 10 primeiros trechos apresentados quando o usuário pesquisou pelos termos ``\textit{defesa de dissertação}''.

\vspace{0.5 cm}


% é possível analisar o grupo ao qual cada trecho pertence para encontrar outros resultados similares. 



%\begin{longtable}{|p{17.6cm}|}
%\hline 
%\textbf{Descritores:}  planos, ensino, sistema, disciplinas, algoritmos\\ 
%\hline 
%\end{longtable} 

\begin{description}
\item[Palavras descritoras: ] orientada, meses, defesa, prazo, dissertação
\end{description}


\noindent
\textbf{Trechos apresentados pelo sistema:}
\begin{longtable}{|p{17.5cm}|}
\hline 
%1 & 
Também foi discutida a situação de alunos que recebem a bolsa no decorrer do curso e apresentam o exame de defesa antes do término do prazo da bolsa. A duração da bolsa oferecida pela CAPES é de 24 meses e foi apresentada a proposta do aluno continuar com a bolsa caso solicite prorrogação do prazo de defesa da dissertação, desde que mantenha os requisitos descritos na norma 5 e não tenha nenhum aluno na lista de espera do ranking de bolsas elaborado pela Comissão de Bolsas de Estudos do PPGCCS. Foi decidido retirar os seguintes requisitos da norma 5: 1. b) Não ter completado 12 (doze) meses corridos a contar da data de sua primeira matrícula no curso de Mestrado, exceto no caso de renovação. 3. d) ultrapassar 24 (vinte e quatro) meses no Programa, ou 3. e) não tiver cumprido a Norma 6 sobre obrigatoriedade de solicitação de Bolsa FAPESP.

 \\ \hline 
%2 &
(II) Aprovado o pedido de prorrogação de prazo para defesa de dissertação do aluno Daniel Ianegitz Vieira por mais 06 meses com prazo final para fevereiro de 2016. A partir de agora o formulário para solicitação de prorrogação de prazo para defesa de dissertação possui a opção de prorrogar a bolsa de estudos: No último mês de vigência da bolsa de estudos, será consultada a lista de espera por bolsas do PPGCCS. A solicitação de prorrogação de bolsa só será analisada caso não existam candidatos na fila de espera e se a vigência da bolsa não tiver atingido 24 meses. Neste caso, se a prorrogação for atendida, o novo prazo de vigência da bolsa será estendido para coincidir com a data prevista da defesa ou até que se complete 24 meses de bolsa (o que ocorrer primeiro);
 \\ \hline 
%3 &
(V) Homologado o Relatório de Defesa de Dissertação da aluna Talita dos Reis Lopes Berbel, orientada pela Profa. Dra. Sahudy Montenegro González e do aluno Tiago Vanderlei de Arruda, orientado pela Profa. Dra. Tiemi Christine Sakata; (VI) Aprovado o pedido de agendamento de exame de defesa de dissertação do aluno Marcus Vinicius Sandri, orientado pelo Prof. Dr. Fábio Luciano Verdi com participação do Prof. Dr. Christian Rodolfo Esteve Rothemberg da UNICAMP e da Profa. Dra. Yeda Regina Venturini como membros da banca examinadora;
 \\ \hline 
%4 &
(III) aprovado o pedido de prorrogação de prazo para defesa de dissertação do aluno Tiago Pasqualini da Silva, orientado pelo Prof. Dr. Tiago Agostinho de Almeida por mais 06 meses com prazo final para setembro de 2015;
 \\ \hline 
%5 &
(III) Homologados os relatórios de defesa de dissertação dos alunos Anderson Parra de Paula, Marcus Vinícius Sandri, Rogério Colpani e Wilton Moreira Ferraz Junior.

 \\ \hline 
%6 &
%(I.IV) A Profa. Dra. Katti Faceli apresenta a disciplina Inteligência Artificial.
% \\ \hline 
%7 &
%(I.VII) Prof. Dr. Siovani Cintra Felipussi apresenta as disciplinas Noções Básicas de Economia, Gestão de Pequenas Empresas, Probabilidade e Estatística e Geometria Analítica e Álgebra Linear.
% \\ \hline 
%8 &
%(I.VIII) Após discussão sobre os objetivos, ementas, distribuição de créditos e requisitos, as fichas de caracterização para o perfil 3, a serem oferecidas no primeiro período letivo de 2009, foram aprovadas como segue: Disciplina CR Teo Lab CH Introdução aos Sistemas de Informação 04 04 00 60 Algoritmos e Complexidade 04 03 01 60 Inteligência Artificial 04 03 01 60 Arquitetura e Organização de Computadores 04 04 00 60 Estrutura de Dados 2 04 03 01 60 Noções Básicas de Economia 02 02 00 30 Gestão de Pequenas Empresas 02 02 00 30 Probabilidade e Estatística 04 04 00 60 Geometria Analítica e Álgebra Linear 04 04 00 60 A disciplina Geometria Analítica e Álgebra Linear teve sua ficha de caracterização aprovada com ressalva.
% \\ \hline 
%9 &
%Após discussão o espelhamento é aprovado por unanimidade como segue: Disciplina CR Teo Lab CH Algoritmos e Programação de Computadores 08 04 04 120 Cálculo Diferencial e Integral 1 04 03 01 60 Física para Computação 04 03 01 60 Geometria Analítica e Álgebra Linear 04 04 00 60 Lógica para Computação 04 03 01 60 (III) DISCUSSÃO SOBRE O REAPROVEITAMENTO DE VAGAS: (II.I) O presidente discorre brevemente sobre a distribuição de vagas já discutida em reunião anterior.
% \\ \hline 
%10 &
%A avaliação dos planos de ensino de cada disciplina oferecida no curso de Ciência da Computação foi atribuída para cada professor, as avaliações foram devidamente realizadas e os pareceres efetuados pelos respectivos professores bem como os planos de ensino na íntegra foram submetidos à avaliação do conselho que decidiu por unanimidade.
% \\ \hline 

\end{longtable} 


