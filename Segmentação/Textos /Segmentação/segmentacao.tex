
\documentclass{sig-alternate-05-2015}
\usepackage[portuguese]{babel}
%\usepackage{multirow}
%\usepackage{adjustbox}
%\usepackage{graphicx}
%\usepackage{array}
%\usepackage{tabulary}
% \usepackage{pgfplotstable}
% recommended:
%\usepackage{booktabs}
%\usepackage{array}
%\usepackage{colortbl}
%\usepackage{emptypage}
%\newcolumntype{l}[1]{>{\centering\arraybackslash}p{#1}}
% \usepackage{showframe}   %% just for demo
% \usepackage[brazilian,hyperpageref]{backref}	 % Paginas com as citações na bibl
% \usepackage[alf]{abntex2cite}	% Citações padrão ABNT
\usepackage[utf8]{inputenc}		% Codificacao do documento (conversão automática dos acentos)

\usepackage{rotating}
\usepackage{tabularx}
\usepackage{multicol}

\begin{document}

\title{Segmentação topical automática de atas de reunião}


\numberofauthors{1} 

\author{
\alignauthor Ovídio José Francisco\\
       \email{ovidiojf@gmail.com}
}

\maketitle

%\begin{abstract}   
%
%\end{abstract}

\section*{RESUMO}



\keywords{}

\begingroup
\let\clearpage\relax

\section{Introdução}
	\label{sec:introducao}
	Frequentemente atas de reunião tem a característica de apresentar um texto com poucas quebras de parágrafo e sem marcações de estrutura, como capítulos, seções ou quaisquer indicações sobre o tema do texto. 


% Definição 

A tarefa de segmentação textual consiste dividir um texto em partes que contenham um significado relativamente independente. Em outras palavras, é identificar as posições onde há uma mudança significativa de tópicos.

% Usos
É útil em aplicações que trabalham com textos sem quebras de assunto, ou seja, não apresentam parágrafos, seções ou capítulos, como transcrições automáticas de áudio e grandes documentos que contêm assuntos não idênticos como atas de reunião e noticias.


% Interesses
O interesse por segmentação textual tem crescido em em aplicações voltadas a recuperação de informação %citar o [15] ...
e sumarização de textos. % ... e [2] do "Efficient Linear T S"
Essa técnica pode ser usada para aprimorar o acesso a informação quando essa é solicitada por um usuário por meio de uma consulta, onde é possível oferecer porções menores de texto mais relevante ao invés de exibir um documento maior que pode conter informações menos pertinente. A sumarização de texto também pode ser aprimorada ao processar segmentos separados por tópicos ao invés de documentos inteiros.




% As Atas
Assim, esse trabalho trata da adaptação e avaliação de algoritmos tradicionais ao contexto de documentos em português do Brasil, com ênfase especial nas atas de reuniões.


%As atas, como frequentemente são, apresentam-se como uma sucessão de tópicos. Assim, o objetivo desse trabalho é identificar, automaticamente, onde há a mudança de um tópico para seus adjacentes.


% Diversas aplicações fazem uso de segmentação textual, incluindo 

% Entre as principais mais frequentes de segmentação textual estão a tra



%É principalmente utilizada em aplicações que processam textos longos como transcrições de áudio e documentos longos, além de aprimorar técnicas de sumarização e information retrievel.



% Usos:
%	* quando não há identificações
%	* em transcrições de áudio
%	* em documentos longos
% 	* text summarization (ver a referencia [2] de Efficient Linear Text...)




%Isto é, dado um texto, identificar onde há mudança de tópicos.


% Interest in automatic text segmentation has blossomed over the last few years, with applications ranging from information retrieval to text summariza-tion to story segmentation of video feeds. [A Critique and Improvement of an Evaluation Metric for Text Segmentation]



%Em outras palavras é identificar divisões entre unidade de informação sucessivas

%A tarefa de segmentação textual consiste em encontrar pontos onde há mudança de tópicos no texto.



%[ The task of linear text segmentation is to split a large text document into shorter fragments, usually blocks of consecutive sentences. ]


% **Segmentação é identificar divisiões entre unidades de informação sucessivas (Beeferman, Berger, and Lafferty (1997))**

%   [Text segmentation is the task of determining the positions at which topics change in a stream of text]






\section{Trabalhos Relacionados}
	\label{sec:trabalhos}
	% Ideia básica dos algorítmos (Coesão léxica ) como **presuposto básico**

Os principais algoritmos de segmentação textual baseiam-se na ideia de coesão léxica entre assuntos. Isto é, a mudança de tópicos é acompanhada de uma proporcional mudança de vocabulário. A partir disso, vários algoritmos foram propostos. Dessa forma, assumem o pressuposto que um segmento pode ser identificado e delimitado pela análise das palavras que o compõe.







% A coesão léxica é um termômetro para as mudanças de tópicos, e portanto, um indicador para quebras de segmento.

 
 
% Nesse artigo, os principais serão analisados na perspectiva de atas de reunião.


%Os principais algoritmos de segmentação textual assumem o pressuposto que um segmento pode ser identificado e delimitado pela análise de seu vocabulário





%Os entre os principais trabalhos relacionados a segmentação textual estão o \textit{TextTiling} e o \textit{C99}



%\subsubsection{TextTiling}
%	O algoritmo TextTiling, proposto por 
	
%
%\subsubsection{C99}



Entre os mais influentes podemos citar o \textit{TextTiling}~\cite{Hearst1994} 




Semelhante a esse trabalho, outras abordagens foram propostas como ...

\cite{Banerjee200657} faz uma adaptação do \textit{TextTiling} ao contexto das conversas em reuniões com múltiplos participantes.  



%%%%%%%
% C99 %
%%%%%%%

Choi \cite{Choi2000} apresenta um trabalho que usa \textit{cosine} como medida similaridade e apresenta um esquema de ranking em seu algoritmo, o C99.
%
Embora muitos dos melhores trabalho utilizarem matrizes de similaridades, o autor traz obervações.
%
Ele aponta que para pequenos segmentos, o cálculo de suas similaridades não é confiável. Pois uma ocorrência adicional de uma palavra causa um impacto desproporcional no cálculo.
%
Além disso, o estilo da escrita pode não ser constante em todo o texto. Choi sugere que, por exemplo, textos iniciais dedicados a introdução costumam apresentar menor coesão do que trechos dedicados a um tópico específico. Portanto comparar a similaridade entre trechos de diferentes regiões, não é apropriado.
% Complexidade O(n²)
Devido a isso, as similaridades não podem ser comparadas em valores absolutos. O autor apresenta um esquema de ranking para contornar esse problema.

Cada valor na matriz similaridade é substituída por seu ranking local. O ranking é o número de elementos vizinhos com similaridade menor, conforme a imagem abaixo.


\begin{equation}
Sim(x,y) = \frac
{\Sigma_j f_{x,j} \times f_{y,j}}
{\sqrt{\Sigma_j f^2_{x,j} \times \Sigma f^2_{x,j}}}
\end{equation}



\begin{equation}
r(x,y) = \frac
{Numero\ de\ elementos\ com\ similaridade\ menor}
{Numero\ de\ elementos\ examinados}
\end{equation}














\section{Análise dos Resultados}
	\label{sec:resultados}
	\section{Análise dos Resultados}
	\label{sec:resultados}





	
\section{TextTilingBR}


\section{Conclusão}
	\label{sec:conclusao}
	\section{Conclusão}
	\label{sec:conclusao}
	
	As atas de reunião, objeto de estudo desse artigo, apresentam características peculiares em relação à discursos em reuniões e textos em geral. Características como segmentos curtos e coesão mais fraca devida ao estilo que evita repetição de palavras e ideias em benefício da leitura por humanos, dificulta o processamento por computadores.

	Os algoritmos \textit{TextTiling} e \textit{C99} foram testados em um conjunto de atas reais coletadas do departamento de computação da UFSCar-Sorocaba. Por meio da análise dos dados chegou-se a um modelo cujos segmentos melhor se aproximaram as amostras de participantes das reuniões. Obteve-se resultados comparáveis aos vistos em discursos longos, o que pode ser justificado pelo estilo peculiar de escrita.	
	
	Em trabalhos futuros, serão investigadas técnicas para descrever os segmentos e com isso aprimorar o acesso ao conteúdo das atas de reunião.

\endgroup



\bibliographystyle{abbrv}
\bibliography{sigproc}  % sigproc.bib is the name of the Bibliography in this case
	
\pagestyle{empty}
 	\label{sec:anexo}
 	\include{project/anexo}
\end{document}
