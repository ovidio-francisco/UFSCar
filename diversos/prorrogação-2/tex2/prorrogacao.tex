
\documentclass[12pt]{article}

%\useoutertheme{infolines}
%\useinnertheme{circles}
\usepackage[utf8]{inputenc}
\usepackage{amsmath}
\usepackage{amsfonts}
\usepackage{amssymb}
% \usepackage{graphicx}
\usepackage{graphicx,epsfig}

\usepackage{textpos} 
\usepackage{listings}
\usepackage{hyperref}
\usepackage{multirow}
\usepackage{multicol}

\newcommand{\logoSENAC}{images/etec-small.png}
\newcommand{\logoUFSCar}{images/LogoUfscar.eps}
%\logo{\includegraphics[width=0.18\textwidth]{images/logoFATEC.png}\vspace{220pt}}


\setbeamertemplate{headline}{%
%\leavevmode%
%  \hbox{%
%    \begin{beamercolorbox}[wd=\paperwidth,ht=2.5ex,dp=1.125ex]{palette quaternary}%
%    \insertsectionnavigationhorizontal{\paperwidth}{}{\hskip0pt plus1filll}
%    \end{beamercolorbox}%
%  }

}

\addtobeamertemplate{frametitle}{}{%
	
	\small{
	
    \begin{textblock*}{\textwidth}(0cm,0.1cm)
		%\insertsectionnavigationhorizontal{\paperwidth}{}{\hskip0pt plus1filll}
%		\textcolor{gray}{\insertsectionhead\insertsubsectionhead}
		 
		
		%\insertsection \insertsubsection
    \end{textblock*}
	}
    




	% \begin{textblock*}{100mm}(0.93\textwidth,-0.55cm)
        % \includegraphics[width=0.15\textwidth]{\logoUFSCar}
    % \end{textblock*}




%    \begin{textblock*}{100mm}(0.93\textwidth,-0.75cm)
%        \includegraphics[width=0.1\textwidth]{\logoSENAC}
%    \end{textblock*}
}


\institute[]{
  %\inst{1}
  \includegraphics[width=0.25\textwidth]{\logoUFSCar}
} 

\setbeamercovered{transparent} 
\setbeamertemplate{navigation symbols}{} 
\setbeamercovered{invisible}




% \setbeamertemplate{footline}{% 
  % \hfill% 
  % \usebeamercolor[fg]{page number in head/foot}% 
  % \usebeamerfont{page number in head/foot}% 
  % \insertframenumber%
  % %\,/\,\inserttotalframenumber
  % \kern1em\vskip2pt% 
% }




% https://tex.stackexchange.com/questions/333587/beamer-frame-number-without-total/333684
\setbeamertemplate{footline}{
  \hfill%
  \usebeamercolor[fg]{page number in head/foot}%
  \usebeamerfont{page number in head/foot}%
  \insertframenumber%
  % \setbeamertemplate{page number in head/foot}[framenumber]%
  \usebeamertemplate*{page number in head/foot}\kern1em\vskip5pt%
}



%gets rid of bottom navigation bars
% \setbeamertemplate{footline}[frame number]{}
% \setbeamertemplate{footline}{\insertframenumber }



% \setbeamertemplate {footline}{\quad\hfill\insertframenumber\strut\quad}




% \setbeamertemplate{sidebar right}{}
% \setbeamertemplate{footline}{%
% \hfill\usebeamertemplate***{navigation symbols}
% \hspace{1cm}\insertframenumber{}}



% Define o caminho das figuras
\graphicspath{{images/}}

%\AtBeginSection[] 
%{
%    \begin{frame}
%	    \frametitle{Roteiro}
%	    \tableofcontents[currentsection]
%    \end{frame}
%}


%https://bloerg.net/2012/06/21/customizing-the-frametitle-of-beamer-presentation.html




\AtBeginSection[] 
{
    \begin{frame}
	    \frametitle{Roteiro}
	    \tableofcontents[currentsection]
    \end{frame}
}












%%%%%%%%%%%%%%%%%%%%%%%%%%%%%%%%%%%%%%%%%%%%%%%%%%%%%%%%%%%%%%%%%%%%%%%%%%%%%%%%%%%%
%%%%%%%%%%%%%%%%%%%%%%%%%%%%%%%%%%%%%%%%%%%%%%%%%%%%%%%%%%%%%%%%%%%%%%%%%%%%%%%%%%%%


\begin{document}

		\thispagestyle{empty}
	\singlespacing

	\noindent
	\ovalbox{

		\begin{minipage}{0.8\linewidth}
			{\Large\bf Universidade Federal de São Carlos}\\
			{\sc  Programa de Pós-Graduação em Ciência da Computação}\\
			{\sc Campus Sorocaba}
		\end{minipage} 
		\hfill 
		
		\begin{minipage}{0.19\linewidth}
				\epsfxsize=2.50cm
				\centerline{\epsffile{ufscar_logo.eps}}
		\end{minipage}
	}

	\vi
	\vi
	\vi

	\begin{center}
		\Large Solicitação de Prorrogação de Prazo de Defesa\\
		\vi
		\large Relatório de Atividades Desenvolvidas

	\end{center}
	\vi
	
	
% - Artigo ENIAC
% - Softwares
% 	- Ferramenta para segmentação textual automática
%		- Ajustes no preprocessamento
%		- Visualização
%		- Configuração de parâmetros com base em testes estatísticos
%	- Ferramenta para segmentação e rotulação manual
%	- Ferramenta para segmentação e extração de tópicos 
% - Revisão Bibliográfica 
% 	- Segmentação textual
%	- 
% - Experimento com especialistas

\newcommand{\urlatas}{http://www.ppgccs.net/?page\_id=1150}

\newcommand{\urleniac}{http://www.bracis2017.ufu.br/eniac-encontro-nacional-de-inteligencia-artificial-e-computacional\#tab-0-0}

Aprovado como aluno regular em agosto de 2015, concluí as disciplinas exigidas para obtenção dos créditos e fui qualificado em agosto do ano seguinte. 

O objetivo do trabalho é aplicar técnicas de mineração de texto em atas de reunião para fornecer históricos de menções a assuntos de interesse por meio de um sistema de busca. O desenvolvimento envolve basicamente 
o tratamento de uma base de documentos e a identificação dos assuntos nela contidos por meio de técnicas de segmentação textual e extração de tópicos. Por meio de uma interface, o sistema receberá uma consulta do usuário para proceder a busca e devolver um histórico de menções relacionadas a intenção do usuário.


% Identificar assuntos relevantes em atas de reunião
% Identificar mudanças de assuntos e identificar do que se trata o assunto.
% Receber consultas dos usuários
% Envolve:
	% Preprocessamento
	% Segmentação textual
	% Extração de toópicos
%

Segue as principais atividades desenvolvidas:

\begin{description}
	
	 
	\item[Segmentação e rotulação manual:] A fim de obter referências para os algoritmos de segmentação e  extração de tópicos, foram obtidas atas de reuniões do Conselhos de Pós Graduação, Conselho de Cursos e Conselho de Departamento de da UFSCar-Sorocaba. Um conjunto de seis documentos foi oferecidos à profissionais que participam de reuniões desse departamento para dividir cada documento em segmentos e rotulá-los. Para isso, desenvolveu-se um \textit{software} que permitiu aos voluntários visualizar um documento, selecionar o texto correspondente a um assunto e removê-lo para então indicar quais palavras melhor indicavam o tópico. O software possibilitou a segmentação e rotulação das atas e ao final gerou-se um aquivo contendo os dados coletados, o qual foi tratado e serviu como referência para a avaliação dos algoritmos de segmentação.
	Os arquivos gerados foram tratados para remoção de ruídos e ajustes para que os segmentos sempre terminem em uma sentença reconhecida pelo algoritmo, uma vez que as medidas de avaliação de segmentos recebem sentenças como unidade mínima de informação.
	
	

	\item[Ferramentas:] Como parte do sistema proposto, desenvolveu-se uma ferramenta que visa a segmentação automática de documentos permitindo a configuração do preprocessamento, a detecção de sentenças e os parâmetros dos algoritmos de segmentação. Oferece uma interface gráfica para configuração e visualização dos segmentos extraídos bem como as \textit{features} selecionadas na etapa de preprocessamento. 
Os algoritmos de segmentação foram avaliados comparando-os com atas segmentadas manualmente por participantes das reuniões. Por meio de análises estatísticas chegou-se ao modelo que melhor segmenta os textos no contexto das atas de reunião. 		
	A biblioteca desenvolvida para a ferramenta está disponibilizada para estudo e pode ser aproveitada na etapa de preprocessamento em outros projetos.
% Utilizada nos testes.	

	
	\item[Sistema proposto:] Atualmente o sistema proposto recebe um conjunto de documentos divide cada um em segmentos com assunto relativamente independente. Em seguida, utiliza técnicas de extração de tópicos para fornecer as palavras que podem melhor representar o tópico de cada segmento. Para isso, possui a implementação das principais técnicas de extração de tópicos como LDA (\textit{Latent Dirichlet Allocation}) e PLSA (\textit{Probabilistic Latent Semantic Analysis}). Após a extração, os segmentos rotulados ficam disponíveis para consulta ou armazenamento. A segmentação dos documentos e a extração de tópicos permite a identificação trechos com um assunto de interesse.
	
	
	
	\item[Submissão de Artigo ao ENIAC:] Como parte da revisão bibliográfica e desenvolvimento do sistema proposto, a segmentação automática das atas resultou em um artigo submetido ao ENIAC -- Encontro Nacional de Inteligência Artificial e Computacional. 
%	
No artigo são descritos as principais técnicas empregadas nessa tarefa, bem como métodos mais utilizados para avaliação de segmentadores. Mostra também como o pré-processamento e configurações de parâmetros podem ser ajustados para esse tipo de documento. Para isso, utilizou-se a segmentação manual como referência de segmentação ideal a qual foi comparada com os segmentos extraídos automaticamente. Por meio de testes estatísticos descritos no artigo, chegou-se a um modelo que melhor identifica mudanças de assunto no contexto das atas.
	
	
\end{description}







%	\item[Sistema proposto:] Iniciou-se o desenvolvimento de um  sistema que é capaz de receber um conjunto de atas, segmentá-las em trechos um assunto relativamente independente e fornecer, por meio de técnicas de extração tópicos, as palavras que podem representar o assunto de cada trecho. 



% onde os voluntários segmentaram e rotularam o texto. O que resultou em um conjunto de referências para avaliação das ferramentas propostas.

%	\item[Revisão bibliográfica:] Foi iniciada uma revisão bibliográfica sobre as principais técnicas de mineração textual como extração automática de tópicos e extração de conhecimento em documentos com vistas ao desenvolvimento do sistema proposto.


%	\item[Redação da dissertação: ] A dissertação contém os primeiros textos introdutórios referentes à mineração de textos, segmentação textos, extração de tópicos e resultados obtidos com as ferramentas já desenvolvidas.












	\newpage
		\thispagestyle{empty}
	\singlespacing

	\noindent
	\ovalbox{

		\begin{minipage}{0.8\linewidth}
			{\Large\bf Universidade Federal de São Carlos}\\
			{\sc  Programa de Pós-Graduação em Ciência da Computação}\\
			{\sc Campus Sorocaba}
		\end{minipage} 
		\hfill 
		
		\begin{minipage}{0.19\linewidth}
				\epsfxsize=2.50cm
				\centerline{\epsffile{ufscar_logo.eps}}
		\end{minipage}
	}

	\vi
	\vi
	\vi

	\begin{center}
		\Large Solicitação de Prorrogação de Prazo de Defesa\\
		\vi
		\large Plano de atividades e cronograma

	\end{center}
	\vi
	
% Plano de atividades para o período estendido e cronograma (até 1 página);	


Para a conclusão dos trabalhos e defesa, são necessárias as seguintes tarefas:

\begin{description}

	\item[1 - Módulo de Preparação e Manutenção:] Deve receber uma coleção de documentos, realizar a etapa de  preprocessamento que entregará cada texto dividido em sub-documentos. Em seguida, cada sub-documento deve receber rótulos os quais irão compor uma estrutura de dados que será utilizada para consulta.
%	
Para isso, serão empregadas técnicas de segmentação textual como BayesSeg e técnicas de extração de tópicos como o LDA (\textit{Latent Dirichlet Allocation}) e PLSA (\textit{Probabilistic Latent Semantic Analysis}) para atribuir tópicos aos documentos.


%	Também terá a responsabilidade de incorporar novos documentos à estrutura de dados, bem como recalibrar os algoritmos.
	
		
	\item[2 - Módulo de Busca:] O sistema deve permitir ao usuário buscar por menções sobre um assunto, bem como encontrar e navegar pelos documentos. Assim, o sistema necessita de um módulo de busca por assuntos. Nesse módulo, a \textit{string} de entrada do usuário deve ser tratada e comparada à base de dados dos documentos já processados, para então exibir ao usuário os trechos correspondentes à busca. 
	%
Deve ainda ser implementada a busca aproveitando o agrupamento dos sub-documentos em tópicos. Para isso, serão empregadas as técnicas de recuperação de informação e
extração de tópicos da literatura para rankear e agrupar os sub-documentos.

	
	
	\item[3 - Avaliação do Sistema:] O sistema final deve ser avaliado junto a usuários a fim de avaliar a eficiência do sistema em suas respostas bem como funcionalidades do ponto de vista de experiência do usuário. 
%
Para isso, novamente será necessário a ajuda de voluntários que se enquadram no perfil de usuários alvo.
%	
	Após avaliação, uma eventual otimização das técnicas deve ser considerada para o aprimoramento das ferramentas.
	
	
	\item[4 - Conclusão da Dissertação:] Redigir o texto da dissertação.
	
		

\end{description}



 \begin{center}
     \begin{longtable}{|c|c|c|c|c|c|c|c|c|c|c|c|}
     \caption{Cronograma das atividades}\label{table:cronograma}\\ \hline
   
     Atividades & 
     Fevereiro  & 
     Março      & 
     Abril      & 
     Maio       & 
     Junho      &
     Julho    
     \\ \hline
 	
%         A   S   O   N   D   J  	
     1  & X & X &   &   &   &   \\ \hline
     2  &   & X & X & X &   &   \\ \hline
     3  &   &   &   & X & X &   \\ \hline
     4  &   &   & X & X & X &   \\ \hline

     \end{longtable}
 \end{center}





%	\item[2 - Módulo de Classificação:] Os segmentos extraídos devem ser automaticamente classificados em, por exemplo, \emph{decisão}, \emph{menção}, \emph{irrelevante}. %, a fim de 
%Para isso, a partir de um conjunto de treinamento, será construído um modelo de classificação o qual irá induzir uma função capaz de identificar se um documento pertence a determinada classe.

%	\item[Otimizações do Sistema:] 


%	À medida que novos documentos são inseridos ao sistema, faz-se necessário rotinas para a inserção desses na base dados, bem como recalibragem dos algoritmos. % ver com rafael se o extrator de tópicos carece de ser rodado novamente na base toda.
% ver com o rafael se as palavras que extrator fornece podem ser chamadas de rótulos.
	


%Para isso, um classificador deve ser treinado com os rótulos obtidos com os participantes das reuniões.



%\begin{description}

	
%	\item[Implementação do sistema:] O módulo de extração de tópicos necessita de otimizações quanto quanta a eficiência. O módulo de pesquisa necessita de uma interface com o usuário. O sistema após concluído deve ser avaliado comparando-o com a rotulação manual fornecida pelos voluntários.
	
%	\item[Avaliação:] 
	

%\end{description}



%Tarefas como a segmentação textual e a extração de tópicos demandam estudo mais aprofundado. A implementação de ferramentas para teste e rotulação exigiram tempo exta para seu desenvolvimento. 


%	\item[1 - Revisão bibliográfica:] A literatura referente a extração de tópicos deve ser estudada com maior profundidade a fim de aperfeiçoar a dissertação e obter referências de técnicas que compõem o estado-da-arte nesse assunto. %--> outras alternativas surgiram segmentadores que extraem tópicos


%	\item[3 - Conclusão da dissertação:] Os resultados obtidos com a versão final do sistema devem ser incorporados ao texto da dissertação bem como uma revisão da literatura.






	
	\newpage
	
	
	
	
	
	
%	Atualmente o sistema proposto recebe um conjunto de documentos divide cada um em segmentos e fornece as palavras que podem melhor representar o tópico de cada segmento.	
%Deve ainda permitir ao usuário buscar por menções a decisões sobre um assunto. Bem como encontrar e navegar pelos documentos.
%Para isso, o sistema necessita de um módulo de busca por assuntos. Nesse módulo, a string de entrada do usuário deve ser tratada e comparada à uma base de dados dos documentos já processados. 	A base de dados deve ser construída a partir de documentos originais e por meio de técnicas de segmentação, extração de tópicos e classificação, irá conter dados que descrevem os assuntos mais relevantes do texto.
%As técnicas utilizadas devem ser avaliadas junto aos rótulos fornecidos pelo classificador e extrator, obtendo assim uma perspectiva dessas técnicas no contexto das atas. O sistema final deve ser experimentado por voluntários a fim de avaliar sua eficiência.
	
\end{document}

