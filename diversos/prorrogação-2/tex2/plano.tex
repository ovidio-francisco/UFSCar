	\thispagestyle{empty}
	\singlespacing

	\noindent
	\ovalbox{

		\begin{minipage}{0.8\linewidth}
			{\Large\bf Universidade Federal de São Carlos}\\
			{\sc  Programa de Pós-Graduação em Ciência da Computação}\\
			{\sc Campus Sorocaba}
		\end{minipage} 
		\hfill 
		
		\begin{minipage}{0.19\linewidth}
				\epsfxsize=2.50cm
				\centerline{\epsffile{ufscar_logo.eps}}
		\end{minipage}
	}

	\vi
	\vi
	\vi

	\begin{center}
		\Large Solicitação de Prorrogação de Prazo de Defesa\\
		\vi
		\large Plano de atividades e cronograma

	\end{center}
	\vi
	
% Plano de atividades para o período estendido e cronograma (até 1 página);	


Para a conclusão dos trabalhos e defesa, são necessárias as seguintes tarefas:

\begin{description}

	\item[1 - Módulo de Preparação e Manutenção:] Deve receber uma coleção de documentos, realizar a etapa de  preprocessamento que entregará cada texto dividido em sub-documentos. Em seguida, cada sub-documento deve receber rótulos os quais irão compor uma estrutura de dados que será utilizada para consulta.
%	
Para isso, serão empregadas técnicas de segmentação textual como BayesSeg e técnicas de extração de tópicos como o LDA (\textit{Latent Dirichlet Allocation}) e PLSA (\textit{Probabilistic Latent Semantic Analysis}) para atribuir tópicos aos documentos.


%	Também terá a responsabilidade de incorporar novos documentos à estrutura de dados, bem como recalibrar os algoritmos.
	
		
	\item[2 - Módulo de Busca:] O sistema deve permitir ao usuário buscar por menções sobre um assunto, bem como encontrar e navegar pelos documentos. Assim, o sistema necessita de um módulo de busca por assuntos. Nesse módulo, a \textit{string} de entrada do usuário deve ser tratada e comparada à base de dados dos documentos já processados, para então exibir ao usuário os trechos correspondentes à busca. 
	%
Deve ainda ser implementada a busca aproveitando o agrupamento dos sub-documentos em tópicos. Para isso, serão empregadas as técnicas de recuperação de informação e
extração de tópicos da literatura para rankear e agrupar os sub-documentos.

	
	
	\item[3 - Avaliação do Sistema:] O sistema final deve ser avaliado junto a usuários a fim de avaliar a eficiência do sistema em suas respostas bem como funcionalidades do ponto de vista de experiência do usuário. 
%
Para isso, novamente será necessário a ajuda de voluntários que se enquadram no perfil de usuários alvo.
%	
	Após avaliação, uma eventual otimização das técnicas deve ser considerada para o aprimoramento das ferramentas.
	
	
	\item[4 - Conclusão da Dissertação:] Redigir o texto da dissertação.
	
		

\end{description}



 \begin{center}
     \begin{longtable}{|c|c|c|c|c|c|c|c|c|c|c|c|}
     \caption{Cronograma das atividades}\label{table:cronograma}\\ \hline
   
     Atividades & 
     Fevereiro  & 
     Março      & 
     Abril      & 
     Maio       & 
     Junho      &
     Julho    
     \\ \hline
 	
%         A   S   O   N   D   J  	
     1  & X & X &   &   &   &   \\ \hline
     2  &   & X & X & X &   &   \\ \hline
     3  &   &   &   & X & X &   \\ \hline
     4  &   &   & X & X & X &   \\ \hline

     \end{longtable}
 \end{center}





%	\item[2 - Módulo de Classificação:] Os segmentos extraídos devem ser automaticamente classificados em, por exemplo, \emph{decisão}, \emph{menção}, \emph{irrelevante}. %, a fim de 
%Para isso, a partir de um conjunto de treinamento, será construído um modelo de classificação o qual irá induzir uma função capaz de identificar se um documento pertence a determinada classe.

%	\item[Otimizações do Sistema:] 


%	À medida que novos documentos são inseridos ao sistema, faz-se necessário rotinas para a inserção desses na base dados, bem como recalibragem dos algoritmos. % ver com rafael se o extrator de tópicos carece de ser rodado novamente na base toda.
% ver com o rafael se as palavras que extrator fornece podem ser chamadas de rótulos.
	


%Para isso, um classificador deve ser treinado com os rótulos obtidos com os participantes das reuniões.



%\begin{description}

	
%	\item[Implementação do sistema:] O módulo de extração de tópicos necessita de otimizações quanto quanta a eficiência. O módulo de pesquisa necessita de uma interface com o usuário. O sistema após concluído deve ser avaliado comparando-o com a rotulação manual fornecida pelos voluntários.
	
%	\item[Avaliação:] 
	

%\end{description}



%Tarefas como a segmentação textual e a extração de tópicos demandam estudo mais aprofundado. A implementação de ferramentas para teste e rotulação exigiram tempo exta para seu desenvolvimento. 


%	\item[1 - Revisão bibliográfica:] A literatura referente a extração de tópicos deve ser estudada com maior profundidade a fim de aperfeiçoar a dissertação e obter referências de técnicas que compõem o estado-da-arte nesse assunto. %--> outras alternativas surgiram segmentadores que extraem tópicos


%	\item[3 - Conclusão da dissertação:] Os resultados obtidos com a versão final do sistema devem ser incorporados ao texto da dissertação bem como uma revisão da literatura.





