Uma vez a coesão léxica é pressuposto de muitas abordagens em segmentação textual, fez-se uma análise desses documentos quantoa a similaridade dos termos ao longo do texto. Verificou-se que a técnica de janelas deslizantes empregada pelo TextTiling encontra os vales que indicam transições entre segmentos, contudo ao comparar esses vales com a segmentação de referência, nota-se que a maioria dos limites coincide  ou estão próximos aos vales, porém há casos onde a referência indica limites em trechos com alta coesão léxica e outros onde a queda da coesão, indicada por vales, não conincide nenhum limite de referência. 

O que sugere que  

os segmentos apresentados por esse algoritmo 