 \documentclass[10pt,a4paper]{article}
 \usepackage[utf8]{inputenc}
 \usepackage{amsmath}
 \usepackage{amsfonts}
 \usepackage{amssymb}
 \begin{document}

 
 
O modelo probabilístico é baseado no princípio da ordenação probabilística (\textit{Probability Ranking Principle}) onde dada um consulta $q$ e um documento $d_j$ relevante a $q$, o modelo tenta estimar a probabilidade do usuário encontrar o documento $d_j$. O modelo assumente que para uma consulta $q$ há um conjunto de documentos $R$ que contém exatamente os documentos relevantes e nenhum outro, sendo este um conjunto resposta ideal que maximiza a probabilidade do usuário encontrar um documento $d_j$ relevante a $q$. 

Seja $\overline{R_q}$ o complemento de $R$ de forma que $\overline{R_q}$ contém todos os documentos não relevantes à consulta $q$. 
Seja $P(R_q|d_j)$ a probabilidade do documento $d_j$ ser relevante à consulta $q$ e $P(\overline{R_q}|d_j)$ a probabilidade de $d_j$ não ser relevante à $q$. 



\newcommand{\piRel} {P(k_i|R_q)}
\newcommand{\piNRel}{P(k_i|\overline{R_q})}
\newcommand{\termoi}{k_i}

Seja $p_i = \piRel$ a probabilidade do termo $\termoi$ ocorrer em um documento relevante à consulta $q$, 
e $s_i    = \piNRel$ a probabilidade do termo $\termoi$ estar presente em um documento não relevante.



A similaridade entre um documento $d_j$ e uma consulta $q$ é definida por:



\begin{equation}
	sim(d_j, q) = \frac{P(R_q|dj)}{P(\overline{R_q}|dj)} 
%	
	=
%	
	\prod_{i:d_i=1} \frac{p_i}{s_i} 
	\cdot
	\prod_{i:d_i=0} \frac{1 - p_i}{1 - s_i}
	\label{equ:simprob}
\end{equation}



A fim de obter-se uma estimativa numéricas das probabilidades, o modelo probabilístico clássico atribui valores binários aos pesos os quais indicam a presença ou ausência de um termo, isto é, $w_{ij} \in \{0,1\}$ e $w_{iq} \in \{0,1\}$. 
O modelo assume o documento como uma combinação de palavras e seus pesos. 
O modelo também supõe que os termos ocorrem independentemente no documento, ou seja, a ocorrência de um termo não influencia a ocorrência de outro. 
Partindo dessas suposições, a Equação~\ref{equ:} passa por transformações que incluem aplicação da regra de Bayes e simplificações matemáticas, e chega-se a Equação~\ref{equ:} conhecida como equação de Robertson-Spark Jones a qual é considerada a expressão clássica para ranqueamento no modelo probabilístico. Detalhes da dedução dessa equação pode ser encontrada em~\cite{}.


\begin{equation}
	sim(d_j,q) = \sum_{i=1}^{t} w_{i,j} \cdot w_{i,q}  \cdot \sigma_{i,R}
\end{equation}

 \end{document}
 
 
 
 
 
 
 
 
 
 
 
 
 
 
 
 
 
 
 
 
 
 
 