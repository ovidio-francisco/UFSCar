 \documentclass[10pt,a4paper]{article}
 \usepackage[utf8]{inputenc}
 \usepackage{amsmath}
 \usepackage{amsfonts}
 \usepackage{amssymb}
 \begin{document}
 
O modelo probabilístico é baseado no princípio da ordenação probabilística (\textit{Probability Ranking Principle}) onde dada um consulta q e um documento $d_j$ <perfeitamente> relevante a $q$, o modelo tenta estimar a probabiliade do usuário encontrar o documento $d_j$. O modelo assumente que para uma consulta $q$ há um conjunto de documentos $R$ que contém exatamente os documentos relevantes e nenhum outro, sendo este um conjunto resposta ideal que maximiza a probablidade do usuário encontrar um documento $d_j$ relevante a $q$. 

Seja $\overline{R_q}$ o complemento de $R$ de forma que $\bar{R_q}$ contém todos os documentos não relevantes à consulta $q$. Seja $P(R_q|d_j)$ a probabilidade do documento $d_j$ ser relevante à consulta $q$ e $P(\overline{R_q}|d_j)$ a probabilidade de $d_j$ não ser relevante à $q$. A similaridade entre um documento $d_j$ e uma consulta $q$ é definada por:


\begin{equation}
	sim(d_j, q) = \frac{P(R_q|dj}{P(\overline{R_q}|dj}
\end{equation}




 \end{document}