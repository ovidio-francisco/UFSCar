\documentclass[10pt,a4paper]{article}
\usepackage[utf8]{inputenc}
\usepackage{amsmath}
\usepackage{amsfonts}
\usepackage{amssymb}
\begin{document}

% === PK ===

Considerando o conceito de \textit{near misses}, algumas soluções foram propostas. Proposta por~\cite{Beeferman1999}, P$_k$ atribui valores parciais a \textit{near misses}, ou seja, limites sempre receberão um peso proporcional à sua proximidade, desde que dentro de um janela de tamanho~$k$.  Para isso, esse método move uma janela de tamanho $k$ ao longo do texto. A cada passo verifica, na referência e na hipótese, se as extremidades (a primeira e última sentença) da janela estão ou não dentro do mesmo segmento, então, penaliza o algoritmo caso este não concorde com a referência. Ou seja, dado duas palavras de distância $k$, o algoritmo é penalizado caso não concorde com a segmentação de referência se as palavras estão ou não no mesmo segmento. 

Dadas uma segmentação de referência $ref$ e uma segmentação automática $hyp$, ambas com $N$ sentenças, P$_k$ é computada como: 


\begin{equation}
P_k(ref,hyp) = \frac{1}{N - k}
\sum_{i=1}^{N-k } 
(
\delta_{ref}(i, i+k) 
\bar{\oplus}
\delta_{hyp}(i, i+k) 
)
\end{equation}


onde $\delta_S(i,j)$ é a função indicadora que retora 1 se as sentenças i e j estão no mesmo segmento e 0 caso contrário, $\bar{\oplus}$ é o operador \texttt{XNOR} (ou exclusivo) que retorna 1 se ambos os argumentos forem diferentes. 
%
%
O valor de $k$ é calculado como a metade da média dos comprimentos dos segmentos reais. Como resultado, é retornado a
dissimilaridade entre as segmentação calculada pela contagem de discrepâncias divida pela quantidade de segmentações analisadas. Essa medida pode ser interpretada como a probabilidade de duas sentenças extraídas aleatoriamente pertencerem ao mesmo segmento.  


% TODO: Rafael: Não dá pra fazer uma fórmula matemática pra deixar o funcionamento dessa medida mais claro?





\end{document}