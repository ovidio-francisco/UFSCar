\documentclass[10pt,a4paper]{article}
\usepackage[utf8]{inputenc}
\usepackage{amsmath}
\usepackage{amsfonts}
\usepackage{amssymb}
\begin{document}


% \textbf{Abordagens probabilísticas}

% ==========  TextSeg Utiyama and Isahara  ==========


Desenvolveu-se também abordagens probabilísticas para segmentação textual, por exemplo, o método proposto por~\cite{Utiyama2001} encontra a segmentação por meio de um modelo estatístico. Dado um texto representado por um conjunto de palavras 
$W = \{w_1, w_2, \dots, w_n\}$ e um conjunto de segmentos $S = \{s_1, s_2, \dots, s_m\}$ que segmenta $W$, a probabilidade da segmentação S é dada por:

\begin{equation}
	P(S|W) = \frac{P(W|S)P(S)}{P(W)}
\end{equation}

Com isso, é possível encontrar a sequência de segmentos mais provável $\hat{S} = arg max_S~P(W|S) P(S)$. Nesse trabalho assume-se que os segmentos são estaticamente independentes entre si e as palavras nos segmentos são independentes dado o segmento que as contém. Essa simplificação permite decompor o termo $P(W|S)$ em um produtório de ocorrência de das palavras dado um segmento.  

\begin{equation}
	P(W|S) = \prod_{i=1}^m \prod_{j=1}^n P(w_j^i|S_i)
\end{equation}

Onde $P(w_j^i|S_i)$ é a probabilidade da j-ésima palavra ocorrer no segmento $S_i$. Seja $f_i(w_j)$ a frequência da j-ésima palavra no i-ésimo segmento, $n_i$ é o número de palavras em $S_i$ e $k$ é o número de palavas diferentes em $W$. Calcula-se: 

\begin{equation}
	P(w_j^i|S_i) = \frac{f_i(w_j) + 1}{n_i + k}
\end{equation}

A suposição de independência entre segmentos e as palavras neles contidas, são é verificada no mundo real. Para segmentos muito pequenos a estimativa das probabilidades das palavras pode ser afetada, além disso, o modelo não leva em conta a importância relativa das palavras~\cite{Malioutov:2006a}.

\begin{equation}
	NCorte = \frac{corte(A,B)}{vol(A} + \frac{corte(A,B)}{vol(B)}
\end{equation}

 

\end{document}