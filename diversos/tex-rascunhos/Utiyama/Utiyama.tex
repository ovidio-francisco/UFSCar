\documentclass[10pt,a4paper]{article}
\usepackage[utf8]{inputenc}
\usepackage{amsmath}
\usepackage{amsfonts}
\usepackage{amssymb}
\begin{document}

Desenvolveu-se também abordagens probabilísticas para segmentação textual, por exemplo, o método proposto por~\cite{UtiamaXXXX} encontra a segmentação por meio de um modelo estatístico.
%
Dado um texto representado por um conjunto de palavras 
$W = \{w_1, w_2, w_n\}$ e um conjunto de segmentos $S = \{s_1, s_2, s_m\}$ que segmenta $W$, a probabilidade da segmentação S é dada por:





%a maior probabilidade ...


\end{document}